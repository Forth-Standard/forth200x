\chapter{Glossary} % 6

\section{Core words} % 6.1
\wordlist{core}

\begin{info}
\subsection{Core words}
\end{info}

\begin{newword}{10}{!}[store]
	\stack{x a-addr}{}

	Store $x$ at \emph{a-addr}.

\item[See:]
	\xref{usage:aaddr}{3.3.3.1 Address alignment}.
\end{newword}


\begin{newword}[num]{30}{\#}[number-sign]
	\stack{ud$_1$}{ud$_2$}

	Divide \emph{ud}$_1$ by the number in \word{BASE} giving the
	quotient \emph{ud}$_2$ and the remainder $n$. ($n$ is the least
	significant digit of \emph{ud}$_1$.) Convert $n$ to external
	form and add the resulting character to the beginning of the
	pictured numeric output string. An ambiguous condition exists
	if \word{num} executes outside of a \word{num-end} \word{num-start}
	delimited number conversion.

\item[See:]
	\wref{core:num-end}{\#>},
	\wref{core:numS}{\#S},
	\wref{core:num-start}{<\#}.
\end{newword}


\begin{newword}[num-end]{40}{\#>}[number-sign-greater]
	\stack{xd}{c-addr $u$}

	Drop \emph{xd}. Make the pictured numeric output string
	available as a character string. \emph{c-addr} and $u$
	specify the resulting character string. A program may replace
	characters within the string.

\item[See:]
	\wref{core:num}{\#},
	\wref{core:numS}{\#S},
	\wref{core:num-start}{<\#}.
\end{newword}


\begin{newword}[numS]{50}{\#S}[number-sign-s]
	\stack{ud$_1$}{ud$_2$}

	Convert one digit of \emph{ud}$_1$ according to the rule for
	\word{num}. Continue conversion until the quotient is zero.
	\emph{ud}$_2$ is zero. An ambiguous condition exists if
	\word{numS} executes outside of a \word{num-start} \word{num-end}
	delimited number conversion.

\item[See:]
	\wref{core:num}{\#},
	\wref{core:num-end}{\#>},
	\wref{core:num-start}{<\#}.
\end{newword}


\begin{newword}{0070}{'}[tick]
	\stack{``\arg{spaces}name''}{xt}

	Skip leading space delimiters. Parse \emph{name} delimited by
	a space. Find \emph{name} and return \emph{xt}, the execution
	token for \emph{name}. An ambiguous condition exists if
	\emph{name} is not found. When interpreting,
	\texttt{' xyz EXECUTE} is equivalent to \texttt{xyz}.

\item[See:]
	\xref{usage:interpret}{3.4 The Forth text interpreter},
	\xref{usage:parsing}{3.4.1 Parsing},
	\rref{core:POSTPONE}{POSTPONE},
	\rref{core:[']}{[']},
	\xref{diff:immediate}{D.6.7 Immediacy}.

	\begin{rationale} % A.6.1.0070 '
		Typical use: {\ldots} \word{'} \emph{name}.

		Many Forth systems use a state-smart tick. Many do not.
		ANS Forth follows the usage of Forth 83.

		See:
			\xref{rat:interpret}{A.3.4.3..2 Interpretation semantics},
			\rref{core:FIND}{FIND}.
	\end{rationale}
\end{newword}


\begin{newword*}[p]{0080}{(}[paren]
\item[Compilation:]
	Perform the execution semantics given below.

\item[Execution:]
	\stack{``ccc\arg{paren}''}{}

	Parse \emph{ccc} delimited by \texttt{p} (right parenthesis).
	\texttt{p} is an immediate word.

	The number of characters in \emph{ccc} may be zero to the
	number of characters in the parse area.

\item[See:]	
	\xref{usage:parsing}{3.4.1 Parsing},
	\xref{file:p}{11.6.1.0080 (}.

	\begin{rationale} % A.6.1.0080 (
		Typical use: {\ldots} \word{p} \emph{ccc}\texttt{)} {\ldots}
	\end{rationale}
\end{newword*}


\begin{newword}{0090}{*}[star]
	\stack{$n_1$\textbar$u_1$ $n_2$\textbar$u_2$}{$n_3$\textbar$u_3$}

	Multiply $n_1$\textbar$u_1$ by $n_2$\textbar$u_2$ giving the
	product $n_3$\textbar$u_3$.
\end{newword}


\begin{newword}{0100}{*/}[star-slash]
	\stack{$n_1$ $n_2$ $n_3$}{$n_4$}

	Multiply $n_1$ by $n_2$ producing the intermediate double-cell
	result $d$. Divide $d$ by $n_3$ giving the single-cell quotient
	$n_4$. An ambiguous condition exists if $n_3$ is zero or if the
	quotient $n_4$ lies outside the range of a signed number. If $d$
	and $n_3$ differ in sign, the implementation-defined result
	returned will be the same as that returned by either the phrase
	\word{toR} \word{M*} \word{Rfrom} \word{FM/MOD} \word{SWAP}
	\word{DROP} or the phrase \word{toR} \word{M*} \word{Rfrom}
	\word{SM/REM} \word{SWAP} \word{DROP}.

\item[See:]
	\xref{usage:div}{3.2.2.1 Integer division}.
\end{newword}


\begin{newword}{0110}{*/MOD}[star-slash-mod]
	\stack{$n_1$ $n_2$ $n_3$}{$n_4$ $n_5$}

	Multiply $n_1$ by $n_2$ producing the intermediate double-cell
	result $d$. Divide $d$ $n_3$ producing the single-cell remainder
	$n_4$ and the single-cell quotient $n_5$. An ambiguous condition
	exists if $n_3$ is zero, or if the quotient $n_5$ lies outside
	the range of a single-cell signed integer. If $d$ and $n_3$
	differ in sign, the implementation-defined result returned will
	be the same as that returned by either the phrase
	\word{toR} \word{M*} \word{Rfrom} \word{FM/MOD} or the phrase
	\word{toR} \word{M*} \word{Rfrom} \word{SM/REM}.

\item[See:]
	\xref{usage:div}{3.2.2.1 Integer division}.
\end{newword}


\begin{newword}{0120}{+}[plus]
	\stack{$n_1$\textbar$u_1$ $n_2$\textbar$u_2$}{$n_3$\textbar$u_3$}

	Add $n_2$\textbar$u_2$ to $n_1$\textbar$u_1$, giving the sum
	$n_3$\textbar$u_3$.

\item[See:]
	\xref{usage:aaddr}{3.3.3.1 Address alignment}.
\end{newword}


\begin{newword}{0130}{+!}[plus-store]
	\stack{$n$\textbar$u$ a-addr}{}

	Add $n$\textbar$u$ to the single-cell number at \emph{a-addr}.

\item[See:]
	\xref{usage:aaddr}{3.3.3.1 Address alignment}.
\end{newword}


\begin{newword*}{0140}{+LOOP}[plus-loop]
\item[Interpretation:]
	Interpretation semantics for this word are undefined.

\item[Compilation:]
	\stack[C]{do-sys}{}

	Append the run-time semantics given below to the current
	definition. Resolve the destination of all unresolved
	occurrences of \word{LEAVE} between the location given
	by \emph{do-sys} and the next location for a transfer of
	control, to execute the words following \word{+LOOP}.

\item[Run-time:]
	\stack{$n$}{}
	\stack[R]{loop-sys$_1$}{\textbar~loop-sys$_2$}

	An ambiguous condition exists if the loop control parameters
	are unavailable. Add $n$ to the loop index. If the loop index
	did not cross the boundary between the loop limit minus one and
	the loop limit, continue execution at the beginning of the loop.
	Otherwise, discard the current loop control parameters and
	continue execution immediately following the loop.

\item[See:]
	\wref{core:DO}{DO},
	\wref{core:I}{I},
	\wref{core:LEAVE}{LEAVE}.

	\begin{rationale} % A.6.1.0140 +LOOP
		Typical use:
			\word{:} \texttt{X} ~{\ldots} limit first \word{DO}
				{\ldots} step \word{+LOOP}
			\word{;}
	\end{rationale}
\end{newword*}


\begin{newword}{0150}{,}[comma]
	\stack{$x$}{}

	Reserve one cell of data space and store $x$ in the cell. If the
	data-space pointer is aligned when \word{,} begins execution, it
	will remain aligned when \word{,} finishes execution.
	An ambiguous condition exists if the data-space pointer is not
	aligned prior to execution of \word{,}.

\item[See:]
	\xref{usage:dataspace}{3.3.3 Data space},
	\xref{usage:aaddr}{3.3.3.1 Address alignment}.

	\begin{rationale} % A.6.1.0150 ,
		The use of \word{,} (comma) for compiling execution tokens is
		not portable.

		See: \wref{core:COMPILE,}{COMPILE,}.
	\end{rationale}
\end{newword}


\begin{newword}{0160}{-}[minus]
	\stack{$n_1$\textbar$u_1$ $n_2$\textbar$u_2$}{$n_3$\textbar$u_3$}

	Subtract $n_2$\textbar$u_2$ from $n_1$\textbar$u_1$, giving the
	difference $n_3$\textbar$u_3$.

\item[See:]
	\xref{usage:aaddr}{3.3.3.1 Address alignment}.
\end{newword}


\begin{newword}{0180}{.}[dot]
	\stack{$n$}{}

	Display $n$ in free field format.

\item[See:]
	\xref{usage:digits}{3.2.1.2 Digit conversion},
	\xref{usage:dot}{3.2.1.3 Free-field number display}.
\end{newword}


\begin{newword*}[.q]{0190}{."}[dot-quote]
\item[Interpretation:]
	Interpretation semantics for this word are undefined.

\item[Compilation:]
	\stack{``ccc\arg{quote}''}{}

	Parse \emph{ccc} delimited by \texttt{"} (double-quote).
	Append the run-time semantics given below to the current
	definition.

\item[Run-time:]
	\stack{}{}

	Display \emph{ccc}.

\item[See:]
	\xref{usage:parsing}{3.4.1 Parsing},
	\wref{core:.p}{.(}.

	\begin{rationale} % A.6.1.0190 ."
		Typical use:
			\word{:} \texttt{X} {\ldots}
				\word{.q} \emph{ccc}\texttt{"} {\ldots}
			\word{;}

		An implementation may define interpretation semantics for
		\word{.q} if desired. In one plausible implementation,
		interpreting \word{.q} would display the delimited message.
		In another plausible implementation, interpreting \word{.q}
		would compile code to display the message later. In still
		another plausible implementation, interpreting \word{.q} would
		be treated as an exception. Given this variation a Standard
		Program may not use \word{.q} while interpreting. Similarly,
		a Standard Program may not compile \word{POSTPONE} \word{.q}
		inside a new word, and then use that word while interpreting.
	\end{rationale}
\end{newword*}


\begin{newword}{0230}{/}[slash]
	\stack{$n_1$ $n_2$}{$n_3$}

	Divide $n_1$ by $n_2$, giving the single-cell quotient $n_3$.
	An ambiguous condition exists if $n_2$ is zero. If $n_1$ and
	$n_2$ differ in sign, the implementation-defined result returned
	will be the same as that returned by either the phrase
	\word{toR} \word{StoD} \word{Rfrom} \word{FM/MOD}
	\word{SWAP} \word{DROP} or the phrase \word{toR} \word{StoD}
	\word{Rfrom} \word{SM/REM} \word{SWAP} \word{DROP}.

\item[See:]
	\xref{usage:div}{3.2.2.1 Integer division}.
\end{newword}


\begin{newword}{0240}{/MOD}[slash-mod]
	\stack{$n_1$ $n_2$}{$n_3$ $n_4$}

	Divide $n_1$ by $n_2$, giving the single-cell remainder $n_3$
	and the single-cell quotient $n_4$. An ambiguous condition exists
	if $n_2$ is zero. If $n_1$ and $n_2$ differ in sign, the
	implementation-defined result returned will be the same as that
	returned by either the phrase
	\word{toR} \word{StoD} \word{Rfrom} \word{FM/MOD}
	or the phrase
	\word{toR} \word{StoD} \word{Rfrom} \word{SM/REM}.

\item[See:]
	\xref{usage:div}{3.2.2.1 Integer division}.
\end{newword}


\begin{newword}[0less]{0250}{0<}[zero-less]
	\stack{$n$}{flag}

	\emph{flag} is true if and only if $n$ is less than zero.
\end{newword}


\begin{newword}{0270}{0=}[zero-equals]
	\stack{$x$}{flag}

	\emph{flag} is true if and only if $x$ is equal to zero.
\end{newword}


\begin{newword}{0290}{1+}[one-plus]
	\stack{$n_1$\textbar$u_1$}{$n_2$\textbar$u_2$}

	Add one (1) to $n_1$\textbar$u_1$ giving the sum
	$n_2$\textbar$u_2$.
\end{newword}


\begin{newword}{0300}{1-}[one-minus]
	\stack{$n_1$\textbar$u_1$}{$n_2$\textbar$u_2$}

	Subtract one (1) from $n_1$\textbar$u_1$ giving the difference
	$n_2$\textbar$u_2$.
\end{newword}


\begin{newword}{0310}{2!}[two-store]
	\stack{$x_1$ $x_2$ a-addr}{}

	Store the cell pair $x_1$ $x_2$ at \emph{a-addr}, with $x_2$ at
	\emph{a-addr} and $x_1$ at the next consecutive cell. It is
	equivalent to the sequence
	\word{SWAP} \word{OVER} \word{!} \word{CELL+} \word{!}.

\item[See:]
	\xref{usage:aaddr}{3.3.3.1 Address alignment}.
\end{newword}


\begin{newword}{0320}{2*}[two-star]
	\stack{$x_1$}{$x_2$}

	$x_2$ is the result of shifting $x_1$ one bit toward the
	most-significant bit, filling the vacated least-significant
	bit with zero.

	\begin{rationale} % A.6.1.0320 2*
		Historically, \word{2*} has been implemented on
		two's-complement machines as a logical left-shift instruction.
		Multiplication by two is an efficient side-effect on these
		machines. However, shifting implies a knowledge of the
		significance and position of bits in a cell. While the name
		implies multiplication, most implementors have used a hardware
		left shift to implement \word{2*}.
	\end{rationale}
\end{newword}


\begin{newword}{0330}{2/}[two-slash]
	\stack{$x_1$}{$x_2$}

	$x_2$ is the result of shifting $x_1$ one bit toward the
	least-significant bit, leaving the most-significant bit
	unchanged.

	\begin{rationale} % A.6.1.0330 2/
		This word has the same common usage and misnaming implications
		as \word{2*}. It is often implemented on two's-complement
		machines with a hardware right shift that propagates the sign
		bit.
	\end{rationale}
\end{newword}


\begin{newword}{0350}{2@}[two-fetch]
	\stack{a-addr}{$x_1$ $x_2$}

	Fetch the cell pair $x_1$ $x_2$ stored at \emph{a-addr}. $x_2$
	is stored at \emph{a-addr} and $x_1$ at the next consecutive
	cell. It is equivalent to the sequence
	\word{DUP} \word{CELL+} \word{@} \word{SWAP} \word{@}.

\item[See:]
	\xref{usage:aaddr}{3.3.3.1  Address alignment},
	\wref{core:2!}{2!}.

	\begin{rationale} % A.6.1.0350 2@
		With \word{2@} the storage order is specified by the Standard.
	\end{rationale}
\end{newword}


\begin{newword}{0370}{2DROP}[two-drop]
	\stack{$x_1$ $x_2$}{}

	Drop cell pair $x_1$ $x_2$ from the stack.
\end{newword}

\begin{newword}{0380}{2DUP}[two-dupe]
	\stack{$x_1$ $x_2$}{$x_1$ $x_2$ $x_1$ $x_2$}

	Duplicate cell pair $x_1$ $x_2$.
\end{newword}


\begin{newword}{0400}{2OVER}[two-over]
	\stack{$x_1$ $x_2$ $x_3$ $x_4$}{$x_1$ $x_2$ $x_3$ $x_4$ $x_1$ $x_2$}

	Copy cell pair $x_1$ $x_2$ to the top of the stack.
\end{newword}


\begin{newword}{0430}{2SWAP}[two-swap]
	\stack{$x_1$ $x_2$ $x_3$ $x_4$}{$x_3$ $x_4$ $x_1$ $x_2$}

	Exchange the top two cell pairs.
\end{newword}


\begin{newword}{0450}{:}[colon]
	\stack[C]{``\arg{spaces}name''}{colon-sys}

	Skip leading space delimiters. Parse \emph{name} delimited by a
	space. Create a definition for \emph{name}, called a ``colon
	definition''. Enter compilation state and start the current
	definition, producing \emph{colon-sys}. Append the initiation
	semantics given below to the current definition.

	The execution semantics of \emph{name} will be determined by the
	words compiled into the body of the definition. The current
	definition shall not be findable in the dictionary until it is
	ended (or until the execution of \word{DOES} in some systems).

\item[Initiation:]
	\stack{i*x}{i*x}
	\stack[R]{}{nest-sys}

	Save implementation-dependent information \emph{nest-sys} about
	the calling definition. The stack effects \emph{i*x} represent
	arguments to \emph{name}.

\item[\emph{name} Execution:]
	~\stack{i*x}{j*x}

	Execute the definition \emph{name}. The stack effects \emph{i*x}
	and \emph{j*x} represent arguments to and results from
	\emph{name}, respectively.

\item[See:]
	\xref{usage:interpret}{3.4 The Forth text interpreter},
	\xref{usage:parsing}{3.4.1 Parsing},
	\xref{usage:compilation}{3.4.5 Compilation},
	\wref{core:DOES}{DOES>},
	\wref{core:[}{[},
	\wref{core:]}{]},
	\wref{tools:;CODE}{;CODE}.

	\begin{rationale} % A.6.1.0450 :
		Typical use:
			\word{:} \emph{name} {\ldots} \word{;}

		In Forth 83, this word was specified to alter the search order.
		This specification is explicitly removed in this Standard. We
		believe that in most cases this has no effect; however, systems
		that allow many search orders found the Forth-83 behavior of
		colon very undesirable.

		Note that colon does not itself invoke the compiler. Colon sets
		compilation state so that later words in the parse area are
		compiled.
	\end{rationale}
\end{newword}


\begin{newword*}{460}{;}[semicolon]
\item[Interpretation:]
	Interpretation semantics for this word are undefined.

\item[Compilation:]
	\stack[C]{colon-sys}{}

	Append the run-time semantics below to the current definition.
	End the current definition, allow it to be found in the
	dictionary and enter interpretation state, consuming
	\emph{colon-sys}. If the data-space pointer is not aligned,
	reserve enough data space to align it.

\item[Run-time:]
	\stack{}{}
	\stack[R]{nest-sys}{}

	Return to the calling definition specified by \emph{nest-sys}.

\item[See:]
	\xref{usage:command}{3.4 The Forth text interpreter},
	\xref{usage:compilation}{3.4.5 Compilation}.

	\begin{rationale} % A.6.1.0460 ;
		Typical use:
			\word{:} \emph{name} {\ldots} \word{;}

		One function performed by both \word{;} and \word[tools]{;CODE}
		is to allow the current definition to be found in the
		dictionary. If the current definition was created by
		\word{:NONAME} the current definition has no definition name
		and thus cannot be found in the dictionary. If \word{:NONAME}
		is implemented the Forth compiler must maintain enough
		information about the current definition to allow \word{;} and
		\word[tools]{;CODE} to determine whether or not any action must
		be taken to allow it to be found.
	\end{rationale}
\end{newword*}


\begin{newword}[less]{0480}{<}[less-than]
	\stack{$n_1$ $n_2$}{flag}

	\emph{flag} is true if and only if $n_1$ is less than $n_2$.

\item[See:]
	\wref{core:Uless}{U<}.
\end{newword}

\begin{newword}[num-start]{0490}{<\#}[less-number-sign]
	\stack{}{}

	Initialize the pictured numeric output conversion process.

\item[See:]
	\wref{core:num}{\#},
	\wref{core:num-end}{\#>},
	\wref{core:numS}{\#S}.
\end{newword}


\begin{newword}{0530}{=}[equals]
	\stack{$x_1$ $x_2$}{flag}

	\emph{flag} is true if and only if $x_1$ is bit-for-bit the
	same as $x_2$.
\end{newword}


\begin{newword}[more]{0540}{>}[greater-than]
	\stack{$n_1$ $n_2$}{flag}

	\emph{flag} is true if and only if $n_1$ is greater than $n_2$.

\item[See:]
	\wref{core:Umore}{U>}.
\end{newword}


\begin{newword}[toBODY]{0550}{>BODY}[to-body]
	\stack{xt}{a-addr}

	\emph{a-addr} is the data-field address corresponding to
	\emph{xt}. An ambiguous condition exists if \emph{xt} is not
	for a word defined via \word{CREATE}.

\item[See:]
	\xref{usage:dataspace}{3.3.3 Data space}.

	\begin{rationale} % A.6.1.0550 >BODY
		\emph{a-addr} is the address that \word{HERE} would have
		returned had it been executed immediately after the execution
		of the \word{CREATE} that defined \emph{xt}.
	\end{rationale}
\end{newword}


\begin{newword}[toIN]{0560}{>IN}[to-in]
	\stack{}{a-addr}

	\emph{a-addr} is the address of a cell containing the offset in
	characters from the start of the input buffer to the start of
	the parse area.
\end{newword}


\begin{newword}[toNUMBER]{0570}{>NUMBER}[to-number]
	\stack{ud$_1$ c-addr$_1$ $u_1$}{ud$_2$ c-addr$_2$ $u_2$}

	\emph{ud}$_2$ is the unsigned result of converting the
	characters within the string specified by \emph{c-addr}$_1$
	$u_1$ into digits, using the number in \word{BASE}, and adding
	each into \emph{ud}$_1$ after multiplying \emph{ud}$_1$ by the
	number in \word{BASE}. Conversion continues left-to-right until
	a character that is not convertible, including any ``+'' or
	``-'', is encountered or the string is entirely converted.
	\emph{c-addr}$_2$ is the location of the first unconverted
	character or the first character past the end of the string if
	the string was entirely converted. $u_2$ is the number of
	unconverted characters in the string. An ambiguous condition
	exists if \emph{ud}$_2$ overflows during the conversion.

\item[See:]
	\xref{usage:digits}{3.2.1.2 Digit conversion}.
\end{newword}


\begin{newword*}[toR]{0580}{>R}[to-r]
\item[Interpretation:]
	Interpretation semantics for this word are undefined.

\item[Execution:]
	\stack{$x$}{}
	\stack[R]{}{$x$}

	Move $x$ to the return stack.

\item[See:]
	\xref{usage:returnstack}{3.2.3.3 Return stack},
	\wref{core:Rfrom}{R>},
	\wref{core:R@}{R@},
	\wref{core:2toR}{2>R},
	\wref{core:2Rfrom}{2R>},
	\wref{core:2R@}{2R@}.
\end{newword*}


\begin{newword}{0630}{?DUP}[question-dupe]
	\stack{$x$}{$0$~\textbar~$x$ $x$}

	Duplicate $x$ if it is non-zero.
\end{newword}


\begin{newword}{0650}{@}[fetch]
	\stack{a-addr}{$x$}

	$x$ is the value stored at \emph{a-addr}.

\item[See:]
	\xref{usage:aaddr}{3.3.3.1 Address alignment}.
\end{newword}


\begin{newword}{0670}{ABORT}
	\stack{i*x}{}
	\stack[R]{j*x}{}

	Empty the data stack and perform the function of \word{QUIT},
	which includes emptying the return stack, without displaying
	a message.

\item[See:]
	\wref{exception:ABORT}{ABORT}.
\end{newword}


\begin{newword*}[ABORTq]{0680}{ABORT"}[abort-quote]
\item[Interpretation:]
	Interpretation semantics for this word are undefined.
\item[Compilation:]
	\stack{``ccc\arg{quote}''}{}

	Parse \emph{ccc} delimited by a \texttt{"} (double-quote).
	Append the run-time semantics given below to the current
	definition.

\item[Run-time:]
	\stack{i*x $x_1$}{\textbar~i*x}
	\stack[R]{j*x}{\textbar~j*x}

	Remove $x_1$ from the stack. If any bit of $x_1$ is not zero,
	display \emph{ccc} and perform an implementation-defined abort
	sequence that includes the function of \word{ABORT}.

\item[See:]
	\xref{usage:parsing}{3.4.1 Parsing},
	\wref{exception:ABORTq}{9.6.2.0680 ABORT"}.

	\begin{rationale} % A.6.1.0680 ABORT"
		Typical use:
			\word{:} \texttt{X} {\ldots}
				\emph{test} \word{ABORTq} \emph{ccc}\texttt{"}
			{\ldots} \word{;}
	\end{rationale}
\end{newword*}


\begin{newword}{0690}{ABS}[abs]
	\stack{$n$}{$u$}

	$u$ is the absolute value of $n$.
\end{newword}


\begin{newword}{0695}{ACCEPT}
	\stack{c-addr $+n_1$}{$+n_2$}

	Receive a string of at most $+n_1$ characters. An ambiguous
	condition exists if $+n_1$ is zero or greater than 32,767.
	Display graphic characters as they are received. A program that
	depends on the presence or absence of non-graphic characters in
	the string has an environmental dependency. The editing
	functions, if any, that the system performs in order to
	construct the string are implementation-defined.

	Input terminates when an implementation-defined line terminator
	is received. When input terminates, nothing is appended to the
	string, and the display is maintained in an
	implementation-defined way.

	$+n_2$ is the length of the string stored at \emph{c-addr}.

	\begin{rationale} % A.6.1.0695 ACCEPT
		Previous standards specified that collection of the input
		string terminates when either a ``return'' is received or when
		$+n_1$ characters have been received. Terminating when $+n_1$
		characters have been received is difficult, expensive, or
		impossible to implement in some system environments.
		Consequently, a number of existing implementations do not
		comply with this requirement. Since line-editing and collection
		functions are often implemented by system components beyond the
		control of the Forth implementation, this Standard imposes no
		such requirement. A Standard Program may only assume that it
		can receive an input string with \word{ACCEPT} or \word{EXPECT}.
		The detailed sequence of user actions necessary to prepare and
		transmit that line are beyond the scope of this Standard.

		Specification of a non-zero, positive integer count ($+n_1$)
		for \word{ACCEPT} allows some implementors to continue their
		practice of using a zero or negative value as a flag to trigger
		special behavior. Insofar as such behavior is outside the
		Standard, Standard Programs cannot depend upon it, but the
		Technical Committee doesn't wish to preclude it unnecessarily.
		Since actual values are almost always small integers, no
		functionality is impaired by this restriction.

		\word{ACCEPT} and \word{EXPECT} perform similar functions.
		\word{ACCEPT} is recommended for new programs, and future use
		of \word{EXPECT} is discouraged.

		It is recommended that all non-graphic characters be reserved
		for editing or control functions and not be stored in the input
		string.

		Commonly, when the user is preparing an input string to be
		transmitted to a program, the system allows the user to edit
		that string and correct mistakes before transmitting the final
		version of the string. The editing function is supplied
		sometimes by the Forth system itself, and sometimes by external
		system software or hardware. Thus, control characters and
		functions may not be available on all systems. In the usual
		case, the end of the editing process and final transmission of
		the string is signified by the user pressing a ``Return'' or
		``Enter'' key.

		As in previous standards, \word{EXPECT} returns the input
		string immediately after the requested number of characters
		are entered, as well as when a line terminator is received.
		The ``automatic termination after specified count of characters
		have been entered'' behavior is widely considered undesirable
		because the user ``loses control'' of the input editing process
		at a potentially unknown time (the user does not necessarily
		know how many characters were requested from \word{EXPECT}).
		Thus \word{EXPECT} and \word{SPAN} have been made obsolescent
		and exist in the Standard only as a concession to existing
		implementations. If \word{EXPECT} exists in a Standard System
		it must have the ``automatic termination'' behavior.

		\word{ACCEPT} does not have the ``automatic termination''
		behavior of \word{EXPECT}. However, because external system
		hardware and software may perform the \word{ACCEPT} function,
		when a line terminator is received the action of the cursor,
		and therefore the display, is implementation-defined. It is
		recommended that the cursor remain immediately following the
		entered text after a line terminator is received.
	\end{rationale}
\end{newword}


\begin{newword}{0705}{ALIGN}
	\stack{}{}

	If the data-space pointer is not aligned, reserve enough space
	to align it.

\item[See:]	
	\xref{usage:dataspace}{3.3.3 Data space},
	\xref{usage:aaddr}{3.3.3.1 Address alignment}.

	\begin{rationale} % A.6.1.0705 ALIGN
		In this Standard we have attempted to provide transportability
		across various CPU architectures. One of the frequent causes
		of transportability problems is the requirement of cell-aligned
		addresses on some CPUs. On these systems, \word{ALIGN} and
		\word{ALIGNED} may be required to build and traverse data
		structures built with \word{C,}. Implementors may define these
		words as no-ops on systems for which they aren't functional.
	\end{rationale}
\end{newword}


\begin{newword}{0706}{ALIGNED}
	\stack{addr}{a-addr}

	\emph{a-addr} is the first aligned address greater than or equal
	to \emph{addr}.

\item[See:]
	\xref{usage:aaddr}{3.3.3.1 Address alignment}.

	\begin{rationale} % A.6.1.0706 ALIGNED
		See: \rref{core:ALIGN}{ALIGN}.
	\end{rationale}
\end{newword}


\begin{newword}{0710}{ALLOT}
	\stack{n}{}

	If $n$ is greater than zero, reserve $n$ address units of data
	space. If $n$ is less than zero, release $|n|$ address units
	of data space. If $n$ is zero, leave the data-space pointer
	unchanged.

	If the data-space pointer is aligned and $n$ is a multiple of
	the size of a cell when \word{ALLOT} begins execution, it will
	remain aligned when \word{ALLOT} finishes execution.

	If the data-space pointer is character aligned and $n$ is a
	multiple of the size of a character when \word{ALLOT} begins
	execution, it will remain character aligned when \word{ALLOT}
	finishes execution.

\item[See:]
	\xref{usage:dataspace}{3.3.3 Data space}.
\end{newword}


\begin{newword}{0720}{AND}
	\stack{ $x_1$ $x_2$ }{ $x_3$ }

	$x_3$ is the bit-by-bit logical ``and'' of $x_1$ with $x_2$.
\end{newword}


\begin{newword}{0750}{BASE}
	\stack{}{a-addr}

	\emph{a-addr} is the address of a cell containing the current
	number-conversion radix \{\{2...36\}\}.
\end{newword}


\begin{newword*}{0760}{BEGIN}
\item[Interpretation:]
	Interpretation semantics for this word are undefined.

\item[Compilation:]
	\stack[C]{}{dest}

	Put the next location for a transfer of control, \emph{dest},
	onto the control flow stack. Append the run-time semantics
	given below to the current definition.

\item[Run-time:]
	\stack{}{}

	Continue execution.

\item[See:]
	\xref{usage:controlstack}{3.2.3.2 Control-flow stack},
	\wref{core:REPEAT}{REPEAT},
	\wref{core:UNTIL}{UNTIL},
	\wref{core:WHILE}{WHILE}.

	\begin{rationale} % A.6.1.0760 BEGIN
		Typical use:

		\tab \word{:} \texttt{X} {\ldots}
			\word{BEGIN} {\ldots} \emph{test} \word{UNTIL}
		\word{;}

		or

		\tab \word{:} \texttt{X} {\ldots}
			\word{BEGIN} {\ldots}
			\emph{test} \word{WHILE} {\ldots}
			\word{REPEAT}
		\word{;}
	\end{rationale}
\end{newword*}


\begin{newword}{0770}{BL}[b-l]
	\stack{}{char}

	\emph{char} is the character value for a space.

	\begin{rationale} % A.6.1.0770 BL
		Because space is used throughout Forth as the standard
		delimiter, this word is the only way a program has to find and
		use the system value of ``space''. The value of a space
		character can not be obtained with \word{CHAR}, for instance.
	\end{rationale}
\end{newword}


\begin{newword}{0850}{C!}[c-store]
	\stack{char c-addr}{}

	Store \emph{char} at \emph{c-addr}. When character size is
	smaller than cell size, only the number of low-order bits
	corresponding to character size are transferred.

\item[See:]
	\xref{usage:aaddr}{3.3.3.1 Address alignment}
\end{newword}


\begin{newword}{0860}{C,}[c-comma]
	\stack{char}{}

	Reserve space for one character in the data space and store
	\emph{char} in the space. If the data-space pointer is
	character aligned when \word{C,} begins execution, it will
	remain character aligned when \word{C,} finishes execution.
	An ambiguous condition exists if the data-space pointer is
	not character-aligned prior to execution of \word{C,}.

\item[See:]
	\xref{usage:dataspace}{3.3.3 Data space},
	\xref{usage:aaddr}{3.3.3.1 Address alignment}.
\end{newword}


\begin{newword}{0870}{C@}[c-fetch]
	\stack{c-addr}{char}

	Fetch the character stored at \emph{c-addr}. When the cell
	size is greater than character size, the unused high-order
	bits are all zeroes.

\item[See:]
	\xref{usage:aaddr}{3.3.3.1 Address alignment}.
\end{newword}


\begin{newword}{0880}{CELL+}[cell-plus]
	\stack{a-addr$_1$}{a-addr$_2$}

	Add the size in address units of a cell to \emph{a-addr}$_1$,
	giving \emph{a-addr}$_2$.

\item[See:]
	\xref{usage:aaddr}{3.3.3.1 Address alignment}.

	\begin{rationale} % A.6.1.0880 CELL+
		As with \word{ALIGN} and \word{ALIGNED}, the words \word{CELLS}
		and \word{CELL+} were added to aid in transportability across
		systems with different cell sizes. They are intended to be used
		in manipulating indexes and addresses in integral numbers of
		cell-widths. Example:
		\begin{quote}\ttfamily
			\word[double]{2VARIABLE} DATA

			0 100 DATA \word{2!}

			DATA \word{@} \word{.} 100

			DATA \word{CELL+} \word{@} \word{.} 0
		\end{quote}
	\end{rationale}
\end{newword}


\begin{newword}{0890}{CELLS}
	\stack{$n_1$}{$n_2$}

	$n_2$ is the size in address units of $n_1$ cells.

	\begin{rationale} % A.6.1.0890 CELLS
		See: \rref{core:CELL+}{CELL+}.

		Example:
			\word{CREATE} \texttt{NUMBERS} ~
			\texttt{100} \word{CELLS} \word{ALLOT} \\
		(Allots space in the array \texttt{NUMBERS} for 100 cells
		of data.)
	\end{rationale}
\end{newword}


\begin{newword}{0895}{CHAR}[char]
	\stack{``\arg{spaces}name''}{char}

	Skip leading space delimiters. Parse \emph{name} delimited by
	a space. Put the value of its first character onto the stack.

\item[See:]
	\xref{usage:parsing}{3.4.1 Parsing},
	\wref{core:[CHAR]}{[CHAR]}.

	\begin{rationale} % A.6.1.0895 CHAR
		Typical use: {\ldots}
			\word{CHAR} \texttt{A} \word{CONSTANT} \texttt{"A"} {\ldots}
	\end{rationale}
\end{newword}


\begin{newword}{0897}{CHAR+}[char-plus]
	\stack{c-addr$_1$}{c-addr$_2$}

	Add the size in address units of a character to
	\emph{c-addr}$_1$, giving \emph{c-addr}$_2$.

\item[See:]
	\xref{usage:aaddr}{3.3.3.1 Address alignment}.
\end{newword}


\begin{newword}{0898}{CHARS}[chars]
	\stack{$n_1$}{$n_2$}

	$n_2$ is the size in address units of $n_1$ characters.
\end{newword}


\begin{newword}{0950}{CONSTANT}
	\stack{x ``\arg{spaces}name''}{}

	Skip leading space delimiters. Parse \emph{name} delimited by
	a space. Create a definition for \emph{name} with the execution
	semantics defined below.

	\emph{name} is referred to as a ``constant''.

\item[\emph{name} Execution:]
	\stack{}{x}

	Place $x$ on the stack.

\item[See:]
	\xref{usage:parsing}{3.4.1 Parsing}.

	\begin{rationale} % A.6.1.0950 CONSTANT
		Typical use: {\ldots}
			\word{DECIMAL} \texttt{10} \word{CONSTANT} \texttt{TEN}
		{\ldots}
	\end{rationale}
\end{newword}


\begin{newword}{0980}{COUNT}
	\stack{c-addr$_1$}{c-addr$_2$ $u$}

	Return the character string specification for the counted
	string stored at \emph{c-addr}$_1$. \emph{c-addr}$_2$ is the
	address of the first character after \emph{c-addr}$_1$. $u$ is
	the contents of the character at \emph{c-addr}$_1$, which is
	the length in characters of the string at \emph{c-addr}$_2$.
\end{newword}


\begin{newword}{0990}{CR}[c-r]
	\stack{}{}

	Cause subsequent output to appear at the beginning of the next
	line.
\end{newword}


\begin{newword}{1000}{CREATE}
	\stack{ ``\arg{spaces}name''}{}

	Skip leading space delimiters. Parse \emph{name} delimited by a
	space. Create a definition for \emph{name} with the execution
	semantics defined below. If the data-space pointer is not
	aligned, reserve enough data space to align it. The new
	data-space pointer defines \emph{name}'s data field.
	\word{CREATE} does not allocate data space in \emph{name}'s
	data field.

\item[\emph{name} Execution:]
	\stack{}{a-addr}

	\emph{a-addr} is the address of \emph{name}'s data field. The
	execution semantics of \emph{name} may be extended by using
	\word{DOES}.

\item[See:]	
	\xref{usage:dataspace}{3.3.3 Data space},
	\wref{core:DOES}{DOES>}.

	\begin{rationale} % A.6.1.1000 CREATE
		The data-field address of a word defined by \word{CREATE} is
		given by the data-space pointer immediately following the
		execution of \word{CREATE}.

		Reservation of data field space is typically done with
		\word{ALLOT}.

		Typical use: {\ldots}
			\word{CREATE} \texttt{SOMETHING} {\ldots}
	\end{rationale}
\end{newword}


\begin{newword}{1170}{DECIMAL}
	\stack{}{}

	Set the numeric conversion radix to ten (decimal).
\end{newword}


\begin{newword}{1200}{DEPTH}
	\stack{}{$+n$}

	$+n$ is the number of single-cell values contained in the data
	stack before $+n$ was placed on the stack.
\end{newword}


\begin{newword*}{1240}{DO}
\item[Interpretation:]	
	Interpretation semantics for this word are undefined.

\item[Compilation:]
	\stack[C]{}{do-sys}

	Place \emph{do-sys} onto the control-flow stack. Append the
	run-time semantics given below to the current definition. The
	semantics are incomplete until resolved by a consumer of
	\emph{do-sys} such as \word{LOOP}.

\item[Run-time:]
	\stack{$n_1$\textbar$u_1$ $n_2$\textbar$u_2$}{}
	\stack[R]{}{loop-sys}

	Set up loop control parameters with index $n_2$\textbar$u_2$ and
	limit $n_1$\textbar$u_1$. An ambiguous condition exists if
	$n_1$\textbar$u_1$ and $n_2$\textbar$u_2$ are not both the same
	type. Anything already on the return stack becomes unavailable
	until the loop-control parameters are discarded.

\item[See:]
	\xref{usage:controlstack}{3.2.3.2 Control-flow stack},
	\wref{core:+LOOP}{+LOOP},
	\wref{core:LOOP}{LOOP}.

	\begin{rationale} % A.6.1.1240 DO
		Typical use:

		\tab \word{:} \texttt{X} {\ldots}
			\emph{limit} \emph{first} \word{DO}
				{\ldots}
			\word{LOOP}
		\word{;}

		or

		\tab \word{:} \texttt{X} {\ldots}
			\emph{limit} \emph{first} \word{DO}
				{\ldots}
			\emph{step} \word{+LOOP}
		\word{;}
	\end{rationale}
\end{newword*}


\begin{newword*}[DOES]{1250}{DOES>}[does]
\item[Interpretation:]
	Interpretation semantics for this word are undefined.

\item[Compilation:]
	\stack[C]{colon-sys$_1$}{colon-sys$_2$}

	Append the run-time semantics below to the current definition.
	Whether or not the current definition is rendered findable in
	the dictionary by the compilation of \word{DOES} is
	implementation defined. Consume \emph{colon-sys}$_1$ and
	produce \emph{colon-sys}$_2$. Append the initiation semantics
	given below to the current definition.

\item[Run-time:]
	\stack{}{}
	\stack[R]{nest-sys$_1$}{}

	Replace the execution semantics of the most recent definition,
	referred to as \emph{name}, with the \emph{name} execution
	semantics given below. Return control to the calling definition
	specified by \emph{nest-sys}$_1$. An ambiguous condition exists
	if \emph{name} was not defined with \word{CREATE} or a
	user-defined word that calls \word{CREATE}.

\item[Initiation:]
	\stack{i*x}{i*x a-addr}
	\stack[R]{}{nest-sys$_2$}

	Save implementation-dependent information \emph{nest-sys}$_2$
	about the calling definition. Place \emph{name}'s data field
	address on the stack. The stack effects \emph{i*x} represent
	arguments to \emph{name}.

\item[\emph{name} Execution:]
	\stack{i*x}{j*x}

	Execute the portion of the definition that begins with the
	initiation semantics appended by the \word{DOES} which modified
	\emph{name}. The stack effects \emph{i*x} and \emph{j*x}
	represent arguments to and results from \emph{name},
	respectively.

\item[See:]
	\wref{core:CREATE}{CREATE}.

	\begin{rationale} % A.6.1.1250 DOES>
		Typical use:
			\word{:} \texttt{X} {\ldots} \word{DOES} {\ldots} \word{;}

		Following \word{DOES}, a Standard Program may not make any
		assumptions regarding the ability to find either the name of
		the definition containing the \word{DOES} or any previous
		definition whose name may be concealed by it. \word{DOES}
		effectively ends one definition and begins another as far as
		local variables and control-flow structures are concerned.
		The compilation behavior makes it clear that the user is not
		entitled to place \word{DOES} inside any control-flow
		structures.
	\end{rationale}
\end{newword*}


\begin{newword}{1260}{DROP}
	\stack{x}{}

	Remove $x$ from the stack.
\end{newword}


\begin{newword}{1290}{DUP}[dupe]
	\stack{x}{x x}

	Duplicate $x$.
\end{newword}


\begin{newword*}{1310}{ELSE}
\item[Interpretation:]
	Interpretation semantics for this word are undefined.

\item[Compilation:]
	\stack[C]{orig$_1$}{orig$_2$}

	Put the location of a new unresolved forward reference
	\emph{orig}$_2$ onto the control flow stack. Append the run-time
	semantics given below to the current definition. The semantics
	will be incomplete until \emph{orig}$_2$ is resolved (e.g., by
	\word{THEN}). Resolve the forward reference \emph{orig}$_1$ using
	the location following the appended run-time semantics.

\item[Run-time:]
	\stack{}{}

	Continue execution at the location given by the resolution of
	\emph{orig}$_2$.

\item[See:]
	\wref{core:IF}{IF},
	\wref{core:THEN}{THEN}.

	\begin{rationale} % A.6.1.1310 ELSE
		Typical use:
			\word{:} \texttt{X} {\ldots}
				\emph{test} \word{IF} {\ldots}
				\word{ELSE} {\ldots} \word{THEN}
			\word{;}
	\end{rationale}
\end{newword*}


\begin{newword}{1320}{EMIT}
	\stack{x}{}

	If $x$ is a graphic character in the implementation-defined
	character set, display $x$. The effect of \word{EMIT} for all
	other values of $x$ is implementation-defined.

	When passed a character whose character-defining bits have a
	value between hex 20 and 7E inclusive, the corresponding
	standard character, specified by \xref{usage:ASCII}{3.1.2.1
	Graphic characters}, is displayed. Because different output
	devices can respond differently to control characters, programs
	that use control characters to perform specific functions have
	an environmental dependency. Each EMIT deals with only one
	character.

\item[See:]
	\wref{core:TYPE}{TYPE}.
\end{newword}

\begin{newword}{1345}{ENVIRONMENT?}[environment-query]
	\stack{c-addr u}{false~\textbar~i*x true}

	\emph{c-addr} is the address of a character string and $u$ is
	the string's character count. $u$ may have a value in the range
	from zero to an implementation-defined maximum which shall not
	be less than 31. The character string should contain a keyword
	from \xref{usage:env}{3.2.6 Environmental queries} or the
	optional word sets to be checked for correspondence with an
	attribute of the present environment. If the system treats the
	attribute as unknown, the returned flag is \emph{false};
	otherwise, the flag is \emph{true} and the \emph{i*x} returned
	is of the type specified in the table for the attribute queried.

	\begin{rationale} % A.6.1.1345 ENVIRONMENT?
		In a Standard System that contains only the Core word set,
		effective use of \word{ENVIRONMENT?} requires either its use
		within a definition, or the use of user-supplied auxiliary
		definitions. The Core word set lacks both a direct method for
		collecting a string in interpretation state (\wref{file:Sq}{S"}
		is in an optional word set) and also a means to test the
		returned flag in interpretation state (e.g. the optional
		\wref{tools:[IF]}{[IF]}).

		The combination of
		\wref{core:ENVIRONMENT?}{ENVIRONMENT?},
		\wref{file:Sq}{S"},
		\wref{tools:[IF]}{[IF]},
		\wref{tools:[ELSE]}{[ELSE]}, and
		\wref{tools:[THEN]}{[THEN]}
		constitutes an effective suite of words for conditional
		compilation that works in interpretation state.
	\end{rationale}
\end{newword}


\begin{newword}{1360}{EVALUATE}
	\stack{i*x c-addr u}{j*x}

	Save the current input source specification. Store minus-one
	(-1) in \word{SOURCE-ID} if it is present. Make the string
	described by \emph{c-addr} and $u$ both the input source and
	input buffer, set \word{toIN} to zero, and interpret. When the
	parse area is empty, restore the prior input source
	specification. Other stack effects are due to the words
	\word{EVALUATE}d.

	\begin{rationale} % A.6.1.1360 EVALUATE
		The Technical Committee is aware that this function is
		commonly spelled \texttt{EVAL}. However, there exist
		implementations that could suffer by defining the word as is
		done here. We also find \word{EVALUATE} to be more readable
		and explicit. There was some sentiment for calling this
		\texttt{INTERPRET}, but that too would have undesirable
		effects on existing code. The longer spelling was not deemed
		significant since this is not a word that should be used
		frequently in source code.
	\end{rationale}
\end{newword}


\begin{newword}{1370}{EXECUTE}
	\stack{i*x xt}{j*x}

	Remove \emph{xt} from the stack and perform the semantics
	identified by it. Other stack effects are due to the word
	\word{EXECUTE}d.

\item[See:]
	\wref{core:'}{'},
	\wref{core:[']}{[']}.
\end{newword}


\begin{newword*}{1380}{EXIT}
\item[Interpretation:]
	Interpretation semantics for this word are undefined.

\item[Execution:]
	\stack{}{}
	\stack[R]{nest-sys}{}

	Return control to the calling definition specified by
	\emph{nest-sys}. Before executing \word{EXIT} within a do-loop,
	a program shall discard the loop-control parameters by executing
	\word{UNLOOP}.

\item[See:]
	\xref{usage:returnstack}{3.2.3.3 Return stack},
	\wref{core:UNLOOP}{UNLOOP}.

	\begin{rationale} % A.6.1.1380 EXIT
		Typical use:
			\word{:} \texttt{X} {\ldots}
				\emph{test} \word{IF}
				{\ldots} \word{EXIT} \word{THEN}
			{\ldots} \word{;}
	\end{rationale}
\end{newword*}


\begin{newword}{1540}{FILL}
	\stack{c-addr u char}{}

	If $u$ is greater than zero, store \emph{char} in each of $u$
	consecutive characters of memory beginning at \emph{c-addr}.
\end{newword}


\begin{newword}{1550}{FIND}
	\stack{c-addr}{c-addr 0~\textbar~xt 1~\textbar~xt -1}

	Find the definition named in the counted string at \emph{c-addr}.
	If the definition is not found, return \emph{c-addr} and zero.
	If the definition is found, return its execution token \emph{xt}.
	If the definition is immediate, also return one (1), otherwise
	also return minus-one (-1). For a given string, the values
	returned by \word{FIND} while compiling may differ from those
	returned while not compiling.

\item[See:]
	\xref{usage:find}{3.4.2 Finding definition names},
	\rref{core:'}{'},
	\rref{core:[']}{[']},
	\rref{core:POSTPONE}{POSTPONE},
	\xref{diff:immediate}{D.6.7 Immediacy}.

	\begin{rationale} % A.6.1.1550 FIND
		One of the more difficult issues which the Committee took on
		was the problem of divorcing the specification of
		implementation mechanisms from the specification of the
		Forth language. Three basic implementation approaches can be
		quickly enumerated:

		\begin{enumerate}
		\item[1)] Threaded code mechanisms.
			These are the traditional approaches to implementing Forth,
			but other techniques may be used.

		\item[2)] Subroutine threading with ``macro-expansion'' (code
			copying). Short routines, like the code for \word{DUP},
			are copied into a definition rather than compiling a
			\texttt{JSR} reference.

		\item[3)] Native coding with optimization.
			This may include stack optimization (replacing such phrases
			as \word{SWAP} \word{ROT} \word{+} with one or two machine
			instructions, for example), parallelization (the trend in
			the newer RISC chips is to have several functional subunits
			which can execute in parallel), and so on.
		\end{enumerate}

		The initial requirement (inherited from Forth 83) that
		compilation addresses be compiled into the dictionary
		disallowed type 2 and type 3 implementations.

		Type 3 mechanisms and optimizations of type 2 implementations
		were hampered by the explicit specification of immediacy or
		non-immediacy of all standard words. \word{POSTPONE} allowed
		de-specification of immediacy or non-immediacy for all but a
		few Forth words whose behavior must be \word{STATE}-independent.

		One type 3 implementation, Charles Moore's cmForth, has both
		compiling and interpreting versions of many Forth words. At the
		present, this appears to be a common approach for type 3
		implementations. The Committee felt that this implementation
		approach must be allowed. Consequently, it is possible that
		words without interpretation semantics can be found only during
		compilation, and other words may exist in two versions: a
		compiling version and an interpreting version. Hence the values
		returned by \word{FIND} may depend on \word{STATE}, and \word{'}
		and \word{[']} may be unable to find words without
		interpretation semantics.
	\end{rationale}
\end{newword}


\begin{newword}{1561}{FM/MOD}[f-m-slash-mod]
	\stack{$d_1$ $n_1$}{$n_2$ $n_3$}

	Divide $d_1$ by $n_1$, giving the floored quotient $n_3$ and the
	remainder $n_2$. Input and output stack arguments are signed. An
	ambiguous condition exists if $n_1$ is zero or if the quotient
	lies outside the range of a single-cell signed integer.

\item[See:]
	\xref{usage:div}{3.2.2.1 Integer division},
	\wref{core:SM/REM}{6.1.2214 SM/REM},
	\wref{core:UM/MOD}{6.1.2370 UM/MOD}.

	\begin{rationale} % A.6.1.1561 FM/MOD
		By introducing the requirement for ``floored'' division,
		Forth 83 produced much controversy and concern on the part of
		those who preferred the more common practice followed in other
		languages of implementing division according to the behavior
		of the host CPU, which is most often symmetric (rounded toward
		zero). In attempting to find a compromise position, this
		Standard provides primitives for both common varieties, floored
		and symmetric (see \word{SM/REM}). \word{FM/MOD} is the floored
		version.

		The Technical Committee considered providing two complete sets
		of explicitly named division operators, and declined to do so
		on the grounds that this would unduly enlarge and complicate
		the Standard. Instead, implementors may define the normal
		division words in terms of either \word{FM/MOD} or
		\word{SM/REM} providing they document their choice. People
		wishing to have explicitly named sets of operators are
		encouraged to do so. \word{FM/MOD} may be used, for example,
		to define:

		\begin{quote}\ttfamily
			\word{:} /\_MOD \word{p} n1 n2 -- n3 n4)
				\word{toR} \word{StoD} \word{Rfrom} \word{FM/MOD}
			\word{;}

			\word{:} /\_ \word{p} n1 n2 -- n3)
				/\_MOD \word{SWAP} \word{DROP}
			\word{;}

			\word{:} \_MOD \word{p} n1 n2 -- n3)
				/\_MOD \word{DROP}
			\word{;}

			\word{:} */\_MOD \word{p} n1 n2 n3 -- n4 n5)
				\word{toR} \word{M*} \word{Rfrom} \word{FM/MOD}
			\word{;}

			\word{:} */\_ \word{p} n1 n2 n3 -- n4 )
				*/\_MOD \word{SWAP} \word{DROP}
			\word{;}
		\end{quote}
	\end{rationale}
\end{newword}


\begin{newword}{1650}{HERE}
	\stack{}{addr}

	\emph{addr} is the data-space pointer.

\item[See:]
	\xref{usage:contiguous}{3.3.3.2 Contiguous regions}.
\end{newword}


\begin{newword}{1670}{HOLD}
	\stack{char}{}

	Add \emph{char} to the beginning of the pictured numeric output
	string. An ambiguous condition exists if \word{HOLD} executes
	outside of a \word{num-start} \word{num-end} delimited number conversion.
\end{newword}


\begin{newword*}{1680}{I}
\item[Interpretation:]
	Interpretation semantics for this word are undefined.

\item[Execution:]
	\stack{}{$n$\textbar$u$}
	\stack[R]{loop-sys}{loop-sys}

	$n$\textbar$u$ is a copy of the current (innermost) loop index.
	An ambiguous condition exists if the loop control parameters
	are unavailable.
\end{newword*}


\begin{newword*}{1700}{IF}
\item[Interpretation:]
	Interpretation semantics for this word are undefined.

\item[Compilation:]
	\stack[C]{}{orig}

	Put the location of a new unresolved forward reference
	\emph{orig} onto the control flow stack. Append the run-time
	semantics given below to the current definition. The semantics
	are incomplete until \emph{orig} is resolved, e.g., by
	\word{THEN} or \word{ELSE}.

\item[Run-time:]
	\stack{x}{}

	If all bits of $x$ are zero, continue execution at the location
	specified by the resolution of \emph{orig}.

\item[See:]
	\xref{usage:controlstack}{3.2.3.2 Control flow stack},
	\wref{core:ELSE}{ELSE},
	\wref{core:THEN}{THEN}.

	\begin{rationale} % A.6.1.1700 IF
		Typical use:

		\tab \word{:} \texttt{X} {\ldots}
			\emph{test} \word{IF} {\ldots} \word{THEN}
		{\ldots} \word{;}

		or

		\tab \word{:} \texttt{X} {\ldots}
			\emph{test} \word{IF}
			{\ldots} \word{ELSE} {\ldots} \word{THEN}
		{\ldots} \word{;}
	\end{rationale}
\end{newword*}


\begin{newword}{1710}{IMMEDIATE}
	\stack{}{}

	Make the most recent definition an immediate word. An ambiguous
	condition exists if the most recent definition does not have a
	name.

\item[See:]
	\xref{diff:immediate}{D.6.7 Immediacy}.

	\begin{rationale} % A.6.1.1710 IMMEDIATE
		Typical use:
			\word{:} \texttt{X}
			{\ldots} \word{;} \word{IMMEDIATE}
	\end{rationale}
\end{newword}


\begin{newword}{1720}{INVERT}
	\stack{$x_1$}{$x_2$}

	Invert all bits of $x_1$, giving its logical inverse $x_2$.

\item[See:]
	\wref{core:NEGATE}{NEGATE},
	\wref{core:0=}{0=}.

	\begin{rationale} % A.6.1.1720 INVERT
		The word \texttt{NOT} was originally provided in Forth as a
		flag operator to make control structures readable. Under its
		intended usage the following two definitions would produce
		identical results:

		\begin{quote}\ttfamily
			\word{:} ONE \word{p} flag -- ) \\
			\tab \word{IF}
					\word{.q} true"
				\word{ELSE}
					\word{.q} false"
				\word{THEN}
			\word{;}

			\word{:} TWO ( flag -- ) \\
			\tab NOT \word{IF}
					\word{.q} false"
				\word{ELSE}
					\word{.q} true"
				\word{THEN}
			\word{;}
		\end{quote}

		This was common usage prior to the Forth-83 Standard which
		redefined \texttt{NOT} as a cell-wide one's-complement
		operation, functionally equivalent to the phrase \texttt{-1}
		\word{XOR}. At the same time, the data type manipulated by
		this word was changed from a flag to a cell-wide collection of
		bits and the standard value for true was changed from ``1''
		(rightmost bit only set) to ``-1'' (all bits set). As these
		definitions of \word{TRUE} and \texttt{NOT} were incompatible
		with their previous definitions, many Forth users continue to
		rely on the old definitions. Hence both versions are in common
		use.

		Therefore, usage of \texttt{NOT} cannot be standardized at
		this time. The two traditional meanings of \texttt{NOT} ---
		that of negating the sense of a flag and that of doing a one's
		complement operation --- are made available by \word{0=} and
		\word{INVERT}, respectively.
	\end{rationale}
\end{newword}


\begin{newword*}{1730}{J}
\item[Interpretation:]
	Interpretation semantics for this word are undefined.

\item[Execution:]
	\stack{}{n\textbar u}
	\stack[R]{loop-sys$_1$ loop-sys$_2$}{loop-sys$_1$ loop-sys$_2$}

	$n$\textbar$u$ is a copy of the next-outer loop index. An
	ambiguous condition exists if the loop control parameters of the
	next-outer loop, \emph{loop-sys}$_1$, are unavailable.

	\begin{rationale} % A.6.1.1730 J
		\word{J} may only be used with a nested
		\word{DO} {\ldots} \word{LOOP},
		\word{DO} {\ldots} \word{+LOOP},
		\word{?DO} {\ldots} \word{LOOP}, or
		\word{?DO} {\ldots} \word{+LOOP},
		for example, in the form:

		\tab \word{:} \texttt{X}
			{\ldots} \word{DO}
				{\ldots} \word{DO}
					{\ldots} \word{J} {\ldots}
				\word{LOOP}
			{\ldots} \word{+LOOP}
		{\ldots} \word{;}
	\end{rationale}
\end{newword*}


\begin{newword}{1750}{KEY}
	\stack{}{char}

	Receive one character \emph{char}, a member of the
	implementation-defined character set. Keyboard events that do
	not correspond to such characters are discarded until a valid
	character is received, and those events are subsequently
	unavailable.

	All standard characters can be received. Characters received by
	\word{KEY} are not displayed.

	Any standard character returned by \word{KEY} has the numeric
	value specified in \xref{usage:ASCII}{3.1.2.1 Graphic characters}.
	Programs that require the ability to receive control characters
	have an environmental dependency.

\item[See:]
	\wref{facility:EKEY}{EKEY},
	\wref{facility:EKEY?}{KEY?}.
\end{newword}


\begin{newword*}{1760}{LEAVE}
\item[Interpretation:]
	Interpretation semantics for this word are undefined.

\item[Execution:]
	\stack{}{}
	\stack[R]{loop-sys}{}

	Discard the current loop control parameters. An ambiguous
	condition exists if they are unavailable. Continue execution
	immediately following the innermost syntactically enclosing
	\word{DO}\ldots\word{LOOP} or \word{DO}\ldots\word{+LOOP}.

\item[See:]
	\xref{usage:returnstack}{3.2.3.3 Return stack},
	\wref{core:+LOOP}{+LOOP},
	\wref{core:LOOP}{LOOP}.

	\begin{rationale} % A.6.1.1760 LEAVE
		Note that \word{LEAVE} immediately exits the loop. No words
		following \word{LEAVE} within the loop will be executed.
		Typical use:

		\tab \word{:} \texttt{X} {\ldots} \word{DO}
			{\ldots} \word{IF}
				{\ldots} \word{LEAVE}
			\word{THEN}
		\word{;}
	\end{rationale}
\end{newword*}


\begin{newword*}{1780}{LITERAL}
\item[Interpretation:]
	Interpretation semantics for this word are undefined.

\item[Compilation:]
	\stack{x}{}

	Append the run-time semantics given below to the current
	definition.

\item[Run-time:]
	\stack{}{x}

	Place $x$ on the stack.

	\begin{rationale} % A.6.1.1780 LITERAL
		Typical use:
			\word{:} \texttt{X} {\ldots}
				\word{[} \texttt{x} \word{]} \word{LITERAL}
			{\ldots} \word{;}
	\end{rationale}
\end{newword*}


\begin{newword*}{1800}{LOOP}
\item[Interpretation:]
	Interpretation semantics for this word are undefined.

\item[Compilation:]
	\stack[C]{do-sys}{}

	Append the run-time semantics given below to the current
	definition. Resolve the destination of all unresolved
	occurrences of \word{LEAVE} between the location given by
	\emph{do-sys} and the next location for a transfer of control,
	to execute the words following the \word{LOOP}.

\item[Run-time:]
	\stack{}{}
	\stack[R]{loop-sys$_1$}{~\textbar~loop-sys$_2$}

	An ambiguous condition exists if the loop control parameters
	are unavailable. Add one to the loop index. If the loop index
	is then equal to the loop limit, discard the loop parameters
	and continue execution immediately following the loop. Otherwise
	continue execution at the beginning of the loop.

\item[See:]
	\wref{core:DO}{DO},
	\wref{core:I}{I},
	\wref{core:LEAVE}{LEAVE}.

	\begin{rationale} % A.6.1.1800 LOOP
		Typical use:

		\tab \word{:} \texttt{X} {\ldots}
			\emph{limit} \emph{first} \word{DO}
				{\ldots}
			\word{LOOP}
		{\ldots} \word{;}

		or

		\tab \word{:} \texttt{X} {\ldots}
			\emph{limit} \emph{first} \word{?DO}
				{\ldots}
			\word{LOOP}
		{\ldots} \word{;}
	\end{rationale}
\end{newword*}


\begin{newword}{1805}{LSHIFT}[l-shift]
	\stack{$x_1$ $u$}{$x_2$}

	Perform a logical left shift of $u$ bit-places on $x_1$, giving
	$x_2$. Put zeroes into the least significant bits vacated by the
	shift. An ambiguous condition exists if $u$ is greater than or
	equal to the number of bits in a cell.
\end{newword}


\begin{newword}{1810}{M*}[m-star]
	\stack{ $n_1$ $n_2$}{d}

	$d$ is the signed product of $n_1$ times $n_2$.

	\begin{rationale} % A.6.1.1810 M*
		This word is a useful early step in calculation, going to
		extra precision conveniently. It has been in use since the
		Forth systems of the early 1970's.
	\end{rationale}
\end{newword}


\begin{newword}{1870}{MAX}
	\stack{$n_1$ $n_2$}{ $n_3$}

	$n_3$ is the greater of $n_1$ and $n_2$.
\end{newword}


\begin{newword}{1880}{MIN}
	\stack{$n_1$ $n_2$}{$n_3$}

	$n_3$ is the lesser of $n_1$ and $n_2$.
\end{newword}


\begin{newword}{1890}{MOD}
	\stack{$n_1$ $n_2$}{$n_3$}

	Divide $n_1$ by $n_2$, giving the single-cell remainder $n_3$.
	An ambiguous condition exists if $n_2$ is zero. If $n_1$ and
	$n_2$ differ in sign, the implementation-defined result returned
	will be the same as that returned by either the phrase
	\word{toR} \word{StoD} \word{Rfrom} \word{FM/MOD} \word{DROP}
	or the phrase
	\word{toR} \word{StoD} \word{Rfrom} \word{SM/REM} \word{DROP}.

\item[See:]
	\xref{usage:div}{3.2.2.1 Integer division}.
\end{newword}


\begin{newword}{1900}{MOVE}
	\stack{addr$_1$ addr$_2$ u}{}

	If $u$ is greater than zero, copy the contents of u consecutive
	address units at \emph{addr}$_1$ to the $u$ consecutive address
	units at \emph{addr}$_2$. After \word{MOVE} completes, the $u$
	consecutive address units at \emph{addr}$_2$ contain exactly
	what the $u$ consecutive address units at \emph{addr}$_1$
	contained before the move.

\item[See:]
	\wref{string:CMOVE}{CMOVE},
	\wref{string:CMOVEtop}{CMOVE>}.

	\begin{rationale} % A.6.1.1900 MOVE
		\word[string]{CMOVE} and \word[string]{CMOVEtop} are the primary
		move operators in Forth 83. They specify a behavior for moving
		that implies propagation if the move is suitably invoked. In
		some hardware, this specific behavior cannot be achieved using
		the best move instruction. Further, \word[string]{CMOVE} and
		\word[string]{CMOVEtop} move characters; ANS Forth needs a move
		instruction capable of dealing with address units. Thus
		\word{MOVE} has been defined and added to the Core word set,
		and \word[string]{CMOVE} and \word[string]{CMOVEtop} have been
		moved to the String word set.
	\end{rationale}
\end{newword}


\begin{newword}{1910}{NEGATE}
	\stack{$n_1$}{$n_2$}

	Negate $n_1$, giving its arithmetic inverse $n_2$.

\item[See:]
	\wref{core:INVERT}{INVERT},
	\wref{core:0=}{0=}.
\end{newword}


\begin{newword}{1980}{OR}
	\stack{$x_1$ $x_2$}{$x_3$}

	$x_3$ is the bit-by-bit inclusive-or of $x_1$ with $x_2$.
\end{newword}


\begin{newword}{1990}{OVER}
	\stack{$x_1$ $x_2$}{$x_1$ $x_2$ $x_1$}

	Place a copy of $x_1$ on top of the stack.
\end{newword}


\begin{newword*}{2033}{POSTPONE}
\item[Interpretation:]
	Interpretation semantics for this word are undefined.

\item[Compilation:]
	\stack{ ``\arg{spaces}name''}{}

	Skip leading space delimiters. Parse \emph{name} delimited by
	a space. Find \emph{name}. Append the compilation semantics of
	\emph{name} to the current definition. An ambiguous condition
	exists if \emph{name} is not found.

\item[See:]
	\xref{usage:parsing}{3.4.1 Parsing}.

	\begin{rationale} % A.6.1.2033 POSTPONE
		Typical use:

		\tab \word{:} \texttt{ENDIF}
			\word{POSTPONE} \word{THEN}
		\word{;} \word{IMMEDIATE}

		\tab \word{:} \texttt{X} {\ldots}
			\word{IF} {\ldots} \texttt{ENDIF}
		{\ldots} \word{;}

		\word{POSTPONE} replaces most of the functionality of
		\texttt{COMPILE} and  \word{[COMPILE]}. \texttt{COMPILE} and
		\word{[COMPILE]} are used for the same purpose: postpone the
		compilation behavior of the next word in the parse area.
		\texttt{COMPILE} was designed to be applied to non-immediate
		words and \word{[COMPILE]} to immediate words. This burdens
		the programmer with needing to know which words in a system
		are immediate. Consequently, Forth standards have had to
		specify the immediacy or non-immediacy of all words covered by
		the Standard. This unnecessarily constrains implementors.

		A second problem with \texttt{COMPILE} is that some
		programmers have come to expect and exploit a particular
		implementation, namely:

		\tab \word{:} \texttt{COMPILE} \word{Rfrom}
			\word{DUP} \word{@} \word{,} \word{CELL+} \word{toR}
		\word{;}

		This implementation will not work on native code Forth systems.
		In a native code Forth using inline code expansion and peephole
		optimization, the size of the object code produced varies; this
		information is difficult to communicate to a ``dumb''
		\texttt{COMPILE}. A ``smart'' (i.e., immediate) \texttt{COMPILE}
		would not have this problem, but this was forbidden in previous
		standards.

		For these reasons, \texttt{COMPILE} has not been included in
		the Standard and \word{[COMPILE]} has been moved in favor of
		\word{POSTPONE}. Additional discussion can be found in Hayes,
		J.R., ``Postpone'', \emph{Proceedings of the 1989 Rochester
		Forth Conference}.
	\end{rationale}
\end{newword*}


\begin{newword}{2050}{QUIT}
	\stack{}{}
	\stack[R]{i*x}{}

	Empty the return stack, store zero in \word{SOURCE-ID} if it is
	present, make the user input device the input source, and enter
	interpretation state. Do not display a message. Repeat the
	following:
	\begin{itemize}
	\item Accept a line from the input source into the input buffer,
		set \word{toIN} to zero, and interpret.
	\item Display the implementation-defined system prompt if in
		interpretation state, all processing has been completed,
		and no ambiguous condition exists.
	\end{itemize}

\item[See:]
	\xref{usage:command}{3.4 The Forth text interpreter}.
\end{newword}


\begin{newword*}[Rfrom]{2060}{R>}[r-from]
\item[Interpretation:]
	Interpretation semantics for this word are undefined.

\item[Execution:]
	\stack{}{x}
	\stack[R]{x}{}

	Move $x$ from the return stack to the data stack.

\item[See:]
	\xref{usage:returnstack}{3.2.3.3 Return stack},
	\wref{core:toR}{>R},
	\wref{core:R@}{R@},
	\wref{core:2toR}{2>R},
	\wref{core:2Rfrom}{2R>},
	\wref{core:2R@}{2R@}.
\end{newword*}


\begin{newword*}{2070}{R@}[r-fetch]
\item[Interpretation:]
	Interpretation semantics for this word are undefined.

\item[Execution:]
	\stack{}{x}
	\stack[R]{x}{x}

	Copy $x$ from the return stack to the data stack.

\item[See:]
	\xref{usage:returnstack},
	\wref{core:toR}{>R},
	\wref{core:Rfrom}{R>},
	\wref{core:2toR}{2>R},
	\wref{core:2Rfrom}{2R>},
	\wref{core:2R@}{2R@}.
\end{newword*}


\begin{newword*}{2120}{RECURSE}
\item[Interpretation:]
	Interpretation semantics for this word are undefined.

\item[Compilation:]
	\stack{}{}

	Append the execution semantics of the current definition to
	the current definition. An ambiguous condition exists if
	\word{RECURSE} appears in a definition after \word{DOES}.

\item[See:]
	\wref{core:DOES}{DOES>},
	\wref{core:RECURSE}{RECURSE}.

	\begin{rationale} % A.6.1.2120 RECURSE
		Typical use:
			\word{:} \texttt{X} {\ldots} \word{RECURSE} {\ldots} \word{;}

		This is Forth's recursion operator; in some implementations it
		is called \texttt{MYSELF}. The usual example is the coding of
		the factorial function.

		\begin{quote}\ttfamily
		\word{:} FACTORIAL \word{p} +n1 -- +n2) \\
		\tab \word{DUP} 2 \word{less} \word{IF}~
				\word{DROP} 1 \word{EXIT}~
			\word{THEN} \\
		\tab \word{DUP} 1- ~\word{RECURSE}~ \word{*} \\
		\word{;}
		\end{quote}

		$n_2 = n_1(n_1-1)(n_1-2)\cdots(2)(1)$, the product of $n_1$
		with all positive integers less than itself (as a special case,
		zero factorial equals one). While beloved of computer scientists,
		recursion makes unusually heavy use of both stacks and should
		therefore be used with caution. See alternate definition in
		\rref{core:REPEAT}{REPEAT}.
	\end{rationale}
\end{newword*}


\begin{newword*}{2140}{REPEAT}
\item[Interpretation:]
	Interpretation semantics for this word are undefined.

\item[Compilation:]
	\stack[C]{orig dest}{}

	Append the run-time semantics given below to the current
	definition, resolving the backward reference \emph{dest}.
	Resolve the forward reference \emph{orig} using the location
	following the appended run-time semantics.

\item[Run-time:]
	\stack{}{}

	Continue execution at the location given by \emph{dest}.

\item[See:]
	\wref{core:BEGIN}{BEGIN},
	\wref{core:WHILE}{WHILE}.

	\begin{rationale} % A.6.1.2140 REPEAT
		Typical use:
		\begin{quote}\ttfamily
			\word{:} FACTORIAL \word{p} +n1 -- +n2 ) \\
			\tab \word{DUP} 2 \word{less} \word{IF}~
				\word{DROP} 1 \word{EXIT}~ \word{THEN} \\
			\tab \word{DUP} \\
			\tab \word{BEGIN}~ \word{DUP} 2 \word{more} \word{WHILE} \\
			\tab~~ 1- ~\word{SWAP} \word{OVER} \word{*}~ \word{SWAP} \\
			\tab \word{REPEAT} \word{DROP} \\
			\word{;}
		\end{quote}
	\end{rationale}
\end{newword*}


\begin{newword}{2160}{ROT}[rote]
	\stack{$x_1$ $x_2$ $x_3$}{$x_2$ $x_3$ $x_1$}

	Rotate the top three stack entries.
\end{newword}


\begin{newword}{2162}{RSHIFT}[r-shift]
	\stack{$x_1$ u}{$x_2$}

	Perform a logical right shift of $u$ bit-places on $x_1$, giving
	$x_2$. Put zeroes into the most significant bits vacated by the
	shift. An ambiguous condition exists if $u$ is greater than or
	equal to the number of bits in a cell.
\end{newword}


\begin{newword*}[Sq]{2165}{S"}[s-quote]
\item[Interpretation:]
	Interpretation semantics for this word are undefined.

\item[Compilation:]
	\stack{``ccc\arg{quote}''}{}

	Parse \emph{ccc} delimited by \texttt{"} (double-quote). Append
	the run-time semantics given below to the current definition.

\item[Run-time:]
	\stack{}{c-addr u}

	Return \emph{c-addr} and $u$ describing a string consisting of
	the characters \emph{ccc}. A program shall not alter the
	returned string.

\item[See:]
	\xref{usage:parsing}{3.4.1 Parsing},
	\wref{core:Cq}{C"},
	\wref{file:Sq}{S"}.

	\begin{rationale} % A.6.1.2165 S"
		Typical use:
			\word{:} \texttt{X} {\ldots}
				\word{Sq} \emph{ccc}\texttt{"}
			{\ldots} \word{;}

		This word is found in many systems under the name \texttt{"}
		(quote). However, current practice is almost evenly divided on
		the use of \texttt{"}, with many systems using the execution
		semantics given here, while others return the address of a
		counted string. We attempt here to satisfy both camps by
		providing two words, \word{Sq} and the Core Extension word
		\word{Cq} so that users may have whichever behavior they expect
		with a simple renaming operation.
	\end{rationale}
\end{newword*}


\begin{newword}[StoD]{2170}{S>D}[s-to-d]
	\stack{n}{d}

	Convert the number $n$ to the double-cell number $d$ with the
	same numerical value.
\end{newword}


\begin{newword}{2210}{SIGN}
	\stack{n}{}

	If $n$ is negative, add a minus sign to the beginning of the
	pictured numeric output string. An ambiguous condition exists
	if \word{SIGN} executes outside of a \word{num-start} \word{num-end}
	delimited number conversion.
\end{newword}


\begin{newword}{2214}{SM/REM}[s-m-slash-rem]
	\stack{$d_1$ $n_1$}{$n_2$ $n_3$}

	Divide $d_1$ by $n_1$, giving the symmetric quotient $n_3$ and
	the remainder $n_2$. Input and output stack arguments are signed.
	An ambiguous condition exists if $n_1$ is zero or if the quotient
	lies outside the range of a single-cell signed integer.

\item[See:]
	\xref{usage:div}{3.2.2.1 Integer division},
	\wref{core:FM/MOD}{FM/MOD},
	\wref{core:UM/MOD}{UM/MOD}.

	\begin{rationale} % A.6.1.2214 SM/REM
		See the previous discussion of division under \word{FM/MOD}.
		\word{SM/REM} is the symmetric-division primitive, which allows
		programs to define the following symmetric-division operators:

		\begin{quote}\ttfamily
			\word{:} /-REM \word{p} n1 n2 -- n3 n4 )
				\word{toR} \word{StoD} \word{Rfrom} \word{SM/REM}
			\word{;}

			\word{:} /- \word{p} n1 n2 -- n3 )
				/-REM \word{SWAP} \word{DROP}
			\word{;}

			\word{:} -REM \word{p} n1 n2 -- n3 )
				/-REM \word{DROP}
			\word{;}

			\word{:} */-REM \word{p} n1 n2 n3 -- n4 n5 )
				\word{toR} \word{M*} \word{Rfrom} \word{SM/REM}
			\word{;}

			\word{:} */- \word{p} n1 n2 n3 -- n4 )
				*/-REM \word{SWAP} \word{DROP}
			\word{;}
		\end{quote}
	\end{rationale}
\end{newword}


\begin{newword}{2216}{SOURCE}
	\stack{}{c-addr u}

	\emph{c-addr} is the address of, and $u$ is the number of
	characters in, the input buffer.

	\begin{rationale} % A.6.1.2216 SOURCE
		\word{SOURCE} simplifies the process of directly accessing the
		input buffer by hiding the differences between its location
		for different input sources. This also gives implementors more
		flexibility in their implementation of buffering mechanisms
		for different input sources. The committee moved away from an
		input buffer specification consisting of a collection of
		individual variables, declaring \word{TIB} and \word{numTIB}
		obsolescent.

		\word{SOURCE} in this form exists in F83, polyFORTH, LMI's
		Forths and others. In conventional systems it is equivalent to
		the phrase

		\tab \word[block]{BLK} \word{@} \word{IF}
			\word[block]{BLK} \word{@} \word[block]{BLOCK} 1024
		\word{ELSE}
			\word{TIB} \word{numTIB} \word{@}
		\word{THEN}
	\end{rationale}
\end{newword}


\begin{newword}{2220}{SPACE}
	\stack{}{}

	Display one space.
\end{newword}


\begin{newword}{2230}{SPACES}
	\stack{n}{}

	If $n$ is greater than zero, display $n$ spaces.
\end{newword}


\begin{newword}{2250}{STATE}
	\stack{}{a-addr}

	\emph{a-addr} is the address of a cell containing the
	compilation-state flag. \word{STATE} is \emph{true} when in
	compilation state, \emph{false} otherwise. The \emph{true} value
	in \word{STATE} is non-zero, but is otherwise
	implementation-defined. Only the following standard words alter
	the value in \word{STATE}:
	\word{:} (colon),
	\word{;} (semicolon),
	\word{ABORT},
	\word{QUIT},
	\word{:NONAME},
	\word{[} (left-bracket), and
	\word{]} (right-bracket).

\item[Note:]
	A program shall not directly alter the contents of \word{STATE}.

\item[See:]
	\xref{usage:command}{3.4 The Forth text interpreter},
	\wref{core::}{:},
	\wref{core:;}{;}
	\wref{core:ABORT}{ABORT},
	\wref{core:QUIT}{QUIT},
	\wref{core:[}{[},
	\wref{core:]}{]},
	\wref{core::NONAME}{:NONAME},
	\wref{tools:STATE}{STATE}.

	\begin{rationale} % A.6.1.2250 STATE
		Although
		\word{EVALUATE},
		\word[block]{LOAD},
		\word[file]{INCLUDE-FILE}, and
		\word[file]{INCLUDED}
		are not listed as words which alter \word{STATE}, the text
		interpreted by any one of these words could include one or
		more words which explicitly alter \word{STATE}.
		\word{EVALUATE},
		\word[block]{LOAD},
		\word[file]{INCLUDE-FILE}, and
		\word[file]{INCLUDED}
		do not in themselves alter \word{STATE}.

		\word{STATE} does not nest with text interpreter nesting. For
		example, the code sequence:

		\tab \word{:} \texttt{FOO}~
			\word{Sq} \texttt{]"} \word{EVALUATE}
		\word{;}
		\qquad
		\texttt{FOO}

		will leave the system in compilation state. Similarly, after
		\word[block]{LOAD}ing a block containing \word{]}, the system
		will be in compilation state.

		Note that \word{]} does not affect the parse area and that the
		only effect that \word{:} has on the parse area is to parse a
		word. This entitles a program to use these words to set the
		state with known side-effects on the parse area. For example:

		\tab \word{:} \texttt{NOP}~
			\word{:} \word{POSTPONE} \word{;} \word{IMMEDIATE}
		\word{;}

		\tab \texttt{NOP} \word{ALIGN} \\
		\tab \texttt{NOP} \word{ALIGNED}

		Some non-ANS Forth compliant systems have \word{]} invoke a
		compiler loop in addition to setting \word{STATE}. Such a
		system would inappropriately attempt to compile the second
		use of \texttt{NOP}.

		Also note that nothing in the Standard prevents a program from
		finding the execution tokens of \word{]} or \word{[} and using
		these to affect \word{STATE}. These facts suggest that
		implementations of \word{]} will do nothing but set \word{STATE}
		and a single interpreter/compiler loop will monitor \word{STATE}.
	\end{rationale}
\end{newword}


\begin{newword}{2260}{SWAP}
	\stack{$x_1$ $x_2$}{$x_2$ $x_1$}

	Exchange the top two stack items.
\end{newword}


\begin{newword*}{2270}{THEN}
\item[Interpretation:]
	Interpretation semantics for this word are undefined.

\item[Compilation:]
	\stack[C]{orig}{}

	Append the run-time semantics given below to the current
	definition. Resolve the forward reference \emph{orig} using
	the location of the appended run-time semantics.

\item[Run-time:]
	\stack{}{}

	Continue execution.

\item[See:]
	\wref{core:ELSE}{ELSE},
	\wref{core:IF}{IF}.

	\begin{rationale} % A.6.1.2270 THEN
		Typical use:

		\tab \word{:} \texttt{X} {\ldots}
			\emph{test} \word{IF} {\ldots} \word{THEN}
		{\ldots} \word{;}

		or

		\tab \word{:} \texttt{X} {\ldots}
			\emph{test} \word{IF} {\ldots} \word{ELSE} {\ldots} \word{THEN}
		{\ldots} \word{;}
	\end{rationale}
\end{newword*}


\begin{newword}{2310}{TYPE}
	\stack{c-addr u}{}

	If $u$ is greater than zero, display the character string
	specified by \emph{c-addr} and $u$.

	When passed a character in a character string whose
	character-defining bits have a value between hex 20 and 7E
	inclusive, the corresponding standard character, specified
	by \xref{usage:ASCII}{3.1.2.1 graphic characters}, is displayed.
	Because different output devices can respond differently to
	control characters, programs that use control characters to
	perform specific functions have an environmental dependency.

\item[See:]
	\wref{core:EMIT}{EMIT}.
\end{newword}


\begin{newword}{2320}{U.}[u-dot]
	\stack{u}{}

	Display $u$ in free field format.
\end{newword}


\begin{newword}[Uless]{2340}{U<}[u-less-than]
	\stack{$u_1$ $u_2$}{flag}

	\emph{flag} is true if and only if $u_1$ is less than $u_2$.

\item[See:]
	\wref{core:less}{<}.
\end{newword}


\begin{newword}{2360}{UM*}[u-m-star]
	\stack{$u_1$ $u_2$}{ud}

	Multiply $u_1$ by $u_2$, giving the unsigned double-cell
	product \emph{ud}. All values and arithmetic are unsigned.
\end{newword}


\begin{newword}{2370}{UM/MOD}[u-m-slash-mod]
	\stack{ud $u_1$}{$u_2$ $u_3$}

	Divide \emph{ud} by $u_1$, giving the quotient $u_3$ and the
	remainder $u_2$. All values and arithmetic are unsigned. An
	ambiguous condition exists if $u_1$ is zero or if the quotient
	lies outside the range of a single-cell unsigned integer.

\item[See:]
	\xref{usage:div}{3.2.2.1 Integer division},
	\wref{core:FM/MOD}{FM/MOD},
	\wref{core:SM/REM}{SM/REM}.
\end{newword}


\begin{newword*}{2380}{UNLOOP}
\item[Interpretation:]
	Interpretation semantics for this word are undefined.

\item[Execution:]
	\stack{}{}
	\stack[R]{loop-sys}{}

	Discard the loop-control parameters for the current nesting
	level. An \word{UNLOOP} is required for each nesting level
	before the definition may be \word{EXIT}ed. An ambiguous
	condition exists if the loop-control parameters are unavailable.

\item[See:]
	\xref{usage:returnstack}{3.2.3.3 Return stack}.

	\begin{rationale} % A.6.1.2380 UNLOOP
		Typical use:
		\begin{quote}
			\word{:} \texttt{X} {\ldots} \\
			\tab \emph{limit} \emph{first} \word{DO} \\
			\tab~~ {\ldots} \emph{test} \word{IF}
				{\ldots} \word{UNLOOP} \word{EXIT} \word{THEN} {\ldots} \\
			\tab \word{LOOP} {\ldots} \\
			\word{;}
		\end{quote}

		\word{UNLOOP} allows the use of \word{EXIT} within the context
		of \word{DO} {\ldots} \word{LOOP} and related do-loop constructs.
		\word{UNLOOP} as a function has been called \texttt{UNDO}.
		\word{UNLOOP} is more indicative of the action: nothing gets
		undone --- we simply stop doing it.
	\end{rationale}
\end{newword*}


\begin{newword*}{2390}{UNTIL}
\item{Interpretation:}
	Interpretation semantics for this word are undefined.

\item[Compilation:]
	\stack[C]{dest}{}

	Append the run-time semantics given below to the current
	definition, resolving the backward reference \emph{dest}.

\item[Run-time:]
	\stack{x}{}

	If all bits of $x$ are zero, continue execution at the
	location specified by \emph{dest}.

\item[See:]
	\wref{core:BEGIN}{BEGIN}.

	\begin{rationale} % A.6.1.2390 UNTIL
		Typical use:
			\word{:} \texttt{X} {\ldots}
				\word{BEGIN} {\ldots} \emph{test} \word{UNTIL}
			{\ldots} \word{;}
	\end{rationale}
\end{newword*}


\begin{newword}{2410}{VARIABLE}
	\stack{``\arg{spaces}name''}{}

	Skip leading space delimiters. Parse \emph{name} delimited by
	a space. Create a definition for \emph{name} with the execution
	semantics defined below. Reserve one cell of data space at an
	aligned address.

	\emph{name} is referred to as a ``variable''.

\item[\emph{name} Execution:]
	\stack{}{a-addr}

	\emph{a-addr} is the address of the reserved cell. A program
	is responsible for initializing the contents of the reserved
	cell.

\item[See:]
	\xref{usage:parsing}{3.4.1 Parsing}.

	\begin{rationale} % A.6.1.2410 VARIABLE
		Typical use:
			{\ldots} \word{VARIABLE} \texttt{XYZ} {\ldots}
	\end{rationale}
\end{newword}


\begin{newword*}{2430}{WHILE}
\item[Interpretation:]
	Interpretation semantics for this word are undefined.

\item[Compilation:]
	\stack[C]{dest}{orig dest}

	Put the location of a new unresolved forward reference
	\emph{orig} onto the control flow stack, under the existing
	\emph{dest}. Append the run-time semantics given below to the
	current definition. The semantics are incomplete until
	\emph{orig} and \emph{dest} are resolved (e.g., by
	\word{REPEAT}).

\item[Run-time:]
	\stack{x}{}

	If all bits of $x$ are zero, continue execution at the location
	specified by the resolution of \emph{orig}.

	\begin{rationale} % A.6.1.2430 WHILE
		Typical use:
			\word{:} \texttt{X} {\ldots}
				\word{BEGIN} {\ldots}
				\emph{test} \word{WHILE}
				{\ldots} \word{REPEAT}
			{\ldots} \word{;}
	\end{rationale}
\end{newword*}


\begin{newword}{2450}{WORD}
	\stack{char ``\arg{chars}ccc\arg{char}''}{c-addr}

	Skip leading delimiters. Parse characters \emph{ccc} delimited
	by \emph{char}.  An ambiguous condition exists if the length of
	the parsed string is greater than the implementation-defined
	length of a counted string.

	\emph{c-addr} is the address of a transient region containing
	the parsed word as a counted string. If the parse area was
	empty or contained no characters other than the delimiter, the
	resulting string has a zero length. A space, not included in the
	length, follows the string. A program may replace characters
	within the string.

\item[Note:]
	The requirement to follow the string with a space is obsolescent
	and is included as a concession to existing programs that use
	\word{CONVERT}. A program shall not depend on the existence of
	the space.

\item[See:]
	\xref{usage:transient}{3.3.3.6 Other transient regions},
	\xref{usage:parsing}{3.4.1 Parsing}.

	\begin{rationale} % A.6.1.2450 WORD
		Typical use: \emph{char} \word{WORD} \emph{ccc}\arg{char}
	\end{rationale}
\end{newword}


\begin{newword}{2490}{XOR}[x-or]
	\stack{$x_1$ $_2$}{$x_3$}

	$x_3$ is the bit-by-bit exclusive-or of $x_1$ with $x_2$.
\end{newword}


\begin{newword*}{2500}{[}[left-bracket]
\item[Interpretation:]
	Interpretation semantics for this word are undefined.

\item[Compilation:]
	Perform the execution semantics given below.

\item[Execution:]
	\stack{}{}

	Enter interpretation state. \word{[} is an immediate word.

\item[See:]
	\xref{usage:command}{The Forth text interpreter},
	\xref{usage:compilation}{3.4.5 Compilation},
	\wref{core:]}{]}.

	\begin{rationale} % A.6.1.2500 [
		Typical use:
			\word{:} \texttt{X} {\ldots}
				\word{[} \texttt{4321} \word{]} \word{LITERAL}
			{\ldots} \word{;}
	\end{rationale}
\end{newword*}


\begin{newword*}{2510}{[']}[bracket-tick]
\item[Interpretation:]
	Interpretation semantics for this word are undefined.

\item[Compilation:]
	\stack{``\arg{spaces}name''}{}

	Skip leading space delimiters. Parse \emph{name} delimited by
	a space. Find \emph{name}. Append the run-time semantics given
	below to the current definition.

	An ambiguous condition exists if \emph{name} is not found.

\item[Run-time:]
	\stack{}{xt}

	Place \emph{name}'s execution token \emph{xt} on the stack. The
	execution token returned by the compiled phrase
	``\word{[']} \texttt{X}'' is the same value returned by
	``\word{'} \texttt{X}'' outside of compilation state.

\item[See:]
	\xref{usage:parsing}{3.4.1 Parsing}
	\rref{core:'}{'}
	\rref{core:POSTPONE}{POSTPONE},
	\xref{diff:immediate}{Immediate}.

	\begin{rationale} % A.6.1.2510 [']
		Typical use:
			\word{:} \texttt{X} {\ldots}
				\word{[']} \emph{name}
			{\ldots} \word{;}

		See:
			\rref{core:FIND}{FIND}.
	\end{rationale}
\end{newword*}


\begin{newword*}{2520}{[CHAR]}[bracket-char]
\item[Interpretation:]
	Interpretation semantics for this word are undefined.

\item[Compilation:]
	\stack{``\arg{spaces}name''}{}

	Skip leading space delimiters. Parse \emph{name} delimited by
	a space. Append the run-time semantics given below to the
	current definition.

\item[Run-time:]
	\stack{}{char}

	Place \emph{char}, the value of the first character of
	\emph{name}, on the stack.

\item[See:]
	\xref{usage:parsing}{3.4.1 Parsing},
	\wref{core:CHAR}{CHAR}.

	\begin{rationale} % A.6.1.2520 [CHAR]
		Typical use:
			\word{:} \texttt{X} {\ldots}
				\word{[CHAR]} \emph{c}
			{\ldots} \word{;}
	\end{rationale}
\end{newword*}


\begin{newword}{2540}{]}[right-bracket]
	\stack{}{}

	Enter compilation state.

\item[See:]
	\xref{usage:command}{3.4 The Forth text interpreter},
	\xref{usage:compilation}{3.4.5 Compilation},
	\wref{core:[}{[}.

	\begin{rationale} % A.6.1.2540 ]
		Typical use:
			\word{:} \texttt{X} {\ldots}
				\word{[} \texttt{4321} \word{]} \word{LITERAL}
			{\ldots} \word{;}
	\end{rationale}
\end{newword}


\section{Core extension words} % 6.2
\extended

\begin{info}

\subsection{Core extension words} % A.6.2

The words in this collection fall into several categories:

\begin{itemize}
\item Words that are in common use but are deemed less essential than
	Core words (e.g., \word{0ne});

\item Words that are in common use but can be trivially defined from
	Core words (e.g., \word{FALSE});

\item Words that are primarily useful in narrowly defined types of
	applications or are in less frequent use (e.g., \word{PARSE});

\item Words that are being deprecated in favor of new words introduced
	to solve specific problems (e.g., \word{CONVERT}).
\end{itemize}

Because of the varied justifications for inclusion of these words,
the Technical Committee does not encourage implementors to offer the
complete collection, but to select those words deemed most valuable to
their clientele.
\end{info}


\begin{newword}[numTIB]{0060}{\#TIB}[number-t-i-b]
	\stack{}{a-addr}

	\emph{a-addr} is the address of a cell containing the number
	of characters in the terminal input buffer.

\item[Note:]
	This word is obsolescent and is included as a concession to
	existing implementations.

	\begin{rationale} % A.6.2.0060 #TIB}
		The function of \word{numTIB} has been superseded by
		\word{SOURCE}.
	\end{rationale}
\end{newword}


\begin{newword*}{0200}{.p}[dot-paren]
\item[Compilation:]
	Perform the execution semantics given below.

\item[Execution:]
	\stack{``ccc\arg{paren}''}{}

	Parse and display \emph{ccc} delimited by \texttt{)} (right
	parenthesis). \word{.p} is an immediate word.

\item[See:]
	\xref{usage:parsing}{3.4.1 Parsing},
	\wref{core:.q}{."}.

	\begin{rationale} % A.6.2.0200 .
		Typical use:
			\word{.p} \emph{ccc}\texttt{)}
	\end{rationale}
\end{newword*}


\begin{newword}{0210}{.R}[dot-r]
	\stack{$n_1$ $n_2$}{}

	Display $_n$ right aligned in a field $n_2$ characters wide. If
	the number of characters required to display $n_1$ is greater
	than $n_2$, all digits are displayed with no leading spaces in
	a field as wide as necessary.

	\begin{rationale} % A.6.2.0210 .R
		In \word{.R}, ``R'' is short for RIGHT.
	\end{rationale}
\end{newword}


\begin{newword}[0ne]{0260}{0<>}[zero-not-equals]
	\stack{x}{flag}

	\emph{flag} is true if and only if $x$ is not equal to zero.
\end{newword}


\begin{newword}[0more]{0280}{0>}[zero-greater]
	\stack{n}{flag}

	\emph{flag} is true if and only if $n$ is greater than zero.
\end{newword}


\begin{newword*}[2toR]{0340}{2>R}[two-to-r]
\item[Interpretation:]
	Interpretation semantics for this word are undefined.

\item[Execution:]
	\stack{$x_1$ $x_2$}{}
	\stack[R]{}{$x_1$ $x_2$}

	Transfer cell pair $x_1$ $x_2$ to the return stack. Semantically
	equivalent to \word{SWAP} \word{toR} \word{toR}.

\item[See:]
	\xref{usage:returnstack}{3.2.3.3 Return stack}
	\wref{core:toR}{>R},
	\wref{core:Rfrom}{R>},
	\wref{core:R@}{R@},
	\wref{core:2Rfrom}{2R>},
	\wref{core:2R@}{2R@}.

	\begin{rationale} % A.6.2.0340 2>R
		Historically, \word{2toR} has been used to implement \word{DO}.
		Hence the order of parameters on the return stack.

		The primary advantage of \word{2toR} is that it puts the top
		stack entry on the top of the return stack. For instance, a
		double-cell number may be transferred to the return stack and
		still have the most significant cell accessible on the top of
		the return stack.
	\end{rationale}
\end{newword*}


\begin{newword*}[2Rfrom]{0410}{2R>}[two-r-from]
\item[Interpretation:]
	Interpretation semantics for this word are undefined.

\item[Execution:]
	\stack{}{$x_1$ $x_2$}
	\stack[R]{$x_1$ $x_2$}{}

	Transfer cell pair $x_1$ $x_2$ from the return stack.
	Semantically equivalent to \word{Rfrom} \word{Rfrom} word{SWAP}.

\item[See:]
	\xref{usage:returnstack}{3.2.3.3 Return stack},
	\wref{core:toR}{>R}
	\wref{core:Rfrom}{R>}
	\wref{core:R@}{R@}
	\wref{core:2toR}{2>R},
	\wref{core:2R@}{2R@}.

	\begin{rationale} % A.6.2.0410 2R>
		Note that \word{2Rfrom} is not equivalent to \word{Rfrom} \word{Rfrom}.
		Instead, it mirrors the action of \word{2toR}
		(see \ref{rat:core:2toR}).
	\end{rationale}
\end{newword*}


\begin{newword*}{0415}{2R@}[two-r-fetch]
\item[Interpretation:]
	Interpretation semantics for this word are undefined.

\item[Execution:]
	\stack{}{$x_1$ $x_2$}
	\stack[R]{$x_1$ $x_2$}{$x_1$ $x_2$}

	Copy cell pair $x_1$ $x_2$ from the return stack.
	Semantically equivalent to \word{Rfrom} \word{Rfrom} \word{2DUP}
	\word{toR} \word{toR} \word{SWAP}.

\item[See:]
	\xref{usage:returnstack}{3.2.3.3 Return stack},
	\wref{core:toR}{>R},
	\wref{core:Rfrom}{R>},
	\wref{core:R@}{R@},
	\wref{core:2toR}{2>R},
	\wref{core:2Rfrom}{2R>}.
\end{newword*}


\begin{newword}{0455}{:NONAME}[colon-no-name]
	\stack[C]{}{colon-sys}
	\stack[S]{}{xt}

	Create an execution token \emph{xt}, enter compilation state
	and start the current definition, producing \emph{colon-sys}.
	Append the initiation semantics given below to the current
	definition.

	The execution semantics of \emph{xt} will be determined by the
	words compiled into the body of the definition. This definition
	can be executed later by using \emph{xt} \word{EXECUTE}.

	If the control-flow stack is implemented using the data stack,
	\emph{colon-sys} shall be the topmost item on the data stack.
	See \xref{usage:controlstack}{3.2.3.2 Control-flow stack}.

\item[Initiation:]
	\stack{i*x}{i*x}
	\stack[R]{}{nest-sys}

	Save implementation-dependent information \emph{nest-sys} about
	the calling definition. The stack effects \emph{i*x} represent
	arguments to \emph{xt}.

\item[\emph{xt} Execution:]
	\stack{i*x}{j*x}

	Execute the definition specified by \emph{xt}. The stack effects
	\emph{i*x} and \emph{j*x} represent arguments to and results
	from \emph{xt}, respectively.


	\begin{rationale} % A.6.2.0455 :NONAME
		\word{:NONAME} allows a user to create an execution token
		with the semantics of a colon definition without an associated
		name. Previously, only \word{:} (colon) could create an
		execution token with these semantics. Thus, Forth code could
		only be compiled using the syntax of \word{:}, that is:

		\tab \word{:} \texttt{NAME} {\ldots} \word{;}

		\word{:NONAME} removes this constraint and places the Forth
		compiler in the hands of the programmer.

		\word{:NONAME} can be used to create application-specific
		programming languages. One technique is to mix Forth code
		fragments with application-specific constructs. The
		application-specific constructs use \word{:NONAME} to compile
		the Forth code and store the corresponding execution tokens
		in data structures.

		The functionality of \word{:NONAME} can be built on any Forth
		system. For years, expert Forth programmers have exploited
		intimate knowledge of their systems to generate unnamed code
		fragments. Now, this function has been named and can be used
		in a portable program.

		For example, \word{:NONAME} can be used to build a table of
		code fragments where indexing into the table allows executing
		a particular fragment. The declaration syntax of the table is:

		\begin{quote}\ttfamily
			\word{:NONAME} {\ldots} code for command 0 {\ldots} \word{;}
			0 CMD \word{!}

			\word{:NONAME} {\ldots} code for command 1 {\ldots} \word{;}
			1 CMD \word{!}

			\tab {\ldots}

			\word{:NONAME} {\ldots} code for command 99 {\ldots} \word{;}
			99 CMD \word{!}

			{\ldots} 5 CMD \word{@} \word{EXECUTE} {\ldots}
		\end{quote}

		The definitions of the table building words are:

		\begin{quote}\ttfamily
			\word{CREATE} CMD-TABLE
				\word{bs} table for command execution tokens \\
			\tab 100 \word{CELLS} \word{ALLOT}

			\word{:} CMD \word{p} n -- a-addr )
				\word{bs} nth element address in table \\
			\tab \word{CELLS} CMD-TABLE \word{+}
			\word{;}
		\end{quote}

		As a further example, a defining word can be created to allow
		performance monitoring. In the example below, the number of
		times a word is executed is counted. \word{:} must first be
		renamed to allow the definition of the new \word{;}.

		\begin{quote}\ttfamily
		  \begin{tabbing}
			\tab \= \tab \= \hspace*{12em} \= \kill
			\word{:} DOCOLON \word{p} -- ) \\
			\+ \word{bs} Modify CREATEd word to execute like a colon def \\
				\+ \word{DOES} \word{p} i*x a-addr -- j*x ) \\
					1 \word{OVER} \word{+!}								\> \word{bs} count executions \\
			\-	\-	\word{CELL+} \word{@} \word{EXECUTE}				\> \word{bs} execute :NONAME definition \\
			\word{;} \\[1.5\parskip]

			\word{:} OLD: \word{:} \word{;}								\>\>\> \word{bs} just an alias \\[1.5\parskip]

			OLD: \word{:} \word{p} "name" -- a-addr xt colon-sys ) \\
			\+ \word{bs} begins an execution-counting colon definition \\
				\word{CREATE} \word{HERE} 0 \word{,}					\>\> \word{bs} storage for execution counter \\
				0 \word{,}												\>\> \word{bs} storage for execution token \\
				DOCOLON													\>\> \word{bs} set run time for CREATEd word \\
			\-	\word{:NONAME}											\>\> \word{bs} begin unnamed colon definition \\
			\word{;}
		  \end{tabbing}
		\end{quote}

		(Note the placement of \word{DOES}: \word{DOES} must modify
		the \word{CREATE}d word and not the \word{:NONAME} definition,
		so \word{DOES} must execute before \word{:NONAME}.)

		\begin{quote}\ttfamily
		  \begin{tabbing}
			\tab \= \tab \= \hspace*{12em} \= \kill
			OLD: \word{;} \word{p} a-addr xt colon-sys -- ) \\
			\+ \word{bs} ends an execution-counting colon definition \\
				\word{POSTPONE} \word{;}										\>\> \word{bs} complete compilation of colon def \\
			\-	\word{SWAP} \word{CELL+} \word{!}								\>\> \word{bs} save execution token \\
			\word{;} \word{IMMEDIATE}
		  \end{tabbing}
		\end{quote}

		The new \word{:} and \word{;} are used just like the standard
		ones to define words:

		\tab {\ldots} \word{:} \texttt{xxx} {\ldots} \word{;}
			{\ldots} \texttt{xxx} {\ldots}

		Now however, these words may be ``ticked'' to retrieve the
		count (and execution token):

		\tab {\ldots} \word{'} \texttt{xxx} \word{toBODY} \word[tools]{?} {\ldots}
	\end{rationale}
\end{newword}


\begin{newword}[ne]{0500}{<>}[not-equals]
	\stack{$x_1$ $x_2$}{flag}

	\emph{flag} is true if and only if $x_1$ is not bit-for-bit the
	same as $x_2$.
\end{newword}


\begin{newword*}{0620}{?DO}[question-do]
\item[Interpretation:]
	Interpretation semantics for this word are undefined.

\item[Compilation:]
	\stack[C]{}{do-sys}

	Put \emph{do-sys} onto the control-flow stack. Append the
	run-time semantics given below to the current definition. The
	semantics are incomplete until resolved by a consumer of
	\emph{do-sys} such as \word{LOOP}.

\item[Run-time:]
	\stack{$n_1$\textbar$u_1$ $n_2$\textbar$u_2$}{}
	\stack[R]{}{loop-sys}

	If $n_1$\textbar$u_1$ is equal to $n_2$\textbar$u_2$,
	continue execution at the location given by the consumer of
	\emph{do-sys}. Otherwise set up loop control parameters with
	index $n_2$\textbar$u_2$ and limit $n_1$\textbar$u_1$ and
	continue executing immediately following \word{?DO}. Anything
	already on the return stack becomes unavailable until the loop
	control parameters are discarded. An ambiguous condition exists
	if $n_1$\textbar$u_1$ and $n_2$\textbar$u_2$ are not both of
	the same type.

\item[See:]
	\xref{usage:controlstack}{3.2.3.2 Control-flow stack},
	\wref{core:+LOOP}{+LOOP},
	\wref{core:DO}{DO},
	\wref{core:I}{I},
	\wref{core:LEAVE}{LEAVE},
	\wref{core:LOOP}{LOOP},
	\wref{core:UNLOOP}{UNLOOP}.

	\begin{rationale} % A.6.2.0620 ?DO
		Typical use:

		\tab \word{:} \texttt{FACTORIAL} \word{p} +n1 -- +n2 )~
			1 \word{SWAP} \word{1+} \word{?DO}~
				\word{I} \word{*}~
			\word{LOOP}
		\word{;}

		This word was added in response to many requests for a
		resolution of the difficulty introduced by Forth-83's DO,
		which on a 16-bit system will loop 65,535 times if given
		equal arguments. As this Standard also encourages 32-bit
		systems, this behavior can be intolerable. The Technical
		Committee considered applying these semantics to DO, but
		declined on the grounds that it might break existing code.
	\end{rationale}
\end{newword*}


\begin{newword*}{0700}{AGAIN}
\item[Interpretation:]
	Interpretation semantics for this word are undefined.

\item[Compilation:]
	\stack[C]{dest}{}

	Append the run-time semantics given below to the current
	definition, resolving the backward reference \emph{dest}.

\item[Run-time:]
	\stack{}{}

	Continue execution at the location specified by \emph{dest}.
	If no other control flow words are used, any program code
	after \word{AGAIN} will not be executed.

\item[See:]
	\wref{core:BEGIN}{BEGIN}.

	\begin{rationale} % A.6.2.0700 AGAIN
		Typical use:
			\word{:} \texttt{X}
				{\ldots} \word{BEGIN} {\ldots} \word{AGAIN}
			{\ldots} \word{;}

		Unless word-sequence has a way to terminate, this is an
		endless loop.
	\end{rationale}
\end{newword*}


\begin{newword*}[Cq]{0855}{C"}[c-quote]
\item[Interpretation:]
	Interpretation semantics for this word are undefined.

\item[Compilation:]
	\stack{``ccc\arg{quote}''}{}

	Parse \emph{ccc} delimited by \texttt{"} (double-quote) and
	append the run-time semantics given below to the current
	definition.

\item[Run-time:]
	\stack{}{c-addr}

	Return \emph{c-addr}, a counted string consisting of the
	characters \emph{ccc}. A program shall not alter the returned
	string.

\item[See:]
	\xref{usage:parsing}{3.4.1 Parsing},
	\wref{core:Sq}{S"},
	\wref{file:Sq}{S"}.

	\begin{rationale} % A.6.2.0855 C"
		Typical use:
			\word{:} \texttt{X} {\ldots}
				\word{Cq} \emph{ccc}\texttt{"}
			{\ldots} \word{;}

		It is easy to convert counted strings to pointer/length but
		hard to do the opposite. \word{Cq} is the only new word that
		uses the ``address of counted string'' stack representation.
		It is provided as an aid to porting existing programs to ANS
		Forth systems. It is relatively difficult to implement \word{Cq}
		in terms of other standard words, considering its ``compile
		string into the current definition'' semantics.

		Users of \word{Cq} are encouraged to migrate their application
		code toward the consistent use of the preferred ``\emph{c-addr}
		$u$'' stack representation with the alternate word \word{Sq}.
		This may be accomplished by converting application words with
		counted string input arguments to use the preferred
		``\emph{c-addr} $u$'' representation, thus eliminating the
		need for \word{Cq}.

		See: \xref{rat:cstring}{A.3.1.3.4 Counted strings}.
	\end{rationale}
\end{newword*}


\begin{newword*}{0873}{CASE}
\item[Interpretation:]
	Interpretation semantics for this word are undefined.

\item[Compilation:]
	\stack[C]{}{case-sys}

	Mark the start of the
	\word{CASE}\ldots\word{OF}\ldots\word{ENDOF}\ldots\word{ENDCASE}
	structure. Append the run-time semantics given below to the
	current definition.

\item[Run-time:]
	\stack{}{}

	Continue execution.

\item[See:]
	\wref{core:ENDCASE}{ENDCASE},
	\wref{core:ENDOF}{ENDOF},
	\wref{core:OF}{OF}.

	\begin{rationale} % A.6.2.0873 CASE
		Typical use:
		\begin{quote}
			\word{:} \texttt{X} {\ldots} \\
			\tab \word{CASE} \\
			\tab~~ \emph{test1} \word{OF} {\ldots} \word{ENDOF} \\
			\tab~~ \emph{testn} \word{OF} {\ldots} \word{ENDOF} \\
			\tab~~ {\ldots} \word{p} default ) \\
			\tab \word{ENDCASE} {\ldots} \\
			\word{;}
		\end{quote}
	\end{rationale}
\end{newword*}


\begin{newword*}{0945}{COMPILE,}[compile-comma]
\item[Interpretation:]
	Interpretation semantics for this word are undefined.

\item[Execution:]
	\stack{xt}{}

	Append the execution semantics of the definition represented
	by \emph{xt} to the execution semantics of the current
	definition.

	\begin{rationale} % A.6.2.0945 COMPILE,
		\word{COMPILE,} is the compilation equivalent of
		\word{EXECUTE}. In many cases, it is possible to compile a
		word by using \word{POSTPONE} without resorting to the use of
		\word{COMPILE,}. However, the use of \word{POSTPONE} requires
		that the name of the word must be known at compile time,
		whereas \word{COMPILE,} allows the word to be located at any
		time. It is sometime possible to use \word{EVALUATE} to compile
		a word whose name is not known until run time. This has two
		possible problems:
		\begin{itemize}
		\item \word{EVALUATE} is slower than \word{COMPILE,} because a
			dictionary search is required.
		\item The current search order affects the outcome of
			\word{EVALUATE}.
		\end{itemize}
		In traditional threaded-code implementations, compilation is
		performed by \word{,} (comma). This usage is not portable; it
		doesn't work for subroutine-threaded, native code, or
		relocatable implementations. Use of \word{COMPILE,} is portable.

		In most systems it is possible to implement \word{COMPILE,} so
		it will generate code that is optimized to the same extent as
		code that is generated by the normal compilation process.
		However, in some implementations there are two different
		``tokens'' corresponding to a particular definition name:
		the normal ``execution token'' that is used while interpreting
		or with \word{EXECUTE}, and another ``compilation token'' that
		is used while compiling. It is not always possible to obtain
		the compilation token from the execution token. In these
		implementations, \word{COMPILE,} might not generate code that
		is as efficient as normally compiled code.
	\end{rationale}
\end{newword*}


\begin{newword}{0970}{CONVERT}
	\stack{ud$_1$ c-addr$_1$}{ud$_2$ c-addr$_2$}

	\emph{ud}$_2$ is the result of converting the characters within
	the text beginning at the first character after \emph{c-addr}$_1$
	into digits, using the number in \word{BASE}, and adding each
	digit to \emph{ud}$_1$ after multiplying \emph{ud}$_1$ by the
	number in \word{BASE}. Conversion continues until a character
	that is not convertible is encountered. \emph{c-addr}$_2$ is
	the location of the first unconverted character. An ambiguous
	condition exists if \emph{ud}$_2$ overflows.

\item[Note:]
	This word is obsolescent and is included as a concession to
	existing implementations. Its function is superseded by
	\wref{core:toNUMBER}{>NUMBER}.

\item[See:]
	\xref{usage:digits}{3.2.1.2 Digit conversion}.

	\begin{rationale} % A.6.2.0970 CONVERT
		\word{CONVERT} may be defined as follows:

		\tab \word{:} \word{CONVERT}~
			\word{CHAR+} \texttt{65535} \word{toNUMBER} \word{DROP}
		\word{;}
	\end{rationale}
\end{newword}


\begin{newword*}{1342}{ENDCASE}[end-case]
\item[Interpretation:]
	Interpretation semantics for this word are undefined.

\item[Compilation:]
	\stack[C]{case-sys}{}

	Mark the end of the
	\word{CASE}\ldots\word{OF}\ldots\word{ENDOF}\ldots\word{ENDCASE}
	structure. Use \emph{case-sys} to resolve the entire structure.
	Append the run-time semantics given below to the current
	definition.

\item[Run-time:]
	\stack{x}{}

	Discard the case selector $x$ and continue execution.

\item[See:]
	\wref{core:CASE}{CASE},
	\wref{core:ENDOF}{ENDOF},
	\wref{core:OF}{OF}.

	\begin{rationale} % A.6.2.1342 ENDCASE
		Typical use:
		\begin{quote}
			\word{:} \texttt{X} {\ldots} \\
			\tab \word{CASE} \\
			\tab~~ \emph{test1} \word{OF} {\ldots} \word{ENDOF} \\
			\tab~~ \emph{testn} \word{OF} {\ldots} \word{ENDOF} \\
			\tab~~ {\ldots} \word{p} default ) \\
			\tab \word{ENDCASE} {\ldots} \\
			\word{;}
		\end{quote}
	\end{rationale}
\end{newword*}


\begin{newword*}{1343}{ENDOF}[end-of]
\item[Interpretation:]
	Interpretation semantics for this word are undefined.

\item[Compilation:]
	\stack[C]{case-sys$_1$ of-sys}{case-sys$_2$}

	Mark the end of the \word{OF}\ldots\word{ENDOF} part of the
	\word{CASE} structure. The next location for a transfer of
	control resolves the reference given by \emph{of-sys}. Append
	the run-time semantics given below to the current definition.
	Replace \emph{case-sys}$_1$ with \emph{case-sys}$_2$ on the
	control-flow stack, to be resolved by \word{ENDCASE}.

\item[Run-time:]
	\stack{}{}

	Continue execution at the location specified by the consumer
	of \emph{case-sys}$_2$.

\item[See:]
	\wref{core:CASE}{CASE},
	\wref{core:ENDCASE}{ENDCASE},
	\wref{core:OF}{OF}.

	\begin{rationale} % A.6.2.1343 ENDOF
		Typical use:
		\begin{quote}
			\word{:} \texttt{X} {\ldots} \\
			\tab \word{CASE} \\
			\tab~~ \emph{test1} \word{OF} {\ldots} \word{ENDOF} \\
			\tab~~ \emph{testn} \word{OF} {\ldots} \word{ENDOF} \\
			\tab~~ {\ldots} \word{p} default ) \\
			\tab \word{ENDCASE} {\ldots} \\
			\word{;}
		\end{quote}
	\end{rationale}
\end{newword*}


\begin{newword}{1350}{ERASE}
	\stack{addr u}{}

	If $u$ is greater than zero, clear all bits in each of $u$
	consecutive address units of memory beginning at \emph{addr}.
\end{newword}


\begin{newword}{1390}{EXPECT}
	\stack{c-addr +n}{}

	Receive a string of at most $+n$ characters. Display graphic
	characters as they are received. A program that depends on
	the presence or absence of non-graphic characters in the
	string has an environmental dependency. The editing functions,
	if any, that the system performs in order to construct the
	string of characters are implementation-defined.

	Input terminates when an implementation-defined line terminator
	is received or when the string is $+n$ characters long. When
	input terminates, nothing is appended to the string and the
	display is maintained in an implementation-defined way.

	Store the string at \emph{c-addr} and its length in \word{SPAN}.

\item[Note:]
	This word is obsolescent and is included as a concession to
	existing implementations. Its function is superseded
	by \wref{core:ACCEPT}{ACCEPT}.

	\begin{rationale} % A.6.2.1390 EXPECT
		Specification of positive integer counts ($+n$) for
		\word{EXPECT} allows some implementors to continue their
		practice of using a zero or negative value as a flag to
		trigger special behavior. Insofar as such behavior is outside
		the Standard, Standard Programs cannot depend upon it, but
		the Technical Committee doesn't wish to preclude it
		unnecessarily. Since actual values are almost always small
		integers, no functionality is impaired by this restriction.
	\end{rationale}
\end{newword}


\begin{newword}{1485}{FALSE}
	\stack{}{false}

	Return a \emph{false} flag.

\item[See:]
	\xref{usage:flags}{3.1.3.1 Flags}
\end{newword}


\begin{newword}{1660}{HEX}
	\stack{}{}

	Set contents of \word{BASE} to sixteen.
\end{newword}


\begin{newword}{1850}{MARKER}
	\stack{``\arg{spaces}name''}{}

	Skip leading space delimiters. Parse \emph{name} delimited by
	a space. Create a definition for \emph{name} with the execution
	semantics defined below.

\item[\emph{name} Execution:]
	\stack{}{}

	Restore all dictionary allocation and search order pointers to
	the state they had just prior to the definition of \emph{name}.
	Remove the definition of \emph{name} and all subsequent
	definitions. Restoration of any structures still existing that
	could refer to deleted definitions or deallocated data space is
	not necessarily provided. No other contextual information such
	as numeric base is affected.

\item[See:]
	\xref{usage:parsing}{3.4.1 Parsing},
	\wref{tools:FORGET}{FORGET}.

	\begin{rationale} % A.6.2.1850 MARKER
		As dictionary implementations have gotten more elaborate
		and in some cases have used multiple address spaces,
		\word[tools]{FORGET} has become prohibitively difficult or
		impossible to implement on many Forth systems. \word{MARKER}
		greatly eases the problem by making it possible for the
		system to remember ``landmark information'' in advance that
		specifically marks the spots where the dictionary may at some
		future time have to be rearranged.
	\end{rationale}
\end{newword}


\begin{newword}{1930}{NIP}
	\stack{$x_1$ $x_2$}{$x_2$}

	Drop the first item below the top of stack.
\end{newword}


\begin{newword*}{1950}{OF}
\item[Interpretation:]
	Interpretation semantics for this word are undefined.

\item[Compilation:]
	\stack[C]{}{of-sys}

	Put \emph{of-sys} onto the control flow stack. Append the
	run-time semantics given below to the current definition.
	The semantics are incomplete until resolved by a consumer
	of \emph{of-sys} such as \word{ENDOF}.

\item[Run-time:]
	\stack{$x_1$ $x_2$}{$x_1$}

	If the two values on the stack are not equal, discard the
	top value and continue execution at the location specified
	by the consumer of \emph{of-sys}, e.g., following the next
	\word{ENDOF}. Otherwise, discard both values and continue
	execution in line.

\item[See:]
	\wref{core:CASE}{CASE},
	\wref{core:ENDCASE}{ENDCASE},
	\wref{core:ENDOF}{ENDOF}.

	\begin{rationale} % A.6.2.1950 OF
		Typical use:
		\begin{quote}
			\word{:} \texttt{X} {\ldots} \\
			\tab \word{CASE} \\
			\tab~~ \emph{test1} \word{OF} {\ldots} \word{ENDOF} \\
			\tab~~ \emph{testn} \word{OF} {\ldots} \word{ENDOF} \\
			\tab~~ {\ldots} \word{p} default ) \\
			\tab \word{ENDCASE} {\ldots} \\
			\word{;}
		\end{quote}
	\end{rationale}
\end{newword*}


\begin{newword}{2000}{PAD}
	\stack{}{c-addr}

	\emph{c-addr} is the address of a transient region that can
	be used to hold data for intermediate processing.

\item[See:]
	\xref{usage:transient}{3.3.3.6 Other transient regions}.

	\begin{rationale} % A.6.2.2000 PAD
		\word{PAD} has been available as scratch storage for strings
		since the earliest Forth implementations. It was brought to
		our attention that many programmers are reluctant to use
		\word{PAD}, fearing incompatibilities with system uses.
		\word{PAD} is specifically intended as a programmer convenience,
		however, which is why we documented the fact that no standard
		words use it.
	\end{rationale}
\end{newword}


\begin{newword}{2008}{PARSE}
	\stack{char ``ccc\arg{char}''}{c-addr u}

	Parse \emph{ccc} delimited by the delimiter \emph{char}.

	\emph{c-addr} is the address (within the input buffer) and
	$u$ is the length of the parsed string. If the parse area
	was empty, the resulting string has a zero length.

\item[See:]
	\xref{usage:parsing}{3.4.1 Parsing}.

	\begin{rationale} % A.6.2.2008 PARSE
		Typical use: \emph{char} \word{PARSE} \emph{ccc}\arg{char}

		The traditional Forth word for parsing is \word{WORD}.
		\word{PARSE} solves the following problems with \word{WORD}:

		\begin{enumerate}
		\item \word{WORD} always skips leading delimiters. This
			behavior is appropriate for use by the text interpreter,
			which looks for sequences of non-blank characters, but is
			inappropriate for use by words like \word{p} , \word{.p},
			and \word{.q}. Consider the following (flawed) definition
			of \word{.p}:

			\tab \word{:} \word{.p} ~
				\word{[CHAR]} \texttt{)} ~
				\word{WORD} \word{COUNT} \word{TYPE}
			\word{;} ~ \word{IMMEDIATE}

			This works fine when used in a line like:

			\tab \word{.p} \texttt{HELLO)} \quad \texttt{5} \word{.}

			but consider what happens if the user enters an empty
			string:

			\tab \word{.p} \texttt{)} \quad \texttt{5} \word{.}

			The definition of \word{.p} shown above would treat the
			\texttt{)} as a leading delimiter, skip it, and continue
			consuming characters until it located another \texttt{)}
			that followed a non-\texttt{)} character, or until the
			parse area was empty. In the example shown, the
			\texttt{5} \word{.}
			would be treated as part of the string to be printed.

			With \word{PARSE}, we could write a correct definition of
			\word{.p}:

			\tab \word{:} \word{.p} ~
				\word{[CHAR]} \texttt{)} ~
				\word{PARSE} \word{TYPE}
			\word{;} ~ \word{IMMEDIATE}

			This definition avoids the ``empty string'' anomaly.

		\item \word{WORD} returns its result as a counted string.
			This has four bad effects:

			\begin{enumerate}
			\item The characters accepted by \word{WORD} must be
				copied from the input buffer into a temporary buffer,
				in order to make room for the count character that
				must be at the beginning of the counted string. The
				copy step is inefficient, compared to \word{PARSE},
				which leaves the string in the input buffer and doesn't
				need to copy it anywhere.

			\item \word{WORD} must be careful not to store too many
				characters into the temporary buffer, thus overwriting
				something beyond the end of the buffer. This adds to
				the overhead of the copy step. (\word{WORD} may have
				to scan a lot of characters before finding the trailing
				delimiter.)

			\item The count character limits the length of the string
				returned by \word{WORD} to 255 characters (longer
				strings can easily be stored in blocks!). This
				limitation does not exist for \word{PARSE}.

			\item The temporary buffer is typically overwritten by the
				next use of \word{WORD}. This introduces a temporal
				dependency; the value returned by \word{WORD} is only
				valid for a limited duration. \word{PARSE} has a
				temporal dependency, too, related to the lifetime of
				the input buffer, but that is less severe in most
				cases than \word{WORD}'s temporal dependency.
			\end{enumerate}

			The behavior of \word{WORD} with respect to skipping
			leading delimiters is useful for parsing blank-delimited
			names. Many system implementations include an additional
			word for this purpose, similar to \word{PARSE} with respect
			to the ``\emph{c-addr} $u$'' return value, but without an
			explicit delimiter argument (the delimiter set is implicitly
			``white space''), and which does skip leading delimiters. A
			common description for this word is:

			\begin{quote}
				\texttt{PARSE-WORD} \qquad
					( ``\arg{spaces}name'' -{}- \emph{c-addr} $u$ ) \\

				Skip leading spaces and parse \emph{name} delimited by
				a space. \emph{c-addr} is the address within the input
				buffer and $u$ is the length of the selected string.
				If the parse area is empty, the resulting string has a
				zero length.
			\end{quote}

			If both \word{PARSE} and \texttt{PARSE-WORD} are present,
			the need for \word{WORD} is largely eliminated.
		\end{enumerate}
	\end{rationale}
\end{newword}


\begin{newword}{2030}{PICK}
	\stack{$x_u${\ldots}$x_1$ $x_0$ u}{$x_u${\ldots}$x_1$ $x_0$ $x_u$}

	Remove $u$. Copy the $x_u$ to the top of the stack. An ambiguous
	condition exists if there are less than $u+2$ items on the stack
	before \word{PICK} is executed.

	\begin{rationale} % A.6.2.2030 PICK}
		\texttt{0} \word{PICK} is equivalent to \word{DUP} and
		\texttt{1} \word{PICK} is equivalent to \word{OVER}.
	\end{rationale}
\end{newword}


\begin{newword}{2040}{QUERY}
	\stack{}{}

	Make the user input device the input source. Receive input into
	the terminal input buffer, replacing any previous contents. Make
	the result, whose address is returned by \word{TIB}, the input
	buffer. Set \word{toIN} to zero.

\item[Note:]
	This word is obsolescent and is included as a concession to
	existing implementations.

	\begin{rationale} % A.6.2.2040 QUERY
		The function of \word{QUERY} may be performed with \word{ACCEPT}
		and \word{EVALUATE}.
	\end{rationale}
\end{newword}


\begin{newword}{2125}{REFILL}
	\stack{}{flag}

	Attempt to fill the input buffer from the input source,
	returning a true flag if successful.

	When the input source is the user input device, attempt to
	receive input into the terminal input buffer. If successful,
	make the result the input buffer, set \word{toIN} to zero, and
	return \emph{true}. Receipt of a line containing no characters
	is considered successful. If there is no input available from
	the current input source, return \emph{false}.

	When the input source is a string from \word{EVALUATE}, return
	\emph{false} and perform no other action.

\item[See:]
	\wref{block:REFILL}{REFILL},
	\wref{file:REFILL}{REFILL}.

	\begin{rationale} % A.6.2.2125 REFILL
		This word is a useful generalization of \word{QUERY}.
		Re-defining \word{QUERY} to meet this specification would
		have broken existing code. \word{REFILL} is designed to behave
		reasonably for all possible input sources. If the input source
		is coming from the user, as with \word{QUERY}, \word{REFILL}
		could still return a false value if, for instance, a
		communication channel closes so that the system knows that no
		more input will be available.
	\end{rationale}
\end{newword}


\begin{newword}{2148}{RESTORE-INPUT}
	\stack{$x_n$ {\ldots} $x_1$ n}{flag}

	Attempt to restore the input source specification to the state
	described by $x_1$ through $x_n$. \emph{flag} is true if the
	input source specification cannot be so restored.

	An ambiguous condition exists if the input source represented
	by the arguments is not the same as the current input source.

\item[See:]
	\xref{rational:SAVE-INPUT}{A.6.2.2182 SAVE-INPUT}.
\end{newword}


\begin{newword}{2150}{ROLL}
	\stack{$x_u$ $x_{u-1}$ {\ldots} $x_0$ u}{$x_{u-1}$ {\ldots} $x_0$ $x_u$}

	Remove $u$. Rotate $u+1$ items on the top of the stack. An
	ambiguous condition exists if there are less than $u+2$ items
	on the stack before \word{ROLL} is executed.

	\begin{rationale} % A.6.2.2150 ROLL
		\texttt{2} \word{ROLL} is equivalent to \word{ROT},
		\texttt{1} \word{ROLL} is equivalent to \word{SWAP} and
		\texttt{0} \word{ROLL} is a null operation.
	\end{rationale}
\end{newword}


\begin{newword}{2182}{SAVE-INPUT}
	\stack{}{$x_n$ {\ldots} $x_1$ n}

	$x_1$ through $x_n$ describe the current state of the input
	source specification for later use by \word{RESTORE-INPUT}.

	\begin{rationale} % A.6.2.2182 SAVE-INPUT
		\word{SAVE-INPUT} and \word{RESTORE-INPUT} allow the same
		degree of input source repositioning within a text file as is
		available with \word[block]{BLOCK} input. \word{SAVE-INPUT}
		and \word{RESTORE-INPUT} ``hide the details'' of the operations
		necessary to accomplish this repositioning, and are used the
		same way with all input sources. This makes it easier for
		programs to reposition the input source, because they do not
		have to inspect several variables and take different action
		depending on the values of those variables.

		\word{SAVE-INPUT} and \word{RESTORE-INPUT} are intended for
		repositioning within a single input source; for example, the
		following scenario is NOT allowed for a Standard Program:

		\begin{quote}\ttfamily
			\word{:} XX \\
			\tab \word{SAVE-INPUT} ~ \word{CREATE} \\
			\tab \word{Sq} RESTORE-INPUT" \word{EVALUATE} \\
			\tab \word{ABORTq} couldn't restore input" \\
			\word{;}
		\end{quote}

		This is incorrect because, at the time \word{RESTORE-INPUT} is
		executed, the input source is the string via \word{EVALUATE},
		which is not the same input source that was in effect when
		\word{SAVE-INPUT} was executed.

		The following code is allowed:

		\begin{quote}\ttfamily
			\word{:} XX \\
			\tab \word{SAVE-INPUT} ~ \word{CREATE} \\
			\tab \word{Sq} \word{.p} Hello)" \word{EVALUATE} \\
			\tab \word{RESTORE-INPUT} \word{ABORTq} couldn't restore input" \\
			\word{;}
		\end{quote}

		After \word{EVALUATE} returns, the input source specification
		is restored to its previous state, thus \word{SAVE-INPUT} and
		\word{RESTORE-INPUT} are called with the same input source in
		effect.

		In the above examples, the \word{EVALUATE} phrase could have
		been replaced by a phrase involving \word[file]{INCLUDE-FILE}
		and the same rules would apply.

		The Standard does not specify what happens if a program
		violates the above rules. A Standard System might check for
		the violation and return an exception indication from
		\word{RESTORE-INPUT}, or it might fail in an unpredictable
		way.

		The return value from \word{RESTORE-INPUT} is primarily
		intended to report the case where the program attempts to
		restore the position of an input source whose position cannot
		be restored. The keyboard might be such an input source.

		Nesting of \word{SAVE-INPUT} and \word{RESTORE-INPUT} is
		allowed. For example, the following situation works as
		expected:

		\begin{quote}\ttfamily
			\word{:} XX \\
			\tab \word{SAVE-INPUT} \\
			\tab~~ \word{Sq} f1" \word[file]{INCLUDED} \\
			\tab~~ \word{bs} The file "f1" includes: \\
			\tab~~ \word{bs} ~~ {\ldots} SAVE-INPUT {\ldots} RESTORE-INPUT {\ldots} \\
			\tab~~ \word{bs} End of file "f1" \\
			\tab \word{RESTORE-INPUT} ~ \word{ABORTq} couldn't restore input" \\
			\word{;}
		\end{quote}

		In principle, \word{RESTORE-INPUT} could be implemented to
		``always fail'', e.g.:

		\begin{quote}\ttfamily
			\word{:} \word{RESTORE-INPUT} \word{p} x1 {\ldots} xn n -- flag ) \\
			\tab 0 \word{?DO} \word{DROP} \word{LOOP} \word{TRUE} \\
			\word{;}
		\end{quote}

		Such an implementation would not be useful in most cases. It
		would be preferable for a system to leave \word{SAVE-INPUT}
		and \word{RESTORE-INPUT} undefined, rather than to create a
		useless implementation. In the absence of the words, the
		application programmer could choose whether or not to create
		``dummy'' implementations or to work-around the problem in
		some other way.

		Examples of how an implementation might use the return values
		from \word{SAVE-INPUT} to accomplish the save/restore function:

		\begin{center}
		  \begin{tabular}{lllll}
			\hline\hline
			Input Source & \multicolumn{4}{l}{possible stack values} \\
			\hline
			block			& \word{toIN} \word{@} & \word[block]{BLK} \word{@} & \texttt{2} \\
			\word{EVALUATE}	& \word{toIN} \word{@} & \texttt{1} \\
			keyboard		& \word{toIN} \word{@} & \texttt{1} \\
			text file		& \word{toIN} \word{@} & \texttt{lo-pos} & \texttt{hi-pos} & \texttt{3} \\
			\hline\hline
		  \end{tabular}
		\end{center}

		These are examples only; a Standard Program may not assume any
		particular meaning for the individual stack items returned by
		\word{SAVE-INPUT}.
	\end{rationale}
\end{newword}


\begin{newword}{2218}{SOURCE-ID}[source-i-d]
	\stack{}{0 \textbar -1 }

	Identifies the input source as follows:
	\begin{center}
		\begin{tabular}{cl}
		\hline\hline
		\word{SOURCE-ID} & Input source \\
		\hline
		-1	& String (via \word{EVALUATE}) \\
		 0	& User input device \\
		\hline\hline
		\end{tabular}
	\end{center}

\item[See:]
	\wref{file:SOURCE-ID}{SOURCE-ID}.
\end{newword}


\begin{newword}{2240}{SPAN}
	\stack{}{a-addr}

	\emph{a-addr} is the address of a cell containing the count of
	characters stored by the last execution of \word{EXPECT}.

\item[Note:]
	This word is obsolescent and is included as a concession to
	existing implementations.
\end{newword}


\begin{newword}{2290}{TIB}[t-i-b]
	\stack{}{c-addr}

	\emph{c-addr} is the address of the terminal input buffer.

\item[Note:]
	This word is obsolescent and is included as a concession to
	existing implementations.

	\begin{rationale} % A.6.2.2290 TIB
		The function of \word{TIB} has been superseded by
		\word{SOURCE}.
	\end{rationale}
\end{newword}


\begin{newword*}{2295}{TO}
\item[Interpretation:]
	\stack{x ``\arg{spaces}name''}{}

	Skip leading spaces and parse \emph{name} delimited by a space.
	Store $x$ in \emph{name}. An ambiguous condition exists if
	\emph{name} was not defined by \word{VALUE}.

\item[Compilation:]
	\stack{``\arg{spaces}name''}{}

	Skip leading spaces and parse \emph{name} delimited by a space.
	Append the run-time semantics given below to the current
	definition. An ambiguous condition exists if \emph{name} was not
	defined by \word{VALUE}.

\item[Run-time:]
	\stack{x}{}

	Store $x$ in \emph{name}.

\item[Note:]
	An ambiguous condition exists if either \word{POSTPONE} or
	\word{[COMPILE]} is applied to \word{TO}.

\item[See:]
	\wref{core:VALUE}{VALUE},
	\wref{local:TO}{TO}.

	\begin{rationale} % A.6.2.2295 TO
		Historically, some implementations of \word{TO} have not
		explicitly parsed. Instead, they set a mode flag that is
		tested by the subsequent execution of \emph{name}. ANS Forth
		explicitly requires that \word{TO} must parse, so that
		\word{TO}'s effect will be predictable when it is used at the
		end of the parse area.

		Typical use: \texttt{x} \word{TO} \emph{name}
	\end{rationale}
\end{newword*}


\begin{newword}{2298}{TRUE}
	\stack{}{true}

	Return a \emph{true} flag, a single-cell value with all
	bits set.

\item[See:]
	\xref{usage:flags}{3.1.3.1 Flags}.

	\begin{rationale} % A.6.2.2298 TRUE
		\word{TRUE} is equivalent to the phrase
		\texttt{0} \word{0=}. 
	\end{rationale}
\end{newword}


\begin{newword}{2300}{TUCK}
	\stack{$x_1$ $x_2$}{$x_2$ $x_1$ $x_2$}

	Copy the first (top) stack item below the second stack item.
\end{newword}


\begin{newword}{2330}{U.R}[u-dot-r]
	\stack{u n}{}

	Display $u$ right aligned in a field $n$ characters wide. If
	the number of characters required to display $u$ is greater
	than $n$, all digits are displayed with no leading spaces in
	a field as wide as necessary.
\end{newword}


\begin{newword}[Umore]{2350}{U>}[u-greater-than]
	\stack{$u_1$ $u_2$}{flag}

	\emph{flag} is true if and only if $u_1$ is greater than $u_2$.

\item[See:]
	\wref{core:more}{>}.
\end{newword}


\begin{newword}{2395}{UNUSED}
	\stack{}{u}

	$u$ is the amount of space remaining in the region addressed
	by \word{HERE}, in address units.
\end{newword}


\begin{newword}{2405}{VALUE}
	\stack{x ``\arg{spaces}name''}{}

	Skip leading space delimiters. Parse \emph{name} delimited by
	a space. Create a definition for \emph{name} with the execution
	semantics defined below, with an initial value equal to $x$.

	\emph{name} is referred to as a ``value''.

\item[\emph{name} Execution:]
	\stack{}{x}

	Place $x$ on the stack. The value of $x$ is that given when
	\emph{name} was created, until the phrase $x$ \word{TO}
	\emph{name} is executed, causing a new value of $x$ to be
	associated with \emph{name}.

\item[See:]
	\xref{usage:parsing}{3.4.1 Parsing}.

	\begin{rationale} % A.6.2.2405 VALUE
		Typical use:
		\begin{quote}\ttfamily
			0 \word{VALUE} DATA

			\word{:} EXCHANGE \word{p} n1 -- n2 )
				DATA \word{SWAP} \word{TO} DATA
			\word{;}
		\end{quote}
		\texttt{EXCHANGE} leaves $n_1$ in \texttt{DATA} and returns the
		prior value $n_2$.
	\end{rationale}
\end{newword}


\begin{newword}{2440}{WITHIN}
	\stack{$n_1$\textbar$u_1$ $n_2$\textbar$u_2$ $n_3$\textbar$u_3$}{flag}

	Perform a comparison of a test value $n_1$\textbar$u_1$ with a
	lower limit $n_2$\textbar$u_2$ and an upper limit
	$n_3$\textbar$u_3$, returning \emph{true} if either
	($n_2$\textbar$u_2 <  n_3$\textbar$u_3$ and
	($n_2$\textbar$u_2 <= n_1$\textbar$u_1$ and
	 $n_1$\textbar$u_1 <  n_3$\textbar$u_3$)) or
	($n_2$\textbar$u_2 >  n_3$\textbar$u_3$ and
	($n_2$\textbar$u_2 <= n_1$\textbar$u_1$ or
	 $n_1$\textbar$u_1 <  n_3$\textbar$u_3$)) is true, returning
	\emph{false} otherwise. An ambiguous condition exists
	$n_1$\textbar$u_1$, $n_2$\textbar$u_2$, and
	$n_3$\textbar$u_3$ are not all the same type.

	\begin{rationale} % A.6.2.2440 WITHIN
		We describe \word{WITHIN} without mentioning circular number
		spaces (an undefined term) or providing the code. Here is a
		number line with the overflow point ($o$) at the far right and
		the underflow point ($u$) at the far left:
		\begin{center}
			$u$\rule[.5ex]{15em}{.4pt}$o$
		\end{center}
		There are two cases to consider: either the
		$n_2|u_2\ldots n_3|u_3$ range straddles the overflow/underflow
		points or it does not. Lets examine the non-straddle case
		first:
		\begin{center}
			$u$\rule[.5ex]{5em}{.4pt}\makebox[5em]{[\dotfill)}\rule[.5ex]{5em}{.4pt}$o$
		\end{center}
		The [ denotes $n_2|u_2$, the ) denotes $n_3|u_3$, and the dots
		and [ are numbers \word{WITHIN} the range. $n_3|u_3$ is greater
		than $n_2|u_2$, so the following tests will determine if
		$n_1|u_1$ is \word{WITHIN} $n_2|u_2$ and $n_3|u_3$:
		\begin{center}
			$n_2|u_2 \le n_1|u_1$ ~ and ~ $n_1|u_1 < n_3|u_3$.
		\end{center}
		In the case where the comparison range straddles the
		overflow/underflow points:
		\begin{center}
			$u$\makebox[5em]{\dotfill)}\rule[.5ex]{5em}{.4pt}\makebox[5em]{[\dotfill}$o$
		\end{center}
		$n_3|u_3$ is less than $n_2|u_2$ and the following tests will
		determine if $n_1|u_1$ is \word{WITHIN} $n_2|u_2$ and
		$n_3|u_3$:
		\begin{center}
			$n_2|u_2 \le n_1|u_1$ ~ or ~ $n_1|u_1 < n_3|u_3$.
		\end{center}
		\word{WITHIN} must work for both signed and unsigned arguments.
		One obvious implementation does not work:
		\begin{quote}\ttfamily
			\word{:} \word{WITHIN} \word{p} test low high -- flag ) \\
			\tab \word{toR} ~ \word{OVER} \word{less} \word{0=} \word{p} test flag1 )
				~ \word{SWAP} \word{Rfrom} \word{less} \word{p} flag1 flag2 )
				\word{AND} \\
			\word{;}
		\end{quote}
		Assume two's-complement arithmetic on a 16-bit machine, and
		consider the following test:

		\tab \texttt{33000 ~ 32000 ~ 34000 ~ WITHIN}

		The above implementation returns \emph{false} for that test,
		even though the unsigned number 33000 is clearly within the
		range \{\{32000 {\ldots} 34000\}\}.

		The problem is that, in the incorrect implementation, the
		signed comparison \word{less} gives the wrong answer when 32000
		is compared to 33000, because when those numbers are treated
		as signed numbers, 33000 is treated as negative 32536, while
		32000 remains positive.

		Replacing \word{less} with \word{Uless} in the above implementation
		makes it work with unsigned numbers, but causes problems with
		certain signed number ranges; in particular, the test:
		\begin{quote}\ttfamily
			1 ~ -5 ~ 5 ~ WITHIN
		\end{quote}
		would give an incorrect answer.

		For two's-complement machines that ignore arithmetic overflow
		(most machines), the following implementation works in all
		cases:
		\begin{quote}\ttfamily
			\word{:} \word{WITHIN} \word{p} test low high -- flag )	~
				\word{OVER} \word{-} \word{toR} \word{-} \word{Rfrom} \word{Uless} ~
			\word{;}
		\end{quote}
	\end{rationale}
\end{newword}


\begin{newword*}{2530}{[COMPILE]}[bracket-compile]
\item{Interpretation:}
	Interpretation semantics for this word are undefined.

\item[Compilation:]
	\stack{``\arg{spaces}name''}{}

	Skip leading space delimiters. Parse \emph{name} delimited by
	a space. Find \emph{name}. If \emph{name} has other than default
	compilation semantics, append them to the current definition;
	otherwise append the execution semantics of \emph{name}. An
	ambiguous condition exists if \emph{name} is not found.

\item[See:]
	\xref{usage:parsing}{3.4.1 Parsing}.

	\begin{rationale} % A.6.2.2530 [COMPILE]
		Typical use:
			\word{:} \texttt{name2} {\ldots}
				\word{[COMPILE]} \texttt{name1}
			{\ldots} \word{;} ~ \word{IMMEDIATE}
	\end{rationale}
\end{newword*}


\begin{newword*}[bs]{2535}{\bs}[backslash]
\item[Compilation:]
	Perform the execution semantics given below.

\item[Execution:]
	\stack{``ccc\arg{eol}''}{}

	Parse and discard the remainder of the parse area.
	\word{bs} is an immediate word.

\item[See:]
	\wref{block:bs}{\bs}.

	\begin{rationale} % A.6.2.2535 \
		Typical use:
			\texttt{5} \word{CONSTANT} \texttt{THAT}
			~ \word{bs} ~ \texttt{THIS IS A COMMENT ABOUT THAT}
	\end{rationale}
\end{newword*}
