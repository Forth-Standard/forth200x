\chapter{The optional Double-Number word set} % 8
\wordlist{double}

\begin{info}
\section{The optional Double-Number word set}

Forth systems on 8-bit and 16-bit processors often find it necessary
to deal with double-length numbers. But many Forths on small embedded
systems do not, and many users of Forth on systems with a cell size of
32-bits or more find that the necessity for double-length numbers is
much diminished. Therefore, we have factored the words that manipulate
double-length entities into this optional word set.

Please note that the naming convention used in this word set conveys
some important information:

\begin{enumerate}
\item[1.]
	Words whose names are of the form \texttt{2}\emph{xxx} deal
	with cell pairs, where the relationship between the cells is
	unspecified. They may be two-vectors, double-length numbers, or
	any pair of cells that it is convenient to manipulate together.

\item[2.]
	Words with names of the form \texttt{D}\emph{xxx} deal
	specifically with double-length integers.

\item[3.]
	Words with names of the form \texttt{M}\emph{xxx} deal with
	some combination of single and double integers. The order in
	which these appear on the stack is determined by long-standing
	common practice.
\end{enumerate}

Refer to \ref{rat:types} for a discussion of data types in Forth.
\end{info}

\section{Introduction} % 8.1

Sixteen-bit Forth systems often use double-length numbers. However, many
Forths on small embedded systems do not, and many users of Forth on systems
with a cell size of 32 bits or more find that the use of double-length
numbers is much diminished. Therefore, the words that manipulate
double-length entities have been placed in this optional word set.

\section{Additional terms and notation} % 8.2

None.

\section{Additional usage requirements} % 8.3

\subsection{Environmental queries} % 8.3.1

Append table \ref{double:env} to table \ref{table:env}.

See: \xref{usage:env}{3.2.6 Environmental queries}.

\begin{table}[ht]
  \begin{center}
	\caption{Environmental Query Strings}
	\label{double:env}
	\begin{tabular}{p{9em}rcp{0.42\textwidth}}
		\hline\hline
		\multicolumn{2}{l}{String \hfill Value data type} & Constant? & Meaning \\
		\hline
		\texttt{DOUBLE}		& \emph{flag}	& no	&
			double-number word set present \\
		\texttt{DOUBLE-EXT}	& \emph{flag}	& no	&
			double-number extensions word set present \\
		\hline\hline
	\end{tabular}
  \end{center}
\end{table}

\subsection{Text interpreter input number conversion} % 8.3.2

When the text interpreter processes a number that is immediately
followed by a decimal point and is not found as a definition name,
the text interpreter shall convert it to a double-cell number.

For example, entering \word[core]{DECIMAL} \texttt{1234} leaves the
single-cell number \texttt{1234} on the stack, and entering
\word[core]{DECIMAL} \texttt{1234.} leaves the double-cell number
\texttt{1234} \texttt{0} on the stack.

See: \xref{usage:numbers}{3.4.1.3 Text interpreter input number
	conversion}.


\section{Additional documentation requirements} % 8.4

\subsection{System documentation} % 8.4.1

\subsubsection{Implementation-defined options} % 8.4.1.1

\begin{itemize}
\item no additional requirements.
\end{itemize}

\subsubsection{Ambiguous conditions} % 8.4.1.2

\begin{itemize}
\item $d$ outside range of $n$ in \wref{double:DtoS}{D>S}.
\end{itemize}

\subsubsection{Other system documentation} % 8.4.1.3

\begin{itemize}
\item no additional requirements.
\end{itemize}

\subsection{Program documentation} % 8.4.2

\begin{itemize}
\item no additional requirements.
\end{itemize}


\section{Compliance and labeling} % 8.5

\subsection{ANS Forth systems} % 8.5.1

The phrase ``Providing the Double-Number word set'' shall be
appended to the label of any Standard System that provides all
of the Double-Number word set.

The phrase ``Providing \emph{name(s)} from the Double-Number
Extensions word set'' shall be appended to the label of any
Standard System that provides portions of the Double-Number
Extensions word set.

The phrase ``Providing the Double-Number Extensions word set''
shall be appended to the label of any Standard System that
provides all of the Double-Number and Double-Number Extensions
word sets.

\subsection{ANS Forth programs} % 8.5.2

The phrase ``Requiring the Double-Number word set'' shall be
appended to the label of Standard Programs that require the
system to provide the Double-Number word set.

The phrase ``Requiring \emph{name(s)} from the Double-Number
Extensions word set'' shall be appended to the label of Standard
Programs that require the system to provide portions of the
Double-Number Extensions word set.

The phrase ``Requiring the Double-Number Extensions word set''
shall be appended to the label of Standard Programs that require
the system to provide all of the Double-Number and Double-Number
Extensions word sets.

\section{Glossary} % 8.6

\begin{info}
\subsection{Glossary}
\end{info}

\subsection{Double-Number words} % 8.6.1

\begin{newword}{0360}{2CONSTANT}[two-constant]
	\stack{$x_1$ $x_2$ ``\arg{spaces}name''}{}

	Skip leading space delimiters. Parse \emph{name} delimited by
	a space. Create a definition for \emph{name} with the execution
	semantics defined below.

	\emph{name} is referred to as a ``two-constant''.

\item[\emph{name} Execution:]
	\stack{}{$x_1$ $x_2$}

	Place cell pair $x_1$ $x_2$ on the stack.

\item[See:]
	\xref{usage:parsing}{3.4.1 Parsing}.

	\begin{rationale} % A.8.6.1.0360 2CONSTANT
		Typical use:
			\texttt{x1} \texttt{x2} \word{2CONSTANT} \emph{name}
	\end{rationale}
\end{newword}


\begin{newword*}{0390}{2LITERAL}[two-literal]
\item[Interpretation:]
	Interpretation semantics for this word are undefined.

\item[Compilation:]
	\stack{$x_1$ $x_2$}{}

	Append the run-time semantics below to the current definition.

\item[Run-time:]
	\stack{}{$x_1$ $x_2$}

	Place cell pair $x_1$ $x_2$ on the stack.

	\begin{rationale} % A.8.6.1.0390 2LITERAL
		\setwordlist{core}
		Typical use:
			\word{:} \texttt{X} {\ldots}
				\word{[} \texttt{x1} \texttt{x2} \word{]} \word[double]{2LITERAL}
			{\ldots} \word{;}
		\setwordlist{double}
	\end{rationale}
\end{newword*}


\begin{newword}{0440}{2VARIABLE}[two-variable]
	\stack{``\arg{spaces}name''}{}

	Skip leading space delimiters. Parse \emph{name} delimited by a
	space. Create a definition for \emph{name} with the execution
	semantics defined below. Reserve two consecutive cells of data
	space.

	\emph{name} is referred to as a ``two-variable''.

\item[\emph{name} Execution:]
	\stack{}{a-addr}

	\emph{a-addr} is the address of the first (lowest address)
	cell of two consecutive cells in data space reserved by
	\word{2VARIABLE} when it defined \emph{name}. A program is
	responsible for initializing the contents.

\item[See:]
	\xref{usage:parsing}{3.4.1 Parsing},
	\wref{core:VARIABLE}{VARIABLE}.

	\begin{rationale} % A.8.6.1.0440 2VARIABLE
		Typical use:
			\word{2VARIABLE} \emph{name}
	\end{rationale}
\end{newword}


\begin{newword}{1040}{D+}[d-plus]
	\stack{$d_1$\textbar ud$_1$ $d_2$\textbar ud$_2$}{$d_3$\textbar ud$_3$}

	Add $d_2$\textbar\emph{ud}$_2$ to $d_1$\textbar\emph{ud}$_1$,
	giving the sum $d_3$\textbar\emph{ud}$_3$.
\end{newword}


\begin{newword}{1050}{D-}[d-minus]
	\stack{$d_1$\textbar ud$_1$ $d_2$\textbar ud$_2$}{$d_3$\textbar ud$_3$}

	Subtract $d_2$\textbar\emph{ud}$_2$ from $d_1$\textbar\emph{ud}$_1$,
	giving the difference $d_3$\textbar\emph{ud}$_3$.
\end{newword}


\begin{newword}{1060}{D.}[d-dot]
	\stack{d}{}

	Display $d$ in free field format.
\end{newword}


\begin{newword}{1070}{D.R}[d-dot-r]
	\stack{d n}{}

	Display $d$ right aligned in a field $n$ characters wide.
	If the number of characters required to display $d$ is greater
	than $n$, all digits are displayed with no leading spaces in
	a field as wide as necessary.

	\begin{rationale} % A.8.6.1.1070 D.R
		In \word{D.R}, the ``R'' is short for RIGHT.
	\end{rationale}
\end{newword}


\begin{newword}[D0less]{1075}{D0<}[d-zero-less]
	\stack{d}{flag}

	\emph{flag} is true if and only if $d$ is less than zero.
\end{newword}


\begin{newword}{1080}{D0=}[d-zero-equals]
	\stack{xd}{flag}

	\emph{flag} is true if and only if \emph{xd} is equal to zero.
\end{newword}


\begin{newword}{1090}{D2*}[d-two-star]
	\stack{xd$_1$}{xd$_2$}

	\emph{xd}$_2$ is the result of shifting \emph{xd}$_1$ one bit
	toward the most-significant bit, filling the vacated
	least-significant bit with zero.

	\begin{rationale} % A.8.6.1.1090 D2*
		See: \rref{core:2*}{2*} for applicable discussion.
	\end{rationale}
\end{newword}


\begin{newword}{1100}{D2/}[d-two-slash]
	\stack{xd$_1$}{xd$_2$}

	\emph{xd}$_2$ is the result of shifting \emph{xd}$_1$ one bit
	toward the least-significant bit, leaving the most-significant
	bit unchanged.

	\begin{rationale} % A.8.6.1.1100 D2/
		See: \rref{core:2/}{2/} for applicable discussion.
	\end{rationale}
\end{newword}


\begin{newword}[Dless]{1110}{D<}[d-less-than]
	\stack{$d_1$ $d_2$}{flag}

	\emph{flag} is true if and only if $d_1$ is less than $d_2$.
\end{newword}


\begin{newword}{1120}{D=}[d-equals]
	\stack{xd$_1$ xd$_2$}{flag}

	\emph{flag} is true if and only if \emph{xd}$_1$ is bit-for-bit
	the same as \emph{xd}$_2$.
\end{newword}


\begin{newword}[DtoS]{1140}{D>S}[d-to-s]
	\stack{d}{n}

	$n$ is the equivalent of $d$. An ambiguous condition exists if
	$d$ lies outside the range of a signed single-cell number.

	\begin{rationale} % A.8.6.1.1140 D>S
		There exist number representations, e.g., the sign-magnitude
		representation, where reduction from double- to single-precision
		cannot simply be done with \word[core]{DROP}. This word,
		equivalent to \word[core]{DROP} on two's complement systems,
		desensitizes application code to number representation and
		facilitates portability.
	\end{rationale}
\end{newword}


\begin{newword}{1160}{DABS}[d-abs]
	\stack{d}{ud}

	\emph{ud} is the absolute value of $d$.
\end{newword}


\begin{newword}{1210}{DMAX}[d-max]
	\stack{$d_1$ $d_2$}{$d_3$}

	$d_3$ is the greater of $d_1$ and $d_2$.
\end{newword}


\begin{newword}{1220}{DMIN}[d-min]
	\stack{$d_1$ $d_2$}{$d_3$}

	$d_3$ is the lesser of $d_1$ and $d_2$.
\end{newword}


\begin{newword}{1230}{DNEGATE}[d-negate]
	\stack{$d_1$}{$d_2$}

	$d_2$ is the negation of $d_1$.
\end{newword}


\begin{newword}{1820}{M*/}[m-star-slash]
	\stack{$d_1$ $n_1$ $+n_2$}{$d_2$}

	Multiply $d_1$ by $n_1$ producing the triple-cell intermediate
	result $t$. Divide $t$ by $+n_2$ giving the double-cell quotient
	$d_2$. An ambiguous condition exists if $+n_2$ is zero or
	negative, or the quotient lies outside of the range of a
	double-precision signed integer.

	\begin{rationale} % A.8.6.1.1820 M*/
		\word{M*/} was once described by Chuck Moore as the most
		useful arithmetic operator in Forth. It is the main workhorse
		in most computations involving double-cell numbers. Note that
		some systems allow signed divisors. This can cost a lot in
		performance on some CPUs. The requirement for a positive
		divisor has not proven to be a problem.
	\end{rationale}
\end{newword}


\begin{newword}{1830}{M+}[m-plus]
	\stack{$d_1$\textbar ud$_1$ n}{$d_2$\textbar ud$_2$}

	Add $n$ to $d_1$\textbar\emph{ud}$_1$, giving the sum
	$d_2$\textbar\emph{ud}$_2$.

	\begin{rationale} % A.8.6.1.1830 M+
		\word{M+} is the classical method for integrating.
	\end{rationale}
\end{newword}


\subsection{Double-Number extension words} % 8.6.2
\extended

\begin{newword}{0420}{2ROT}[two-rote]
	\stack{$x_1$ $x_2$ $x_3$ $x_4$ $x_5$ $x_6$}{$x_3$ $x_4$ $x_5$ $x_6$ $x_1$ $x_2$}

	Rotate the top three cell pairs on the stack bringing cell pair
	$x_1$ $x_2$ to the top of the stack.
\end{newword}


\begin{newword}[DUless]{1270}{DU<}[d-u-less]
	\stack{ud$_1$ ud$_2$}{flag}

	\emph{flag} is true if and only if \emph{ud}$_1$ is less than
	\emph{ud}$_2$.
\end{newword}
