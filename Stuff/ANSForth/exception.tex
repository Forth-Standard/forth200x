\chapter{The optional Exception word set} % 9
\wordlist{exception}

\begin{info}
\section{The optional Exception word set}

\word{CATCH} and \word{THROW} provide a reliable mechanism for
handling exceptions, without having to propagate exception flags
through multiple levels of word nesting. It is similar in spirit
to the ``non-local return'' mechanisms of many other languages,
such as C's \texttt{setjmp()} and \texttt{longjmp()}, and LISP's
\texttt{CATCH} and \texttt{THROW}. In the Forth context, \word{THROW}
may be described as a ``multi-level \word[core]{EXIT}'', with
\word{CATCH} marking a location to which a \word{THROW} may return.

Several similar Forth ``multi-level \word[core]{EXIT}''
exception-handling schemes have been described and used in past years.
It is not possible to implement such a scheme using only standard words
(other than \word{CATCH} and \word{THROW}), because there is no portable
way to ``unwind'' the return stack to a predetermined place.

\word{THROW} also provides a convenient implementation technique for
the standard words \word{ABORT} and \word{ABORTq}, allowing an
application to define, through the use of \word{CATCH}, the behavior
in the event of a system \word{ABORT}.

This sample implementation of \word{CATCH} and \word{THROW} uses the
non-standard words described below. They or their equivalents are
available in many systems. Other implementation strategies, including
directly saving the value of \word[core]{DEPTH}, are possible if such
words are not available.

\begin{quote}
\texttt{SP@} \stack{}{addr}
	returns the address corresponding to the top of data stack.

\texttt{SP!} \stack{addr}{}
	sets the stack pointer to \emph{addr}, thus restoring the stack
	depth to the same depth that existed just before \emph{addr} was
	acquired by executing \texttt{SP@}.

\texttt{RP@} \stack{}{addr}
	returns the address corresponding to the top of return stack.

\texttt{RP!} \stack{addr}{}
	sets the return stack pointer to \emph{addr}, thus restoring the
	return stack depth to the same depth that existed just before
	\emph{addr} was acquired by executing \texttt{RP@}.

	\setwordlist{core}\ttfamily
	\begin{tabbing}
	\tab \= \tab \= \hspace{7em} \= ( saved-sp ) \= \kill
	\word{VARIABLE} HANDLER ~ 0 HANDLER \word{!} ~ \word{bs} last exception handler \\[\parskip]

	\+ \word{:} \word[exception]{CATCH} ~ \word{p} xt -{}- exception\# | 0 ) \word{bs} return addr on stack \\
		SP@ \word{toR}				\>\> \word{p} xt )	\> \word{bs} save data stack pointer \\
		HANDLER \word{@} \word{toR}	\>\> \word{p} xt )	\> \word{bs} and previous handler \\
		RP@ HANDLER \word{!}		\>\> \word{p} xt )	\> \word{bs} set current handler \\
		\word{EXECUTE}				\>\> \word{p} )		\> \word{bs} execute returns if no THROW \\
		\word{Rfrom} HANDLER \word{!}	\>\> \word{p} )		\> \word{bs} restore previous handler \\
		\word{Rfrom} \word{DROP}		\>\> \word{p} )		\> \word{bs} discard saved stack ptr \\
	\-	0							\>\> \word{p} 0 )	\> \word{bs} normal completion \\
	\word{;} \\[\parskip]

	\+ \word{:} \word[exception]{THROW} \word{p} ??? exception\# -{}- ??? exception\# ) \\
		\+ \word{?DUP} \word{IF}			\>\> \word{p} exc\# )	\> \word{bs} 0 THROW is no-op \\
			HANDLER \word{@} RP!			\> \word{p} exc\# ) \> \word{bs} restore prev return stack \\
			\word{Rfrom} HANDLER \word{!}		\> \word{p} exc\# ) \> \word{bs} restore prev handler \\
			\word{Rfrom} \word{SWAP} \word{toR} \> \word{p} saved-sp ) \> \word{bs} exc\# on return stack \\
			SP! \word{DROP} \word{Rfrom}		\> \word{p} exc\# ) 	\> \word{bs} restore stack \\
			\word{bs} Return to the caller of CATCH because return \\
			\word{bs} stack is restored to the state that existed \\
		\-	\word{bs} when CATCH began execution \\
	\- \word{THEN} \\
	\word{;}
	\end{tabbing}
\end{quote}

In a multi-tasking system, the \texttt{HANDLER} variable should be in
the per-task variable area (i.e., a user variable).

This sample implementation does not explicitly handle the case in
which \word[exception]{CATCH} has never been called (i.e., the
\word[exception]{ABORT} behavior). One solution is to add the following
code after the \word{IF} in \word[exception]{THROW}:
\begin{quote}\ttfamily
	HANDLER \word{@} \word{0=} \word{IF}
		\word{p} empty the stack ) \word{QUIT}
	\word{THEN}
\end{quote}
Another solution is to execute \word[exception]{CATCH} within
\word{QUIT}, so that there is always an ``exception handler of last
resort'' present. For example:
\begin{quote}\ttfamily
	\word{:} \word{QUIT} \\
	\tab \word{p} empty the return stack and ) \\
	\tab \word{p} set the input source to the user input device )\\
	\tab \word{POSTPONE} \word{[} \\
	\tab \word{BEGIN} \\
	\tab~~ \word{REFILL} \\
	\tab \word{WHILE} \\
	\tab~~ \word{[']} INTERPRET ~ \word[exception]{CATCH}\\
	\tab~~ \word{CASE} \\
	\tab\tab ~0 \word{OF} \word{STATE} \word{@} \word{0=} \word{IF}
		\word{.q} OK" \word{THEN} \word{CR} ~\word{ENDOF} \\
	\tab\tab -1 \word{OF} \word{p} Aborted) \word{ENDOF} \\
	\tab\tab -2 \word{OF} \word{p} display message from \word[exception]{ABORTq} ) \word{ENDOF} \\
	\tab\tab \word{p} default ) \word{DUP} \word{.q} Exception \# " ~ \word{.} \\
	\tab~~ \word{ENDCASE} \\
	\tab \word{REPEAT} \word[tools]{BYE} \\
	\word{;}
\end{quote}

This example assumes the existance of a system-implementation word
\texttt{INTERPRET} that embodies the text interpreter semantics
described in \xref{usage:command}{3.4 The Forth text interpreter}.
Note that this implementation of \word{QUIT} automatically handles
the emptying of the stack and return stack, due to
\word[exception]{THROW}'s inherent restoration of the data and return
stacks. Given this definition of \word{QUIT}, it's easy to define:

\tab \word{:} \word[exception]{ABORT}
	\texttt{-1} \word[exception]{THROW}
\word{;}

In systems with other stacks in addition to the data and return stacks,
the implementation of \word[exception]{CATCH} and \word[exception]{THROW}
must save and restore those stack pointers as well. Such an ``extended
version'' can be built on top of this basic implementation. For example,
with another stack pointer accessed with \texttt{FP@} and \texttt{FP!}
only \word[exception]{CATCH} needs to be redefined:

\begin{quote}\ttfamily
	\word{:} \word[exception]{CATCH}
		\word{p} \texttt{xt -{}- exception\# | 0 )} \\
	\tab FP@ \word{toR} \word[exception]{CATCH} \word{Rfrom}
		\word{OVER} \word{IF} FP! \word{ELSE} \word{DROP} \word{THEN}
	\word{;}
\end{quote}

\setwordlist{exception}
No change to \word{THROW} is necessary in this case. Note that, as
with all redefinitions, the redefined version of \word{CATCH} will
only be available to definitions compiled after the redefinition of
\word{CATCH}.

\word{CATCH} and \word{THROW} provide a convenient way for an
implementation to ``clean up'' the state of open files if an
exception occurs during the text interpretation of a file with
\word[file]{INCLUDE-FILE}. The implementation of
\word[file]{INCLUDE-FILE} may guard (with \word{CATCH}) the word
that performs the text interpretation, and if \word{CATCH} returns
an exception code, the file may be closed and the exception
re\word{THROW}n so that the files being included at an outer nesting
level may be closed also. Note that the Standard allows, but does not
require, \word[file]{INCLUDE-FILE} to close its open files if an
exception occurs. However, it does require \word[file]{INCLUDE-FILE}
to unnest the input source specification if an exception is
\word{THROW}n.
\end{info}

\setwordlist{exception}
\section{Introduction} % 9.1

\section{Additional terms and notation} % 9.2
None.

\section{Additional usage requirements} % 9.3

\begin{info}
\subsection{Additional usage requirements}

One important use of an exception handler is to maintain program
control under many conditions which \word{ABORT}. This is practicable
only if a range of codes is reserved. Note that an application may
overload many standard words in such a way as to \word{THROW}
ambiguous conditions not normally \word{THROW}n by a particular
system.
\end{info}

\subsection{THROW values} % 9.3.1
\label{exception:throw}

The \word{THROW} values \{-255{\ldots}-1\} shall be used only as
assigned by this Standard. The values \{-4095{\ldots}-256\}
shall be used only as assigned by a system.

If the File-Access or Memory-Allocation word sets are implemented,
it is recommended that the non-zero values of \emph{ior} lie within
the range of system \word{THROW} values, as defined above. In an
operating-system environment, this can sometimes be accomplished
by ``biasing'' the range of operating-system exception codes to fall
within the \word{THROW} range.

Programs shall not define values for use with \word{THROW} in the
range \{-4095{\ldots}-1\}.

\subsection{Exception frame} % 9.3.2

An exception frame is the implementation-dependent set of
information recording the current execution state necessary for
the proper functioning of \word{CATCH} and \word{THROW}. It often
includes the depths of the data stack and return stack.

\subsection{Exception stack} % 9.3.3

A stack used for the nesting of exception frames by \word{CATCH}
and \word{THROW}. It may be, but need not be, implemented using
the return stack.

\subsection{Environmental queries} % 9.3.4

Append table \ref{exception:env} to table \ref{table:env}.

See: \xref{usage:env}{3.2.6 Environmental queries}.

\begin{table}[ht]
  \begin{center}
	\caption{Environmental Query Strings}
	\label{exception:env}
	\begin{tabular}{p{9em}rcp{0.42\textwidth}}
		\hline\hline
		\multicolumn{2}{l}{String \hfill Value data type} & Constant? & Meaning \\
		\hline
		\texttt{EXCEPTION}		& \emph{flag}	& no	&
			Exception word set present \\
		\texttt{EXCEPTION-EXT}	& \emph{flag}	& no	&
			Exception extensions word set present \\
		\hline\hline
	\end{tabular}
  \end{center}
\end{table}

\subsection{Possible actions on an ambiguous condition} % 9.3.5
\label{exception:ambiguous}

A system choosing to execute \word{THROW} when detecting one of the
ambiguous conditions listed in table \ref{table:throw} shall use the
throw code listed there.

See: \xref{usage:ambiguous}{3.4.4 Possible actions on an ambiguous condition}.

\begin{table}[ht]
  \begin{center}
	\caption{\word{THROW} code assignments}
	\label{table:throw}
	\small
	\begin{tabular}{r@{}r@{}r@{}l@{~~}rrr@{}l}
	\hline\hline
%	\multicolumn{2}{c}{Code} & Reserved for & %a & b & c \\
%	\multicolumn{2}{c}{Code} & \\
	\multicolumn{3}{c}{Code} & Reserved for &\multicolumn{3}{c}{Code} & Reserved for \\
	\hline
	& -1&& \word{ABORT}						&&-30&& obsolescent feature \\
	& -2&& \word{ABORTq}					&&-31&& \word[core]{toBODY} used on non-\word[core]{CREATE}d definition \\
	& -3&& stack overflow					&&-32&& invalid name argument (e.g., \word[core]{TO} xxx) \\
	& -4&& stack underflow					&&-33&& block read exception \\
	& -5&& return stack overflow			&&-34&& block write exception \\
	& -6&& return stack underflow			&&-35&& invalid block number \\
	& -7&& do-loops nested too deeply during execution
	 										&&-36&& invalid file position \\
	& -8&& dictionary overflow				&&-37&& file I/O exception \\
	& -9&& invalid memory address			&&-38&& non-existent file \\
	&-10&& division by zero					&&-39&& unexpected end of file \\
	&-11&& result out of range				&&-40&& invalid \word[core]{BASE} for floating point conversion \\
	&-12&& argument type mismatch			&&-41&& loss of precision \\
	&-13&& undefined word					&&-42&& floating-point divide by zero \\
	&-14&& interpreting a compile-only word	&&-43&& floating-point result out of range \\
	&-15&& invalid \word[tools]{FORGET}		&&-44&& floating-point stack overflow \\
	&-16&& attempt to use zero-length string as a name
											&&-45&& floating-point stack underflow \\
	&-17&& pictured numeric output string overflow
											&&-46&& floating-point invalid argument \\
	&-18&& parsed string overflow			&&-47&& compilation word list deleted \\
	&-19&& definition name too long			&&-48&& invalid \word[core]{POSTPONE} \\
	&-20&& write to a read-only location	&&-49&& search-order overflow \\
	&-21&& unsupported operation			&&-50&& search-order underflow \\
	&	&& (e.g., \word[facility]{AT-XY} on a too-dumb terminal)
											&&-51&& compilation word list changed \\
	&-22&& control structure mismatch		&&-52&& control-flow stack overflow \\
	&-23&& address alignment exception		&&-53&& exception stack overflow \\
	&-24&& invalid numeric argument			&&-54&& floating-point underflow \\
	&-25&& return stack imbalance			&&-55&& floating-point unidentified fault \\
	&-26&& loop parameters unavailable		&&-56&& \word[core]{QUIT} \\
	&-27&& invalid recursion				&&-57&& exception in sending or receiving a character \\
	&-28&& user interrupt					&&-58&& \word[tools]{[IF]}, \word[tools]{[ELSE]}, or \word[tools]{[THEN]} exception \\
	&-29&& compiler nesting					\\
	\hline\hline
	\end{tabular}
  \end{center}
\end{table}

\subsection{Exception handling} % 9.3.6

There are several methods of coupling \word{CATCH} and \word{THROW}
to other procedural nestings. The usual nestings are the execution
of definitions, use of the return stack, use of loops,
instantiation of locals and nesting of input sources (i.e., with
\word[block]{LOAD}, \word[core]{EVALUATE}, or
\word[file]{INCLUDE-FILE}).

When a \word{THROW} returns control to a \word{CATCH}, the system
shall un-nest not only definitions, but also, if present, locals
and input source specifications, to return the system to its proper
state for continued execution past the \word{CATCH}.

\begin{info}
\subsubsection{Exception handling}

The method of accomplishing this coupling is implementation dependent.
For example, \word[block]{LOAD} could ``know'' about \word{CATCH} and
\word{THROW} (by using \word{CATCH} itself, for example), or
\word{CATCH} and \word{THROW} could ``know'' about \word[block]{LOAD}
(by maintaining input source nesting information in a data structure
known to \word{THROW}, for example). Under these circumstances it is
not possible for a Standard Program to define words such as
\word[block]{LOAD} in a completely portable way.
\end{info}

\section{Additional documentation requirements} % 9.4

\subsection{System documentation} % 9.4.1

\subsubsection{Implementation-defined options} % 9.4.1.1
\begin{itemize}
\item Values used in the system by \wref{exception:CATCH}{CATCH} and
	\wref{exception:THROW}{THROW}
	(\xref{exception:throw}{9.3.1 THROW values},
	 \xref{exception:ambiguous}{9.3.5 Possible actions on an ambiguous
	 condition}).
\end{itemize}

\subsubsection{Ambiguous conditions} % 9.4.1.2
\begin{itemize}
\item no additional requirements.
\end{itemize}

\subsubsection{Other system documentation} % 9.4.1.3
\begin{itemize}
\item no additional requirements.
\end{itemize}

\subsection{Program documentation} % 9.4.2
\begin{itemize}
\item no additional requirements.
\end{itemize}

\section{Compliance and labeling} % 9.5

\subsection{ANS Forth systems} % 9.5.1

The phrase ``Providing the Exception word set'' shall be appended to
the label of any Standard System that provides all of the Exception
word set.

The phrase ``Providing \emph{name(s)} from the Exception Extensions
word set'' shall be appended to the label of any Standard System
that provides portions of the Exception Extensions word set.

The phrase ``Providing the Exception Extensions word set'' shall be
appended to the label of any Standard System that provides all of
the Exception and Exception Extensions word sets.

\subsection{ANS Forth programs} % 9.5.2

The phrase ``Requiring the Exception word set'' shall be appended
to the label of Standard Programs that require the system to provide
the Exception word set.

The phrase ``Requiring \emph{name(s)} from the Exception Extensions
word set'' shall be appended to the label of Standard Programs that
require the system to provide portions of the Exception Extensions
word set.

The phrase ``Requiring the Exception Extensions word set'' shall be
appended to the label of Standard Programs that require the system
to provide all of the Exception and Exception Extensions word sets.

\section{Glossary} % 9.6

\begin{info}
\subsection{Glossary}
\end{info}

\subsection{Exception words} % 9.6.1

\begin{newword}{0875}{CATCH}
	\stack{i*x xt}{j*x 0 ~\textbar~ i*x n}

	Push an exception frame on the exception stack and then execute
	the execution token \emph{xt} (as with \word[core]{EXECUTE}) in
	such a way that control can be transferred to a point just after
	\word{CATCH} if \word{THROW} is executed during the execution of
	\emph{xt}.

	If the execution of \emph{xt} completes normally (i.e., the
	exception frame pushed by this \word{CATCH} is not popped by an
	execution of \word{THROW}) pop the exception frame and return
	zero on top of the data stack, above whatever stack items would
	have been returned by \emph{xt} \word[core]{EXECUTE}. Otherwise,
	the remainder of the execution semantics are given by
	\word{THROW}.
\end{newword}


\begin{newword}{2275}{THROW}
	\stack{k*x n}{k*x ~\textbar~ i*x n}

	If any bits of $n$ are non-zero, pop the topmost exception
	frame from the exception stack, along with everything on the
	return stack above that frame. Then restore the input source
	specification in use before the corresponding \word{CATCH}
	and adjust the depths of all stacks defined by this Standard
	so that they are the same as the depths saved in the exception
	frame ($i$ is the same number as the $i$ in the input arguments
	to the corresponding \word{CATCH}), put $n$ on top of the data
	stack, and transfer control to a point just after the
	\word{CATCH} that pushed that exception frame.

	If the top of the stack is non zero and there is no exception
	frame on the exception stack, the behavior is as follows:

	If $n$ is minus-one (-1), perform the function of
	\wref{core:ABORT}{ABORT} (the version of \word[core]{ABORT} in
	the Core word set), displaying no message.

	If $n$ is minus-two, perform the function of
	\wref{core:ABORTq}{ABORT"} (the version of \word[core]{ABORTq}
	in the Core word set), displaying the characters \emph{ccc}
	associated with the \word{ABORTq} that generated the
	\word{THROW}.

	Otherwise, the system may display an implementation-dependent
	message giving information about the condition associated with
	the \word{THROW} code $n$. Subsequently, the system shall
	perform the function of \wref{core:ABORT}{ABORT} (the version
	of \word[core]{ABORT} in the Core word set).

\begin{rationale} % A.9.6.1.2275 THROW
	If \word{THROW} is executed with a non zero argument, the effect
	is as if the corresponding \word{CATCH} had returned it. In that
	case, the stack depth is the same as it was just before \word{CATCH}
	began execution. The values of the \emph{i*x} stack arguments could
	have been modified arbitrarily during the execution of \emph{xt}.
	In general, nothing useful may be done with those stack items, but
	since their number is known (because the stack depth is deterministic),
	the application may \word[core]{DROP} them to return to a predictable
	stack state.

	Typical use:
	\setwordlist{core}
	\begin{quote}\ttfamily
		\word{:} could-fail \word{p} -- char ) \\
		\tab \word{KEY} \word{DUP} \word{[CHAR]} Q \word{=}~
			\word{IF}~ 1 \word[exception]{THROW}
			\word{THEN}
		\word{;}

		\word{:} do-it \word{p} a b -- c) ~
			\word{2DROP} could-fail \word{;}

		\word{:} try-it \word{p} --) \\
		\tab 1 2 \word{[']} do-it~ \word[exception]{CATCH}~ \word{IF} \\
		\tab~~ \word{p} x1 x2 ) \word{2DROP}
			\word{.q}  There was an exception" \word{CR} \\
		\tab \word{ELSE}
			\word{.q} The character was " \word{EMIT} \word{CR} \\
		\tab \word{THEN} \\
		\word{;}

		\word{;} retry-it \word{p} -- ) \\
		\tab \word{BEGIN}  1 2 \word{[']} do-it \word[exception]{CATCH}~
			\word{WHILE} \\
		\tab~~ \word{p} x1 x2) \word{2DROP}
			\word{.q} Exception, keep trying" \word{CR} \\
		\tab \word{REPEAT} \word{p} char ) \\
		\tab \word{.q} The character was " \word{EMIT} \word{CR} \\
		\word{;}
	\end{quote}
\end{rationale}
\end{newword}


\subsection{Exception extension words} % 9.6.2
\extended

\begin{newword}{0670}{ABORT}
	Extend the semantics of \wref{core:ABORT}{ABORT} to be:

	\stack{i*x}{}
	\stack[R]{j*x}{}

	Perform the function of \texttt{-1} \word{THROW}.

\item[See:]
	\wref{core:ABORT}{ABORT}.
\end{newword}


\begin{newword}[ABORTq]{0680}{ABORT"}[abort-quote]
	Extend the semantics of \wref{core:ABORTq}{ABORT"} to be:

\item[Interpretation:]
	Interpretation semantics for this word are undefined.

\item[Compilation:]
	\stack{``ccc\arg{quote}''}{}

	Parse \emph{ccc} delimited by a \texttt{"} (double-quote).
	Append the run-time semantics given below to the current
	definition.

\item[Run-time:]
	\stack{i*x $x_1$}{~\textbar~ i*x}
	\stack[R]{j*x}{~\textbar~j*x}

	Remove $x_1$ from the stack. If any bit of $x_1$ is not zero,
	perform the function of \texttt{-2} \word{THROW}, displaying
	\emph{ccc} if there is no exception frame on the exception
	stack.

\item[See:]
	\xref{usage:parsing}{3.4.1 Parsing},
	\wref{core:ABORTq}{ABORT"}.
\end{newword}
