\chapter{The optional Facility word set} % 10
\wordlist{facility}

\begin{info}
\section{The optional Facility word set}
\end{info}

\section{Introduction} % 10.1

\section{Additional terms and notation} % 10.2
None.

\section{Additional usage requirements} % 10.3

\subsection{Character types} % 10.3.1
Programs that use more than seven bits of a character by
\word{EKEY} have an environmental dependency.

See: \xref{usage:char}{3.1.2 Character types}.

\subsection{Environmental queries} % 10.3.2
Append table \ref{facility:env} to table \ref{table:env}.

See: \xref{usage:env}{3.2.6 Environmental queries}.

\begin{table}[h]
  \begin{center}
	\caption{Environmental Query Strings}
	\label{facility:env}
	\begin{tabular}{p{9em}rcp{0.42\textwidth}}
		\hline\hline
		\multicolumn{2}{l}{String \hfill Value data type} & Constant? & Meaning \\
		\hline
		\texttt{FACILITY}		& \emph{flag}	& no	&
			facility word set present \\
		\texttt{FACILITY-EXT}	& \emph{flag}	& no	&
			facility extensions word set present \\
		\hline\hline
	\end{tabular}
  \end{center}
\end{table}

\section{Additional documentation requirements} % 10.4

\subsection{System documentation} % 10.4.1

\subsubsection{Implementation-defined options} % 10.4.1.1

\begin{itemize}
\item encoding of keyboard events \wref{facility:EKEY}{EKEY});
\item duration of a system clock tick;
\item repeatability to be expected from execution of
	\wref{facility:MS}{MS}.
\end{itemize}

\subsubsection{Ambiguous conditions} % 10.4.1.2

\begin{itemize}
\item \wref{facility:AT-XY}{AT-XY} operation can't be performed on
	user output device.
\end{itemize}

\subsubsection{Other system documentation} % 10.4.1.3

\begin{itemize}
\item no additional requirements.
\end{itemize}

\subsection{Program documentation} % 10.4.2

\subsubsection{Environmental dependencies} % 10.4.2.1

\begin{itemize}
\item using more than seven bits of a character in
	\wref{facility:EKEY}{EKEY}.
\end{itemize}

\subsubsection{Other program documentation} % 10.4.2.2

\begin{itemize}
\item no additional requirements.
\end{itemize}

\section{Compliance and labeling} % 10.5

\subsection{ANS Forth systems} % 10.5.1

The phrase ``Providing the Facility word set'' shall be appended to
the label of any Standard System that provides all of the Facility
word set.

The phrase ``Providing \emph{name(s)} from the Facility Extensions
word set'' shall be appended to the label of any Standard System
that provides portions of the Facility Extensions word set.

The phrase ``Providing the Facility Extensions word set'' shall be
appended to the label of any Standard System that provides all of
the Facility and Facility Extensions word sets.

\subsection{ANS Forth programs} % 10.5.2

The phrase ``Requiring the Facility word set'' shall be appended to
the label of Standard Programs that require the system to provide
the Facility word set.

The phrase ``Requiring \emph{name(s)} from the Facility Extensions
word set'' shall be appended to the label of Standard Programs that
require the system to provide portions of the Facility Extensions
word set.

The phrase ``Requiring the Facility Extensions word set'' shall be
appended to the label of Standard Programs that require the system
to provide all of the Facility and Facility Extensions word sets.

\section{Glossary} % 10.6

\begin{info}
\subsection{Glossary}
\end{info}

\subsection{Facility words} % 10.6.1

\begin{newword}{0742}{AT-XY}[at-x-y]
	\stack{$u_1$ $u_2$}{}

	Perform implementation-dependent steps so that the next
	character displayed will appear in column $u_1$, row $u_2$ of
	the user output device, the upper left corner of which is
	column zero, row zero. An ambiguous condition exists if the
	operation cannot be performed on the user output device with
	the specified parameters.

	\begin{rationale} % A.10.6.1.0742 AT-XY
		Most implementors supply a method of positioning a cursor on
		a CRT screen, but there is great variance in names and stack
		arguments. This version is supported by at least one major
		vendor.
	\end{rationale}
\end{newword}


\begin{newword}{1755}{KEY?}[key-question]
	\stack{}{flag}

	If a character is available, return \emph{true}. Otherwise,
	return \emph{false}. If non-character keyboard events are
	available before the first valid character, they are discarded
	and are subsequently unavailable. The character shall be
	returned by the next execution of \word[core]{KEY}.

	After \word{KEY?} returns with a value of \emph{true},
	subsequent executions of \word{KEY?} prior to the execution
	of \word[core]{KEY} or \word{EKEY} also return \emph{true},
	without discarding keyboard events.

	\begin{rationale} % A.10.6.1.1755 KEY?
		The Technical Committee has gone around several times on the
		stack effects. Whatever is decided will violate somebody's
		practice and penalize some machine. This way doesn't interfere
		with type-ahead on some systems, while requiring the
		implementation of a single-character buffer on machines where
		polling the keyboard inevitably results in the destruction of
		the character.

		Use of \word[core]{KEY} or \word{KEY?} indicates that the
		application does not wish to bother with non-character events,
		so they are discarded, in anticipation of eventually receiving
		a valid character. Applications wishing to handle non-character
		events must use \word{EKEY} and \word{EKEY?}. It is possible
		to mix uses of \word{KEY?}/\word[core]{KEY} and
		\word{EKEY?}/\word{EKEY} within a single application, but
		the application must use \word{KEY?} and \word[core]{KEY} only
		when it wishes to discard non-character events until a valid
		character is received.
	\end{rationale}
\end{newword}


\begin{newword}{2005}{PAGE}
	\stack{}{}

	Move to another page for output. Actual function depends on the
	output device. On a terminal, \word{PAGE} clears the screen and
	resets the cursor position to the upper left corner. On a
	printer, \word{PAGE} performs a form feed.
\end{newword}


\subsection{Facility extension words} % 10.6.2
\extended

\begin{newword}{1305}{EKEY}[e-key]
	\stack{}{u}

	Receive one keyboard event $u$. The encoding of keyboard events
	is implementation defined.

\item[See:]
	\wref{facility:KEY?}{KEY?},
	\wref{core:KEY}{KEY}.

	\begin{rationale} % A.10.6.2.1305 EKEY
		\word{EKEY} provides a standard word to access a system-dependent
		set of ``raw'' keyboard events, including events corresponding
		to members of the standard character set, events corresponding
		to other members of the implementation-defined character set,
		and keystrokes that do not correspond to members of the
		character set.

		\word{EKEY} assumes no particular numerical correspondence
		between particular event code values and the values representing
		standard characters. On some systems, this may allow two
		separate keys that correspond to the same standard character
		to be distinguished from one another.

		In systems that combine both keyboard and mouse events into a
		single ``event stream'', the single number returned by
		\word{EKEY} may be inadequate to represent the full range of
		input possibilities. In such systems, a single ``event record''
		may include a time stamp, the x,y coordinates of the mouse
		position, the keyboard state, and the state of the mouse
		buttons. In such systems, it might be appropriate for \word{EKEY}
		to return the address of an ``event record'' from which the
		other information could be extracted.

		Also, consider a hypothetical Forth system running under
		MS-DOS on a PC-compatible computer. Assume that the
		implementation-defined character set is the ``normal'' 8-bit
		PC character set. In that character set, the codes from 0 to
		127 correspond to ASCII characters. The codes from 128 to 255
		represent characters from various non-English languages,
		mathematical symbols, and some graphical symbols used for line
		drawing. In addition to those characters, the keyboard can
		generate various other ``scan codes'', representing such
		non-character events as arrow keys and function keys.

		There may be multiple keys, with different scan codes,
		corresponding to the same standard character. For example,
		the character representing the number ``1'' often appears both
		in the row of number keys above the alphabetic keys, and also
		in the separate numeric keypad.

		When a program asks the MS-DOS operating system for a keyboard
		event, it receives either a single non-zero byte, representing
		a character, or a zero byte followed by a ``scan code'' byte,
		representing a non-character keyboard event (e.g., a function
		key).

		\word{EKEY} represents each keyboard event as a single number,
		rather than as a sequence of numbers. For the system described
		above, the following would be a reasonable implementation of
		\word{EKEY} and related words:

		\begin{quote}
			The \texttt{MAX-CHAR} environmental query would return 255.

			Assume the existence of a word
			\texttt{DOS-KEY} \stack{}{char}
			which executes the MS-DOS ``Direct STDIN Input'' system call
			(Interrupt 21h, Function 07h) and a word
			\texttt{DOS-KEY?} \stack{}{flag}
			which executes the MS-DOS ``Check STDIN Status'' system call
			(Interrupt 21h, Function 0Bh).

			\setwordlist{core}\ttfamily
			\begin{tabbing}
			\tab \= \tab \= \tab \= \tab \= \hspace{7em} \= \kill
			\word{:} \word[facility]{EKEY?}~ \word{p} -- flag )~
				DOS-KEY?~ \word{0ne} \word{;} \\[\parskip]

			\word{:} \word[facility]{EKEY}~ \word{p} -- u )~
				DOS-KEY~ \word{?DUP} \word{0=} \word{IF}~
					DOS-KEY 256 \word{+}~ \word{THEN} \word{;} \\[\parskip]

			\+ \word{:} \word[facility]{EKEYtoCHAR} \word{p} u -- u false | char true ) \\
				\+ \word{DUP} 255 \word{more} \word{IF} 		\>\>\>\>	\word{p} u ) \\
					\word{DUP} 259 \word{=} \word{IF} 		\>\>\>		\word{bs} 259 is Ctrl-@ (ASCII NUL) \\
					\> \word{DROP} 0 \word{TRUE} \word{EXIT}\>\>		\word{bs} so replace with character \\
				\- \word{THEN} \word{FALSE} \word{EXIT}		\>\>\>		\word{bs} otherwise extended character \\
			\- \word{THEN} \word{TRUE}						\>\>\>\>	\word{bs} normal extended ASCII char. \\
			\word{;} \\[\parskip]

			\word{VARIABLE} PENDING-CHAR ~ -1 PENDING-CHAR \word{!} \\[\parskip]

			\+ \word{:} \word[facility]{KEY?}~ \word{p} -- flag ) \\
				\+ PENDING-CHAR \word{@} \word{0less} \word{IF} \\
					\+ \word{BEGIN}~ \word[facility]{EKEY?} \word{WHILE} \\
						\+ \word[facility]{EKEY} \word[facility]{EKEYtoCHAR} \word{IF} \\
						\-	PENDING-CHAR \word{!}~ \word{TRUE} \word{EXIT} \\
					\- \word{THEN} \word{DROP} \\
				\- \word{REPEAT}~ \word{FALSE} \word{EXIT} \\
			\- \word{THEN}~ \word{TRUE} \\
			\word{;} \\[\parskip]

			\+ \word{:} \word{KEY}~ \word{p} -- char ) \\
				\+ PENDING-CHAR \word{@} \word{0less} \word{IF} \\
					\word{BEGIN}~ \word[facility]{EKEY}~ \word[facility]{EKEYtoCHAR} \word{0=} \word{WHILE} \\
					\> \word{DROP} \\
				\-	\word{REPEAT}~ \word{EXIT} \\
			\- \word{THEN}~ PENDING-CHAR \word{@}~ -1 PENDING-CHAR \word{!} \\
			\word{;}
			\end{tabbing}
		\end{quote}

		\setwordlist{facility}
		This is a full-featured implementation, providing the
		application program with an easy way to either handle
		non-character events (with \word{EKEY}), or to ignore them
		and to only consider ``real'' characters (with
		\word[core]{KEY}).

		Note that \word{EKEY} maps scan codes from 0 to 255 into
		numbers from 256 to 511. \word{EKEY} maps the number 259,
		representing the keyboard combination Ctrl-Shift-@, to the
		character whose numerical value is 0 (ASCII NUL). Many ASCII
		keyboards generate ASCII NUL for Ctrl-Shift-@, so we use that
		key combination for ASCII NUL (which is otherwise unavailable
		from MS-DOS, because the zero byte signifies that another
		scan-code byte follows).

		One consequence of using the ``Direct STDIN Input'' system call
		(function 7) instead of the ``STDIN Input'' system call
		(function 8) is that the normal DOS ``Ctrl-C interrupt'' behavior
		is disabled when the system is waiting for input (Ctrl-C would
		still cause an interrupt while characters are being output). On
		the other hand, if the ``STDIN Input'' system call (function 8)
		were used to implement \word{EKEY}, Ctrl-C interrupts would be
		enabled, but Ctrl-Shift-@ would also cause an interrupt, because
		the operating system would treat the second byte of the 0,3
		sequence as a Ctrl-C, even though the 3 is really a scan code
		and not a character. One ``best of both worlds'' solution is to
		use function 8 for the first byte received by \word{EKEY}, and
		function 7 for the scan code byte. For example:

		\begin{quote}
			\setwordlist{core}\ttfamily
			\begin{tabbing}
			\tab \= \tab \= \tab \= \tab \= \hspace{7em} \= \kill
			\+ \word{:} \word[facility]{EKEY}~ \word{p} -- u ) \\
				\+ DOS-KEY-FUNCTION-8~ \word{?DUP}~ \word{0=}~ \word{IF} \\
					DOS-KEY-FUNCTION-7~ \word{DUP} 3~ \word{=}~ \word{IF} \\
					\> \word{DROP} 0~ \word{ELSE}~ 256 \word{+}\\
				\- \word{THEN} \\
			\- \word{THEN} \\
			\word{;}
			\end{tabbing}
		\end{quote}
		\setwordlist{facility}

		Of course, if the Forth implementor chooses to pass Ctrl-C
		through to the program, without using it for its usual
		interrupt function, then DOS function 7 is appropriate in both
		cases (and some additional care must be taken to prevent a
		typed-ahead Ctrl-C from interrupting the Forth system during
		output operations).

		A Forth system might also choose a simpler implementation of
		\word[core]{KEY}, without implementing \word{EKEY}, as follows:

		\setwordlist{core}
		\begin{quote}\ttfamily
			\word{:} \word{KEY}~~ \word{p} -- char )~
				DOS-KEY~
			\word{;}

			\word{:} \word[facility]{KEY?}~ \word{p} -- flag )~
				DOS-KEY?~ \word{0ne}~
			\word{;}
		\end{quote}
		\setwordlist{facility}

		The disadvantages of the simpler version are:

		\begin{enumerate}
		\item An application program that uses \word[core]{KEY},
			expecting to receive only valid characters, might receive a
			sequence of bytes (e.g., a zero byte followed by a byte with
			the same numerical value as the letter ``A'' that appears to
			contain a valid character, even though the user pressed a key
			(e.g., function key 4) that does not correspond to any valid
			character.

		\item An application program that wishes to handle non-character
			events will have to execute \word[core]{KEY} twice if it
			returns zero the first time. This might appear to be a
			reasonable and easy thing to do. However, such code is not
			portable to other systems that do not use a zero byte as an
			``escape'' code. Using the \word{EKEY} approach, the
			algorithm for handling keyboard events can be the same for
			all systems; the system dependencies can be reduced to a
			table or set of constants listing the system-dependent key
			codes used to access particular application functions.
			Without \word{EKEY}, the algorithm, not just the table, is
			likely to be system dependent.
		\end{enumerate}

		Another approach to \word{EKEY} on MS-DOS is to use the BIOS
		``Read Keyboard Status'' function (Interrupt 16h, Function 01h)
		or the related ``Check Keyboard'' function (Interrupt 16h,
		Function 11h). The advantage of this function is that it allows
		the program to distinguish between different keys that correspond
		to the same character (e.g. the two ``1'' keys). The disadvantage
		is that the BIOS keyboard functions read only the keyboard. They
		cannot be ``redirected'' to another ``standard input'' source,
		as can the DOS STDIN Input functions.
	\end{rationale}
\end{newword}


\begin{newword}[EKEYtoCHAR]{1306}{EKEY>CHAR}[e-key-to-char]
	\stack{u}{u false ~\textbar~ char true}

	If the keyboard event $u$ corresponds to a character in the
	implementation-defined character set, return that character and
	\emph{true}. Otherwise return $u$ and \emph{false}.

	\begin{rationale} % A.10.6.2.1306 EKEY>CHAR
		\word{EKEYtoCHAR} translates a keyboard event into the
		corresponding member of the character set, if such a
		correspondence exists for that event.

		It is possible that several different keyboard events may
		correspond to the same character, and other keyboard events
		may correspond to no character.
	\end{rationale}
\end{newword}


\begin{newword}{1307}{EKEY?}[e-key-question]
	\stack{}{flag}

	If a keyboard event is available, return \emph{true}. Otherwise
	return \emph{false}. The event shall be returned by the next
	execution of \word{EKEY}.

	After \word{EKEY?} returns with a value of \emph{true},
	subsequent executions of \word{EKEY?} prior to the execution of
	\word[core]{KEY}, \word{KEY?} or \word{EKEY} also return
	\emph{true}, referring to the same event.
\end{newword}


\begin{newword}{1325}{EMIT?}[emit-question]
	\stack{}{flag}

	\emph{flag} is true if the user output device is ready to
	accept data and the execution of \word[core]{EMIT} in place of
	\word{EMIT?} would not have suffered an indefinite delay. If
	the device status is indeterminate, \emph{flag} is true.

	\begin{rationale} % A.10.6.2.1325 EMIT?
		An indefinite delay is a device related condition, such as
		printer off-line, that requires operator intervention before
		the device will accept new data.
	\end{rationale}
\end{newword}


\begin{newword}{1905}{MS}
	\stack{u}{}

	Wait at least $u$ milliseconds.

\item[Note:]
	The actual length and variability of the time period depends
	upon the implementation-defined resolution of the system clock
	and upon other system and computer characteristics beyond the
	scope of this Standard.

	\begin{rationale} % A.10.6.2.1905 MS
		Although their frequencies vary, every system has a clock.
		Since many programs need to time intervals, this word is
		offered. Use of milliseconds as an internal unit of time is
		a practical ``least common denominator'' external unit. It
		is assumed implementors will use ``clock ticks'' (whatever
		size they are) as an internal unit and convert as appropriate.
	\end{rationale}
\end{newword}


\begin{newword}[TIMEandDATE]{2292}{TIME\&DATE}[time-and-date]
	\stack{}{$+n_1$ $+n_2$ $+n_3$ $+n_4$ $+n_5$ $+n_6$}

	Return the current time and date.
	$+n_1$ is the second \{0{\ldots}59\},
	$+n_2$ is the minute \{0{\ldots}59\},
	$+n_3$ is the hour \{0{\ldots}23\},
	$+n_4$ is the day \{1{\ldots}31\},
	$+n_5$ is the month \{1{\ldots}12\}, and
	$+n_6$ is the year (e.g., 1991).

	\begin{rationale} % A.10.6.2.2292 TIME&DATE
		Most systems have a real-time clock/calendar.
		This word gives portable access to it.
	\end{rationale}
\end{newword}
