\chapter{The optional File-Access word set} % 11
\wordlist{file}

\begin{info}
\section{The optional File-Access word set}

Many Forth systems support access to a host file system, and many of
these support interpretation of Forth from source text files. The
Forth-83 Standard did not address host OS files. Nevertheless, a degree
of similarity exists among modern implementations.

For example, files must be opened and closed, created and deleted.
Forth file-system implementations differ mostly in the treatment and
disposition of the exception codes, and in the format of the
file-identification strings. The underlying mechanism for creating
file-control blocks might or might not be visible. We have chosen to
keep it invisible.

Files must also be read and written. Text files, if supported, must
be read and written one line at a time. Interpretation of text files
implies that they are somehow integrated into the text interpreter
input mechanism. These and other requirements have shaped the
file-access extensions word set.

Most of the existing implementations studied use simple English words
for common host file functions: \texttt{OPEN}, \texttt{CLOSE},
\texttt{READ}, etc. Although we would have preferred to do likewise,
there were so many minor variations in implementation of these words
that adopting any particular meaning would have broken much existing
code. We have used names with a suffix \texttt{-FILE} for most of these
words. We encourage implementors to conform their single-word primitives
to the ANS behaviors, and hope that if this is done on a widespread
basis we can adopt better definition names in a future standard.

Specific rationales for members of this word set follow.


\end{info}

\section{Introduction} % 11.1

These words provide access to mass storage in the form of ``files''
under the following assumptions:

\begin{itemize}
\item files are provided by a host operating system;
\item file names are represented as character strings;
\item the format of file names is determined by the host operating
	system;
\item an open file is identified by a single-cell file identifier
	(\emph{fileid});
\item file-state information (e.g., position, size) is managed by
	the host operating system;
\item file contents are accessed as a sequence of characters;
\item file read operations return an actual transfer count, which
	can differ from the requested transfer count.
\end{itemize}

\section{Additional terms} % 11.2

\begin{description}
\item[file-access method:]
	A permissible means of accessing a file, such as ``read/write''
	or ``read only''.

\item[file position:]
	The character offset from the start of the file.

\item[input file:]
	The file, containing a sequence of lines, that is the input source.
\end{description}

\section{Additional usage requirements} % 11.3

\begin{info}
\subsection{Additional usage requirements}
\end{info}

\subsection{Data types} % 11.3.1

Append table \ref{file:types} to table \ref{table:datatypes}.

\begin{table}[h]
  \begin{center}
	\caption{Data types}
	\label{file:types}
	\begin{tabular}{llr}
	\hline\hline
	\emph{Symbol} & \emph{Data type} & \emph{Size on stack} \\
	\hline
	\emph{ior}		& I/O results			& 1 cell \\
	\emph{fam}		& file access method	& 1 cell \\
	\emph{fileid}	& file identifier		& 1 cell \\
	\hline\hline
	\end{tabular}
  \end{center}
\end{table}

\subsubsection{File identifiers} % 11.3.1.1

File identifiers are implementation-dependent single-cell values
that are passed to file operators to designate specific files.
Opening a file assigns a file identifier, which remains valid
until closed.

\subsubsection{I/O results} % 11.3.1.2
\label{file:ior}

I/O results are single-cell numbers indicating the result of I/O
operations. A value of zero indicates that the I/O operation
completed successfully; other values and their meanings are
implementation-defined. Reaching the end of a file shall be
reported as zero.

An I/O exception in the execution of a File-Access word that can
return an I/O result shall not cause a \word[exception]{THROW};
exception indications are returned in the \emph{ior}.

\subsubsection{File access methods} % 11.3.1.3

File access methods are implementation-defined single-cell
values.

\subsubsection{File names} % 11.3.1.4
\label{file:names}

A character string containing the name of the file. The file name
may include an implementation-dependent path name. The format of
file names is implementation defined.

\subsection{Blocks in files} % 11.3.2
\label{file:blocks}

If the File-Access word set is implemented, the Block word set
shall be implemented. Blocks may, but need not, reside in files.
When they do:
\begin{itemize}
\item Block numbers may be mapped to one or more files by
	implementation-defined means. An ambiguous condition exists
	if a requested block number is not currently mapped;
\item An \word[block]{UPDATE}d block that came from a file shall
	be transferred back to the same file.
\end{itemize}

\begin{info}
\subsubsection{Blocks in files}

Many systems reuse file identifiers; when a file is closed, a
subsequently opened file may be given the same identifier. If the
original file has blocks still in block buffers, these will be
incorrectly associated with the newly opened file with disastrous
results. The block buffer system must be flushed to avoid this.
\end{info}


\subsection{Environmental queries} % 11.3.3

Append table \ref{file:env} to table \ref{table:env}.

See: \xref{usage:env}{3.2.6 Environmental queries}.

\begin{table}[ht]
  \begin{center}
	\caption{Environmental Query Strings}
	\label{file:env}
	\begin{tabular}{p{9em}rcp{0.42\textwidth}}
		\hline\hline
		\multicolumn{2}{l}{String \hfill Value data type} & Constant? & Meaning \\
		\hline
		\texttt{FILE}		& \emph{flag}		& no	&
			file word set present \\
		\texttt{FILE-EXT}	& \emph{flag}		& no	&
			file extensions word set present \\
		\hline\hline
	\end{tabular}
  \end{center}
\end{table}

\subsection{Input source} % 11.3.4
\label{file:source}

The File-Access word set creates another input source for the text
interpreter. When the input source is a text file, \word[block]{BLK}
shall contain zero, \word{SOURCE-ID} shall contain the \emph{fileid}
of that text file, and the input buffer shall contain one line of
the text file.

Input with \word{INCLUDED}, \word{INCLUDE-FILE}, \word[block]{LOAD}
and \word[block]{EVALUATE} shall be nestable in any order to at least
eight levels.

A program that uses more than eight levels of input-file nesting has
an environmental dependency. See:
\xref{usage:inbuf}{3.3.3.5 Input buffers},
\xref{wordlist:exception}{9. Optional Exception word set}.

\subsection{Other transient regions} % 11.3.5
\label{file:buffers}

The list of words using memory in transient regions is extended to
include \wref{file:Sq}{S"}. See: \xref{usage:transient}{3.3.3.6
Other transient regions}.

\subsection{Parsing} % 11.3.6
\label{file:parsing}

When parsing from a text file using a space delimiter, control
characters shall be treated the same as the space character.

Lines of at least 128 characters shall be supported. A program that
requires lines of more than 128 characters has an environmental
dependency.

A program may reposition the parse area within the input buffer by
manipulating the contents of \word[core]{toIN}. More extensive
repositioning can be accomplished using \word[core]{SAVE-INPUT} and
\word[core]{RESTORE-INPUT}.

See: \xref{usage:parsing}{3.4.1 Parsing}.

\section{Additional documentation requirements} % 11.4

\subsection{System documentation} % 11.4.1

\subsubsection{Implementation-defined options} % 11.4.1.1
\begin{itemize}
\item file access methods used by
	\wref{file:BIN}{BIN},
	\wref{file:CREATE-FILE}{CREATE-FILE},
	\wref{file:OPEN-FILE}{OPEN-FILE},
	\wref{file:R/O}{R/O},
	\wref{file:R/W}{R/W}, and
	\wref{file:W/O}{W/O};
\item file exceptions;
\item file line terminator (\wref{file:READ-LINE}{READ-LINE});
\item file name format (\xref{file:names}{11.3.1.4 File names});
\item information returned by \wref{file:FILE-STATUS}{FILE-STATUS};
\item input file state after an exception
	(\wref{file:INCLUDE-FILE}{INCLUDE-FILE},
	 \wref{file:INCLUDED}{INCLUDED});
\item \emph{ior} values and meaning (\xref{file:ior}{11.3.1.2 I/O results});
\item maximum depth of file input nesting
	(\xref{file:source}{11.3.4 Input source});
\item maximum size of input line (\xref{file:parsing}{11.3.6 Parsing});
\item methods for mapping block ranges to files
	(\xref{file:blocks}{11.3.2 Blocks in files});
\item number of string buffers provided (\wref{file:Sq}{S"});
\item size of string buffer used by \wref{file:Sq}{S"}.
\end{itemize}

\subsubsection{Ambiguous conditions} % 11.4.1.2
\begin{itemize}
\item attempting to position a file outside its boundaries
	(\wref{file:REPOSITION-FILE}{REPOSITION-FILE});
\item attempting to read from file positions not yet written
	(\wref{file:READ-FILE}{READ-FILE}, \wref{file:READ-LINE}{READ-LINE});
\item \emph{fileid} is invalid (\wref{file:INCLUDE-FILE}{INCLUDE-FILE});
\item I/O exception reading or closing \emph{fileid}
	(\wref{file:INCLUDE-FILE}{INCLUDE-FILE},
	 \wref{file:INCLUDED}{INCLUDED});
\item named file cannot be opened
	(\wref{file:INCLUDED}{INCLUDED});
\item requesting an unmapped block number
	(\xref{file:blocks}{11.3.2 Blocks in files});
\item using \wref{file:SOURCE-ID}{SOURCE-ID} when
	\wref{block:BLK}{BLK} is not zero.
\end{itemize}

\subsubsection{Other system documentation} % 11.4.1.3
\begin{itemize}
\item no additional requirements.
\end{itemize}

\subsection{Program documentation} % 11.4.2

\subsubsection{nvironmental dependencies} % 11.4.2.1
\begin{itemize}
\item requiring lines longer than 128 characters
	(\xref{file:parsing}{11.3.6 Parsing});
\item using more than eight levels of input-file nesting
	(\xref{file:source}{11.3.4 Input source}).
\end{itemize}

\subsubsection{Other program documentation} % 11.4.2.2
\begin{itemize}
\item no additional requirements.
\end{itemize}

\section{Compliance and labeling} % 11.5

\subsection{ANS Forth systems} % 11.5.1

The phrase ``Providing the File Access word set'' shall be appended
to the label of any Standard System that provides all of the File
Access word set.

The phrase ``Providing \emph{name(s)} from the File Access Extensions
word set'' shall be appended to the label of any Standard System that
provides portions of the File Access Extensions word set.

The phrase ``Providing the File Access Extensions word set'' shall
be appended to the label of any Standard System that provides all of
the File Access and File Access Extensions word sets.

\subsection{ANS Forth programs} % 11.5.2

The phrase ``Requiring the File Access word set'' shall be appended
to the label of Standard Programs that require the system to provide
the File Access word set.

The phrase ``Requiring \emph{name(s)} from the File Access Extensions
word set'' shall be appended to the label of Standard Programs that
require the system to provide portions of the File Access Extensions
word set.

The phrase ``Requiring the File Access Extensions word set'' shall be
appended to the label of Standard Programs that require the system to
provide all of the File Access and File Access Extensions word sets.


\section{Glossary} % 11.6

\begin{info}
\subsection{Glossary}
\end{info}

\subsection{File Access words} % 11.6.1

\begin{newword}[p]{0080}{(}[paren]
	\stack{``ccc\arg{paren}''}{}

	Extend the semantics of \wref{core:p}{(} to include:

	When parsing from a text file, if the end of the parse area is
	reached before a right parenthesis is found, refill the input
	buffer from the next line of the file, set \word[core]{toIN} to
	zero, and resume parsing, repeating this process until either a
	right parenthesis is found or the end of the file is reached.
\end{newword}


\begin{newword}{0765}{BIN}
	\stack{fam$_1$}{fam$_2$}

	Modify the implementation-defined file access method
	\emph{fam}$_1$ to additionally select a ``binary'', i.e., not
	line oriented, file access method, giving access method
	\emph{fam}$_2$.

\item[See:]
	\wref{file:R/O}{R/O},
	\wref{file:R/W}{R/W},
	\wref{file:W/O}{W/O}.

	\begin{rationale} % A.11.6.1.0765 BIN
		Some operating systems require that files be opened in a
		different mode to access their contents as an unstructured
		stream of binary data rather than as a sequence of lines.

		The arguments to \word{READ-FILE} and \word{WRITE-FILE} are
		arrays of character storage elements, each element consisting
		of at least 8 bits. The Technical Committee intends that, in
		\word{BIN} mode, the contents of these storage elements can be
		written to a file and later read back without alteration. The
		Technical Committee has declined to address issues regarding
		the impact of ``wide'' characters on the File and Block word
		sets.
	\end{rationale}
\end{newword}


\begin{newword}{0900}{CLOSE-FILE}
	\stack{fileid}{ior}

	Close the file identified by \emph{fileid}. \emph{ior} is the
	implementation-defined I/O result code.
\end{newword}


\begin{newword}{1010}{CREATE-FILE}
	\stack{c-addr u fam}{fileid ior}

	Create the file named in the character string specified by
	\emph{c-addr} and $u$, and open it with file access method
	\emph{fam}. The meaning of values of \emph{fam} is
	implementation defined. If a file with the same name already
	exists, recreate it as an empty file.

	If the file was successfully created and opened, \emph{ior} is
	zero, \emph{fileid} is its identifier, and the file has been
	positioned to the start of the file.

	Otherwise, \emph{ior} is the implementation-defined I/O result
	code and \emph{fileid} is undefined.

	\begin{rationale} % A.11.6.1.1010 CREATE-FILE
		Typical use:

		\tab \word[core]{:} \texttt{X} {\ldots}
				\word{Sq} \texttt{TEST.FTH"} \word{R/W}
				\word{CREATE-FILE}
				\word[core]{ABORTq} \texttt{CREATE-FILE FAILED"}
			\word[core]{;}
	\end{rationale}
\end{newword}


\begin{newword}{1190}{DELETE-FILE}
	\stack{c-addr u}{ior}

	Delete the file named in the character string specified by
	\emph{c-addr} $u$. \emph{ior} is the implementation-defined
	I/O result code.
\end{newword}


\begin{newword}{1520}{FILE-POSITION}
	\stack{fileid}{ud ior}

	\emph{ud} is the current file position for the file identified
	by \emph{fileid}. \emph{ior} is the implementation-defined I/O
	result code. \emph{ud} is undefined if \emph{ior} is non-zero.
\end{newword}


\begin{newword}{1522}{FILE-SIZE}
	\stack{fileid}{ud ior}

	\emph{ud} is the size, in characters, of the file identified by
	\emph{fileid}. \emph{ior} is the implementation-defined I/O
	result code. This operation does not affect the value returned
	by \word{FILE-POSITION}. \emph{ud} is undefined if \emph{ior}
	is non-zero.
\end{newword}


\begin{newword}{1717}{INCLUDE-FILE}
	\stack{i*x fileid}{j*x}

	Remove \emph{fileid} from the stack. Save the current input
	source specification, including the current value of
	\word{SOURCE-ID}. Store \emph{fileid} in \word{SOURCE-ID}.
	Make the file specified by \emph{fileid} the input source. Store
	zero in \word[block]{BLK}. Other stack effects are due to the
	words included.

	Repeat until end of file: read a line from the file, fill the
	input buffer from the contents of that line, set \word[core]{toIN}
	to zero, and interpret.

	Text interpretation begins at the file position where the next
	file read would occur.

	When the end of the file is reached, close the file and restore
	the input source specification to its saved value.

	An ambiguous condition exists if \emph{fileid} is invalid, if
	there is an I/O exception reading \emph{fileid}, or if an I/O
	exception occurs while closing \emph{fileid}. When an ambiguous
	condition exists, the status (open or closed) of any files that
	were being interpreted is implementation-defined.

\item[See:]
	\xref{file:source}{11.3.4 Input source}.

	\begin{rationale} % A.11.6.1.1717 INCLUDE-FILE
		Here are two implementation alternatives for saving the input
		source specification in the presence of text file input:

		\begin{enumerate}
		\item[1)] Save the file position (as returned by
			\word{FILE-POSITION}) of the beginning of the line being
			interpreted. To restore the input source specification,
			seek to that position and re-read the line into the input
			buffer.

		\item[2)] Allocate a separate line buffer for each active text
			input file, using that buffer as the input buffer. This
			method avoids the ``seek and reread'' step, and allows the
			use of ``pseudo-files'' such as pipes and other
			sequential-access-only communication channels. 
		\end{enumerate}
	\end{rationale}
\end{newword}


\begin{newword}{1718}{INCLUDED}
	\stack{i*x c-addr u}{j*x}

	Remove \emph{c-addr} $u$ from the stack. Save the current input
	source specification, including the current value of
	\word{SOURCE-ID}. Open the file specified by \emph{c-addr} $u$,
	store the resulting \emph{fileid} in \word{SOURCE-ID}, and make
	it the input source. Store zero in \word[block]{BLK}. Other
	stack effects are due to the words included.

	Repeat until end of file: read a line from the file, fill the
	input buffer from the contents of that line, set \word[core]{toIN}
	to zero, and interpret.

	Text interpretation begins at the file position where the next
	file read would occur.

	When the end of the file is reached, close the file and restore
	the input source specification to its saved value.

	An ambiguous condition exists if the named file can not be
	opened, if an I/O exception occurs reading the file, or if an
	I/O exception occurs while closing the file. When an ambiguous
	condition exists, the status (open or closed) of any files that
	were being interpreted is implementation-defined.

\item[See:]
	\wref{file:INCLUDE-FILE}{INCLUDE-FILE}.

	\begin{rationale} % A.11.6.1.1718 INCLUDED
		Typical use:
			{\ldots} \word{Sq} \texttt{filename"} \word{INCLUDED} {\ldots}
	\end{rationale}
\end{newword}


\begin{newword}{1970}{OPEN-FILE}
	\stack{c-addr u fam}{fileid ior}

	Open the file named in the character string specified by
	\emph{c-addr} $u$, with file access method indicated by
	\emph{fam}. The meaning of values of \emph{fam} is
	implementation 	defined.

	If the file is successfully opened, \emph{ior} is zero,
	\emph{fileid} is its identifier, and the file has been
	positioned to the start of the file.

	Otherwise, \emph{ior} is the implementation-defined I/O
	result code and \emph{fileid} is undefined.

	\begin{rationale} % A.11.6.1.1970 OPEN-FILE
		Typical use:

		\tab \word[core]{:} \texttt{X} {\ldots}
				\word{Sq} \texttt{ TEST.FTH"} \word{R/W}
				\word{OPEN-FILE}~ \word[core]{ABORTq} \texttt{OPEN-FILE FAILED"}
		{\ldots} \word[core]{;}
	\end{rationale}
\end{newword}


\begin{newword}{2054}{R/O}[r-o]
	\stack{}{fam}

	\emph{fam} is the implementation-defined value for selecting
	the ``read only'' file access method.

\item[See:]	
	\wref{file:CREATE-FILE}{CREATE-FILE},
	\wref{file:OPEN-FILE}{OPEN-FILE}.
\end{newword}


\begin{newword}{2056}{R/W}[r-w]
	\stack{}{fam}

	\emph{fam} is the implementation-defined value for selecting
	the ``read/write'' file access method.

\item[See:]
	\wref{file:CREATE-FILE}{CREATE-FILE},
	\wref{file:OPEN-FILE}{OPEN-FILE}.
\end{newword}


\begin{newword}{2080}{READ-FILE}
	\stack{c-addr $u_1$ fileid}{$u_2$ ior}

	Read $u_1$ consecutive characters to \emph{c-addr} from the
	current position of the file identified by \emph{fileid}.

	If $u_1$ characters are read without an exception, \emph{ior}
	is zero and $u_2$ is equal to $u_1$.

	If the end of the file is reached before $u_1$ characters are
	read, \emph{ior} is zero and $u_2$ is the number of characters
	actually read.

	If the operation is initiated when the value returned by
	\word{FILE-POSITION} is equal to the value returned by
	\word{FILE-SIZE} for the file identified by \emph{fileid},
	\emph{ior} is zero and $u_2$ is zero.

	If an exception occurs, \emph{ior} is the implementation-defined
	I/O result code, and $u_2$ is the number of characters
	transferred to \emph{c-addr} without an exception.

	An ambiguous condition exists if the operation is initiated when
	the value returned by \word{FILE-POSITION} is greater than the
	value returned by \word{FILE-SIZE} for the file identified by
	\emph{fileid}, or if the requested operation attempts to read
	portions of the file not written.

	At the conclusion of the operation, \word{FILE-POSITION} returns
	the next file position after the last character read.

	\begin{rationale} % A.11.6.1.2080 READ-FILE
		A typical sequential file-processing algorithm might look like:
		\begin{quote}\ttfamily\setwordlist{core}
			\begin{tabbing}
			\tab \= \hspace{12em} \= \kill
			\word{BEGIN}					\>\> \word{p} ) \\ 
			\> {\ldots} \word[file]{READ-FILE} \word[exception]{THROW}
											\>	 \word{p} length ) \\
			\word{?DUP} \word{WHILE}		\>\> \word{p} length ) \\
			\> {\ldots}						\>	 \word{p} ) \\
			\word{REPEAT}					\>\> \word{p} )
			\end{tabbing}
		\end{quote}\setwordlist{file}

		In this example, \word[exception]{THROW} is used to handle
		(unexpected) exception conditions, which are reported as
		non-zero values of the \emph{ior} return value from
		\word{READ-FILE}. End-of-file is reported as a zero value of
		the ``length'' return value.
	\end{rationale}
\end{newword}


\begin{newword}{2090}{READ-LINE}
	\stack{c-addr $u_1$ fileid}{$u_2$ flag ior}

	Read the next line from the file specified by \emph{fileid} into
	memory at the address \emph{c-addr}. At most $u_1$ characters
	are read. Up to two implementation-defined line-terminating
	characters may be read into memory at the end of the line, but
	are not included in the count $u_2$. The line buffer provided by
	\emph{c-addr} should be at least $u_1+2$ characters long.

	If the operation succeeded, \emph{flag} is true and \emph{ior} is
	zero. If a line terminator was received before $u_1$ characters
	were read, then $u_2$ is the number of characters, not including
	the line terminator, actually read ($0 <= u_2 <= u_1$). When
	$u_1 = u_2$ the line terminator has yet to be reached.

	If the operation is initiated when the value returned by
	\word{FILE-POSITION} is equal to the value returned by
	\word{FILE-SIZE} for the file identified by \emph{fileid},
	\emph{flag} is false, \emph{ior} is zero, and $u_2$ is zero.
	If \emph{ior} is non-zero, an exception occurred during the
	operation and \emph{ior} is the implementation-defined I/O
	result code.

	An ambiguous condition exists if the operation is initiated when
	the value returned by \word{FILE-POSITION} is greater than the
	value returned by \word{FILE-SIZE} for the file identified by
	\emph{fileid}, or if the requested operation attempts to read
	portions of the file not written.

	At the conclusion of the operation, \word{FILE-POSITION} returns
	the next file position after the last character read.

	\begin{rationale} % A.11.6.1.2090 READ-LINE
		Implementations are allowed to store the line terminator in
		the memory buffer in order to allow the use of line reading
		functions provided by host operating systems, some of which
		store the terminator. Without this provision, a temporary
		buffer might be needed. The two-character limitation is
		sufficient for the vast majority of existing operating
		systems. Implementations on host operating systems whose line
		terminator sequence is longer than two characters may have to
		take special action to prevent the storage of more than two
		terminator characters.

		Standard Programs may not depend on the presence of any such
		terminator sequence in the buffer.

		A typical line-oriented sequential file-processing algorithm
		might look like:

		\begin{quote}\ttfamily\setwordlist{core}
		  \begin{tabbing}
			\tab \= \hspace{12em} \= \kill
			\word{BEGIN}				\>\> \word{p} ) \\
			\> {\ldots} \word[file]{READ-LINE} \word[exception]{THROW}
										\>	 \word{p} length not-eof-flag ) \\
			\word{WHILE} 				\>\> \word{p} length ) \\
			\> {\ldots}					\>	 \word{p} ) \\
			\word{REPEAT} \word{DROP}	\>	 \word{p} ) \\
		  \end{tabbing}
		\end{quote}\setwordlist{file}

		In this example, \word[exception]{THROW} is used to handle
		(unexpected) I/O exception condition, which are reported as
		non-zero values of the ``\emph{ior}'' return value from
		\word{READ-LINE}.

		\word{READ-LINE} needs a separate end-of-file flag because
		empty (zero-length) lines are a routine occurrence, so a
		zero-length line cannot be used to signify end-of-file.
	\end{rationale}
\end{newword}


\begin{newword}{2142}{REPOSITION-FILE}
	\stack{ud fileid}{ior}

	Reposition the file identified by \emph{fileid} to \emph{ud}.
	\emph{ior} is the implementation-defined I/O result code. An
	ambiguous condition exists if the file is positioned outside
	the file boundaries.

	At the conclusion of the operation, \word{FILE-POSITION}
	returns the value \emph{ud}.
\end{newword}


\begin{newword}{2147}{RESIZE-FILE}
	\stack{ud fileid}{ior}

	Set the size of the file identified by \emph{fileid} to
	\emph{ud}. \emph{ior} is the implementation-defined I/O result
	code.

	If the resultant file is larger than the file before the
	operation, the portion of the file added as a result of the
	operation might not have been written.

	At the conclusion of the operation, \word{FILE-SIZE} returns
	the value \emph{ud} and \word{FILE-POSITION} returns an
	unspecified value.

\item[See:]
	\wref{file:READ-FILE}{READ-FILE},
	\wref{file:READ-LINE}{READ-LINE}.
\end{newword}


\begin{newword}[Sq]{2165}{S"}[s-quote]
	Extend the semantics of \wref{core:Sq}{S"} to be:

\item[Interpretation:]
	\stack{``ccc\arg{quote}''}{c-addr u}

	Parse \emph{ccc} delimited by \texttt{"} (double quote). Store
	the resulting string \emph{c-addr} $u$ at a temporary location.
	The maximum length of the temporary buffer is
	implementation-dependent but shall be no less than 80 characters.
	Subsequent uses of \word{Sq} may overwrite the temporary buffer.
	At least one such buffer shall be provided.

\item[Compilation:]
	\stack{``ccc\arg{quote}''}{}

	Parse \emph{ccc} delimited by \texttt{"} (double quote). Append
	the run-time semantics given below to the current definition.

\item[Run-time:]
	\stack{}{c-addr u}

	Return \emph{c-addr} and $u$ that describe a string consisting
	of the characters \emph{ccc}.

\item[See:]
	\xref{usage:parsing}{3.4.1 Parsing},
	\wref{core:Cq}{C"},
	\wref{core:Sq}{S"},
	\xref{file:buffers}{11.3.5 Other transient regions}.

	\begin{rationale} % A.11.6.1.2165 S"
		Typical use:
			{\ldots} \word{Sq} \emph{ccc}\texttt{"} {\ldots}

		The interpretation semantics for \word{Sq} are intended to
		provide a simple mechanism for entering a string in the
		interpretation state. Since an implementation may choose to
		provide only one buffer for interpreted strings, an
		interpreted string is subject to being overwritten by the
		next execution of \word{Sq} in interpretation state. It is
		intended that no standard words other than \word{Sq} should
		in themselves cause the interpreted string to be overwritten.
		However, since words such as \word[core]{EVALUATE},
		\word[block]{LOAD}, \word{INCLUDE-FILE} and
		\word{INCLUDED} can result in the interpretation of arbitrary
		text, possibly including instances of \word{Sq}, the
		interpreted string may be invalidated by some uses of these
		words.

		When the possibility of overwriting a string can arise, it is
		prudent to copy the string to a ``safe'' buffer allocated by
		the application.

		Programs wishing to parse in the fashion of \word{Sq} are
		advised to use \word[core]{PARSE} or \word[core]{WORD}
		\word[core]{COUNT} instead of \word{Sq}, preventing the
		overwriting of the interpreted string buffer.
	\end{rationale}
\end{newword}


\begin{newword}{2218}{SOURCE-ID}[source-i-d]
	\stack{}{0~\textbar~-1~\textbar~fileid}

	Extend \wref{core:SOURCE-ID}{SOURCE-ID} to include text-file
	input as follows:
	\begin{center}
	  \begin{tabular}{cl}
		\hline\hline
		\word{SOURCE-ID} & Input source \\
		\hline
		\emph{fileid}	& Text file ``\emph{fileid}'' \\
		\emph{-1}		& String (via \word[core]{EVALUATE}) \\
		\emph{0}		& User input device \\
		\hline\hline
	  \end{tabular}
	\end{center}

	An ambiguous condition exists if \word{SOURCE-ID} is used when
	\word[block]{BLK} contains a non-zero value.
\end{newword}


\begin{newword}{2425}{W/O}[w-o]
	\stack{}{fam}

	\emph{fam} is the implementation-defined value for selecting
	the ``write only'' file access method.

\item[See:]
	\wref{file:CREATE-FILE}{CREATE-FILE},
	\wref{file:OPEN-FILE}{OPEN-FILE}.
\end{newword}


\begin{newword}{2480}{WRITE-FILE}
	\stack{c-addr u fileid}{ior}

	Write $u$ characters from \emph{c-addr} to the file identified
	by \emph{fileid} starting at its current position. \emph{ior}
	is the implementation-defined I/O result code.

	At the conclusion of the operation, \word{FILE-POSITION} returns
	the next file position after the last character written to the
	file, and \word{FILE-SIZE} returns a value greater than or equal
	to the value returned by \word{FILE-POSITION}.

\item[See:]
	\wref{file:READ-FILE}{READ-FILE},
	\wref{file:READ-LINE}{READ-LINE}.
\end{newword}


\begin{newword}{2485}{WRITE-LINE}
	\stack{c-addr u fileid}{ior}

	Write $u$ characters from \emph{c-addr} followed by the
	implementation-dependent line terminator to the file identified
	by \emph{fileid} starting at its current position. \emph{ior} is
	the implementation-defined I/O result code.

	At the conclusion of the operation, \word{FILE-POSITION} returns
	the next file position after the last character written to the
	file, and \word{FILE-SIZE} returns a value greater than or equal
	to the value returned by \word{FILE-POSITION}.

\item[See:]
	\wref{file:READ-FILE}{READ-FILE},
	\wref{file:READ-LINE}{READ-LINE}.
\end{newword}


\subsection{File-Access extension words} % 11.6.2
\extended

\begin{newword}{1524}{FILE-STATUS}
	\stack{c-addr u}{x ior}

	Return the status of the file identified by the character string
	\emph{c-addr} $u$. If the file exists, \emph{ior} is zero;
	otherwise \emph{ior} is the implementation-defined I/O result
	code. $x$ contains implementation-defined information about the
	file.
\end{newword}


\begin{newword}{1560}{FLUSH-FILE}
	\stack{fileid}{ior}

	Attempt to force any buffered information written to the file
	referred to by \emph{fileid} to be written to mass storage, and
	the size information for the file to be recorded in the storage
	directory if changed. If the operation is successful, \emph{ior}
	is zero. Otherwise, it is an implementation-defined I/O result
	code.
\end{newword}


\begin{newword}{2125}{REFILL}
	\stack{}{flag}

	Extend the execution semantics of \wref{core:REFILL}{REFILL}
	with the following:

	When the input source is a text file, attempt to read the next
	line from the text-input file. If successful, make the result
	the current input buffer, set \word[core]{toIN} to zero, and
	return \emph{true}. Otherwise return \emph{false}.

\item[See:]	
	\wref{core:REFILL}{REFILL},
	\wref{block:REFILL}{REFILL}.
\end{newword}


\begin{newword}{2130}{RENAME-FILE}
	\stack{c-addr$_1$ $u_1$ c-addr$_2$ $u_2$}{ior}

	Rename the file named by the character string \emph{c-addr}$_1$
	$u_1$ to the name in the character string \emph{c-addr}$_2$
	$u_2$. \emph{ior} is the implementation-defined I/O result code.
\end{newword}
