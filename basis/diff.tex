% !TeX root = forth.tex
% !TeX spellcheck = en_US
\cbstart\patch{compatibility}
\annex[Compatibility analysis]{Compatibility analysis of ANS Forth\strike{4}{80}} % D (informative annex)
\cbend
\label{annex:diff}
\setwordlist{core}

\replace{compatibility}{Prior to}{Before} this standard, there were several industry standards for Forth.
The most influential are listed here in chronological order, along
with the major differences between this standard and the most recent,
\replace{compatibility}{ANS Forth}{Forth 94}.

\section{FIG Forth (circa 1978)} % D.1

FIG Forth was a ``model'' implementation of the Forth language
developed by the Forth Interest Group (FIG). In FIG Forth, a
relatively small number of words were implemented in processor-dependent
machine language and the rest of the words were implemented in Forth.
The FIG model was placed in the public domain, and was ported to a wide
variety of computer systems. Because the bulk of the FIG Forth
implementation was the same across all machines, programs written in
FIG Forth enjoyed a substantial degree of portability, even for
``system-level'' programs that directly manipulate the internals
of the Forth system implementation.

FIG Forth implementations were influential in increasing the number
of people interested in using Forth. Many people associate the
implementation techniques embodied in the FIG Forth model with
``the nature of Forth''.

However, FIG Forth was not necessarily representative of commercial
Forth implementations of the same era. Some of the most successful
commercial Forth systems used implementation techniques different
from the FIG Forth ``model''.


\section{Forth 79} % D.2

The Forth-79 Standard resulted from a series of meetings from 1978
to 1980, by the Forth Standards Team, an international group of Forth
users and vendors (interim versions known as Forth 77 and Forth 78
were also released by the group).

Forth 79 described a set of words defined on a 16-bit, twos-complement,
unaligned, linear byte-addressing virtual machine. It prescribed an
implementation technique known as ``indirect threaded code'', and used
the ASCII character set.

The Forth-79 Standard served as the basis for several public domain
and commercial implementations, some of which are still available and
supported today.


\section{Forth 83} % D.3

The Forth-83 Standard, also by the Forth Standards Team, was released
in 1983. Forth 83 attempted to fix some of the deficiencies of Forth
79.

Forth 83 was similar to Forth 79 in most respects. However, Forth 83
changed the definition of several well-defined features of Forth 79.
For example, the rounding behavior of integer division, the base value
of the operands of \word{PICK} and \word{ROLL}, the meaning of the
address returned by \word{'}, the compilation behavior of \word{'},
the value of a ``true'' flag, the meaning of \texttt{NOT}, and the
``chaining'' behavior of words defined by \texttt{VOCABULARY} were all
changed. Forth 83 relaxed the implementation restrictions of Forth 79
to allow any kind of threaded code, but it did not fully allow
compilation to native machine code (this was not specifically prohibited,
but rather was an indirect consequence of another provision).

Many new Forth implementations were based on the Forth-83 Standard, but
few ``strictly compliant'' Forth-83 implementations exist.

Although the incompatibilities resulting from the changes between
Forth 79 and Forth 83 were usually relatively easy to fix, a number
of successful Forth vendors did not convert their implementations to
be Forth 83 compliant. For example, the most successful commercial
Forth for Apple Macintosh computers is based on Forth 79.

\cbstart\patch{compatibility}
\section[Recent developments]{\sout{Recent developments}} % D.4
\label{diff:development}

\sout{%
Since the Forth-83 Standard was published, the computer industry has
undergone rapid and profound changes. The speed, memory capacity, and
disk capacity of affordable personal computers have increased by
factors of more than 100. 8-bit processors have given way to 16-bit
processors, and now 32-bit processors are commonplace.}

\sout{%
The operating systems and programming-language environments of small
systems are much more powerful than they were in the early 80's.}

\sout{%
The personal-computer marketplace has changed from a predominantly
``hobbyist'' market to a mature business and commercial market.}

\sout{%
Improved technology for designing custom microprocessors has resulted
in the design of numerous ``Forth chips'', computers optimized for
the execution of the Forth language.}

\sout{%
The market for ROM-based embedded control computers has grown
substantially.}

\sout{%
In order to take full advantage of this evolving technology, and to
better compete with other programming languages, many recent Forth
implementations have ignored some of the ``rules'' of previous Forth
standards. In particular:}

\begin{itemize}
\item \sout{32-bit Forth implementations are now common.}
\item \sout{Some Forth systems adopt the address-alignment restrictions of
	the hardware on which they run.}
\item \sout{Some Forth systems use native-code generation, microcode
	generation, and optimization techniques, rather than the
	traditional ``threaded code''.}
\item \sout{Some Forth systems exploit segmented addressing architectures,
	placing portions of the Forth ``dictionary'' in different segments.}
\item \sout{More and more Forth systems now run in the environment of another
	``standard'' operating system, using OS text files for source code,
	rather than the traditional Forth ``blocks''.}
\item \sout{Some Forth systems allow external operating system software,
	windowing software, terminal con\-cen\-tratores, or communications
	channels to handle or preprocess user input, resulting in deviations
	from the input editing, character set availability, and screen
	management behavior prescribed by Forth 83.}
\end{itemize}

\sout{%
Competitive pressure from other programming languages (predominantly
``C'') and from other Forth ven\-dors have led Forth vendors to
optimizations that do not fit in well with the ``virtual machine
model'' implied by existing Forth standards.}


\section[ANS Forth approach]{\sout{ANS Forth approach}} % D.5
\label{diff:ans}

\sout{%
The ANS Forth committee addressed the serious fragmentation of the
Forth community caused by the differences between Forth 79 and
Forth 83, and the divergence from either of these two industry
standards caused by marketplace pressures.}

\sout{%
Consequently, the committee has chosen to base its compatibility
decisions not upon a strict comparison with the Forth-83 Standard,
but instead upon consideration of the variety of existing
implementations, especially those with substantial user bases and/or
considerable success in the marketplace.}

\sout{%
The committee feels that, if ANS Forth prescribes stringent
requirements upon the virtual machine model, as did the previous
standards, then many implementors will chose not to comply with
ANS Forth. The committee hopes that ANS Forth will serve to unify
rather than to further divide the Forth community, and thus
has chosen to encompass rather than invalidate popular implementation
techniques.}

\sout{%
Many of the changes from Forth 83 are justified by this rationale.
Most fall into the category that ``an ANS Forth Standard Program may
not assume x'', where ``x'' is an entitlement resulting from the
virtual machine model prescribed by the Forth-83 Standard. The
committee feels that these restrictions are reasonable, especially
considering that a substantial number of existing Forth implementations
do not correctly implement the Forth-83 virtual model, thus the Forth-83
entitlements exist ``in theory'' but not ``in practice''.}

\sout{%
Another way of looking at this is that while ANS Forth acknowledges
the diversity of current Forth practice, it attempts to document the
similarity therein. In some sense, ANS Forth is thus a ``description
of reality'' rather than a ``prescription for a particular virtual
machine''.}

\sout{%
Since there is no previous American National Standard for Forth, the
action requirements prescribed by section 3.4 of X3/SD-9,
``Policy and Guidelines'', regarding previous standards do not apply.
}

\sout{%
The following discussion describes differences between ANS Forth and
Forth 83. In most cases, Forth 83 is representative of Forth 79 and
FIG Forth for the purposes of this discussion. In many of these cases,
however, ANS Forth is more representative of the existing state of the
Forth industry than the previously-published standards.
}

\section[ANS Forth (1994)]{\uline{ANS Forth (1994)}} % D.5

\uline{%
In the mid to late 1980s the computer industry underwent a rapid and
profound change.  The personal-com\-put\-er market matured into a business
and commercial market, while the market for ROM-based embedded control
computers grew substantially.  Improvements in custom processor design
allowed for the de\-vel\-op\-ment of numerous ``Forth chips,'' customized
for the execution of the Forth language.}

\uline{%
In order to take full advantage of evolving technology, many Forth
implementations ignored some of the restrictions imposed by the
implied ``virtual machine'' of previous standards.
The ANS Forth committee was formed in 1987 to address the fragmentation
within the Forth community caused not only by the difference between
Forth 79 and Forth 83 but the exploitation of technical developments.}

\uline{%
The committee undertook a comprehensive review of a variety of existing
implementations, especially those with substantial user bases and/or
considerable success in the market place.  This allowed them to identify
and document features common to these systems, many of which had not been
included in any previous standard.}

\uline{%
The committee chose to move away from prescribing stringent requirements
as previous standards had, with the specification of a virtual machine.
It preferred to describe the operation of the virtual machine, without
reference to its implementation, thus allowing an implementor to take
full advantage of any technical developments while providing the
developer with a complete list of entitlements.}

\uline{%
This required the identification of implicit assumptions made by the
previous standards, making them explicit and abstracting them into
more general concepts where possible.  A good example of this is the
size of an item on the stack.  In previous standards this was assumed
to be 16-bits wide.  This was no longer a valid assumption.  ANS Forth
introduced the concept of the \emph{cell}, allowing the an implementation
to use a stack size most suited to the environment.}

\uline{%
This was the most comprehensive review of Forth systems to date, taking
eighty seven days covering twenty three meetings over eight years.
The inclusive nature of the standard allowed the various factions within
the community to unify in support of ANS Forth, with many systems
providing a comparability layer.}

\uline{%
The American National Standards Institution (ANSI) published the ANS
Forth Standard in 1994 with the title ``\emph{ANSI X3.215-1994
Information Systems --- Programming Language FORTH}''.  This is referenced
throughout this document as Forth 94.}

\section[ISO Forth (1997)]{\uline{ISO Forth (1997)}}
\label{diff:iso}

\uline{%
ANSI submitted the Forth 94 Standard to the
ISO (International Organization for Standardization) and
IEC (International Electrotechnical Commission) joint committee for
consideration as an international standard.
The ISO/IEC adopted the Forth 94 document as an international standard
in 1997, publishing it under the title ``\emph{ISO/IEC 15145:1997
Information technology.  Programming languages.  FORTH}''.
}


\section[Approach of this Standard]{\uline{Approach of this Standard}} % D.6
\label{diff:approach}

\uline{%
During a workshop on the Forth standard at the EuroForth conference in
2004 it was agreed that the Forth 94 required updating.}

\uline{%
A committee was formed and agreed that the process should be as open
as possible, adopting the Usenet RfD/CfV (Request for Discussion/Call
for Votes) process to produce semi-formal proposals for changes to the
standard.  In addition to general discussion on the \texttt{comp.lang.forth}
usenet news group, a moderated mailing list (with public archive) was
created for those who do not follow the news group.
Standards meetings to discuss CfVs were held in public in
conjunction with the EuroForth conference.}

\uline{%
The work of the Forth 94 Committee was the basis of this standard,
informally called Forth 200\emph{x}.  The aim of the Forth 200\emph{x}
committe is to produce a rolling document, with the standard constantly
being updated based on discussion of proposals and the corresponding
votes.  A snapshot document is occasionally produced, with this document
being the first.}

\uline{%
The Forth 200\emph{x} committee defined a procedure for proposals.  In
addition to the formal text of the proposal, they had to include:
the rationale behind the change;
a reference implementation, or a description of the reason a reference
implementation cannot be presented;
unit testing for the proposed change, especially for border conditions.
See \xref[Process]{process} (page \pageref{process}) for a full description.}


\section[Differences from Forth 83]{\sout{Differences from Forth 83}} % D.6

\subsection[Stack width]{\sout{Stack width}} % D.6.1

\sout{%
Forth 83 specifies that stack items occupy 16 bits. This includes
addresses, flags, and numbers. ANS Forth specifies that stack items
are at least 16 bits; the actual size must be documented by the
implementation.}

\begin{description}
\item[\sout{Words affected:}]
	\sout{all arithmetic, logical and addressing operators}

\item[\sout{Reason:}]
	\sout{32-bit machines are becoming commonplace. A 16-bit Forth
	system on a 32-bit machine is not competitive.}

\item[\sout{Impact:}]
	\sout{Programs that assume 16-bit stack width will continue to run on
	16-bit machines; ANS Forth does not require a different stack
	width, but simply allows it. Many programs will be unaffected
	(but see ``address unit'').}

\item[\sout{Transition/Conversion:}]
	\sout{Programs which use bit masks with the high bits set may have to
	be changed, substituting either an implementation-defined bit-mask
	constant, or a procedure to calculate a bit mask in a
	stack-width-independent way. Here are some procedures for
	constructing width-in\-de\-pend\-ent bit masks:}

	\begin{tt}
		\sout{1 \word{CONSTANT} LO-BIT} \\
		\sout{\word{TRUE} 1 \word{RSHIFT}
			\quad \word{INVERT}
			\quad \word{CONSTANT} HI-BIT} \\
		\sout{\word{:} LO-BITS \word{p} n -{}- mask )
			0 \word{SWAP}	0 \word{qDO}
				1 \word{LSHIFT}		LO-BIT \word{OR}
			\word{LOOP}
		\word{;}} \\
		\sout{\word{:} HI-BITS \word{p} n -{}- mask )
			0 \word{SWAP}	0 \word{qDO}
				1 \word{RSHIFT}	HI-BIT \word{OR}
			\word{LOOP}
		\word{;}}
	\end{tt}
\end{description}

\sout{%
Programs that depend upon the ``modulo 65536'' behavior implicit in
16-bit arithmetic operations will need to be rewritten to explicitly
perform the modulus operation in the appropriate places. The committee
believes that such assumptions occur infrequently. Examples: some
checksum or CRC calculations, some random number generators and most
fixed-point fractional math.}

\subsection[Number representation]{\sout{Number representation}} % D.6.2

\sout{%
Forth 83 specifies two's-complement number representation and
arithmetic. ANS Forth also allows one's-complement and
signed-magnitude.}

\begin{description}
\item[\sout{Words affected:}]
\sout{%
	all arithmetic and logical operators,
	\word{LOOP},
	\word{+LOOP}.}

\item[\sout{Reason:}]
\sout{%
	Some computers use one's-complement or signed-magnitude. The
	committee did not wish to force Forth implementations for those
	machines to emulate two's-complement arithmetic, and thus incur
	severe performance penalties. The experience of some committee
	members with such machines indicates that the usage restrictions
	necessary to support their number representations are not overly
	burdensome.}

\item[\sout{Impact:}]
\sout{%
	An ANS Forth Standard Program may declare an ``environmental
	dependency on two's-com\-ple\-ment arithmetic''. This means that the
	otherwise-Standard Program is only guaranteed to work on
	two's-complement machines. Effectively, this is not a severe
	restriction, because the overwhelming majority of current
	computers use two's-complement. The committee knows of no Forth-83
	com\-pli\-ant implementations for non-two's-complement machines at
	present, so existing Forth-83 programs will still work on the same
	class of machines on which they currently work.}

\item[\sout{Transition/Conversion:}]
\sout{%
	Existing programs wishing to take advantage of the possibility of
	ANS Forth Standard Systems on non-two's-complement machines may
	do so by eliminating the use of a\-rith\-me\-tic operators to perform
	logical functions, by deriving bit-mask constants from bit
	operations as described in the section about stack width, by
	restricting the usage range of unsigned numbers to the range of
	positive numbers, and by using the provided operators for
	conversion from single numbers to double numbers.}
\end{description}

\subsection[Address units]{\sout{Address units}} % D.6.3

\sout{%
Forth 83 specifies that each unique address refers to an 8-bit byte
in memory. ANS Forth specifies that the size of the item referred to
by each unique address is implementation-defined, but, by default,
is the size of one character. Forth 83 describes many memory
operations in terms of a number of bytes. ANS Forth describes those
operations in terms of a number of either characters or address
units.}

\begin{description}
\item[\sout{Words affected:}]
	\sout{those with ``address unit'' arguments}

\item[\sout{Reason:}]
\sout{%
	Some machines, including the most popular Forth chip, address
	16-bit memory locations instead of 8-bit bytes.}

\item[\sout{Impact:}]
\sout{%
	Programs may choose to declare an environmental dependency on
	byte addressing, and will continue to work on the class of
	machines for which they now work. In order for a Forth
	implementation on a word-addressed machine to be Forth 83
	compliant, it would have to simulate byte addressing at
	considerable cost in speed and memory efficiency. The committee
	knows of no such Forth-83 implementations for such machines,
	thus an environmental dependency on byte addressing does not
	restrict a Standard Program beyond its current de facto
	restrictions.}

\item[\sout{Transition/Conversion:}]
\sout{%
	The new \word{CHARS} and \word{CHAR+} address arithmetic operators
	should be used for programs that require portability to
	non-byte-addressed machines. The places where such conversion is
	necessary may be identified by searching for occurrences of words
	that accept a number of address units as an argument (e.g.,
	\word{MOVE}, \word{ALLOT}).}
\end{description}

\subsection[Address increment for a cell is no longer two]{\sout{Address increment for a cell is no longer two}} % D.6.4

\sout{%
As a consequence of Forth-83's simultaneous specification of 16-bit
stack width and byte addressing, the number two could reliably be used
in address calculations involving memory arrays containing items from
the stack. Since ANS Forth requires neither 16-bit stack width nor
byte addressing, the number two is no longer necessarily appropriate
for such calculations.}

\begin{description}
\item[\sout{Words affected:}]
\sout{%
	\word{@}	\word{!}	\word{+!}	\texttt{2+}
	\word{2*}	\texttt{2-}	\word{+LOOP}}

\item[\sout{Reason:}]
\sout{%
	See reasons for ``Address Units'' and ``Stack Width''}

\item[\sout{Impact:}]
\sout{%
	In this respect, existing programs will continue to work on
	machines where a stack cell occupies two address units when
	stored in memory. This includes most machines for which
	Forth 83 compliant implementations currently exist. In principle,
	it would also include 16-bit-word-addressed machines with 32-bit
	stack width, but the committee knows of no examples of such
	machines.}

\item[\sout{Transition/Conversion:}]
\sout{%
	The new \word{CELLS} and \word{CELL+} address arithmetic operators
	should be used for portable programs. The places where such
	conversion is necessary may be identified by searching for the
	character ``2'' and determining whether or not it is used as part
	of an address calculation. The following substitutions are
	appropriate within address calculations:}

	\begin{center}
	  \begin{tabular}{cl}
	  \hline\hline
	  \sout{Old} & \sout{New} \\
	  \hline
		\sout{\texttt{2+} or \texttt{2} \word{+}}	& \sout{\word{CELL+}} \\
		\sout{\word{2*}   or \texttt{2} \word{*}}	& \sout{\word{CELLS}} \\
		\sout{\texttt{2-} or \texttt{2} \word{-}}	& \sout{1 \word{CELLS} \word{-}} \\
		\sout{\word{2/}   or \texttt{2} \word{/}}	& \sout{1 \word{CELLS} \word{/}} \\
		\sout{\texttt{2}}							& \sout{1 \word{CELLS}} \\
	  \hline\hline
	  \end{tabular}
	\end{center}
\end{description}

\sout{%
The number ``2'' by itself is sometimes used for address calculations
as an argument to \word{+LOOP}, when the loop index is an address. When
converting the word \word{2/} which operates on negative dividends, one
should be cognizant of the rounding method used.}

\subsection[Address alignment]{\sout{Address alignment}} % D.6.5

\sout{%
Forth 83 imposes no restriction upon the alignment of addresses to
any boundary. ANS Forth specifies that a Standard System may require
alignment of addresses for use with various ``\word{@}'' and
``\word{!}'' operators.}

\begin{description}
\item[\sout{Words Affected:}]
\sout{%
	\word{!}	\word{+!}		\word{2!}	\word{2@}
	\word{@}	\word[tools]{q}	\word{,}}

\item[\sout{Reason:}]
\sout{%
	Many computers have hardware restrictions that favor the use of
	aligned addresses. On some machines, the native memory-access
	instructions will cause an exception trap if used with an
	unaligned address. Even on machines where unaligned accesses do
	not cause exception traps, aligned accesses are usually faster.}

\item[\sout{Impact:}]
\sout{%
	All of the ANS Forth words that return addresses suitable for
	use with aligned ``\word{@}'' and ``\word{!}'' words must return
	aligned addresses. In most cases, there will be no problem.
	Problems can arise from the use of user-defined data structures
	containing a mixture of character data and cell-sized data.}

\sout{%
	Many existing Forth systems, especially those currently in use on
	computers with strong alignment requirements, already require
	alignment. Much existing Forth code that is currently in use on
	such machines has already been converted for use in an aligned
	environment.}

\item[\sout{Transition/Conversion:}]
\sout{%
	There are two possible approaches to conversion of programs for
	use on a system requiring address alignment.}

\sout{%
	The easiest approach is to redefine the system's aligned
	``\word{@}'' and ``\word{!}'' operators so that they do not
	require alignment. For example, on a 16-bit little-endian
	byte-addressed machine, unaligned ``\word{@}'' and ``\word{!}''
	could be defined:}
	\begin{quote}\ttfamily
		\sout{\word{:} \word{@} \word{p} addr -{}- x )
			\word{DUP} \word{C@} \word{SWAP}
			\word{CHAR+} \word{C@} 8
			\word{LSHIFT} \word{OR}
		\word{;}} \\
		\sout{\word{:} \word{!} \word{p} x addr -{}- )
			\word{OVER} 8 \word{RSHIFT}
			\word{OVER} \word{CHAR+}
			\word{C!} \word{C!}
		\word{;}}
	\end{quote}
\sout{%
	These definitions, and similar ones for ``\word{+!}'',
	``\word{2@}'', ``\word{2!}'', ``\word{,}'', and
	``\word[tools]{q}'' as needed, can be compiled before an
	unaligned application, which will then work as expected.}

\sout{%
	This approach may conserve memory if the application uses
	substantial numbers of data structures containing unaligned
	fields.}

\sout{%
	Another approach is to modify the application's source code to
	eliminate unaligned data fields. The ANS Forth words \word{ALIGN}
	and \word{ALIGNED} may be used to force alignment of data fields.
	The places where such alignment is needed may be determined by
	inspecting the parts of the application where data structures
	(other than simple variables) are defined, or by ``smart compiler''
	techniques (see the ``Smart Compiler'' discussion below).}

\sout{%
	This approach will probably result in faster application execution
	speed, at the possible expense of increased memory utilization for
	data structures.}

\sout{%
	Finally, it is possible to combine the preceding techniques by
	identifying exactly those data fields that are unaligned, and
	using ``unaligned'' versions of the memory access operators for
	only those fields. This ``hybrid'' approach affects a compromise
	between execution speed and memory utilization.}
\end{description}


\subsection[Division/modulus rounding direction]{\sout{Division/modulus rounding direction}} % D.6.6

\sout{%
Forth 79 specifies that division rounds toward 0 and the remainder
carries the sign of the dividend. Forth 83 specifies that division
rounds toward negative infinity and the remainder carries the sign
of the divisor. ANS Forth allows either behavior for the division
operators listed below, at the discretion of the implementor, and
provides a pair of division primitives to allow the user to
synthesize either explicit behavior.}

\begin{description}
\item[\sout{Words Affected:}]
\sout{%
	\word{/}		\word{MOD}		\word{/MOD}
	\word{*/MOD}	\word{*/}}

\item[\sout{Reason:}]
\sout{%
	The difference between the division behaviors in Forth 79 and
	Forth 83 was a point of much contention, and many Forth
	implementations did not switch to the Forth 83 behavior. Both
	variants have vocal proponents, citing both application
	requirements and execution efficiency arguments on both sides.
	After extensive debate spanning many meetings, the committee was
	unable to reach a consensus for choosing one behavior over the
	other, and chose to allow either behavior as the default, while
	providing a means for the user to explicitly use both behaviors
	as needed. Since implementors are allowed to choose either
	behavior, they are not required to change the behavior exhibited
	by their current systems, thus preserving correct functioning of
	existing programs that run on those systems and depend on a
	particular behavior. New implementations could choose to supply
	the behavior that is supported by the native CPU instruction set,
	thus maximizing execution speed, or could choose the behavior
	that is most appropriate for the intended application domain of
	the system.}

\item[\sout{Impact:}]
\sout{%
	The issue only affects programs that use a negative dividend with
	a positive divisor, or a positive dividend with a negative divisor.
	The vast majority of uses of division occur with both a positive
	dividend and a positive divisor; in that case, the results are the
	same for both allowed division behaviors.}

\item[\sout{Transition/Conversion:}]
\sout{%
	For programs that require a specific rounding behavior with division
	operands of mixed sign, the division operators used by the program
	may be redefined in terms of one of the new ANS Forth division
	primitives \word{SM/REM} (symmetrical division, i.e., round toward
	zero) or \word{FM/MOD} (floored division, i.e., round toward
	negative infinity). Then the program may be recompiled without
	change. For example, the Forth 83 style division operators may be
	defined by:}
	\begin{quote}\ttfamily
		\sout{\word{:} \word{/MOD}~ \word{p} n1 n2 -{}- n3 n4 ) ~~
			\word{toR} \word{StoD} \word{Rfrom} \word{FM/MOD} \word{;}} \\
		\sout{\word{:} \word{MOD}~~ \word{p} n1 n2 -{}- n3 ) ~~~~~
			\word{/MOD} \word{DROP}  \word{;}} \\
		\sout{\word{:} \word{/}~~~~ \word{p} n1 n2 -{}- n3 ) ~~~~~
			\word{/MOD} \word{SWAP} \word{DROP}  \word{;}} \\
		\sout{\word{:} \word{*/MOD} \word{p} n1 n2 n3 -{}- n4 n5 )
			\word{toR} \word{M*} \word{Rfrom} \word{FM/MOD} \word{;}} \\
		\sout{\word{:} \word{*/}~~~ \word{p} n1 n2 n3 -{}- n4 n5 )
			\word{*/MOD} \word{SWAP} \word{DROP}  \word{;}} \\
	\end{quote}
\end{description}


\subsection[Immediacy]{\sout{Immediacy}} % D.6.7
\label{diff:immediate}

\sout{%
Forth 83 specified that a number of ``compiling words'' are
``immediate'', meaning that they are executed instead of compiled
during compilation. ANS Forth is less specific about most of these
words, stating that their behavior is only defined during compilation,
and specifying their results rather than their specific compile-time
actions.}

\sout{%
To force the compilation of a word that would normally be executed,
Forth 83 provided the words \linebreak \texttt{COMPILE}, used with non-immediate
words, and \word{[COMPILE]}, used with immediate words. ANS Forth
provides the single word \word{POSTPONE}, which is used with both
immediate and non-immediate words, automatically selecting the
appropriate behavior.}

\begin{description}
\item[\sout{Words Affected:}]
\sout{%
	\texttt{COMPILE}	\word{[COMPILE]}
	\word{[']}			\word{'}}

\item[\sout{Reason:}]
\sout{%
	The designation of particular words as either immediate or not
	depends upon the implementation technique chosen for the Forth
	system. With traditional ``threaded code'' implementations, the
	choice was generally quite clear (with the single exception of
	the word \word{LEAVE}), and the standard could specify which words
	should be immediate. However, some of the currently popular
	implementation techniques, such as native-code generation with
	optimization, require the immediacy attribute on a different set
	of words than the set of immediate words of a threaded code
	implementation. ANS Forth, acknowledging the validity of these
	other implementation techniques, specifies the immediacy attribute
	in as few cases as possible.}

\sout{%
	When the membership of the set of immediate words is unclear, the
	decision about whether to use \texttt{COMPILE} or \word{[COMPILE]}
	becomes unclear. Consequently, ANS Forth provides a ``general
	purpose'' replacement word \word{POSTPONE} that serves the purpose
	of the vast majority of uses of both \texttt{COMPILE} and
	\word{[COMPILE]}, without requiring that the user know whether or
	not the ``postponed'' word is immediate.}

\sout{%
	Similarly, the use of \word{'} and \word{[']} with compiling words
	is unclear if the precise compilation behavior of those words is
	not specified, so ANS Forth does not permit a Standard Program to
	use \word{'} or \word{[']} with compiling words.}

\sout{%
	The traditional (non-immediate) definition of the word \texttt{COMPILE}
	has an additional problem. Its traditional definition assumes a
	threaded code implementation technique, and its behavior can only
	be properly described in that context. In the context of ANS Forth,
	which permits other implementation techniques in addition to
	threaded code, it is very difficult, if not impossible, to describe
	the behavior of the traditional \texttt{COMPILE}. Rather than changing
	its behavior, and thus breaking existing code, ANS Forth does not
	include the word \texttt{COMPILE}. This allows existing implementations
	to continue to supply the word \texttt{COMPILE} with its traditional
	behavior, if that is appropriate for the implementation.}

\item[\sout{Impact:}]
\sout{%
	\word{[COMPILE]} remains in ANS Forth, since its proper use does
	not depend on knowledge of whether or not a word is immediate (Use
	of \word{[COMPILE]} with a non-immediate word is and has always
	been a no-op). Whether or not you need to use \word{[COMPILE]}
	requires knowledge of whether or not its target word is immediate,
	but it is always safe to use \word{[COMPILE]}. \word{[COMPILE]}
	is no longer in the (required) core word set, having been moved
	to the Core Extensions word set, but the committee anticipates
	that most vendors will supply it anyway.}

\sout{%
	In nearly all cases, it is correct to replace both \word{[COMPILE]}
	and \texttt{COMPILE} with \word{POSTPONE}. Uses of \word{[COMPILE]}
	and \texttt{COMPILE} that are not suitable for ``mindless'' replacement
	by \word{POSTPONE} are quite infrequent, and fall into the following
	two categories:}

	\begin{itemize}
	\item \sout{%
Use of \word{[COMPILE]} with non-immediate words. This is
		sometimes done with the words \word{'} (tick, which was
		immediate in Forth 79 but not in Forth 83) and \word{LEAVE}
		(which was immediate in Forth 83 but not in Forth 79), in
		order to force the compilation of those words without regard
		to whether you are using a Forth 79 or Forth 83 system.}

	\item \sout{%
Use of the phrase \texttt{COMPILE} \word{[COMPILE]}
		\arg{immediate word} to ``doubly postpone'' an immediate word.}
	\end{itemize}

\item[\sout{Transition/Conversion:}]
\sout{%
	Many ANS Forth implementations will continue to implement both
	\word{[COMPILE]} and \texttt{COMPILE} in forms compatible with
	existing usage. In those environments, no conversion is necessary.}

\sout{%
	For complete portability, uses of \texttt{COMPILE} and \word{[COMPILE]}
	should be changed to \word{POSTPONE}, except in the rare cases
	indicated above. Uses of \word{[COMPILE]} with non-immediate words
	may be left as-is, and the program may declare a requirement for
	the word \word{[COMPILE]} from the Core Extensions word set, or
	the \word{[COMPILE]} before the non-immediate word may be simply
	deleted if the target word is known to be non-immediate.}

\sout{%
	Uses of the phrase \texttt{COMPILE} \word{[COMPILE]}
	\arg{immediate-word} may be handled by introducing an
	``intermediate word'' (\texttt{XX} in the example below) and then
	postponing that word. For example:}
	\begin{quote}\ttfamily
		\sout{\word{:} ABC COMPILE \word{[COMPILE]} \word{IF} \word{;}}
	\end{quote}
\sout{changes to:}
	\begin{quote}\ttfamily
		\sout{\word{:} XX \word{POSTPONE} \word{IF} \word{;}} \\
		\sout{\word{:} ABC \word{POSTPONE} XX \word{;}}
	\end{quote}
\sout{%
	A non-standard case can occur with programs that ``switch out of
	compilation state'' to explicitly compile a thread in the
	dictionary following a \texttt{COMPILE}. For example:}
	\begin{quote}\ttfamily
		\sout{\word{:} XYZ COMPILE \word{[} \word{'} ABC \word{,} \word{]} \word{;}}
	\end{quote}
\sout{%
	This depends heavily on knowledge of exactly how \texttt{COMPILE}
	and the threaded-code implementation works. Cases like this cannot
	be handled mechanically; they must be translated by understanding
	exactly what the code is doing, and rewriting that section according
	to ANS Forth restrictions.}

\sout{%
	Use the phrase \word{POSTPONE} \word{[COMPILE]} to replace
	\word{[COMPILE]} \word{[COMPILE]}.}
\end{description}


\subsection[Input character set]{\sout{Input character set}} % D.6.8

\sout{%
Forth 83 specifies that the full 7-bit ASCII character set is
available through \word{KEY}. ANS Forth restricts it to the graphic
characters of the ASCII set, with codes from hex 20 to hex 7E
inclusive.}

\begin{description}
\item[\sout{Words Affected:}]
\sout{\word{KEY}}

\item[\sout{Reason:}]
\sout{%
	Many system environments ``consume'' certain control characters
	for such purposes as input editing, job control, or flow control.
	A Forth implementation cannot always control this system behavior.}

\item[\sout{Impact:}]
\sout{%
	Standard Programs which require the ability to receive particular
	control characters through \word{KEY} must declare an environmental
	dependency on the input character set.}

\item[\sout{Transition/Conversion:}]
\sout{%
	For maximum portability, programs should restrict their required
	input character set to only the graphic characters. Control
	characters may be handled if available, but complete program
	functionality should be accessible using only graphic characters.}

\sout{%
	As stated above, an environmental dependency on the input character
	set may be declared. Even so, it is recommended that the program
	should avoid the requirement for particularly-troublesome control
	characters, such as control-S and control-Q (often used for flow
	control, sometimes by communication hardware whose presence may be
	difficult to detect), ASCII NUL (difficult to type on many keyboards),
	and the distinction between carriage return and line feed (some
	systems translate carriage returns into line feeds, or vice versa).}
\end{description}


\subsection[Shifting with UM/MOD]{\sout{Shifting with \word{UM/MOD}}} % D.6.9

\sout{%
Given Forth-83's two's-complement nature, and its requirement for
floored (round toward minus infinity) division, shifting is equivalent
to division. Also, two's-complement representation implies that
unsigned division by a power of two is equivalent to logical
right-shifting, so \word{UM/MOD} could be used to perform a logical
right-shift.}

\begin{description}
\item[\sout{Words Affected:}]
\sout{\word{UM/MOD}}

\item[\sout{Reason:}]
\sout{%
	The problem with \word{UM/MOD} is a result of allowing
	non-two's-complement number representations, as already
	described.}

\sout{%
	ANS Forth provides the words \word{LSHIFT} and \word{RSHIFT}
	to perform logical shifts. This is usually more efficient, and
	certainly more descriptive, than the use of \word{UM/MOD} for
	logical shifting.}

\item[\sout{Impact:}]
\sout{%
	Programs running on ANS Forth systems with two's-complement
	arithmetic (the majority of machines), will not experience any
	incompatibility with \word{UM/MOD}. Existing Forth-83 Standard
	programs intended to run on non-two's-complement machines will
	not be able to use \word{UM/MOD} for shifting on a
	non-two's-complement ANS Forth system. This should not affect
	a significant number of existing programs (perhaps none at all),
	since the committee knows of no existing Forth-83 implementations
	on non-two's-complement machines.}

\item[\sout{Transition/Conversion:}]
\sout{%
	A program that requires \word{UM/MOD} to behave as a shift
	operation may declare an environmental dependency on
	two's-complement arithmetic.}

\sout{%
	A program that cannot declare an environmental dependency on
	two's-complement arithmetic may require editing to replace
	incompatible uses of \word{UM/MOD} with other operators defined
	within the application.}
\end{description}

\subsection[Vocabularies / wordlists]{\sout{Vocabularies / wordlists}} % D.6.10

\sout{%
ANS Forth does not define the words \texttt{VOCABULARY},
\texttt{CONTEXT}, and \texttt{CURRENT}, which were present in
Forth 83. Instead, ANS Forth defines a primitive word set for
search order specification and control, including words which have
not existed in any previous standard.}

\sout{%
Forth-83's ``\word[search]{ALSO}/\word[search]{ONLY}'' experimental
search order word set is specified for the most part as the extension
portion of the ANS Forth Search Order word set.}

\begin{description}
\item[\sout{Words Affected:}]
\sout{\texttt{VOCABULARY}	\texttt{ CONTEXT}	\texttt{CURRENT}}

\item[\sout{Reason:}]
\sout{%
	Vocabularies are an area of much divergence among existing systems.
	Considering major ven\-dors' systems and previous standards, there
	are at least 5 different and mutually incompatible behaviors of
	words defined by \texttt{VOCABULARY}. Forth 83 took a step in the
	direction of ``run-time search-order specification'' by declining
	to specify a specific relationship between the hierarchy of
	compiled vocabularies and the run-time search order. Forth 83 also
	specified an experimental mechanism for run-time search-order
	specification, the \word[search]{ALSO}/\word[search]{ONLY} scheme.
	\word[search]{ALSO}/\word[search]{ONLY} was implemented in numerous
	systems, and has achieved some measure of popularity in the Forth
	community.}

\sout{%
	However, several vendors refuse to implement it, citing technical
	limitations. In an effort to address those limitations and thus
	hopefully make \word[search]{ALSO}/\word[search]{ONLY} more
	palatable to its critics, the committee specified a simple
	``primitive word set'' that not only fixes some of the objections
	to \word[search]{ALSO}/\word[search]{ONLY}, but also provides
	sufficient power to implement \word[search]{ALSO}/\word[search]{ONLY}
	and all of the other search-order word sets that are currently
	popular.}

\sout{%
	The Forth 83 \word[search]{ALSO}/\word[search]{ONLY} word set is
	provided as an optional extension to the search-order word set.
	This allows implementors that are so inclined to provide this
	word set, with well-defined standard behavior, but does not
	compel implementors to do so. Some vendors have publicly stated
	that they will not implement \word[search]{ALSO}/\word[search]{ONLY},
	no matter what, and one major vendor stated an unwillingness to
	implement ANS Forth at all if \word[search]{ALSO}/\word[search]{ONLY}
	is mandated. The committee feels that its actions are prudent,
	specifying \word[search]{ALSO}/\word[search]{ONLY} to the extent
	possible without mandating its inclusion in all systems, and also
	providing a primitive search-order word set that vendors may be
	more likely to implement, and which can be used to synthesize
	\word[search]{ALSO}/\word[search]{ONLY}.}

\item[\sout{Transition/Conversion:}]
\sout{%
	Since Forth 83 did not mandate precise semantics for \texttt{VOCABULARY},
	existing Forth-83 Standard programs cannot use it except in a
	trivial way. Programs can declare a dependency on the existence
	of the Search Order word set, and can implement whatever semantics
	are required using that word set's primitives. Forth 83 programs
	that need \word[search]{ALSO}/\word[search]{ONLY} can declare a
	dependency on the Search Order Extensions word set, or can implement
	the extensions in terms of the Search Order word set itself.}
\end{description}


\subsection[Multiprogramming impact]{\sout{Multiprogramming impact}} % D.6.11
\label{diff:multitasking}

\sout{%
Forth 83 marked words with ``multiprogramming impact'' by the letter
``M'' in the first lines of their descriptions. ANS Forth has removed
the ``M'' designation from the word descriptions, moving the dis\-cus\-sion
of multiprogramming impact to this non-normative annex.}

\begin{description}
\item[\sout{Words affected:}]
\sout{none}

\item[\sout{Reason:}]
\sout{%
	The meaning of ``multiprogramming impact'' is precise only in the
	context of a specific model for multiprogramming. Although many
	Forth systems do provide multiprogramming capabilities using a
	particular round-robin, cooperative, block-buffer sharing model,
	that model is not universal. Even assuming the classical model,
	the ``M'' designations did not contain enough information to
	enable writing of applications that interacted in a multiprogrammed
	system.}

\sout{%
	Practically speaking, the ``M'' designations in Forth 83 served
	to document usage rules for block buffer addresses in multiprogrammed
	systems. These addresses often become meaningless after a task
	has relinquished the CPU for any reason, most often for the
	purposes of performing I/O, awaiting an event, or voluntarily
	sharing CPU resources using the word \texttt{PAUSE}. It was
	essential that portable applications respect those usage rules to
	make it practical to run them on multiprogrammed systems; failure
	to adhere to the rules could easily compromise the integrity of
	other applications running on those systems as well as the
	applications actually in error. Thus, ``M'' appeared on all words
	that by design gave up the CPU, with the understanding that other
	words NEVER gave it up.}

\sout{%
	These usage rules have been explicitly documented in the Block
	word set where they are relevant. The ``M'' designations have
	been removed entirely.}

\item[\sout{Impact:}]
\sout{In practice, none.}

\sout{%
	In the sense that any application that depends on multiprogramming
	must consist of at least two tasks that share some resource(s) and
	communicate between themselves, Forth 83 did not contain enough
	information to enable writing of a standard program that DEPENDED
	on multiprogramming. This is also true of ANS Forth.}

\sout{%
	Non-multiprogrammed applications in Forth 83 were required to
	respect usage rules for \word[block]{BLOCK} so that they could be
	run properly on multiprogrammed systems. The same is true of
	ANS Forth.}

\sout{%
	The only difference is the documentation method used to define
	the \word[block]{BLOCK} usage rules. The Technical Committee
	believes that the current method is clearer than the concept of
	``multi\-pro\-gram\-ming impact''.}

\item[\sout{Transition/Conversion:}]
	\sout{none needed.}
\end{description}


\subsection[Words not provided in executable form]{\sout{Words not provided in executable form}} % D.6.12

\sout{%
ANS Forth allows an implementation to supply some words in source
code or ``load as needed'' form, rather than requiring all supplied
words to be available with no additional programmer action.}

\begin{description}
\item[\sout{Words affected:}]
\sout{all}

\item[\sout{Reason:}]
\sout{
	Forth systems are often used in environments where memory space
	is at a premium. Every word included in the system in executable
	form consumes memory space. The committee believes that allowing
	standard words to be provided in source form will increase the
	probability that implementors will provide complete ANS Forth
	implementations even in systems designed for use in constrained
	environments.}

\item[\sout{Impact:}]
\sout{%
	In order to use a Standard Program with a given ANS Forth
	implementation, it may be necessary to precede the program with
	an implementation-dependent ``preface'' to make ``source form''
	words executable. This is similar to the methods that other
	computer languages require for selecting the library routines
	needed by a particular application.}

\sout{%
	In languages like C, the goal of eliminating unnecessary routines
	from the memory image of an application is usually accomplished
	by providing libraries of routines, using a ``linker'' program
	to incorporate only the necessary routines into an executable
	application. The method of invoking and controlling the linker
	is outside the scope of the language definition.}

\item[\sout{Transition/Conversion:}]
\sout{%
	Before compiling a program, the programmer may need to perform
	some action to make the words required by that program available
	for execution.}
\end{description}

% =========================================================
\cbend

\newpage
\cbstart\patch{ed12}
\section{Differences from Forth 94} % D.6
\label{diff:forth94}

\subsection[Removed Obsolete Words]{Removed \sout{Definitions} \uline{Obsolete Words}} % D.7.1

	This standard removes six words that were marked `obsolescent'
	in the \replace{ed12}{ANS Forth}{Forth 94} document.  These are:

\begin{tabular}{rl@{\qquad}rl@{\qquad}rl}
	6.2.0060	& \textbf{\#TIB}
&	6.2.1390	& \textbf{EXPECT}
&	6.2.2240	& \textbf{SPAN} \\
	6.2.0970	& \textbf{CONVERT}
&	6.2.2040	& \textbf{QUERY}
&	6.2.2290	& \textbf{TIB} \\
\end{tabular}

\remove{ed12}{Reuse of these names is strongly discouraged.}

\subsection{Obsolescent features}
\label{diff:94:obsolete}

	This standard has designated the following practice as obsolescent:

\begin{itemize}
\item Requiring floating-point numbers to be kept on the data stack.
	(This has always been an environmental dependency.)

\item Using \word{ENVIRONMENTq} to enquire whether a word set is present.
\end{itemize}

% \uline{Forth 94 declared seven words as `obsolescent', all but \word[tools]{FORGET} have
% been removed from this standard.}
%  
% \begin{description}
% \item[\uline{Words affected:}]
% \uline{\texttt{\#TIB},
%  	\texttt{CONVERT},
%  	\texttt{EXPECT},
%  	\texttt{QUERY},
%  	\texttt{SPAN},
%  	\texttt{TIB},
%  	\word{WORD}.}
%  
% \item[\uline{Reason:}]
% \uline{Obsolescent words have been removed.}
%  
% \item[\uline{Impact:}]
% \uline{\word{WORD} is no longer required to leave a space at the
%  	end of the returned string.}
%  
% \item[\uline{Transition/Conversion:}]
% \uline{The function of \texttt{TIB} and \texttt{\#TIB} has been
%  	superseded by \word{SOURCE}.}
% 
% \uline{The function of \texttt{CONVERT} has been superseded by
% 	\word{toNUMBER}.}
% 
% \uline{The function of \texttt{EXPECT} and \texttt{SPAN} have
% 	been superseded by \word{ACCEPT}.}
% 
% \uline{The function of \texttt{QUERY} may be performed with
% 	\word{ACCEPT} and \word{EVALUATE}.}
% \end{description}
% 
% 
% \subsection{Combined Floating-point/Data Stack Obsolescent} % D.7.2
% 
% The requirement for floating-point numbers to be kept on the data stack
% has been marked as obsolescent.  This was previously an environmental
% dependency/restriction.
% 
% \begin{description}
% \item[Words Affected:]
% 	All floating-point words.
% 
% \item[Reason:]
% 	The developing of software that may be used with either a combined
% 	stack or a separate stack is extremely difficult and costly.  While
% 	some of the systems surveyed provide a combined floating-point/data
% 	stack, they all provide a separate floating-point stack.
% 
% \item[Impact:]
% %	Any code that has an environmental dependency on the uses of a
% %	combined floating-point/data stack will become obsolescent and should
% %	be ported to use a separate floating-point stack.
% %
% %	A system that has an environmental restriction on using a combined
% %	floating-point/data stack will become obsolescent.  Such systems
% %	should consider providing a separate floating-point stack, simulating
% %	a floating-point stack if required.
% 
% 	Forth 94 programs with an environmental dependency on a separate
% 	floating-point stack become standard programs.
% 
% 	Forth 94 programs with an environmental dependency on a combined
% 	stack retain the environmental dependency.
% 
% 	Forth 94 programs (without environmental dependency, i.e., those
% 	working on either kind of system) remain standard programs.
% 
% 	Forth 94 systems that implement a separate floating-point stack
% 	remain standard systems.
% 
% 	Forth 94 systems that implement a combined stack become systems
% 	with an environmental restriction of not providing a separate
% 	floating-point stack, but a combined stack.
% \end{description}
% 
% The combined floating-point/data stack environmental restriction/dependency
% is obsolescent and will be removed from a future standard.
% 
% 
% \subsection{Using \word{ENVIRONMENT?} to inquire whether a word set is present.} % D.7.3
% 
% With the advent of a new standard, it was necessary to review the meaning
% of word set queries.  Compatibility with Forth 94 demands that a word set
% query produce the same result as for Forth 94; i.e., querying for
% \texttt{CORE-EXT} returns true only if all the Forth 94 CORE EXT words
% are present.  The question was how to distinguish between word sets
% described by this and subsequent standards.
% 
% The committee considered adding a year indicator to the word set name
% (``\texttt{CORE-EXT-\snapshot}'') or a providing a general query
% (``\texttt{Forth-\snapshot}'') which could be combined with the
% word set query.  As the committee could find very few examples of the
% word set queries being used, it chose not to update the word set query
% mechanism, but rather to mark it as obsolescent.
% 
% \begin{description}
% \item[Words Affected:]
% 	\word{ENVIRONMENT?}
% 
% \item[Reason:]
% 	The use of the word set query to inquire whether a word set is present
% 	in the system has been marked obsolescent.  If present the query
% 	indicates the word set, as documented in Forth 94, is available.
% 
% \item[Impact:]
% 	Forth 94 did not guarantee the presence of these queries.  Many
% 	systems that provided all the words in a particular word set did not
% 	provide the corresponding query.
% 
% \item[Transition/Conversion:]
% 	There is no equivalent for detecting the presence of a whole word set.
% 	The \wref{tools:[DEFINED]}{} and \wref{tools:[UNDEFINED]}{} words
% 	can be used to detect the availability (or otherwise) of individual words.
% \end{description}
% 
% 
% \subsection{Additional \word{TO} targets} % D.7.4
% 
% The definition of \wref{core:TO}{TO} has been extended to any word
% defined with a user defining word that includes a ``\word{TO}
% \param{name} run-time'' semantics in its definition.
% 
% \begin{description}
% \item[Words affected:]
% 	\word{VALUE},
% 	\word[local]{LOCAL},
% 	\word{TO}.
% 
% \item[Reason:]
% 	The Forth 94 definition of \word{TO} required each new class of
% 	target to redefine \word{TO}.  As \word{TO} is an optional word
% 	this would require $2^n$ new definitions for each new target,
% 	where $n$ is the number of targets.
% 
% 	Using this approach this standard would require 16 different
% 	definitions of \word{TO} to support the four different classes
% 	of target (\word[local]{LOCAL}, \word{VALUE},
% 	\word[floating]{FVALUE}, \word[double]{2VALUE}),
% 	depending on which optional word sets are present.
% 
% 	By redefining \word{TO} in a more general manner, only the one
% 	definition is necessary.  However, it does require any word that
% 	defines a new class of target, to include a
% 	``\word{TO} \param{name} run-time'' semantics definition.
% 
% \item[Impact:]
% 	The Forth 94 definition of \word{TO} required it to parse the input
% 	for a target.  It was required to differentiate between a target
% 	defined with \word{VALUE} or \word[local]{LOCAL}.  The revised definition
% 	requires a system to implement \word{TO} in such a way that
% 	additional types of target can be identified.
% 
% \item[Transition/Conversion:]
% 	None.
% \end{description}
% 
% 
% \subsection{Additional Input/Output Return values} % D.7.5
% 
% All words that return an \param{ior} have a value assigned in
% the \word[exception]{THROW} code table (\xref[9.2 Throw Codes]{table:throw}).
% 
% \begin{description}
% \item[Words affected:]
% 	All words that return an \param{ior} code:\\
% 	\setwordlist{file}
% 	\word[memory]{ALLOCATE},
% 	\word{CLOSE-FILE},
% 	\word{CREATE-FILE},
% 	\word{DELETE-FILE},
% 	\word{FILE-POSITION},
% 	\word{FILE-SIZE},
% 	\word{FILE-STATUS},
% 	\word{FLUSH-FILE},
% 	\word[memory]{FREE},
% 	\word{OPEN-FILE},
% 	\word{READ-FILE},
% 	\word{READ-LINE}, \linebreak
% 	\word{RENAME-FILE},
% 	\word{REPOSITION-FILE},
% 	\word{RESIZE-FILE},
% 	\word[memory]{RESIZE},
% 	\word{WRITE-FILE},
% 	\word{WRITE-LINE}.
% 
% \item[Reason:]
% 	Forth 94 left the error code (\param{ior}) returned by these words
% 	implementation-defined.  This meant that an application had no
% 	entitlement to use the \param{ior} value as a \word[exception]{THROW} code.
% 	This would lead to very clumsy code of the form:
% 	\begin{quote}
% 		\word[memory]{ALLOCATE} ~ \word{IF} \emph{n} \word[exception]{THROW} \word{THEN}
% 	\end{quote} 
% 
% 	However, it was common to see incorrect code, such as:
% 	\begin{quote}
% 		\word[memory]{ALLOCATE} ~ \word[exception]{THROW}
% 	\end{quote}
% 	which could have lead to silent aborts, when the \param{ior} was
% 	-1, or even an incorrect error message.
% 
% 	By returning a specific \param{ior} value, throwing the value
% 	in this manner is now acceptable.  This provides a method of
% 	requesting the error be processed further up the call hierarchy.
% 
% \item[Impact:]
% 	It may be necessary to convert operating system error codes
% 	into throw codes or suitable application error codes.
% \end{description}
% 
% Systems that provide a more detailed \param{ior} error code should
% continue to do so.  The new \param{ior} codes are within the range
% reserved for standard codes, this should not effect any existing
% \param{ior} codes.
% 
% 
% \section{Additions to the standard}
% \label{diff:2012}
% 
% \setwordlist{facility}
% \subsection[EKEY return value]{\word{EKEY} return value.} % D.7.6
% 
% Forth 94 did not provide any means of accessing the non-graphic keys,
% this was intentional as access to such keys represents an environmental
% dependency.  However, it did provide the word \word{EKEY}, the return
% value of which was left implementation-dependent to allow access to
% such additional keys.
% 
% It became clear that a portable way of accessing the non-graphic keys was
% required.
% 
% \begin{description}
% \item[Words Affected:]
% 	\word{EKEY}
% 
% \item[Reason:]
% 	To provide a standard method of accessing the implementation defined
% 	value returned by \word{EKEY}.  While it does not affect the value
% 	returned by \word{EKEY} directly, another word (\word{EKEYtoFKEY})
% 	is required to convert the implement-dependent value into a value
% 	that can be interrogated in a standard manner.
% 
% 	The words given in table \ref{table:fkeys} produce implementation-defined
% 	constant values that may be compared against the value returned by
% 	\word{EKEYtoFKEY} to identify the key press.  These values can be
% 	combined (with an \word{OR}) to the mask values retuned by the
% 	words in table \ref{table:fmasks} to identify the corresponding
% 	key combination.
% 
% \begin{table}[!ht]
% 	\centering
% 	\begin{minipage}[t]{21em}
% 		\caption{Key Constants}
% 		\label{table:fkeys}
% 		\begin{tabular}{ll@{\qquad}ll}\hline\hline
% 			Word & Key
% 			& \multicolumn{1}{c}{Word}
% 			& \multicolumn{1}{c}{Key}
% 			\\\hline
% 			\multicolumn{2}{}{} \\[-2ex]
% \word{K-F1}		& F1	& \word{K-LEFT}		& cursor left \\
% \word{K-F2}		& F2	& \word{K-RIGHT}	& cursor right \\
% \word{K-F3}		& F3	& \word{K-UP}		& cursor up \\
% \word{K-F4}		& F4	& \word{K-DOWN}		& cursor down \\
% \word{K-F5}		& F5	& \word{K-HOME}		& home \emph{or} Pos1 \\
% \word{K-F6}		& F6	& \word{K-END}		& End \\
% \word{K-F7}		& F7	& \word{K-PRIOR}	& PgUp \emph{or} Prior \\
% \word{K-F8}		& F8	& \word{K-NEXT}		& PgDn \emph{or} Next \\
% \word{K-F9}		& F9	& \word{K-INSERT}	& Insert \\
% \word{K-F10}	& F10 	& \word{K-DELETE}	& Delete \\
% \word{K-F11}	& F11	\\
% \word{K-F12}	& F12	\\\hline\hline
% 		\end{tabular}
% 	\end{minipage}
% \qquad
% 	\begin{minipage}[t]{17em}
% 		\caption{Shift Masks}
% 		\label{table:fmasks}
% 		\begin{tabular}{ll}\hline\hline
% 		\multicolumn{1}{c}{Word} & Shift Key \\ \hline
% \word{K-SHIFT-MASK} & Shift \\
% \word{K-CTRL-MASK} 	& Control \\
% \word{K-ALT-MASK} 	& Alternate \emph{or} Meta \\ \hline\hline
% 		\end{tabular}
% 	\end{minipage}
% \end{table}
% 
% \pagebreak[1]
% 	Typical Use:
% 	\begin{tt}
% 		\begin{tabbing}
% \tab \= k-left ~ k-shift-mask or k-ctrl-mask or ~ \= of ... endof\kill
% \ldots{} \word{EKEY} \word{EKEYtoFKEY} \word{IF} \\
% \> \word{CASE} \+ \\
% 	\word{K-UP}	\> \word{OF} \ldots{} \word{ENDOF} \\
% 	\word{K-F1}	\> \word{OF} \ldots{} \word{ENDOF} \\
% 	\word{K-LEFT} ~ \word{K-SHIFT-MASK} \word{OR}
% 		\word{K-CTRL-MASK} \word{OR} \> \word{OF} \ldots{} \word{ENDOF} \\
%  	\ldots{} \\
%  \word{ENDCASE} \- \\
% \word{ELSE} \\
% \> \ldots{} \\
% \word{THEN} \\
% 		\end{tabbing}
% 	\end{tt}
% 
% \item[Transition/Conversion:]
% 	A keyboard may lack some of the keys, or the system the capability
% 	to report them.  A portable program should be written such that it
% 	will also work (although less conveniently or with less functionality)
% 	if these key numbers cannot be produced.
% 
% 	The shift key masks are only useful for recognizing shifted cursor
% 	and function keys, as these are the only keys defined in table
% 	\ref{table:fkeys}. E.g., we cannot recognize \textsf{ALT-T}, as no
% 	return value for \textsf{T} has been defined.
% \end{description}
% 
% Introducing \word{EKEYtoFKEY} leaves implementors free to use the value
% returned by \word{EKEY} in an {im\-ple\-ment\-ation-de\-pend\-ent} manner.
% Many systems return a code which is suitable for comparison, a simple
% implementation of \word{EKEYtoFKEY} on such a system would be:
% 
% \begin{quote}\ttfamily
% \word{:}\ \word{EKEYtoFKEY} \word{p} u1 -{}- u2 flag ) \\
% \tab \word{DUP} \word{EKEYtoCHAR} \word{NIP} \word{0=} \word{;}
% \end{quote}
% 
% =========================================================
\cbend

\vfill
\begin{editor}
This is a list of proposals accepted into the Forth 200\emph{x} document
since we began work in 2005.

\begin{description}
\item[Wording only] ~\\
These proposals changed wording to clarify the text, but did not actually
change the meaning of the document:

\begin{tabular}{*6l}
X:namelen	&% Definition Name Length
X:to 		& % Separation of TO from VALUE and (LOCAL) \\
			% Wording change / documentation structure change 
x:ekey		% Ekey Event Record only 1-3 accepted; also delete last paragraph of Rationale
\end{tabular}
~

\item[New Words] ~\\
These proposals added new words to the document.  These changes are
catered for in by section \xref{diff:new12}:

\begin{tabular}{*6l}
X:ftrunc			&%	Floating point truncation
x:buffer			&%	BUFFER:
x:substitute		&%	Text Substitution
X:xchar			&%	XCHAR wordset
X:synonym		&%	Synonym
X:escaped-strings	\\% Escaped Strings
X:n-to-r 			&% N>R and NR>
x:structures		&%	Structures wording
X:2value			&%	2VALUE
X:fvalue			&%	FVALUE
X:structures		&%	Structures
X:required		\\%	One-time file loading
X:ekeys			&%	EKEY return values
X:defined		&%	[DEFINED] and [UNDEFINED]
X:parse-name	&%	PARSE-NAME
X:deferred		\\%	Deferred words
\end{tabular}
~

\item[Normative Changes] ~\\
These proposals have normative changes that require documentation:

\begin{itemize}
\item x:enhanced-locals --- Minimum number of locals now sixteen
\item x:block --- Multi-tasking Block Buffers
\item x:source-id -- FILE-POSITION during interpretation
\item X:fasinh --- Removed restriction from FASINH
\item X:fatan2 --- Ambiguity in FATAN2
\item x:fp-stack --- Separate FP Stack
\item X:key-ekey --- Chars now unsigned / Added primitive characters
\item X:number-prefix	
\item X:throw-iors --- Lots of new throw codes
\item x:legacy
\item x:wordset-query
\end{itemize}

Descriptions of these changes and their effect on existing programs
need to be included.
\end{description}
\end{editor}

\vfill
\newpage

\cbstart\patch{ed12}
\section[Additional words]{\uline{Additional words}}
\label{diff:new12}

\uline{The following words have been added to the standard:}

\setcounter{subsection}{5}
\subsection[Core word sets]{\uline{Core word sets}}
\uline{The following words have been added to \xref{wordlist:core-ext}:}

\begin{minipage}[t]{0.3\linewidth}
\uline{\wref{core:ACTION-OF}{}} \\
\uline{\wref{core:BUFFER:}{}} \\
\uline{\wref{core:DEFER}{}}
\end{minipage}
\hfill
\begin{minipage}[t]{0.3\linewidth}
\uline{\wref{core:DEFER!}{}} \\
\uline{\wref{core:DEFER@}{}} \\
\uline{\wref{core:HOLDS}{}}
\end{minipage}
\hfill
\begin{minipage}[t]{0.3\linewidth}
\uline{\wref{core:IS}{}} \\
\uline{\wref{core:PARSE-NAME}{}} \\
\uline{\wref{core:Seq}{}}
\end{minipage}


\subsection[{Block word sets}]{\uline{Block word sets}}
\uline{No words have been added to \xref{wordlist:block}.}

\subsection[{Double-Number word sets}]{\uline{Double-Number word sets}}
\uline{The following words have been added to \xref{wordlist:double-ext}:}

\uline{\wref{double:2VALUE}{}}

\subsection[{Exception word sets}]{\uline{Exception word sets}}
\uline{No words have been added to \xref{wordlist:exception}.}

\subsection[{Facility word sets}]{\uline{Facility word sets}}
\uline{The following words have been added to \xref{wordlist:facility-ext}:}

\begin{minipage}[t]{0.35\linewidth}
\uline{\wref{facility:+FIELD}{}} \\
\uline{\wref{facility:BEGIN-STRUCTURE}{}} \\
\uline{\wref{facility:CFIELD:}{}} \\
\uline{\wref{facility:EKEYtoFKEY}{}} \\
\uline{\wref{facility:END-STRUCTURE}{}} \\
\uline{\wref{facility:FIELD:}{}} \\
\uline{\wref{facility:K-ALT-MASK}{}} \\
\uline{\wref{facility:K-CTRL-MASK}{}} \\
\uline{\wref{facility:K-DELETE}{}} \\
\uline{\wref{facility:K-DOWN}{}} \\
\uline{\wref{facility:K-END}{}}
\end{minipage}
\hfill
\begin{minipage}[t]{0.25\linewidth}
\uline{\wref{facility:K-F1}{}} \\
\uline{\wref{facility:K-F10}{}} \\
\uline{\wref{facility:K-F11}{}} \\
\uline{\wref{facility:K-F12}{}} \\
\uline{\wref{facility:K-F2}{}} \\
\uline{\wref{facility:K-F3}{}} \\
\uline{\wref{facility:K-F4}{}} \\
\uline{\wref{facility:K-F5}{}} \\
\uline{\wref{facility:K-F6}{}} \\
\uline{\wref{facility:K-F7}{}} \\
\uline{\wref{facility:K-F8}{}}
\end{minipage}
\hfill
\begin{minipage}[t]{0.35\linewidth}
\uline{\wref{facility:K-F9}{}} \\
\uline{\wref{facility:K-HOME}{}} \\
\uline{\wref{facility:K-INSERT}{}} \\
\uline{\wref{facility:K-LEFT}{}} \\
\uline{\wref{facility:K-NEXT}{}} \\
\uline{\wref{facility:K-PRIOR}{}} \\
\uline{\wref{facility:K-RIGHT}{}} \\
\uline{\wref{facility:K-SHIFT-MASK}{}} \\
\uline{\wref{facility:K-UP}{}}
\end{minipage}

\subsection[{File-Access word sets}]{\uline{File-Access word sets}}
\uline{The following words have been added to \xref{wordlist:file-ext}:}

\begin{minipage}[t]{0.3\linewidth}
\uline{\wref{file:INCLUDE}{}}
\end{minipage}
\hfill
\begin{minipage}[t]{0.3\linewidth}
\uline{\wref{file:REQUIRE}{}}
\end{minipage}
\hfill
\begin{minipage}[t]{0.3\linewidth}
\uline{\wref{file:REQUIRED}{}}
\end{minipage}

\subsection[{Floating-Point word sets}]{\uline{Floating-Point word sets}}
\uline{The following words have been added to \xref{wordlist:floating-ext}:}

\begin{minipage}[t]{0.3\linewidth}
\uline{\wref{floating:DFFIELD:}{}} \\
\uline{\wref{floating:FtoS}{F>S}} \\
\uline{\wref{floating:FFIELD:}{}}
\end{minipage}
\hfill
\begin{minipage}[t]{0.3\linewidth}
\uline{\wref{floating:FTRUNC}{}} \\
\uline{\wref{floating:FVALUE}{}}
\end{minipage}
\hfill
\begin{minipage}[t]{0.3\linewidth}
\uline{\wref{floating:StoF}{S>F}} \\
\uline{\wref{floating:SFFIELD:}{}}
\end{minipage}

\subsection[{Locals word sets}]{\uline{Locals word sets}}
\uline{The following words have been added to \xref{wordlist:local-ext}:}

\uline{\wref{local:b:}{}}

\subsection[{Memory-Allocation word sets}]{\uline{Memory-Allocation word sets}}
\uline{No words have been added to \xref{wordlist:memory}.}

\subsection[{Programming-Tools word sets}]{\uline{Programming-Tools word sets}}
\uline{The following words have been added to the \xref{wordlist:tools-ext}:}

\begin{minipage}[t]{0.3\linewidth}
\uline{\wref{tools:NtoR}{}} \\
\uline{\wref{tools:NRfrom}{}}
\end{minipage}
\hfill
\begin{minipage}[t]{0.3\linewidth}
\uline{\wref{tools:SYNONYM}{}} \\
\uline{\wref{tools:[DEFINED]}{}}
\end{minipage}
\hfill
\begin{minipage}[t]{0.3\linewidth}
\uline{\wref{tools:[UNDEFINED]}{}}
\end{minipage}

\subsection[{Search-Order word sets}]{\uline{Search-Order word sets}}
\uline{No words have been added to \xref{wordlist:search}.}

\subsection[{String word sets}]{\uline{String word sets}}
\uline{The following words have been added to the \xref{wordlist:string-ext}:}

\begin{minipage}[t]{0.3\linewidth}
\uline{\wref{string:REPLACES}{}}
\end{minipage}
\hfill
\begin{minipage}[t]{0.3\linewidth}
\uline{\wref{string:SUBSTITUTE}{}}
\end{minipage}
\hfill
\begin{minipage}[t]{0.3\linewidth}
\uline{\wref{string:UNESCAPE}{}}
\end{minipage}

\subsection[{Extended Character word sets}]{\uline{Extended Character word sets}}
\uline{The Extended Character word set was introduced by Forth-\snapshot.}

\uline{The following words make up \xref{wordlist:xchar}:}

\begin{minipage}[t]{0.25\linewidth}
\uline{\wref{xchar:X-SIZE}{}} \\
\uline{\wref{xchar:XC!+}{}} \\
\uline{\wref{xchar:XC!+q}{}} \\
\uline{\wref{xchar:XC,}{}}
\end{minipage}
\hfill
\begin{minipage}[t]{0.25\linewidth}
\uline{\wref{xchar:XC-SIZE}{}} \\
\uline{\wref{xchar:XC@+}{}} \\
\uline{\wref{xchar:XCHAR+}{}} \\
\uline{\wref{xchar:XEMIT}{}}
\end{minipage}
\hfill
\begin{minipage}[t]{0.4\linewidth}
\uline{\wref{xchar:XKEY}{}} \\
\uline{\wref{xchar:XKEYq}{}} \\
\uline{\wref{xchar:+X/STRING}{}} \\
\uline{\wref{xchar:-TRAILING-GARBAGE}{}}
\end{minipage}

\uline{The following words make up \xref{wordlist:xchar-ext}:}

\begin{minipage}[t]{0.3\linewidth}
\uline{\wref{xchar:CHAR}{}} \\
\uline{\wref{xchar:EKEYtoXCHAR}{}} \\
\uline{\wref{xchar:PARSE}{}}
\end{minipage}
\hfill
\begin{minipage}[t]{0.3\linewidth}
\uline{\wref{xchar:X-WIDTH}{}} \\
\uline{\wref{xchar:XC-WIDTH}{}} \\
\uline{\wref{xchar:XCHAR-}{}}
\end{minipage}
\hfill
\begin{minipage}[t]{0.3\linewidth}
\uline{\wref{xchar:XHOLD}{}} \\
\uline{\wref{xchar:XSTRING-}{}} \\
\uline{\wref{xchar:[CHAR]}{}}
\end{minipage}

\cbend
