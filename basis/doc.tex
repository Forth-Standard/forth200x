\chapter{Documentation requirements}
\label{doc}

When it is impossible or infeasible for a system or program to
define a particular behavior itself, it is permissible to state
that the behavior is unspecifiable and to explain the circumstances
and reasons why this is so.

\section{System documentation} % 4.1
\label{doc:system}

\subsection{Implementation-defined options} % 4.1.1

The implementation-defined items in the following list represent
characteristics and choices left to the discretion of the
implementor, provided that the requirements of this Standard are
met. A system shall document the values for, or behaviors of, each
item.

\begin{itemize}
\item aligned address requirements \xref[Addresses]{usage:addr};

\item behavior of \wref{core:EMIT}{EMIT} for non-graphic characters;

\item character editing of \wref{core:ACCEPT}{ACCEPT} and
	\wref{core:EXPECT}{EXPECT};

\item character set (\xref[Character types]{usage:char},
	\wref{core:EMIT}{EMIT}, \wref{core:KEY}{KEY});

\item character-aligned address requirements
	(\xref[Addresses]{usage:addr});

\item character-set-extensions matching characteristics
	(\xref[Finding definition names]{usage:find});

\item conditions under which control characters match a space
	delimiter (\xref[Delimiters]{usage:delim});

\item format of the control-flow stack
	(\xref[Control-flow stack]{usage:controlstack});

\item conversion of digits larger than thirty-five
	(\xref[Digit conversion]{usage:digits});

\item display after input terminates in \wref{core:ACCEPT}{ACCEPT}
	and \wref{core:EXPECT}{EXPECT};

\item exception abort sequence (as in \wref{core:ABORTq}{ABORT"});

\item input line terminator (\xref[User input device]{usage:input});

\item maximum size of a counted string, in characters
	(\xref[Counted strings]{usage:cstring}, \wref{core:WORD}{WORD});

\item maximum size of a parsed string
	(\xref[Parsing]{usage:parsing});

\item maximum size of a definition name, in characters
	(\xref[Definition names]{usage:names});

\item maximum string length for
	\wref{core:ENVIRONMENTq}{ENVIRONMENT?}, in characters;

\item method of selecting \xref[User input device]{usage:input};

\item method of selecting \xref[User output device]{usage:output};

\item methods of dictionary compilation
	(\xref[The Forth dictionary]{usage:dict});

\item number of bits in one address unit
	(\xref[Addresses]{usage:addr});

\item number representation and arithmetic
	(\xref[Internal number representation]{usage:number});

\item ranges for \param{n}, \param{+n}, \param{u}, \param{d},
	\param{+d}, and \param{ud}
	(\xref[Single-cell types]{usage:cell},
	 \xref[Cell-pair types]{usage:2cell});

\item read-only data-space regions
	(\xref[Data space]{usage:dataspace});

\item size of buffer at \wref{core:WORD}{WORD}
	(\xref[Other transient regions]{usage:transient});

\item size of one cell in address units
	(\xref[Single-cell types]{usage:cell});

\item size of one character in address units
	(\xref[Character types]{usage:char});

\item size of the keyboard terminal input buffer
	(\xref[Input buffers]{usage:inbuf});

\item size of the pictured numeric output string buffer
	(\xref[Other transient regions]{usage:transient});

\item size of the scratch area whose address is returned by
	\wref{core:PAD}{PAD} (\xref[Other transient regions]{usage:transient});

\item system case-sensitivity characteristics
	(\xref[Finding definition names]{usage:find});

\item system prompt (\xref[The Forth text interpreter]{usage:dict},
	\wref{core:QUIT}{QUIT});

\item type of division rounding (\xref[Integer division]{usage:div},
	\wref{core:*/}{*/}, \wref{core:*/MOD}{*/MOD}, \wref{core:/}{/},
	\wref{core:/MOD}{/MOD}, \wref{core:MOD}{MOD});

\item values of \wref{core:STATE}{STATE} when true;

\item values returned after arithmetic overflow
	(\xref[Other integer operations]{usage:intops});

\item whether the current definition can be found after
	\wref{core:DOES}{DOES>} (\wref{core::}{:}).

\end{itemize}



\subsection{Ambiguous conditions} % 4.1.2 ===========================

A system shall document the system action taken upon each of the
general or specific ambiguous conditions identified in this
Standard. See \xref[Possible actions on an ambiguous condition]{usage:ambiguous}.

The following general ambiguous conditions could occur because of a
combination of factors:

\begin{itemize}

\item a \emph{name} is neither a valid definition name nor a valid
	number during text interpretation (\xref[The Forth text
	interpreter]{usage:command});

\item a definition name exceeded the maximum length allowed
	(\xref[Definition names]{usage:names});

\item addressing a region not listed in \xref[Data Space]{usage:dataspace};

\item argument type incompatible with specified input parameter,
	e.g., passing a \emph{flag} to a word expecting an \emph{n}
	(\xref[Data types]{usage:data});

\item attempting to obtain the execution token, (e.g., with
	\wref{core:'}{'}, \wref{core:FIND}{FIND}, etc. of a definition
	with undefined interpretation semantics;

\item dividing by zero
	(\wref{core:*/}{*/},
	 \wref{core:*/MOD}{*/MOD},
	 \wref{core:/}{/},
	 \wref{core:/MOD}{/MOD},
	 \wref{core:FM/MOD}{FM/MOD},
	 \wref{core:MOD}{MOD},
	 \wref{core:SM/REM}{SM/REM},
	 \wref{core:UM/MOD}{UM/MOD},
	 \wref{double:M*/}{M*/});

\item insufficient data-stack space or return-stack space (stack
	overflow);

\item insufficient space for loop-control parameters;

\item insufficient space in the dictionary;

\item interpretating a word with undefined interpretation semantics;

\item modifying the contents of the input buffer or a string literal
	(\xref[Text-literal regions]{usage:"literal},
	\xref[Input buffers]{usage:inbuf});

\item overflow of a pictured numeric output string;

\item parsed string overflow;

\item producing a result out of range, e.g., multiplication
	(using \word{*}) results in a value too big to be represented by
	a single-cell integer
	(\wref{core:*}{*},
	 \wref{core:*/}{*/},
	 \wref{core:*/MOD}{*/MOD},
	 \wref{core:toNUMBER}{>NUMBER},
	 \wref{core:FM/MOD}{FM/MOD},
	 \wref{core:SM/REM}{SM/REM},
	 \wref{core:UM/MOD}{UM/MOD},
	 \wref{core:CONVERT}{CONVERT},
	 \wref{double:M*/}{M*/});

\item reading from an empty data stack or return stack
	(stack underflow);

\item unexpected end of input buffer, resulting in an attempt to
	use a zero-length string as a \emph{name};

\end{itemize}


The following specific ambiguous conditions are noted in the
glossary entries of the relevant words:

\begin{itemize}

\item \word{toIN} greater than size of input buffer
	(\xref[Parsing]{usage:parsing});

\item \wref{core:RECURSE}{RECURSE} appears after
	\wref{core:DOES}{DOES>};

\item argument input source different than current input source for
	\wref{core:RESTORE-INPUT}{RESTORE-INPUT};

\item data space containing definitions is de-allocated
	(\xref[Contiguous regions]{usage:contiguous});

\item data space read/write with incorrect alignment
	(\xref[Address alignment]{usage:aaddr});

\item data-space pointer not properly aligned
	(\wref{core:,}{,}, \wref{core:C,}{C,});

\item less than \param{u}+2 stack items (\wref{core:PICK}{PICK},
	\wref{core:ROLL}{ROLL});

\item loop-control parameters not available
	(\wref{core:+LOOP}{+LOOP},
	 \wref{core:I}{I},
	 \wref{core:J}{J},
	 \wref{core:LEAVE}{LEAVE},
	 \wref{core:LOOP}{LOOP},
	 \wref{core:UNLOOP}{UNLOOP});

\item most recent definition does not have a \emph{name}
	(\wref{core:IMMEDIATE}{IMMEDIATE});

\item \remove{x:to}{\emph{name} not defined by \wref{core:VALUE}{VALUE}
	used by \wref{core:TO}{TO};}

\item \place{x:to}{\wref{core:TO}{TO} applied to a \emph{name} not
	defined by a word with a ``\word{TO} \emph{name} run-time''
	semantics (\wref{core:VALUE}{VALUE} and
	\wref{local:LOCAL}{(LOCAL)});}

\item \emph{name} not found \wref{core:'}{'},
	\wref{core:POSTPONE}{POSTPONE}, \wref{core:[']}{[']},
	\wref{core:[COMPILE]}{[COMPILE]});

\item parameters are not of the same type \wref{core:DO}{DO},
	\wref{core:qDO}{?DO}, \wref{core:WITHIN}{WITHIN});

\item \wref{core:POSTPONE}{POSTPONE}\replace{x:to}{ or}{,}
	\wref{core:[COMPILE]}{[COMPILE]}\place{x:to}{,
	\wref{core:'}{'} or
	\wref{core:[']}{[']}}
	applied to \wref{core:TO}{TO};

\item string longer than a counted string returned by
	\wref{core:WORD}{WORD};

\item $u$ greater than or equal to the number of bits in a cell
	(\wref{core:LSHIFT}{LSHIFT}, \wref{core:RSHIFT}{RSHIFT});

\item word not defined via \wref{core:CREATE}{CREATE}
	(\wref{core:toBODY}{>BODY}, \wref{core:DOES}{DOES>});

\item words improperly used outside
	\wref{core:num-start}{<\num} and \wref{core:num-end}{\num>}
	(\wref{core:num}{\num}, \wref{core:numS}{\num{}S},
	 \wref{core:HOLD}{HOLD}, \wref{core:SIGN}{SIGN}).
\end{itemize}

The following specific ambiguous conditions are have been introduced
as a consequence of specific extensions.

\begin{description}

% ==== X:Deferred ====

\item[\textsf{X:deferrd}] ~
	\begin{itemize}
\item access to a deferred word, a word defined by \wref{core:DEFER}{DEFER},
	which has yet to be associated with an \param{xt}.

\item access to a deferred word, a word defined by \wref{core:DEFER}{DEFER},
	which was not defined by \wref{core:DEFER}{DEFER}.

\item \wref{core:POSTPONE}{POSTPONE}, \wref{core:[COMPILE]}{[COMPILE]},
	\wref{core:[']}{[']} or \wref{core:'}{'} applied to \wref{core:ACTION-OF}{ACTION-OF}
	or \wref{core:IS}{IS}.
	\end{itemize}

% ==== X: Required ====

\item[\textsf{X:required}] \cbstart\marginpar{\tiny x:required} ~
	\begin{itemize}
\item a file is required while it is being \word[file]{REQUIRED}
	(\ref{file:REQUIRED}) or \word[file]{INCLUDED} (\ref{file:INCLUDED}).

\item a marker is defined outside and executed inside a file or vice
	versa, and the file is \word[file]{REQUIRED} (\ref{file:REQUIRED})
	again.

\item the same file is required twice using different names (e.g.,
	through symbolic links), or different files with the same name are
	provided to \wref{file:REQUIRED}{REQUIRED} (by doing some renaming
	between the invocations of \word[file]{REQUIRED}).

\item the stack effect of including with \wref{file:REQUIRED}{REQUIRED}
	the file is not \stack{i*x}{i*x}.
	\end{itemize}
\end{description}
\cbend

\subsection{Other system documentation} % 4.1.3

A system shall provide the following information:

\begin{itemize}

\item list of non-standard words using \wref{core:PAD}{PAD}
	(\xref[Other transient regions]{usage:transient});

\item operator's terminal facilities available;

\item program data space available, in address units;

\item return stack space available, in cells;

\item stack space available, in cells;

\item system dictionary space required, in address units.
\end{itemize}



\section{Program documentation} % 4.2 ===============================
\label{doc:program}

\subsection{Environmental dependencies} % 4.2.1

A program shall document the following environmental dependencies,
where they apply, and should document other known environmental
dependencies:

\begin{itemize}

\item considering the pictured numeric output string buffer a fixed
	area with unchanging access parameters
	(\xref[Other transient regions]{usage:transient});

\item depending on the presence or absence of non-graphic characters
	in a received string (\wref{core:ACCEPT}{ACCEPT},
	\wref{core:EXPECT}{EXPECT});

\item relying on a particular rounding direction
	(\xref[Integer division]{usage:div});

\item requiring a particular number representation and arithmetic
	(\xref[Internal number representation]{usage:number});

\item requiring non-standard words or techniques
	(\xref[Usage requirements]{usage});

\item requiring the ability to send or receive control characters
	(\xref[Control characters]{usage:control},
	\wref{core:KEY}{KEY});

\item using control characters to perform specific functions
	\wref{core:EMIT}{EMIT}, \wref{core:TYPE}{TYPE});

\item using flags as arithmetic operands
	(\xref[Flags]{usage:flags});

\item using lower case for standard definition names or depending
	on the case sensitivity of a system
	(\xref[Definition names]{usage:names});

\item using the graphic character with a value of hex 24
	(\xref[Graphic characters]{usage:ASCII}).

\end{itemize}

\subsection{Other program documentation} % 4.2.2

A program shall also document:

\begin{itemize}

\item minimum operator's terminal facilities required;

\item whether a Standard System exists after the program is loaded.

\end{itemize}
