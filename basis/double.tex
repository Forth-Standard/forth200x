\chapter{The optional Double-Number word set} % 8
\wordlist{double}

\section{Introduction} % 8.1

Sixteen-bit Forth systems often use double-length numbers. However,
many Forths on small embedded systems do not, and many users of Forth
on systems with a cell size of 32 bits or more find that the use of
double-length numbers is much diminished. Therefore, the words that
manipulate double-length entities have been placed in this optional
word set.

\section{Additional terms and notation} % 8.2

None.

\section{Additional usage requirements} % 8.3

\subsection{Environmental queries} % 8.3.1

Append table \ref{double:env} to table \ref{table:env}.

See: \xref[3.2.6 Environmental queries]{usage:env}.

\begin{table}[ht]
  \begin{center}
	\caption{Environmental Query Strings}
	\label{double:env}
	\begin{tabular}{p{9em}rcp{0.42\textwidth}}
		\hline\hline
		\multicolumn{2}{l}{String \hfill Value data type} & Constant? & Meaning \\
		\hline
		\texttt{DOUBLE}		& \emph{flag}	& no	&
			double-number word set present \\
		\texttt{DOUBLE-EXT}	& \emph{flag}	& no	&
			double-number extensions word set present \\
		\hline\hline
	\end{tabular}
  \end{center}
\end{table}

\subsection{Text interpreter input number conversion} % 8.3.2

When the text interpreter processes a number, except a \arg{cnum},
that is immediately followed by a decimal point and is not found
as a definition name, the text interpreter shall convert it to a
double-cell number.

For example, entering \word[core]{DECIMAL} \texttt{1234} leaves the
single-cell number \texttt{1234} on the stack, and entering
\word[core]{DECIMAL} \texttt{1234.} leaves the double-cell number
\texttt{1234} \texttt{0} on the stack.

See: \xref[3.4.1.3 Text interpreter input number conversion]{usage:numbers}.


\section{Additional documentation requirements} % 8.4

\subsection{System documentation} % 8.4.1

\subsubsection{Implementation-defined options} % 8.4.1.1

\begin{itemize}
\item no additional requirements.
\end{itemize}

\subsubsection{Ambiguous conditions} % 8.4.1.2

\begin{itemize}
\item \param{d} outside range of \param{n} in \wref{double:DtoS}{D>S}.
\end{itemize}

\subsubsection{Other system documentation} % 8.4.1.3

\begin{itemize}
\item no additional requirements.
\end{itemize}

\subsection{Program documentation} % 8.4.2

\begin{itemize}
\item no additional requirements.
\end{itemize}


\section{Compliance and labeling} % 8.5

\subsection{ANS Forth systems} % 8.5.1

The phrase ``Providing the Double-Number word set'' shall be
appended to the label of any Standard System that provides all
of the Double-Number word set.

The phrase ``Providing \emph{name(s)} from the Double-Number
Extensions word set'' shall be appended to the label of any
Standard System that provides portions of the Double-Number
Extensions word set.

The phrase ``Providing the Double-Number Extensions word set''
shall be appended to the label of any Standard System that
provides all of the Double-Number and Double-Number Extensions
word sets.

\subsection{ANS Forth programs} % 8.5.2

The phrase ``Requiring the Double-Number word set'' shall be
appended to the label of Standard Programs that require the
system to provide the Double-Number word set.

The phrase ``Requiring \emph{name(s)} from the Double-Number
Extensions word set'' shall be appended to the label of Standard
Programs that require the system to provide portions of the
Double-Number Extensions word set.

The phrase ``Requiring the Double-Number Extensions word set''
shall be appended to the label of Standard Programs that require
the system to provide all of the Double-Number and Double-Number
Extensions word sets.

\section{Glossary} % 8.6

\subsection{Double-Number words} % 8.6.1

\begin{worddef}{0360}{2CONSTANT}[two-constant]
\item \stack{x_1 x_2 "<spaces>name"}{}

	Skip leading space delimiters. Parse \param{name} delimited by
	a space. Create a definition for \param{name} with the execution
	semantics defined below.

	\param{name} is referred to as a ``two-constant''.

\execute[name]
	\stack{}{x_1 x_2}

	Place cell pair \param{x_1 x_2} on the stack.

\see \xref[3.4.1 Parsing]{usage:parsing}.

	\begin{defer}
	\rationale % A.8.6.1.0360 2CONSTANT
		Typical use:
			\texttt{x1} \texttt{x2} \word{2CONSTANT} \emph{name}
	\end{defer}
\end{worddef}


\begin{worddef}{0390}{2LITERAL}[two-literal]
\interpret
	Interpretation semantics for this word are undefined.

\compile
	\stack{x_1 x_2}{}

	Append the run-time semantics below to the current definition.

\runtime
	\stack{}{x_1 x_2}

	Place cell pair \param{x_1 x_2} on the stack.

	\begin{defer}
	\rationale % A.8.6.1.0390 2LITERAL
		\setwordlist{core}
		Typical use:
			\word{:} \texttt{X} {\ldots}
				\word{[} \texttt{x1} \texttt{x2} \word{]} \word[double]{2LITERAL}
			{\ldots} \word{;}
		\setwordlist{double}
	\end{defer}
\end{worddef}


\begin{worddef}{0440}{2VARIABLE}[two-variable]
\item \stack{"<spaces>name"}{}

	Skip leading space delimiters. Parse \param{name} delimited by a
	space. Create a definition for \param{name} with the execution
	semantics defined below. Reserve two consecutive cells of data
	space.

	\param{name} is referred to as a ``two-variable''.

\execute[name]
	\stack{}{a-addr}

	\param{a-addr} is the address of the first (lowest address)
	cell of two consecutive cells in data space reserved by
	\word{2VARIABLE} when it defined \param{name}. A program is
	responsible for initializing the contents.

\see \xref[3.4.1 Parsing]{usage:parsing},
	\wref{core:VARIABLE}{VARIABLE}.

	\begin{defer}
	\rationale % A.8.6.1.0440 2VARIABLE
		Typical use:
			\word{2VARIABLE} \param{name}
	\end{defer}
\end{worddef}


\begin{worddef}{1040}{D+}[d-plus]
\item \stack{d_1|ud_1 d_2|ud_2}{d_3|ud_3}

	Add \param{d_2|ud_2} to \param{d_1|ud_1}, giving the sum
	\param{d_3|ud_3}.
\end{worddef}


\begin{worddef}{1050}{D-}[d-minus]
\item \stack{d_1|ud_1 d_2|ud_2}{d_3|ud_3}

	Subtract \param{d_2|ud_2} from \param{d_1|ud_1}, giving the
	difference \param{d_3|ud_3}.
\end{worddef}


\begin{worddef}{1060}{D.}[d-dot]
\item \stack{d}{}

	Display \param{d} in free field format.
\end{worddef}


\begin{worddef}{1070}{D.R}[d-dot-r]
\item \stack{d n}{}

	Display \param{d} right aligned in a field \param{n} characters
	wide. If the number of characters required to display \param{d}
	is greater than \param{n}, all digits are displayed with no
	leading spaces in a field as wide as necessary.

	\begin{defer}
	\rationale % A.8.6.1.1070 D.R
		In \word{D.R}, the ``R'' is short for RIGHT.
	\end{defer}
\end{worddef}


\begin{worddef}[D0less]{1075}{D0<}[d-zero-less]
\item \stack{d}{flag}

	\param{flag} is true if and only if \param{d} is less than zero.
\end{worddef}


\begin{worddef}{1080}{D0=}[d-zero-equals]
\item \stack{xd}{flag}

	\param{flag} is true if and only if \param{xd} is equal to zero.
\end{worddef}


\begin{worddef}{1090}{D2*}[d-two-star]
\item \stack{xd_1}{xd_2}

	\param{xd_2} is the result of shifting \param{xd_1} one bit
	toward the most-significant bit, filling the vacated
	least-significant bit with zero.

	\begin{defer}
	\rationale % A.8.6.1.1090 D2*
		See: \rref{core:2*}{2*} for applicable discussion.
	\end{defer}
\end{worddef}


\begin{worddef}{1100}{D2/}[d-two-slash]
\item \stack{xd_1}{xd_2}

	\param{xd_2} is the result of shifting \param{xd_1} one bit
	toward the least-significant bit, leaving the most-significant
	bit unchanged.

	\begin{defer}
	\rationale % A.8.6.1.1100 D2/
		See: \rref{core:2/}{2/} for applicable discussion.
	\end{defer}
\end{worddef}


\begin{worddef}[Dless]{1110}{D<}[d-less-than]
\item \stack{d_1 d_2}{flag}

	\param{flag} is true if and only if \param{d_1} is less than
	\param{d_2}.
\end{worddef}


\begin{worddef}{1120}{D=}[d-equals]
\item \stack{xd_1 xd_2}{flag}

	\param{flag} is true if and only if \param{xd_1} is bit-for-bit
	the same as \param{xd_2}.
\end{worddef}


\begin{worddef}[DtoS]{1140}{D>S}[d-to-s]
\item \stack{d}{n}

	\param{n} is the equivalent of \param{d}. An ambiguous condition
	exists if \param{d} lies outside the range of a signed single-cell
	number.

	\begin{defer}
	\rationale % A.8.6.1.1140 D>S
		There exist number representations, e.g., the sign-magnitude
		representation, where reduction from double- to single-precision
		cannot simply be done with \word[core]{DROP}. This word,
		equivalent to \word[core]{DROP} on two's complement systems,
		desensitizes application code to number representation and
		facilitates portability.
	\end{defer}
\end{worddef}


\begin{worddef}{1160}{DABS}[d-abs]
\item \stack{d}{ud}

	\param{ud} is the absolute value of \param{d}.
\end{worddef}


\begin{worddef}{1210}{DMAX}[d-max]
\item \stack{d_1 d_2}{d_3}

	\param{d_3} is the greater of \param{d_1} and \param{d_2}.
\end{worddef}


\begin{worddef}{1220}{DMIN}[d-min]
\item \stack{d_1 d_2}{d_3}

	\param{d_3} is the lesser of \param{d_1} and \param{d_2}.
\end{worddef}


\begin{worddef}{1230}{DNEGATE}[d-negate]
\item \stack{d_1}{d_2}

	\param{d_2} is the negation of \param{d_1}.
\end{worddef}


\begin{worddef}{1820}{M*/}[m-star-slash]
\item \stack{d_1 n_1 +n_2}{d_2}

	Multiply \param{d_1} by \param{n_1} producing the triple-cell
	intermediate result $t$. Divide $t$ by \param{+n_2} giving the
	double-cell quotient \param{d_2}. An ambiguous condition exists
	if \param{+n_2} is zero or negative, or the quotient lies outside
	of the range of a double-precision signed integer.

	\begin{defer}
	\rationale % A.8.6.1.1820 M*/
		\word{M*/} was once described by Chuck Moore as the most
		useful arithmetic operator in Forth. It is the main workhorse
		in most computations involving double-cell numbers. Note that
		some systems allow signed divisors. This can cost a lot in
		performance on some CPUs. The requirement for a positive
		divisor has not proven to be a problem.
	\end{defer}
\end{worddef}


\begin{worddef}{1830}{M+}[m-plus]
\item \stack{d_1|ud_1 n}{d_2|ud_2}

	Add \param{n} to \param{d_1|ud_1}, giving the sum \param{d_2|ud_2}.

	\begin{defer}
	\rationale % A.8.6.1.1830 M+
		\word{M+} is the classical method for integrating.
	\end{defer}
\end{worddef}


\subsection{Double-Number extension words} % 8.6.2
\extended

\begin{worddef}{0420}{2ROT}[two-rote]
\item \stack{x_1 x_2 x_3 x_4 x_5 x_6}{x_3 x_4 x_5 x_6 x_1 x_2}

	Rotate the top three cell pairs on the stack bringing cell pair
	\param{x_1 x_2} to the top of the stack.
\end{worddef}


\begin{worddef}[DUless]{1270}{DU<}[d-u-less]
\item \stack{ud_1 ud_2}{flag}

	\param{flag} is true if and only if \param{ud_1} is less than
	\param{ud_2}.
\end{worddef}
