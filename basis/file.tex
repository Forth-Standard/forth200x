% !TeX root = forth.tex
% !TeX spellcheck = en_US

\chapter{The optional File-Access word set} % 11
\wordlist{file}

\section{Introduction} % 11.1

These words provide access to mass storage in the form of ``files''
under the following assumptions:

\begin{itemize}
\item files are provided by a host operating system;
\item file names are represented as character strings;
\item the format of file names is determined by the host operating
	system;
\item an open file is identified by a single-cell file identifier
	(\emph{fileid});
\item file-state information (e.g., position, size) is managed by
	the host operating system;
\item file contents are accessed as a sequence of characters;
\item file read operations return an actual transfer count, which
	can differ from the requested transfer count.
\end{itemize}

\section{Additional terms} % 11.2

\begin{description}
\item[file-access method:]
	A permissible means of accessing a file, such as ``read/write''
	or ``read only''.

\item[file position:]
	The character offset from the start of the file.

\item[input file:]
	The file, containing a sequence of lines, that is the input source.
\end{description}

\section{Additional usage requirements} % 11.3

\subsection{Data types} % 11.3.1

Append table \ref{file:types} to table \ref{table:datatypes}.

\begin{table}[h]
  \begin{center}
	\caption{Data types}
	\label{file:types}
	\begin{tabular}{llr}
	\hline\hline
	\emph{Symbol} & \emph{Data type} & \emph{Size on stack} \\
	\hline
	\emph{fam}		& file access method	& 1 cell \\
	\emph{fileid}	& file identifier		& 1 cell \\
	\hline\hline
	\end{tabular}
  \end{center}
\end{table}

\subsubsection{File identifiers} % 11.3.1.1

File identifiers are implementation-dependent single-cell values
that are passed to file operators to designate specific files.
Opening a file assigns a file identifier, which remains valid
until closed.

\setcounter{subsubsection}{2}
\subsubsection{File access methods (11.3.1.3)} % 11.3.1.3

File access methods are implementation-defined single-cell
values.

\subsubsection{File names} % 11.3.1.4
\label{file:names}

A character string containing the name of the file. The file name
may include an implementation-dependent path name. The format of
file names is implementation defined.

\subsection{Blocks in files} % 11.3.2
\label{file:blocks}

\remove{ed14}{If the File-Access word set is implemented, the Block word set
shall be implemented.} Blocks may, but need not, reside in files.
When they do:
\begin{itemize}
\item Block numbers may be mapped to one or more files by
	implementation-defined means. An ambiguous condition exists
	if a requested block number is not currently mapped;
\item An \word[block]{UPDATE}d block that came from a file shall
	be transferred back to the same file.
\end{itemize}


\subsection{Input source} % 11.3.4
\label{file:source}

The File-Access word set creates another input source for the text
interpreter. When the input source is a text file, \word[block]{BLK}
shall contain zero, \word{SOURCE-ID} shall contain the \emph{fileid}
of that text file, and the input buffer shall contain one line of
the text file.  During text interpretation from a text file, the
value returned by \word{FILE-POSITION} for the \param{fileid} returned
by \word{SOURCE-ID} is undefined.  A standard program
shall not call \word{REPOSITION-FILE} on the \param{fileid} returned
by \word{SOURCE-ID}.

Input with \word{INCLUDED}, \word{INCLUDE-FILE}, \word[block]{LOAD}
and \word[block]{EVALUATE} shall be nestable in any order to at least
eight levels.

A program that uses more than eight levels of input-file nesting has
an environmental dependency. See:
\xref[3.3.3.5 Input buffers]{usage:inbuf},
\xref[9. Optional Exception word set]{wordlist:exception}.

\subsection{Other transient regions} % 11.3.5
\label{file:buffers}

\remove{X:interpret-escaped-s}{%
The list of words using memory in transient regions is extended to
include \wref{file:Sq}{S"}. See:} \xref[3.3.3.6 Other transient
regions]{usage:transient}.

\place{X:interpret-escaped-s}{%
The system provides transient buffers for \word{Sq} and \word{Seq}
strings. These buffers shall be no less than 80 characters in length, and
there shall be at least two buffers. The system should be able to store
two strings defined by sequential use of \word{Sq} or \word{Seq}.
RAM-limited systems may have environmental restrictions on the number
of buffers and their lifetimes.}

\subsection{Parsing} % 11.3.6
\label{file:parsing}

When parsing from a text file using a space delimiter, control
characters shall be treated the same as the space character.

Lines of at least 128 characters shall be supported. A program that
requires lines of more than 128 characters has an environmental
dependency.

A program may reposition the parse area within the input buffer by
manipulating the contents of \word[core]{toIN}. More extensive
repositioning can be accomplished using \word[core]{SAVE-INPUT} and
\word[core]{RESTORE-INPUT}.

See: \xref[3.4.1 Parsing]{usage:parsing}.

\section{Additional documentation requirements} % 11.4

\subsection{System documentation} % 11.4.1

\subsubsection{Implementation-defined options} % 11.4.1.1
\label{file:impopt}

\begin{itemize}
\item file access methods used by
	\wref{file:BIN}{BIN},
	\wref{file:CREATE-FILE}{CREATE-FILE}, \\
	\wref{file:OPEN-FILE}{OPEN-FILE},
	\wref{file:R/O}{R/O},
	\wref{file:R/W}{R/W} and
	\wref{file:W/O}{W/O};
\item file exceptions;
\item file line terminator (\wref{file:READ-LINE}{READ-LINE});
\item file name format (\xref[11.3.1.4 File names]{file:names});
\item information returned by \wref{file:FILE-STATUS}{FILE-STATUS};
\item input file state after an exception
	(\wref{file:INCLUDE-FILE}{INCLUDE-FILE},
	 \wref{file:INCLUDED}{INCLUDED});
\item maximum depth of file input nesting
	(\xref[11.3.4 Input source]{file:source});
\item maximum size of input line (\xref[11.3.6 Parsing]{file:parsing});
\item methods for mapping block ranges to files
	(\xref[11.3.2 Blocks in files]{file:blocks});
\item number of string buffers provided (\replace{X:interpret-escape-s}{\wref{file:Sq}{S"}}{ \xref{file:buffers}});
\item size of string buffer used by \replace{X:interpret-escape-s}{\wref{file:Sq}{S"}}{ \xref{file:buffers}}.
\end{itemize}

\subsubsection{Ambiguous conditions} % 11.4.1.2
\begin{itemize}
\item attempting to position a file outside its boundaries
	(\wref{file:REPOSITION-FILE}{REPOSITION-FILE});
\item attempting to read from file positions not yet written
	(\wref{file:READ-FILE}{READ-FILE}, \\ \wref{file:READ-LINE}{READ-LINE});
\item \emph{fileid} is invalid (\wref{file:INCLUDE-FILE}{INCLUDE-FILE});
\item I/O exception reading or closing \emph{fileid}
	(\wref{file:INCLUDE-FILE}{INCLUDE-FILE},
	 \wref{file:INCLUDED}{INCLUDED});
\item named file cannot be opened
	(\wref{file:INCLUDED}{INCLUDED});
\item requesting an unmapped block number
	(\xref[11.3.2 Blocks in files]{file:blocks});
\item using \wref{file:SOURCE-ID}{SOURCE-ID} when
	\wref{block:BLK}{BLK} is not zero.

% ==== X:Required ====

\item a file is required while it is being \word{REQUIRED}
	(\ref{file:REQUIRED}) or \word{INCLUDED} (\ref{file:INCLUDED}).
\item a marker is defined outside and executed inside a file or
	vice versa, and the file is \word{REQUIRED}
	(\ref{file:REQUIRED}) again.
\item the same file is required twice using different names
	(e.g., through symbolic links), or different files with the
	same name are provided to \wref{file:REQUIRED}{REQUIRED} (by doing
	some renaming between the invocations of \word{REQUIRED}).
\item the stack effect of including with
	\wref{file:REQUIRED}{REQUIRED} the file is not \stack{i*x}{i*x}.

\end{itemize}

\subsubsection{Other system documentation} % 11.4.1.3
\begin{itemize}
\item no additional requirements.
\end{itemize}

\subsection{Program documentation} % 11.4.2

\subsubsection{Environmental dependencies} % 11.4.2.1
\begin{itemize}
\item requiring lines longer than 128 characters
	(\xref[11.3.6 Parsing]{file:parsing});
\item using more than eight levels of input-file nesting
	(\xref[11.3.4 Input source]{file:source}).
\end{itemize}

\subsubsection{Other program documentation} % 11.4.2.2
\begin{itemize}
\item no additional requirements.
\end{itemize}

\section{Compliance and labeling} % 11.5

\subsection{Forth-\snapshot{} systems} % 11.5.1

The phrase ``Providing the File Access word set'' shall be appended
to the label of any Standard System that provides all of the File
Access word set.

The phrase ``Providing \emph{name(s)} from the File Access Extensions
word set'' shall be appended to the label of any Standard System that
provides portions of the File Access Extensions word set.

The phrase ``Providing the File Access Extensions word set'' shall
be appended to the label of any Standard System that provides all of
the File Access and File Access Extensions word sets.

\subsection{Forth-\snapshot{} programs} % 11.5.2

The phrase ``Requiring the File Access word set'' shall be appended
to the label of Standard Programs that require the system to provide
the File Access word set.

The phrase ``Requiring \emph{name(s)} from the File Access Extensions
word set'' shall be appended to the label of Standard Programs that
require the system to provide portions of the File Access Extensions
word set.

The phrase ``Requiring the File Access Extensions word set'' shall be
appended to the label of Standard Programs that require the system to
provide all of the File Access and File Access Extensions word sets.


\section{Glossary} % 11.6

\subsection{File Access words} % 11.6.1

\enlargethispage{2ex}
\begin{worddef}[p]{0080}{(}[paren]
\item \stack{"ccc<paren>"}{}

	Extend the semantics of \wref{core:p}{(} to include:

	When parsing from a text file, if the end of the parse area is
	reached before a right parenthesis is found, refill the input
	buffer from the next line of the file, set \word[core]{toIN} to
	zero, and resume parsing, repeating this process until either a
	right parenthesis is found or the end of the file is reached.

	\begin{testing}
		\test{\word{p} 1 2 3 \\
		\tab[2.4] 4 5 6 \\
		\tab[2.4] 7 8 9 ) 11 22 33}{11 22 33}
	\end{testing}
\end{worddef}


\enlargethispage{4ex}
\begin{worddef}{0765}{BIN}
\item \stack{fam_1}{fam_2}

	Modify the implementation-defined file access method
	\param{fam_1} to additionally select a \linebreak
	``binary'', i.e., not line oriented, file access method,
	giving access method \param{fam_2}.

\see \wref{file:R/O}{R/O},
	\wref{file:R/W}{R/W},
	\wref{file:W/O}{W/O},
	\rref{file:BIN}{}.

	\begin{rationale} % A.11.6.1.0765 BIN
		Some operating systems require that files be opened in a
		different mode to access their contents as an unstructured
		stream of binary data rather than as a sequence of lines.

		The arguments to \word{READ-FILE} and \word{WRITE-FILE} are
		arrays of character storage elements, each element consisting
		of at least 8 bits. The committee intends that, in
		\word{BIN} mode, the contents of these storage elements can be
		written to a file and later read back without alteration.
	\end{rationale}
\end{worddef}

\vspace*{-1ex}
\begin{worddef}{0900}{CLOSE-FILE}
\item \stack{fileid}{ior}

	Close the file identified by \param{fileid}. \param{ior} is the
	implementation-defined I/O result code.
\end{worddef}

\vspace*{-1ex}
\begin{worddef}{1010}{CREATE-FILE}
\item \stack{c-addr u fam}{fileid ior}

	Create the file named in the character string specified by
	\param{c-addr} and \param{u}, and open it with file access method
	\param{fam}. The meaning of values of \param{fam} is
	implementation defined. If a file with the same name already
	exists, recreate it as an empty file.

	If the file was successfully created and opened, \param{ior} is
	zero, \param{fileid} is its identifier, and the file has been
	positioned to the start of the file.

	Otherwise, \param{ior} is the implementation-defined I/O result
	code and \param{fileid} is undefined.

\see \rref{file:CREATE-FILE}{}.

	\begin{rationale} % A.11.6.1.1010 CREATE-FILE
		Typical use:

		\tab \word[core]{:} \texttt{X} {\ldots}
				\word{Sq} \texttt{TEST.FTH"} \word{R/W}
				\word{CREATE-FILE}
				\word[core]{ABORTq} \texttt{CREATE-FILE FAILED"}
			\word[core]{;}
	\end{rationale}

	\begin{testing}\ttfamily
		\word{:} fn1 \word{Sq} fatest1.txt" \word{;} \\
		\word{VARIABLE} fid1

		\test{fn1 \word{R/W} \word{CREATE-FILE} \word{SWAP} fid1 \word{!}}{0} \\
		\test{fid1 \word{@} \word{CLOSE-FILE}}{0}
	\end{testing}
\end{worddef}

\vspace*{-1ex}
\begin{worddef}{1190}{DELETE-FILE}
\item \stack{c-addr u}{ior}

	Delete the file named in the character string specified by
	\param{c-addr u}. \param{ior} is the {im\-ple\-ment\-ation-de\-fin\-ed}
	I/O result code.

	\begin{testing}
		\test{fn2 \word{DELETE-FILE}}{0} \\
		\test{fn2 \word{R/W} \word{BIN} \word{OPEN-FILE} \word{SWAP} \word{DROP}}{0} \\
		\test{fn2 \word{DELETE-FILE}}{0}
	\end{testing}
\end{worddef}

\vspace*{-1ex}
\begin{worddef}{1520}{FILE-POSITION}
\item \stack{fileid}{ud ior}

	\param{ud} is the current file position for the file identified
	by \param{fileid}. \param{ior} is the implementation-defined I/O
	result code. \param{ud} is undefined if \param{ior} is non-zero.
\end{worddef}

\vspace*{-1ex}
\begin{worddef}{1522}{FILE-SIZE}
\item \stack{fileid}{ud ior}

	\param{ud} is the size, in characters, of the file identified by
	\param{fileid}. \param{ior} is the {im\-ple\-ment\-ation-de\-fin\-ed} I/O
	result code. This operation does not affect the value returned
	by
	\hyperref{}{file}{FILE-POSITION}{\textbf{\texttt{FILE-}}}
	\hyperref{}{file}{FILE-POSITION}{\textbf{\texttt{POSITION}}}.
	\param{ud} is undefined if \param{ior} is non-zero.

	\begin{testing}\ttfamily
		\word{:} cbuf buf bsize 0 \word{FILL} \word{;} \\
		\word{:} fn2 \word{Sq} fatest2.txt" \word{;} \\
		\word{VARIABLE} fid2 \\
		\word{:} setpad \word{PAD} 50 0 \word{DO} \word{I} \word{OVER} \word{C!} \word{CHAR+} \word{LOOP} \word{DROP} \word{;}

		setpad

		\textdf{\textbf{Note:} If anything else is defined
			\texttt{setpad} must be called again as the pad may move}

		\test{fn2 \word{R/W} \word{BIN} \word{CREATE-FILE} \word{SWAP} fid2 \word{!}}{0} \\
		\test{\word{PAD} 50 fid2 \word{@} \word{WRITE-FILE} fid2 \word{@} \word{FLUSH-FILE}}{0 0} \\
		\test{fid2 \word{@} \word{FILE-SIZE}}{50. 0} \\
		\test{0. fid2 \word{@} \word{REPOSITION-FILE}}{0} \\
		\test{cbuf buf 29 fid2 \word{@} \word{READ-FILE}}{29 0} \\
		\test{\word{PAD} 29 buf 29 \word[string]{COMPARE}}{0} \\
		\test{\word{PAD} 30 buf 30 \word[string]{COMPARE}}{1} \\
		\test{cbuf buf 29 fid2 \word{@} \word{READ-FILE}}{21 0} \\
		\test{\word{PAD} 29 \word{+} 21 buf 21 \word[string]{COMPARE}}{0} \\
		\test{fid2 \word{@} \word{FILE-SIZE} \word{DROP} fid2 \word{@} \word{FILE-POSITION} \word{DROP} \word[double]{D=}}{<TRUE>} \\
		\test{buf 10 fid2 \word{@} \word{READ-FILE}}{0 0} \\
		\test{fid2 \word{@} \word{CLOSE-FILE}}{0}
	\end{testing}
\end{worddef}


\begin{worddef}{1717}{INCLUDE-FILE}
\item \stack{i*x fileid}{j*x}

	Remove \param{fileid} from the stack. Save the current input
	source specification, including the current value of
	\word{SOURCE-ID}. Store \param{fileid} in \word{SOURCE-ID}.
	Make the file specified by \param{fileid} the input source. Store
	zero in \word[block]{BLK}. Other stack effects are due to the
	words included.

	Repeat until end of file: read a line from the file, fill the
	input buffer from the contents of that line, set \word[core]{toIN}
	to zero, and interpret.

	Text interpretation begins at the file position where the next
	file read would occur.

	When the end of the file is reached, close the file and restore
	the input source specification to its saved value.

	An ambiguous condition exists if \param{fileid} is invalid, if
	there is an I/O exception reading \param{fileid}, or if an I/O
	exception occurs while closing \param{fileid}. When an ambiguous
	condition exists, the status (open or closed) of any files that
	were being interpreted is {im\-ple\-ment\-ation-de\-fin\-ed}.

\see \xref[11.3.4 Input source]{file:source},
	\rref{file:INCLUDE-FILE}{}.

	\begin{rationale} % A.11.6.1.1717 INCLUDE-FILE
		Here are two implementation alternatives for saving the input
		source specification in the presence of text file input:

		\begin{enumerate}
		\item[1)] Save the file position (as returned by
			\word{FILE-POSITION}) of the beginning of the line being
			interpreted. To restore the input source specification,
			seek to that position and re-read the line into the input
			buffer.

		\item[2)] Allocate a separate line buffer for each active text
			input file, using that buffer as the input buffer. This
			method avoids the ``seek and reread'' step, and allows the
			use of ``pseudo-files'' such as pipes and other
			sequential-access-only communication channels.
		\end{enumerate}
	\end{rationale}
\end{worddef}


\begin{worddef}{1718}{INCLUDED}
\item \stack{i*x c-addr u}{j*x}

	Remove \param{c-addr u} from the stack. Save the current input
	source specification, including the current value of
	\word{SOURCE-ID}. Open the file specified by \param{c-addr u},
	store the resulting \param{fileid} in \word{SOURCE-ID}, and make
	it the input source. Store zero in \word[block]{BLK}. Other
	stack effects are due to the words included.

	Repeat until end of file: read a line from the file, fill the
	input buffer from the contents of that line, set \word[core]{toIN}
	to zero, and interpret.

	Text interpretation begins at the start of the file.

	When the end of the file is reached, close the file and restore
	the input source specification to its saved value.

	An ambiguous condition exists if the named file can not be
	opened, if an I/O exception occurs reading the file, or if an
	I/O exception occurs while closing the file. When an ambiguous
	condition exists, the status (open or closed) of any files that
	were being interpreted is implementation-defined.

	\word{INCLUDED} may allocate memory in data space before it
	starts interpreting the file.

\see \wref{file:INCLUDE-FILE}{INCLUDE-FILE},
	\rref{file:INCLUDED}{}.

	\begin{rationale} % A.11.6.1.1718 INCLUDED
		Typical use:
			{\ldots} \word{Sq} \texttt{filename"} \word{INCLUDED} {\ldots}
	\end{rationale}

	\begin{testing} % T.11.6.1.1718 INCLUDED
		See \tref{file:REQUIRED}{REQUIRED}.
	\end{testing}
\end{worddef}


\begin{worddef}{1970}{OPEN-FILE}
\item \stack{c-addr u fam}{fileid ior}

	Open the file named in the character string specified by
	\param{c-addr u}, with file access method indicated by
	\param{fam}. The meaning of values of \param{fam} is
	implementation 	defined.

	If the file is successfully opened, \param{ior} is zero,
	\param{fileid} is its identifier, and the file has been
	positioned to the start of the file.

	Otherwise, \param{ior} is the implementation-defined I/O
	result code and \param{fileid} is undefined.

\see \rref{file:OPEN-FILE}{}.

	\begin{rationale} % A.11.6.1.1970 OPEN-FILE
		Typical use:

		\tab \word[core]{:} \texttt{X} {\ldots}
				\word{Sq} \texttt{ TEST.FTH"} \word{R/W}
				\word{OPEN-FILE}~ \word[core]{ABORTq} \texttt{OPEN-FILE FAILED"}
			{\ldots} \word[core]{;}
	\end{rationale}
\end{worddef}


\begin{worddef}{2054}{R/O}[r-o]
\item \stack{}{fam}

	\param{fam} is the implementation-defined value for selecting
	the ``read only'' file access method.

\see \wref{file:CREATE-FILE}{CREATE-FILE},
	\wref{file:OPEN-FILE}{OPEN-FILE}.
\end{worddef}


\begin{worddef}{2056}{R/W}[r-w]
\item \stack{}{fam}

	\param{fam} is the implementation-defined value for selecting
	the ``read/write'' file access method.

\see \wref{file:CREATE-FILE}{CREATE-FILE},
	\wref{file:OPEN-FILE}{OPEN-FILE}.
\end{worddef}


\begin{worddef}{2080}{READ-FILE}
\item \stack{c-addr u_1 fileid}{u_2 ior}

	Read \param{u_1} consecutive characters to \param{c-addr} from the
	current position of the file identified by \param{fileid}.

	If \param{u_1} characters are read without an exception, \param{ior}
	is zero and \param{u_2} is equal to \param{u_1}.

	If the end of the file is reached before \param{u_1} characters are
	read, \param{ior} is zero and \param{u_2} is the number of characters
	actually read.

	If the operation is initiated when the value returned by
	\word{FILE-POSITION} is equal to the value returned by
	\word{FILE-SIZE} for the file identified by \param{fileid},
	\param{ior} is zero and \param{u_2} is zero.

	If an exception occurs, \param{ior} is the implementation-defined
	I/O result code, and \param{u_2} is the number of characters
	transferred to \param{c-addr} without an exception.

	An ambiguous condition exists if the operation is initiated when
	the value returned by \word{FILE-POSITION} is greater than the
	value returned by \word{FILE-SIZE} for the file identified by
	\param{fileid}, or if the requested operation attempts to read
	portions of the file not written.

	At the conclusion of the operation, \word{FILE-POSITION} returns
	the next file position after the last character read.

\see \rref{file:READ-FILE}{}.

	\begin{rationale} % A.11.6.1.2080 READ-FILE
		A typical sequential file-processing algorithm might look like:
		\begin{quote}\ttfamily\setwordlist{core}
			\begin{tabbing}
			\tab \= \hspace{12em} \= \kill
			\word{BEGIN}					\>\> \word{p} ) \\
			\> {\ldots} \word[file]{READ-FILE} \word[exception]{THROW}
											\>	 \word{p} length ) \\
			\word{qDUP} \word{WHILE}		\>\> \word{p} length ) \\
			\> {\ldots}						\>	 \word{p} ) \\
			\word{REPEAT}					\>\> \word{p} )
			\end{tabbing}
		\end{quote}\setwordlist{file}

		In this example, \word[exception]{THROW} is used to handle
		exception conditions, which are reported as
		non-zero values of the \param{ior} return value from
		\word{READ-FILE}. End-of-file is reported as a zero value of
		the ``length'' return value.
	\end{rationale}
\end{worddef}


\begin{worddef}{2090}{READ-LINE}
\item \stack{c-addr u_1 fileid}{u_2 flag ior}

	Read the next line from the file specified by \param{fileid} into
	memory at the address \param{c-addr}. At most \param{u_1} characters
	are read. Up to two implementation-defined line-terminating
	characters may be read into memory at the end of the line, but
	are not included in the count \param{u_2}. The line buffer provided
	by \param{c-addr} should be at least \param{u_1+2} characters long.

	If the operation succeeded, \param{flag} is true and \param{ior}
	is zero. If a line terminator was received before \param{u_1}
	characters were read, then \param{u_2} is the number of characters,
	not including the line terminator, actually read ($0 <=$ \param{u_2}
	$<=$ \param{u_1}). When \param{u_1} = \param{u_2} the line
	terminator has yet to be reached.

	If the operation is initiated when the value returned by
	\word{FILE-POSITION} is equal to the value returned by
	\word{FILE-SIZE} for the file identified by \param{fileid},
	\param{flag} is false, \param{ior} is zero, and \param{u_2} is zero.
	If \param{ior} is non-zero, an exception occurred during the
	operation and \param{ior} is the implementation-defined I/O
	result code.

	An ambiguous condition exists if the operation is initiated when
	the value returned by \word{FILE-POSITION} is greater than the
	value returned by \word{FILE-SIZE} for the file identified by
	\param{fileid}, or if the requested operation attempts to read
	portions of the file not written.

	At the conclusion of the operation, \word{FILE-POSITION} returns
	the next file position after the last character read.

\see \rref{file:READ-LINE}{}.

	\begin{rationale} % A.11.6.1.2090 READ-LINE
		Implementations are allowed to store the line terminator in
		the memory buffer in order to allow the use of line reading
		functions provided by host operating systems, some of which
		store the terminator. Without this provision, a \replace{ed14}{temporary}{transient}
		buffer might be needed. The two-character limitation is
		sufficient for the vast majority of existing operating
		systems. Implementations on host operating systems whose line
		terminator sequence is longer than two characters may have to
		take special action to prevent the storage of more than two
		terminator characters.

		Standard Programs may not depend on the presence of any such
		terminator sequence in the buffer.

		A typical line-oriented sequential file-processing algorithm
		might look like:

		\begin{quote}\ttfamily\setwordlist{core}
		  \begin{tabbing}
			\tab \= \hspace{12em} \= \kill
			\word{BEGIN}				\>\> \word{p} ) \\
			\> {\ldots} \word[file]{READ-LINE} \word[exception]{THROW}
										\>	 \word{p} length not-eof-flag ) \\
			\word{WHILE} 			\>\> \word{p} length ) \\
			\> {\ldots}				\>	 \word{p} ) \\
			\word{REPEAT} \word{DROP}	\>\> \word{p} ) \\
		  \end{tabbing}
		\end{quote}\setwordlist{file}

		\word{READ-LINE} needs a separate end-of-file flag because
		empty (zero-length) lines are a routine occurrence, so a
		zero-length line cannot be used to signify end-of-file.
	\end{rationale}

	\begin{testing}\ttfamily
		200 \word{CONSTANT} bsize \\
		\word{CREATE} buf bsize \word{ALLOT} \\
		\word{VARIABLE} \#chars

		\test{fn1 \word{R/O} \word{OPEN-FILE} \word{SWAP} fid1 \word{!}}{0} \\
		\test{fid1 \word{@} \word{FILE-POSITION}}{0. 0} \\
		\test{buf 100 fid1 \word{@} \word{READ-LINE} \word{ROT} \word{DUP} \#chars \word{!}}{\\
		\tab[1.2] <TRUE> 0 line1 \word{SWAP} \word{DROP}} \\
		\test{buf \#chars \word{@} line1 \word[string]{COMPARE}}{0} \\
		\test{fid1 \word{@} \word{CLOSE-FILE}}{0}
	\end{testing}
\end{worddef}


\begin{worddef}{2142}{REPOSITION-FILE}
\item \stack{ud fileid}{ior}

	Reposition the file identified by \param{fileid} to \param{ud}.
	\param{ior} is the implementation-defined I/O result code. An
	ambiguous condition exists if the file is positioned outside
	the file boundaries.

	At the conclusion of the operation, \word{FILE-POSITION}
	returns the value \param{ud}.

	\begin{testing}\ttfamily
		\word{:} line2 \word{Sq} Line 2 blah blah blah" \word{;} \\
		\word{:} rl1 buf 100 fid1 \word{@} \word{READ-LINE} \word{;} \\
		\word[double]{2VARIABLE} fp

		\test{fn1 \word{R/W} \word{OPEN-FILE} \word{SWAP} fid1 \word{!}}{0} \\
		\test{fid1 \word{@} \word{FILE-SIZE} \word{DROP} fid1 \word{@} \word{REPOSITION-FILE}}{0} \\
		\test{fid1 \word{@} \word{FILE-SIZE}}{fid1 \word{@} \word{FILE-POSITION}}

		\test{line2 fid1 \word{@} \word{WRITE-FILE}}{0} \\
		\test{10. fid1 \word{@} \word{REPOSITION-FILE}}{0} \\
		\test{fid1 \word{@} \word{FILE-POSITION}}{10. 0}

		\test{0. fid1 \word{@} \word{REPOSITION-FILE}}{0} \\
		\test{rl1}{line1 \word{SWAP} \word{DROP} <TRUE> 0} \\
		\test{rl1}{\word{ROT} \word{DUP} \#chars \word{!}}{<TRUE> 0 line2 \word{SWAP} \word{DROP}} \\
		\test{buf \#chars \word{@} line2 \word[string]{COMPARE}}{0} \\
		\test{rl1}{0 <FALSE> 0}

		\test{fid1 \word{@} \word{FILE-POSITION} \word{ROT} \word{ROT} fp \word{2!}}{0} \\
		\test{fp \word{2@} fid1 \word{@} \word{FILE-SIZE} \word{DROP} \word[double]{D=}}{<TRUE>} \\
		\test{\word{Sq} " fid1 \word{@} \word{WRITE-LINE}}{0} \\
		\test{\word{Sq} " fid1 \word{@} \word{WRITE-LINE}}{0} \\
		\test{fp \word{2@} fid1 \word{@} \word{REPOSITION-FILE}}{0} \\
		\test{rl1}{0 <TRUE>  0} \\
		\test{rl1}{0 <TRUE>  0} \\
		\test{rl1}{0 <FALSE> 0} \\
		\test{fid1 \word{@} \word{CLOSE-FILE}}{0}
	\end{testing}
\end{worddef}


\begin{worddef}{2147}{RESIZE-FILE}
\item \stack{ud fileid}{ior}

	Set the size of the file identified by \param{fileid} to
	\param{ud}. \param{ior} is the implementation-defined I/O result
	code.

	If the resultant file is larger than the file before the
	operation, the portion of the file added as a result of the
	operation might not have been written.

	At the conclusion of the operation, \word{FILE-SIZE} returns
	the value \param{ud} and
	\hyperref{}{file}{FILE-POSITION}{\textbf{\texttt{FILE-}}} \linebreak
	\hyperref{}{file}{FILE-POSITION}{\textbf{\texttt{POSITION}}}
	returns an unspecified value.

\see \wref{file:READ-FILE}{READ-FILE},
	\wref{file:READ-LINE}{READ-LINE}.

	\begin{testing}
		\texttt{setpad} \\
		\test{fn2 \word{R/W} \word{BIN} \word{OPEN-FILE} \word{SWAP} fid2 \word{!}}{0} \\
		\test{37. fid2 \word{@} \word{RESIZE-FILE}}{0} \\
		\test{fid2 \word{@} \word{FILE-SIZE}}{37. 0} \\
		\test{0. fid2 \word{@} \word{REPOSITION-FILE}}{0} \\
		\test{cbuf buf 100 fid2 \word{@} \word{READ-FILE}}{37 0} \\
		\test{\word{PAD} 37 buf 37 \word[string]{COMPARE}}{0} \\
		\test{\word{PAD} 38 buf 38 \word[string]{COMPARE}}{1} \\
		\test{500. fid2 \word{@} \word{RESIZE-FILE}}{0} \\
		\test{fid2 \word{@} \word{FILE-SIZE}}{500. 0} \\
		\test{0. fid2 \word{@} \word{REPOSITION-FILE}}{0} \\
		\test{cbuf buf 100 fid2 \word{@} \word{READ-FILE}}{100 0} \\
		\test{\word{PAD} 37 buf 37 \word[string]{COMPARE}}{0} \\
		\test{fid2 \word{@} \word{CLOSE-FILE}}{0}
	\end{testing}
\end{worddef}


\begin{worddef}[Sq]{2165}{S"}[s-quote]
\item Extend the semantics of \wref{core:Sq}{S"} to be:

\interpret
	\stack{"ccc<quote>"}{c-addr u}

	Parse \param{ccc} delimited by \texttt{"} (double quote). Store
	the resulting string \replace{X:interpret-escape-s}{%
	\param{c-addr u} at a temporary location.
	The maximum length of the temporary buffer is
	implementation-dependent but shall be no less than 80 characters.
	Subsequent uses of \word{Sq} may overwrite the temporary buffer.
	At least one such buffer shall be provided.}{ in a transient buffer
	\param{c-addr u}.}

\compile
	\stack{"ccc<quote>"}{}

	Parse \param{ccc} delimited by \texttt{"} (double quote). Append
	the run-time semantics given below to the current definition.

\runtime
	\stack{}{c-addr u}

	Return \param{c-addr} and \param{u} that describe a string
	consisting of the characters \param{ccc}.

\see \xref[3.4.1 Parsing]{usage:parsing},
	\wref{core:Cq}{C"},
	\wref{core:Sq}{S"},
	\xref[11.3.5 Other transient regions]{file:buffers}, \\
	\place{X:interpret-escape-s}{\wref{file:Seq}{},}
	\xref{rat:file:buffers},
	\rref{file:Sq}{}.

	\begin{rationale} % A.11.6.1.2165 S"
		Typical use:
			{\ldots} \word{Sq} \emph{ccc}\texttt{"} {\ldots}

		The interpretation semantics for \word{Sq} are intended to
		provide a simple mechanism for entering a string in the
		interpretation state. Since an implementation may choose to
		provide only one buffer for interpreted strings, an
		interpreted string is subject to being overwritten by the
		next execution of \word{Sq} in interpretation state. It is
		intended that no standard words other than \word{Sq} should
		in themselves cause the interpreted string to be overwritten.
		However, since words such as \word[core]{EVALUATE},
		\word[block]{LOAD}, \word{INCLUDE-FILE} and
		\word{INCLUDED} can result in the interpretation of arbitrary
		text, possibly including instances of \word{Sq}, the
		interpreted string may be invalidated by some uses of these
		words.

		When the possibility of overwriting a string can arise, it is
		prudent to copy the string to a ``safe'' buffer allocated by
		the application.
	\end{rationale}

	\begin{testing}
		\test{\word{Sq} A String"\word{2DROP}}{}
			\tab \word{bs} \textdf{There is no space between the " and \word{2DROP}}
	\end{testing}
\end{worddef}


\begin{worddef}{2218}{SOURCE-ID}[source-i-d]
\item \stack{}{0 | -1 | fileid}

	Extend \wref{core:SOURCE-ID}{SOURCE-ID} to include text-file
	input as follows:
	\begin{center}
	  \begin{tabular}{cl}
		\hline\hline
		\word{SOURCE-ID} & Input source \\
		\hline
		\emph{fileid}	& Text file ``\emph{fileid}'' \\
		\emph{-1}		& String (via \word[core]{EVALUATE}) \\
		\emph{0}		& User input device \\
		\hline\hline
	  \end{tabular}
	\end{center}

	An ambiguous condition exists if \word{SOURCE-ID} is used when
	\word[block]{BLK} contains a non-zero value.

	\begin{testing}
		\test{\word{SOURCE-ID} \word{DUP} -1 \word{=} \word{SWAP} \word{0=} \word{OR}}{<FALSE>}
	\end{testing}
\end{worddef}

\pagebreak
\begin{worddef}{2425}{W/O}[w-o]
\item \stack{}{fam}

	\param{fam} is the implementation-defined value for selecting
	the ``write only'' file access method.

\see \wref{file:CREATE-FILE}{CREATE-FILE},
	\wref{file:OPEN-FILE}{OPEN-FILE}.
\end{worddef}


\begin{worddef}{2480}{WRITE-FILE}
\item \stack{c-addr u fileid}{ior}

	Write \param{u} characters from \param{c-addr} to the file
	identified by \param{fileid} starting at its current position.
	\param{ior} is the implementation-defined I/O result code.

	At the conclusion of the operation, \word{FILE-POSITION} returns
	the next file position after the last character written to the
	file, and \word{FILE-SIZE} returns a value greater than or equal
	to the value returned by \word{FILE-POSITION}.

\see \wref{file:READ-FILE}{READ-FILE},
	\wref{file:READ-LINE}{READ-LINE}.
\end{worddef}


\begin{worddef}{2485}{WRITE-LINE}
\item \stack{c-addr u fileid}{ior}

	Write \param{u} characters from \param{c-addr} followed by the
	implementation-dependent line terminator to the file identified
	by \param{fileid} starting at its current position. \param{ior} is
	the implementation-defined I/O result code.

	At the conclusion of the operation, \word{FILE-POSITION} returns
	the next file position after the last character written to the
	file, and \word{FILE-SIZE} returns a value greater than or equal
	to the value returned by \word{FILE-POSITION}.

\see \wref{file:READ-FILE}{READ-FILE},
	\wref{file:READ-LINE}{READ-LINE}.

	\begin{testing}\ttfamily
		\word{:} line1 \word{Sq} Line 1" \word{;}

		\test{fn1 \word{W/O} \word{OPEN-FILE} \word{SWAP} fid1 \word{!}}{0} \\
		\test{line1 fid1 \word{@} \word{WRITE-LINE}}{0} \\
		\test{fid1 \word{@} \word{CLOSE-FILE}}{0}
	\end{testing}
\end{worddef}


\subsection{File-Access extension words} % 11.6.2
\extended

\begin{worddef}{1524}{FILE-STATUS}
\item \stack{c-addr u}{x ior}

	Return the status of the file identified by the character string
	\param{c-addr u}. If the file exists, \param{ior} is zero;
	otherwise \param{ior} is the implementation-defined I/O result
	code. \param{x} contains implementation-defined information about
	the file.
\end{worddef}


\pagebreak
\begin{worddef}{1560}{FLUSH-FILE}
\item \stack{fileid}{ior}

	Attempt to force any buffered information written to the file
	referred to by \param{fileid} to be written to mass storage, and
	the size information for the file to be recorded in the storage
	directory if changed. If the operation is successful, \param{ior}
	is zero. Otherwise, it is an implementation-defined I/O result
	code.
\end{worddef}


\begin{worddef}{1714}{INCLUDE}[][X:required]
\item \stack{i*x "name"}{j*x}

	Skip leading white space and parse \param{name} delimited by a
	white space character. Push the address and length of the
	\param{name} on the stack and perform the function of
	\word{INCLUDED}.

\see \wref{file:INCLUDED}{INCLUDED},
	\rref{file:INCLUDE}{}.

	\begin{rationale} % A.11.6.2.---- INCLUDE
		Typical use:
		\begin{quote}
			\word{INCLUDE} \texttt{filename}
		\end{quote}
	\end{rationale}

	\begin{implement} % I.11.6.2.---- INCLUDE
		\word{:} \word{INCLUDE} \word{p} i*x "name" -{}- j*x ) \\
		\tab \word{PARSE-NAME} \word{INCLUDED} \word{;}
	\end{implement}

	\begin{testing} % T.11.6.2.---- INCLUDE
		See \tref{file:REQUIRED}{REQUIRED}.
	\end{testing}
\end{worddef}


\begin{worddef}{2125}{REFILL}
\item \stack{}{flag}

	Extend the execution semantics of \wref{core:REFILL}{REFILL}
	with the following:

	When the input source is a text file, attempt to read the next
	line from the text-input file. If successful, make the result
	the current input buffer, set \word[core]{toIN} to zero, and
	return \emph{true}. Otherwise return \emph{false}.

\see \wref{core:REFILL}{REFILL},
	\wref{block:REFILL}{REFILL}.
\end{worddef}


\begin{worddef}{2130}{RENAME-FILE}
\item \stack{c-addr_1 u_1 c-addr_2 u_2}{ior}

	Rename the file named by the character string \param{c-addr_1 u_1}
	to the name in the character string \param{c-addr_2 u_2}.
	\param{ior} is the implementation-defined I/O result code.

	\begin{testing}\ttfamily
		\word{:} fn3 \word{Sq} fatest3.txt" \word{;} \\
		\word{:} >end fid1 \word{@} \word{FILE-SIZE} \word{DROP} fid1 \word{@} \word{REPOSITION-FILE} \word{;}

		\test{fn3 \word{DELETE-FILE} \word{DROP}}{} \\
		\test{fn1 fn3 \word{RENAME-FILE}}{0} \\
		\word{bs} \textdf{\ Return value is undefined} \\
		\test{fn1 \word{FILE-STATUS} \word{SWAP} \word{DROP} \word{0=}}{<FALSE>} \\
		\test{fn3 \word{FILE-STATUS} \word{SWAP} \word{DROP} \word{0=}}{<TRUE> } \\
		\test{fn3 \word{R/W} \word{OPEN-FILE} \word{SWAP} fid1 \word{!}}{0} \\
		\test{>end}{0} \\
		\test{\word{Sq} Final line" fid1 \word{@} \word{WRITE-LINE}}{0} \\
		\test{fid1 \word{@} \word{FLUSH-FILE}}{0} \tab \word{bs} \textdf{Can only test \word{FLUSH-FILE} doesn't fail} \\
		\test{fid1 \word{@} \word{CLOSE-FILE}}{0}

		\word{bs} \textdf{Tidy the test folder} \\
		\test{fn3 \word{DELETE-FILE} \word{DROP}}{}
	\end{testing}
\end{worddef}


\begin{worddef}{2144}[10]{REQUIRE}[][X:required]
\item \stack{i*x "name"}{i*x}

	Skip leading white space and parse \param{name} delimited by a
	white space character. Push the address and length of the
	\param{name} on the stack and perform the function of
	\word{REQUIRED}.

\see \wref{file:REQUIRED}{REQUIRED},
	\rref{file:REQUIRE}{}.

	\begin{rationale} % A.11.6.2.---- REQUIRE
		Typical use:
		\begin{quote}
			\word{REQUIRE} \texttt{filename}
		\end{quote}
	\end{rationale}

	\begin{implement} % I.11.6.2.---- REQUIRE
		\word{:} \word{REQUIRE} \word{p} i*x "name" -{}- i*x ) \\
		\tab \word{PARSE-NAME} \word{REQUIRED} \word{;}
	\end{implement}

	\begin{testing} % T.11.6.2.---- REQUIRE
		See \tref{file:REQUIRED}{REQUIRED}.
	\end{testing}
\end{worddef}

\enlargethispage{2ex}
\begin{worddef}{2144}[50]{REQUIRED}[][X:required]
\item \stack{i*x c-addr u}{i*x}

	If the file specified by \param{c-addr u} has been \word{INCLUDED}
	or \word{REQUIRED} already, but not between the definition and
	execution of a marker (or equivalent usage of \word[tools]{FORGET}),
	discard \param{c-addr u}; otherwise, perform the function of
	\word{INCLUDED}.

	An ambiguous condition exists if a file is \word{REQUIRED} while
	it is being \word{REQUIRED} or \word{INCLUDED}.

	An ambiguous condition exists, if a marker is defined outside and
	executed inside a file or vice versa, and the file is
	\word{REQUIRED} again.

	An ambiguous condition exists if the same file is \word{REQUIRED}
	twice using different names (e.g., through symbolic links), or
	different files with the same name are \word{REQUIRED} (by doing
	some renaming between the invocations of \word{REQUIRED}).

	An ambiguous condition exists if the stack effect of including the
	file is not \stack{i*x}{i*x}.

\see \rref{file:REQUIRED}{}.

	\begin{rationale} % A.11.6.2.---- REQUIRED
		Typical use:
		\begin{quote}
			\word{Sq} \texttt{filename"} \word{REQUIRED}
		\end{quote}
	\end{rationale}

	\begin{implement}
		\dffamily
		This implementation does not implement the requirements with
		regard to \word{MARKER} and \word[tools]{FORGET} (\word{REQUIRED}
		only includes each file once, whether a marker was executed or
		not), so it is not a correct implementation on systems that
		support these words. It extends the definition of \word{INCLUDED}
		to record the name of files which have been either included or
		required previously. The names are recorded in a linked list
		held in the \texttt{included-names} variable.

		\ttfamily
		\word{:} save-mem	\word{p} addr1 u -{}- addr2 u ) \word{bs} gforth \\
		\word{bs} \textdf{copy a memory block into a newly allocated region in the heap} \\
		\tab \word{SWAP} \word{toR} \\
		\tab \word{DUP} \word[memory]{ALLOCATE} \word[exception]{THROW} \\
		\tab \word{SWAP} \word{2DUP} \word{Rfrom} \word{ROT} \word{ROT} \word{MOVE} \word{;} \\

		\word{:} name-add \word{p} addr u listp -{}- ) \\
		\tab \word{toR} save-mem \word{p} addr1 u ) \\
		\tab 3 \word{CELLS} \word[memory]{ALLOCATE} \word[exception]{THROW} \word{bs} allocate list node \\
		\tab \word{R@} \word{@} \word{OVER} \word{!} \word{bs} \textdf{set next pointer} \\
		\tab \word{DUP} \word{Rfrom} \word{!} \word{bs} \textdf{store current node in list var} \\
		\tab \word{CELL+} \word{2!} \word{;} \\

		\word{:} name-present? \word{p} addr u list -{}- f ) \\
		\tab \word{ROT} \word{ROT} \word{2toR} \word{BEGIN} \word{p} list R: addr u ) \\
		\tab[2] \word{DUP} \\
		\tab \word{WHILE} \\
		\tab[2] \word{DUP} \word{CELL+} \word{2@} \word{2R@} \word[string]{COMPARE} \word{0=} \word{IF} \\
		\tab[3]		\word{DROP} \word{2Rfrom} \word{2DROP} \word{TRUE} \word{EXIT} \\
		\tab[2]	\word{THEN} \\
		\tab[2] \word{@} \\
		\tab \word{REPEAT} \\
		\tab \word{p} \word{DROP} 0 ) \word{2Rfrom} \word{2DROP} \word{;} \\

		\word{:} name-join \word{p} addr u list -{}- ) \\
		\tab \word{toR} \word{2DUP} \word{R@} \word{@} name-present? \word{IF} \\
		\tab[2] \word{Rfrom} \word{DROP} \word{2DROP} \\
	    \tab \word{ELSE} \\
		\tab[2] \word{Rfrom} name-add \\
		\tab \word{THEN} \word{;} \\

		\word{VARIABLE} included-names 0 included-names \word{!} \\

		\word{:} included \word{p} i*x addr u -{}- j*x ) \\
		\tab \word{2DUP} included-names name-join \\
		\tab \word{INCLUDED} \word{;} \\

		\word{:} \word{REQUIRED} \word{p} i*x addr u -{}- i*x ) \\
		\tab \word{2DUP} included-names \word{@} name-present? \word{0=} \word{IF} \\
		\tab[2] included \\
		\tab \word{ELSE} \\
		\tab[2] \word{2DROP} \\
		\tab \word{THEN} \word{;}
	\end{implement}

	\begin{testing} % T.11.6.2.---- REQUIRED
		This test requires two additional files:
			\texttt{required-helper1.fs} and
			\texttt{required-helper2.fs}.
		Both of which hold the text:
		\begin{quote}
			\word{1+}
		\end{quote}
		As for the test themselves:

		\test{0 \\
		\tab \word{Sq} required-helper1.fs" \word{REQUIRED} \word{bs} Increment TOS \\
		\tab \word{REQUIRE} required-helper1.fs				\word{bs} Ignore - already loaded \\
		\tab \word{INCLUDE} required-helper1.fs				\word{bs} Increment TOS \\
		}{2} \\

		\test{0 \\
		\tab \word{INCLUDE} required-helper2.fs				\word{bs} Increment TOS \\
		\tab \word{Sq} required-helper2.fs" \word{REQUIRED} \word{bs} Ignored - already loaded \\
		\tab \word{REQUIRE} required-helper2.fs				\word{bs} Ignored - already loaded \\
		\tab \word{Sq} required-helper2.fs" \word{INCLUDED}	\word{bs} Increment TOS \\
		}{2}
	\end{testing}
\end{worddef}

\begin{worddef*}[Seq]{2266}{S\bs{"}}[s-backslash-quote][X:interpret-escape-s]
\item
	\uline{Extend the semantics of \wref{core:Seq}{} to be:}

\interpret \stack{\uline{"ccc<quote>"}}{\uline{c-addr u}}

\uline{%
	Parse \param{ccc} delimited by \texttt{"} (double quote) according to the
	translation rules given in \wref{core:Seq}{}.  Store the resulting string in a
	transient buffer \param{c-addr u}.
}

\compile \stack{\uline{"ccc<quote>"}}{}

\uline{%
	Parse \param{ccc} delimited by \texttt{"} (double quote) according to the
	translation rules given in \wref{core:Seq}{}.  Append the run-time semantics
	given below to the current definition.
}

\runtime \stack{}{\uline{c-addr u}}

\uline{%
	Return a string \param{c-addr u} describing the translation of \param{ccc}.
	A program shall not alter the returned string.
}

\see \uline{\wref{core:Seq},} \xref[11.3.4 Transient String Buffers]{file:buffers},
	\uline{\wref{file:Sq}{},} \xref[A.11.3.4 Transient String Buffers]{rat:file:buffers}.
\end{worddef*}
