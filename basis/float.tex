% !TeX root = forth.tex
% !TeX spell check = en_US

\chapter{The optional Floating-Point word set} % 12
\wordlist{floating}

\section{Introduction} % 12.1

\section{Additional terms and notation} % 12.2

\subsection{Definition of terms} % 12.2.1
\label{float:terms}

\begin{description}
\item[float-aligned address:]
	The address of a memory location at which a floating-point
	number can be accessed.

\item[double-float-aligned address:]
	The address of a memory location at which a 64-bit IEEE
	double-precision floating-point number can be accessed.

\item[single-float-aligned address:]
	The address of a memory location at which a 32-bit IEEE
	single-precision floating-point number can be accessed.

\item[IEEE floating-point number:]
	A single- or double-precision floating-point number as defined
	in \linebreak \textbf{ANSI/IEEE 754-1985}.
\end{description}

\subsection{Notation} % 12.2.2
\label{float:notation}

\stepcounter{subsubsection}
\subsubsection{Stack notation} % 12.2.2.2

Floating-point stack notation is:
\begin{quote}
	\stack[F]{before}{after}
\end{quote}

A unified stack notation is provided for systems with the environmental
restriction that the floating-point numbers are kept on the data stack.

\section{Additional usage requirements} % 12.3

\subsection{Data types} % 12.3.1
\label{float:datatypes}

Append table \ref{float:types} to table \ref{table:datatypes}.

\begin{table}[ht]
  \begin{center}
	\caption{Data Types}
	\label{float:types}
	\begin{tabular}{llr}
	\hline\hline
	\emph{Symbol} & \emph{Data type} & \emph{Size on stack} \\
	\hline
	\emph{df-addr}	& double-float-aligned address	& 1 cell \\
	\emph{f-addr}	& float-aligned address				& 1 cell \\
	\emph{sf-addr}	& single-float-aligned address	& 1 cell \\
	\hline\hline
	\end{tabular}
  \end{center}
\end{table}

\subsubsection{Addresses} % 12.3.1.1
\label{float:addr}

The set of float-aligned addresses is an implementation-defined
subset of the set of aligned addresses. Adding the size of a
floating-point number to a float-aligned address shall produce
a float-aligned address.

The set of double-float-aligned addresses is an implementation-defined
subset of the set of aligned addresses. Adding the size of a 64-bit
IEEE double-precision floating-point number to a double-float-aligned
address shall produce a double-float-aligned address.

The set of single-float-aligned addresses is an implementation-defined
subset of the set of aligned addresses. Adding the size of a 32-bit
IEEE single-precision floating-point number to a single-float-aligned
address shall produce a single-float-aligned address.

\subsubsection{Floating-point numbers} % 12.3.1.2
\label{float:num}

The internal representation of a floating-point number, including the
format and precision of the significand and the format and range of
the exponent, is implementation defined.

Any rounding or truncation of floating-point numbers is implementation
defined.

\subsection{Floating-point operations} % 12.3.2
\label{float:ops}

``Round to nearest'' means round the result of a floating-point
operation to the representable value nearest the result. If the two
nearest representable values are equally near the result, the one
having zero as its least significant bit shall be delivered.

``Round toward negative infinity'' means round the result of a
floating-point operation to the representable value nearest to and
no greater than the result.

``Round toward zero'' means round the result of a floating-point
operation to the representable value nearest to zero, frequently
referred to as ``truncation''.

\subsection{Floating-point stack} % 12.3.3
\label{float:stack}

A last in, first out list that shall be used by all floating-point
operators.

The width of the floating-point stack is implementation-defined.
The floating-point stack shall be separate from the data and return
stacks.

The size of a floating-point stack shall be at least 6 items.

A program that depends on the floating-point stack being larger than
six items has an environmental dependency.

\subsection{Environmental queries} % 12.3.4

Append table \ref{float:env} to table \ref{table:env}.

See: \xref[3.2.6 Environmental queries]{usage:env}.

\begin{table}[ht]
  \begin{center}
	\caption{Environmental Query Strings}
	\label{float:env}
	\begin{tabular}{p{9em}rcp{0.42\textwidth}}
		\hline\hline
		\multicolumn{2}{l}{String \hfill Value data type} & Constant? & Meaning \\
		\hline
		\texttt{FLOATING-STACK} & \emph{n} & yes &
			the maximum depth of the separate floating-point stack.
			On systems with the environmental restriction of keeping
			floating-point items on the data stack, \emph{n} = 0. \\
		\texttt{MAX-FLOAT}		& \emph{r}			& yes	&
			largest usable floating-point number \\
		\hline\hline
	\end{tabular}
  \end{center}
\end{table}


\subsection{Address alignment} % 12.3.5

Since the address returned by a \word[core]{CREATE}d word is not
necessarily aligned for any particular class of floating-point
data, a program shall align the address (to be float aligned,
single-float aligned, or double-float aligned) before accessing
floating-point data at the address.

See: \xref[3.3.3.1 Address Alignment]{usage:aaddr},
	\xref[12.3.1.1 Addresses]{float:addr}.

\subsection{Variables} % 12.3.6

A program may address memory in data space regions made available
by \word{FVARIABLE}. These regions may be non-contiguous with
regions subsequently allocated with \word[core]{,} (comma) or
\word[core]{ALLOT}. See: \xref[3.3.3.3 Variables]{usage:var}.

\subsection{Text interpreter input number conversion} % 12.3.7
\label{float:conv}

If the Floating-Point word set is present in the dictionary and
the current base is \word[core]{DECIMAL}, the input number-conversion
algorithm shall be extended to recognize floating-point numbers in
this form:

\begin{center}
\begin{tabular}{r@{ \textsf{:=} }l}
Convertible string  & \arg{significand}\arg{exponent} \\
  \arg{significand} & [\arg{sign}]\arg{digits}[\textbf{.}\arg{digits0}] \\
     \arg{exponent} & \textbf{E}[\arg{sign}]\arg{digits0} \\
         \arg{sign} & \{ \textbf{+} {\textbar} \textbf{-} \} \\
       \arg{digits} & \arg{digit}\arg{digits0} \\
      \arg{digits0} & \arg{digit}* \\
        \arg{digit} & \{ \textbf{0} {\textbar} \textbf{1} {\textbar}
						 \textbf{2} {\textbar} \textbf{3} {\textbar}
						 \textbf{4} {\textbar} \textbf{5} {\textbar}
						 \textbf{6} {\textbar} \textbf{7} {\textbar}
						 \textbf{8} {\textbar} \textbf{9} \} \\
\end{tabular}
\end{center}

These are examples of valid representations of floating-point numbers
in program source:

\begin{center}
\texttt{1E \qquad 1.E \qquad 1.E0 \qquad +1.23E-1 \qquad -1.23E+1}
\end{center}

See:
	\xref[3.4.1.3 Text interpreter input number conversion]{usage:numbers},
	\wref{floating:toFLOAT}{>FLOAT}.

\section{Additional documentation requirements} % 12.4

\subsection{System documentation} % 12.4.1

\subsubsection{Implementation-defined options} % 12.4.1.1
\begin{itemize}
\item format and range of floating-point numbers
	(\xref[12.3.1 Data types]{float:datatypes},
	 \wref{floating:REPRESENT}{REPRESENT})
\item results of \wref{floating:REPRESENT}{REPRESENT} when
	\emph{float} is out of range;
\item rounding or truncation of floating-point numbers
	(\xref[12.3.1.2 Floating-point numbers]{float:num});
\item size of floating-point stack
	(\xref[12.3.3 Floating-point stack]{float:stack});
\item width of floating-point stack
	(\xref[12.3.3 Floating-point stack]{float:stack}).
\end{itemize}

\subsubsection{Ambiguous conditions} % 12.4.1.2
\label{float:ambiguous}
\begin{itemize}
\item \word{DF@} or \word{DF!} is used with an address that is not
	double-float aligned;
\item \word{F@} or \word{F!} is used with an address that is not
	float aligned;
\item floating point result out of range
	(e.g., in \wref{floating:F/}{F/});
\item \word{SF@} or \word{SF!} is used with an address that is not
	single-float aligned;
\item \word[core]{BASE} is not decimal
	(\wref{floating:REPRESENT}{REPRESENT},
	 \wref{floating:Fd}{F.},
	 \wref{floating:FEd}{FE.},
	 \wref{floating:FSd}{FS.});
\item both arguments equal zero (\wref{floating:FATAN2}{FATAN2});
\item cosine of argument is zero for \wref{floating:FTAN}{FTAN};
\item \param{d} can't be precisely represented as \emph{float} in
	\wref{floating:DtoF}{D>F};
\item dividing by zero (\wref{floating:F/}{F/});
\item exponent too big for conversion
	(\wref{floating:DF!}{DF!},
	 \wref{floating:DF@}{DF@},
	 \wref{floating:SF!}{SF!},
	 \wref{floating:SF@}{SF@});
\item \emph{float} less than one (\wref{floating:FACOSH}{FACOSH});
\item \emph{float} less than or equal to minus-one
	(\wref{floating:FLNP1}{FLNP1});
\item \emph{float} less than or equal to zero
	(\wref{floating:FLN}{FLN},
	 \wref{floating:FLOG}{FLOG});
\item \emph{float} less than zero
	(\wref{floating:FSQRT}{FSQRT});
\item \emph{float} magnitude greater than one
	(\wref{floating:FACOS}{FACOS},
	 \wref{floating:FASIN}{FASIN},
	 \wref{floating:FATANH}{FATANH});
\item integer part of \emph{float} can't be represented by \param{d} in
	\wref{floating:FtoD}{F>D};
\item string larger than pictured-numeric output area
	(\wref{floating:Fd}{F.},
	 \wref{floating:FEd}{FE.},
	 \wref{floating:FSd}{FS.}).
\item \place{x:stoftos}{\param{n} can't be precisely represented as
	\emph{float} in \wref{floating:StoF}{S>F};}
\item \place{x:stoftos}{integer part of \emph{float} can't be represented
	by \param{n} in \wref{floating:FtoS}{F>S}.}
\end{itemize}

\subsubsection{Other system documentation} % 12.4.1.3
\begin{itemize}
\item no additional requirements.
\end{itemize}

\subsubsection{Environmental restrictions} % 12.4.1.4
\label{float:restrictions}

\begin{itemize}
\item Keeping floating-point numbers on the data stack.
\end{itemize}

\subsection{Program documentation} % 12.4.2

\subsubsection{Environmental dependencies} % 12.4.2.1
\label{float:dependencies}

\begin{itemize}
\item requiring the floating-point stack to be larger than six items
	(\xref[12.3.3 Floating-point stack]{float:stack}).
\item requiring floating-point numbers to be kept on the data stack,
	with $n$ cells per floating point number.
\end{itemize}

\subsubsection{Other program documentation} % 12.4.2.2
\begin{itemize}
\item no additional requirements.
\end{itemize}

\section{Compliance and labeling} % 12.5

\cbstart\patch{F12}
\subsection[Forth-\snapshot systems]{Forth-\snapshot\strike{-1}{25} systems} % 12.5.1
\cbend

The phrase ``Providing the Floating-Point word set'' shall be
appended to the label of any Standard System that provides all of
the Floating-Point word set.

The phrase ``Providing \emph{name(s)} from the Floating-Point
Extensions word set'' shall be appended to the label of any Standard
System that provides portions of the Floating-Point Extensions word
set.

The phrase ``Providing the Floating-Point Extensions word set'' shall
be appended to the label of any Standard System that provides all of
the Floating-Point and Floating-Point Extensions word sets.

\cbstart\patch{F12}
\subsection[Forth-\snapshot programs]{Forth-\snapshot\strike{-1}{25} programs} % 12.5.2
\cbend

The phrase ``Requiring the Floating-Point word set'' shall be
appended to the label of Standard Programs that require the system
to provide the Floating-Point word set.

The phrase ``Requiring \emph{name(s)} from the Floating-Point
Extensions word set'' shall be appended to the label of Standard
Programs that require the system to provide portions of the
Floating-Point Extensions word set.

The phrase ``Requiring the Floating-Point Extensions word set'' shall
be appended to the label of Standard Programs that require the system
to provide all of the Floating-Point and Floating-Point Extensions
word sets.


\section{Glossary} % 12.6

\subsection{Floating-Point words} % 12.6.1


\begin{worddef}[toFLOAT]{0558}{>FLOAT}[to-float]
\item \stack{c-addr u}{true | false}
	\stack[F]{}{r |~} or
	\stack{c-addr u}{r true | false}

	An attempt is made to convert the string specified by \param{c-addr}
	and \param{u} to internal floating-point representation. If the
	string represents a valid floating-point number in the syntax below,
	its value \param{r} and \param{true} are returned. If the string does
	not represent a valid floating-point number only \param{false} is
	returned.

	A string of blanks should be treated as a special case
	representing zero.

	The syntax of a convertible string
	\begin{center}
		\begin{tabular}{r@{ \textsf{:=} }l}
							& \arg{significand}[\arg{exponent}] \\
		\arg{significand}	& [\arg{sign}]\{\arg{digits}[\textbf{.}\arg{digits0}]
								{\textbar} \textbf{.}\arg{digits} \} \\
		\arg{exponent}		& \arg{marker}\arg{digits0} \\
		\arg{marker}		& \{\arg{e-form} {\textbar} \arg{sign-form}\}\\
		\arg{e-form}		& \arg{e-char}[\arg{sign-form}] \\
		\arg{sign-form} 	& \{ \textbf{+} {\textbar} \textbf{-} \} \\
		\arg{e-char}		& \{ \textbf{D} {\textbar} \textbf{d} {\textbar}
								 \textbf{E} {\textbar} \textbf{e} \} \\
		\end{tabular}
	\end{center}

\see \rref{floating:toFLOAT}{}.

	\begin{rationale} % A.12.6.1.0558 >FLOAT
		\word{toFLOAT} enables programs to read floating-point data in
		legible ASCII format. It accepts a much broader syntax than
		does the text interpreter since the latter defines rules for
		composing source programs whereas \word{toFLOAT} defines rules
		for accepting data. \word{toFLOAT} is defined as broadly as is
		feasible to permit input of data from Forth\place{F12}{-\snapshot} systems as well
		as other widely used standard programming environments.

		This is a synthesis of common FORTRAN practice. Embedded spaces
		are explicitly forbidden in much scientific usage, as are other
		field separators such as comma or slash.

		While \word{toFLOAT} is not required to treat a string of blanks
		as zero, this behavior is strongly encouraged, since a future
		version of this standard may include such a requirement.
	\end{rationale}
\end{worddef}


\begin{worddef}[DtoF]{1130}{D>F}[d-to-f]
\item \stack{d}{} \stack[F]{}{r} or
	\stack{d}{r}

	\param{r} is the floating-point equivalent of \param{d}. An ambiguous
	condition exists if \param{d} cannot be precisely represented as a
	floating-point value.
\end{worddef}


\begin{worddef}{1400}{F!}[f-store]
\item \stack{f-addr}{} \stack[F]{r}{} or
	\stack{r f-addr}{}

	Store \param{r} at \param{f-addr}.
\end{worddef}


\begin{worddef}{1410}{F*}[f-star]
\item \stack[F]{r_1 r_2}{r_3} or
	\stack{r_1 r_2}{r_3}

	Multiply \param{r_1} by \param{r_2} giving \param{r_3}.
\end{worddef}


\begin{worddef}{1420}{F+}[f-plus]
\item \stack[F]{r_1 r_2}{r_3} or
	\stack{r_1 r_2}{r_3}

	Add \param{r_1} to \param{r_2} giving the sum \param{r_3}.
\end{worddef}


\begin{worddef}{1425}{F-}[f-minus]
\item \stack[F]{r_1 r_2}{r_3} or
	\stack{r_1 r_2}{r_3}

	Subtract \param{r_2} from \param{r_1}, giving \param{r_3}.
\end{worddef}


\begin{worddef}{1430}{F/}[f-slash]
\item \stack[F]{r_1 r_2}{r_3} or
	\stack{r_1 r_2}{r_3}

	Divide \param{r_1} by \param{r_2}, giving the quotient
	\param{r_3}. An ambiguous condition exists if \param{r_2} is
	zero, or the quotient lies outside 	of the range of a
	floating-point number.
\end{worddef}


\begin{worddef}[F0less]{1440}{F0<}[f-zero-less-than]
\item \stack{}{flag} \stack[F]{r}{} or
	\stack{r}{flag}

	\param{flag} is true if and only if \param{r} is less than
	zero.
\end{worddef}


\begin{worddef}{1450}{F0=}[f-zero-equals]
\item \stack{}{flag} \stack[F]{r}{} or
	\stack{r}{flag}

	\param{flag} is true if and only if \param{r} is equal to
	zero.
\end{worddef}


\begin{worddef}[Fless]{1460}{F<}[f-less-than]
\item \stack{}{flag} \stack[F]{r_1 r_2}{} or
	\stack{r_1 r_2}{flag}

	\param{flag} is true if and only if \param{r_1} is less than
	\param{r_2}.
\end{worddef}


\begin{worddef}[FtoD]{1470}{F>D}[f-to-d]
\item \stack{}{d} \stack[F]{r}{} or
	\stack{r}{d}

	\param{d} is the double-cell signed-integer equivalent of the
	integer portion of \param{r}. The fractional portion of \param{r}
	is discarded. An ambiguous condition exists if the integer portion
	of \param{r} cannot be precisely represented as a double-cell
	signed integer.
\end{worddef}


\begin{worddef}{1472}{F@}[f-fetch]
\item \stack{f-addr}{} \stack[F]{}{r} or
	\stack{f-addr}{r}

	\param{r} is the value stored at \param{f-addr}.
\end{worddef}


\begin{worddef}{1479}{FALIGN}[f-align]
\item \stack{}{}

	If the data-space pointer is not float aligned, reserve enough
	data space to make it so.
\end{worddef}


\begin{worddef}{1483}{FALIGNED}[f-aligned]
\item \stack{addr}{f-addr}

	\param{f-addr} is the first float-aligned address greater than
	or equal to \param{addr}.
\end{worddef}


\begin{worddef}{1492}{FCONSTANT}[f-constant]
\item \stack{"<spaces>name"}{} \stack[F]{r}{} or
	\stack{r "<spaces>name"}{}

	Skip leading space delimiters. Parse \param{name} delimited by a
	space. Create a definition for \param{name} with the execution
	semantics defined below.

	\param{name} is referred to as an ``f-constant''.

\execute[name]
	\stack{}{} \stack[F]{}{r} or
	\stack{}{r}

	Place \param{r} on the floating-point stack.

\see \xref[3.4.1 Parsing]{usage:parsing},
	\rref{floating:FCONSTANT}{}.

	\begin{rationale} % A.12.6.1.1492 FCONSTANT
		Typical use: \texttt{r} \word[floating]{FCONSTANT} \param{name}
	\end{rationale}
\end{worddef}


\begin{worddef}{1497}{FDEPTH}[f-depth]
\item \stack{}{+n}

	\param{+n} is the number of values contained on the
	floating-point stack. If the system has an environmental
	restriction of keeping the floating-point numbers on the
	data stack, \param{+n} is the current number of possible
	floating-point values contained on the data stack.
\end{worddef}


\begin{worddef}{1500}{FDROP}[f-drop]
\item \stack[F]{r}{} or
	\stack{r}{}

	Remove \param{r} from the floating-point stack.
\end{worddef}


\begin{worddef}{1510}{FDUP}[f-dupe]
\item \stack[F]{r}{r r} or
	\stack{r}{r r}

	Duplicate \param{r}.
\end{worddef}


\begin{worddef}{1552}{FLITERAL}[f-literal]
\interpret
	Interpretation semantics for this word are undefined.

\compile
	\stack[F]{r}{} or
	\stack{r}{}

	Append the run-time semantics given below to the current
	definition.

\runtime
	\stack[F]{}{r} or
	\stack{}{r}

	Place \param{r} on the floating-point stack.

\see \rref{floating:FLITERAL}{}.

	\begin{rationale} % A.12.6.1.1552 FLITERAL
		Typical use:
			\word[core]{:} \texttt{X} {\ldots}
				\word[core]{[} {\ldots} \word[core]{p} r ) \word[core]{]}
				\word{FLITERAL} {\ldots}
			\word[core]{;}
	\end{rationale}
\end{worddef}


\begin{worddef}{1555}{FLOAT+}[float-plus]
\item \stack{f-addr_1}{f-addr_2}

	Add the size in address units of a floating-point number to
	\param{f-addr_1}, giving \param{f-addr_2}.
\end{worddef}


\begin{worddef}{1556}{FLOATS}
\item \stack{n_1}{n_2}

	\param{n_2} is the size in address units of \param{n_1} floating-point
	numbers.
\end{worddef}


\begin{worddef}{1558}{FLOOR}
\item \stack[F]{r_1}{r_2} or
	\stack{r_1}{r_2}

	Round \param{r_1} to an integral value using the ``round toward
	negative infinity'' rule, giving \param{r_2}.

\see \xref{float:ops}, \wref{floating:FROUND}{},
	\wref{floating:FTRUNC}{}.
\end{worddef}


\begin{worddef}{1562}{FMAX}[f-max]
\item \stack[F]{r_1 r_2}{r_3} or
	\stack{r_1 r_2}{r_3}

	\param{r_3} is the greater of \param{r_1} and \param{r_2}.
\end{worddef}


\begin{worddef}{1565}{FMIN}[f-min]
\item \stack[F]{r_1 r_2}{r_3} or
	\stack{r_1 r_2}{r_3}

	\param{r_3} is the lesser of \param{r_1} and \param{r_2}.
\end{worddef}


\begin{worddef}{1567}{FNEGATE}[f-negate]
\item \stack[F]{r_1}{r_2} or
	\stack{r_1}{r_2}

	\param{r_2} is the negation of \param{r_1}.
\end{worddef}


\begin{worddef}{1600}{FOVER}[f-over]
\item \stack[F]{r_1 r_2}{r_1 r_2 r_1} or
	\stack{r_1 r_2}{r_1 r_2 r_1}

	Place a copy of \param{r_1} on top of the floating-point stack.
\end{worddef}


\begin{worddef}{1610}{FROT}[f-rote]
\item \stack[F]{r_1 r_2 r_3}{r_2 r_3 r_1} or
	\stack{r_1 r_2 r_3}{r_2 r_3 r_1}

	Rotate the top three floating-point stack entries.
\end{worddef}


\begin{worddef}{1612}{FROUND}[f-round]
\item \stack[F]{r_1}{r_2} or
	\stack{r_1}{r_2}

	Round \param{r_1} to an integral value using the ``round to nearest''
	rule, giving \param{r_2}.

\see \xref[12.3.2 Floating-point operations]{float:ops},
	\wref{floating:FLOOR}{}, \wref{floating:FTRUNC}{}.
\end{worddef}


\begin{worddef}{1620}{FSWAP}[f-swap]
\item \stack[F]{r_1 r_2}{r_2 r_1} or
	\stack{r_1 r_2}{r_2 r_1}

	Exchange the top two floating-point stack items.
\end{worddef}


\begin{worddef}{1630}{FVARIABLE}[f-variable]
\item \stack{"<spaces>name"}{}

	Skip leading space delimiters. Parse \param{name} delimited by a
	space. Create a definition for \param{name} with the execution
	semantics defined below. Reserve \texttt{1} \word{FLOATS}
	address units of data space at a float-aligned address.

	\param{name} is referred to as an ``f-variable''.

\execute[name]
	\stack{}{f-addr}

	\param{f-addr} is the address of the data space reserved by
	\word{FVARIABLE} when it created \param{name}. A program is
	responsible for initializing the contents of the reserved
	space.

\see \xref[3.4.1 Parsing]{usage:parsing},
	\rref{floating:FVARIABLE}{}.

	\begin{rationale} % A.12.6.1.1630 FVARIABLE
		Typical use:
			\word{FVARIABLE} \param{name}
	\end{rationale}
\end{worddef}


\begin{worddef}{2143}{REPRESENT}
\item \stack{c-addr u}{n flag_1 flag_2} \stack[F]{r}{} or
	\stack{r c-addr u}{n flag_1 flag_2}

	At \param{c-addr}, place the character-string external
	representation of the significand of the floating-point number
	\param{r}. Return the decimal-base exponent as \param{n}, the sign as
	\param{flag_1} and ``valid result'' as \param{flag_2}.
	The character string shall consist of the \param{u} most significant
	digits of the significand represented as a decimal fraction with
	the implied decimal point to the left of the first digit, and
	the first digit zero only if all digits are zero. The significand
	is rounded to \param{u} digits following the ``round to nearest'' rule;
	\param{n} is adjusted, if necessary, to correspond to the rounded
	magnitude of the significand. If \param{flag_2} is \emph{true}
	then \param{r} was in the implementation-defined range of floating-point
	numbers. If \param{flag_1} is \emph{true} then \param{r} is negative.

	An ambiguous condition exists if the value of \word[core]{BASE}
	is not decimal ten.

	When \param{flag_2} is \emph{false}, \param{n} and \param{flag_1}
	are implementation defined, as are the contents of \param{c-addr}.
	Under these circumstances, the string at \param{c-addr} shall
	consist of graphic characters.

\see \xref[3.2.1.2 Digit conversion]{usage:digits},
	\wref{core:BASE}{BASE},
	\xref[12.3.2 Floating-point operations]{float:ops}, \\
	\rref{floating:REPRESENT}{}.

	\begin{rationale} % A.12.6.1.2143 REPRESENT
		This word provides a primitive for floating-point display.
		Some floating-point formats, including those specified by
		IEEE-754, allow representations of numbers outside of an
		implementation-defined range. These include plus and minus
		infinities, denormalized numbers, and others. In these cases
		we expect that \word{REPRESENT} will usually be implemented
		to return appropriate character strings, such as ``+infinity''
		or ``nan'', possibly truncated.
	\end{rationale}
\end{worddef}


\subsection{Floating-Point extension words} % 12.6.2
\extended

\begin{worddef}{1203}{DF!}[d-f-store]
\item \stack{df-addr}{} \stack[F]{r}{} or
	\stack{r df-addr}{}

	Store the floating-point number \param{r} as a 64-bit IEEE
	double-precision number at \param{df-addr}. If the significand
	of the internal representation of \param{r} has more precision than
	the IEEE double-precision format, it will be rounded using the
	``round to nearest'' rule. An ambiguous condition exists if the
	exponent of \param{r} is too large to be accommodated in IEEE
	double-precision format.

\see \xref[12.3.1.1 Addresses]{float:addr},
	\xref[12.3.2 Floating-point operations]{float:ops}.
\end{worddef}


\begin{worddef}{1204}{DF@}[d-f-fetch]
\item \stack{df-addr}{} \stack[F]{}{r} or
	\stack{df-addr}{r}

	Fetch the 64-bit IEEE double-precision number stored at
	\param{df-addr} to the floating-point stack as \param{r} in the
	internal representation. If the IEEE double-precision
	significand has more precision than the internal representation
	it will be rounded to the internal representation using the
	``round to nearest'' rule. An ambiguous condition exists if the
	exponent of the IEEE double-precision representation is too
	large to be accommodated by the internal representation.

\see \xref[12.3.1.1 Addresses]{float:addr},
	\xref[12.3.2 Floating-point operations]{float:ops}.
\end{worddef}


\begin{worddef}{1205}{DFALIGN}[d-f-align]
\item \stack{}{}

	If the data-space pointer is not double-float aligned, reserve
	enough data space to make it so.

\see \xref[12.3.1.1 Addresses]{float:addr}.
\end{worddef}


\begin{worddef}{1207}{DFALIGNED}[d-f-aligned]
\item \stack{addr}{df-addr}

	\param{df-addr} is the first double-float-aligned address greater
	than or equal to \param{addr}.

\see \xref[12.3.1.1 Addresses]{float:addr}.
\end{worddef}

% ---------------------------------------------------------

\begin{worddef}{1207}[40]{DFFIELD:}[d-f-field-colon][X:structures]
\item \stack{n_1 "<spaces>name"}{n_2}

	Skip leading space delimiters. Parse \param{name} delimited by
	a space. \param{Offset} is the first double-float aligned value
	greater than or equal to \param{n_1}.  \param{n_2 = offset + 1}
	double-float.

	Create a definition for \param{name} with the execution semantics
	given below.

\execute[name]
\cbstart\patch{ed12}
	\stack{\sout{a-}addr_{\uline{1}}}{\sout{df-}addr_{\uline{2}}}
\cbend

	Add the \param{offset} calculated during the compile time action to
\cbstart\param{\sout{a-}addr_{\uline{1}}}\cbend\ 
	giving the double-float aligned address
\cbstart\param{\sout{df-}addr_{\uline{2}}}.\cbend

\see \wref{facility:+FIELD}{+FIELD},
	\wref{facility:BEGIN-STRUCTURE}{BEGIN-STRUCTURE}, \\
	\wref{facility:END-STRUCTURE}{END-STRUCTURE},
	\rref{facility:FIELD:}{}.
\end{worddef}

% ---------------------------------------------------------

\begin{worddef}{1208}{DFLOAT+}[d-float-plus]
\item \stack{df-addr_1}{df-addr_2}

	Add the size in address units of a 64-bit IEEE double-precision
	number to \param{df-addr_1}, giving \param{df-addr_2}.

\see \xref[12.3.1.1 Addresses]{float:addr}.
\end{worddef}


\begin{worddef}{1209}{DFLOATS}[d-floats]
\item \stack{n_1}{n_2}

	\param{n_2} is the size in address units of \param{n_1} 64-bit
	IEEE double-precision numbers.
\end{worddef}


\begin{worddef}{1415}{F**}[f-star-star]
\item \stack[F]{r_1 r_2}{r_3} or
	\stack{r_1 r_2}{r_3}

	Raise \param{r_1} to the power \param{r_2}, giving the product
	\param{r_3}.
\end{worddef}


\begin{worddef}[Fd]{1427}{F.{}}[f-dot]
\item \stack{}{} \stack[F]{r}{} or
	\stack{r}{}

	Display, with a trailing space, the top number on the
	floating-point stack using fixed-point notation:
	\begin{quote}
		[\textbf{-}] \arg{digits}\textbf{.}\arg{digits0}
	\end{quote}
	An ambiguous condition exists if the value of \word[core]{BASE}
	is not (decimal) ten or if the character string representation
	exceeds the size of the pictured numeric output string buffer.

\see \wref{floating:toFLOAT}{>FLOAT},
	\rref{floating:Fd}{}.

	\begin{rationale} % A.12.6.1.1427 F.
		For example, \texttt{1E3} \replace{ed12}{\textbf{\texttt{F}}}{ \word{Fd}} displays \texttt{1000.} .
	\end{rationale}
\end{worddef}


% ========== x:stoftos ==========

\begin{worddef*}[FtoS]{1471}{F>S}[F to S][x:stoftos]
\item \stack{}{n} \stack[F]{r}{} or \stack{r}{n}

	\uline{\param{n} is the single-cell signed-integer equivalent of the
	integer portion of \param{r}.  The fractional portion of
	\param{r} is discarded.  An ambiguous condition exists if
	the integer portion of \param{r} cannot be precisely
	represented as a single-cell signed integer.}

\note \uline{Rounding the floating-point value prior to calling \word{FtoS} is
	advised.}

\see \uline{\wref{floating:StoF}{}.}

	\begin{implement}
		\uline{\word{:} \word{FtoS} \word{p} r -{}- n )} \\
		\tab \uline{\word{FtoD} \word[double]{DtoS}} \\
		\uline{\word{;}}
	\end{implement}
\end{worddef*}

% ==============================

\begin{worddef}{1474}{FABS}[f-abs]
\item \stack[F]{r_1}{r_2} or
	\stack{r_1}{r_2}

	\param{r_2} is the absolute value of \param{r_1}.
\end{worddef}


\begin{worddef}{1476}{FACOS}[f-a-cos]
\item \stack[F]{r_1}{r_2} or
	\stack{r_1}{r_2}

	\param{r_2} is the principal radian angle whose cosine is
	\param{r_1}. An ambiguous condition exists if \param{|r_1|} is
	greater than one.
\end{worddef}


\begin{worddef}{1477}{FACOSH}[f-a-cosh]
\item \stack[F]{r_1}{r_2} or
	\stack{r_1}{r_2}

	\param{r_2} is the floating-point value whose hyperbolic cosine is
	\param{r_1}. An ambiguous condition exists if \param{r_1} is less
	than one.
\end{worddef}


\begin{worddef}{1484}{FALOG}[f-a-log]
\item \stack[F]{r_1}{r_2} or
	\stack{r_1}{r_2}

	Raise ten to the power \param{r_1}, giving \param{r_2}.
\end{worddef}


\begin{worddef}{1486}{FASIN}[f-a-sine]
\item \stack[F]{r_1}{r_2} or
	\stack{r_1}{r_2}

	\param{r_2} is the principal radian angle whose sine is \param{r_1}.
	An ambiguous condition exists if \param{|r_1|} is greater than one.
\end{worddef}


% ---------------------------------------------------------

\begin{worddef}{1487}{FASINH}[f-a-cinch]
\item \stack[F]{r_1}{r_2} or
	\stack{r_1}{r_2}

	\param{r_2} is the floating-point value whose hyperbolic sine is
	\param{r_1}.
\end{worddef}
% ---------------------------------------------------------


\begin{worddef}{1488}{FATAN}[f-a-tan]
\item \stack[F]{r_1}{r_2} or
	\stack{r_1}{r_2}

	\param{r_2} is the principal radian angle whose tangent is
	\param{r_1}.
\end{worddef}

% ---------------------------------------------------------

\begin{worddef}{1489}{FATAN2}[f-a-tan-two]
\item \stack[F]{r_1 r_2}{r_3} or
	\stack{r_1 r_2}{r_3}

	\param{r_3} is the principal radian angle (between -$\pi$ and $\pi$)
	whose tangent is \param{r_1/r_2}.
	A system that returns false for ``\texttt{-0E 0E 0E \word{Ftilde}}''
	shall return a value (approximating) $-\pi$ when \param{r_1} $=$ 0E
	and \param{r_2} is negative.
	An ambiguous condition exists if \param{r_1} and \param{r_2} are
	zero.

\see \rref{floating:FATAN2}{}.

	\begin{rationale} % A.12.6.1.1489 FATAN2
		\word{FSINCOS} and \word{FATAN2} are a complementary pair of
		operators which convert angles to 2-vectors and vice-versa.
		They are essential to most geometric and physical applications
		since they correctly and unambiguously handle this conversion
		in all cases except null vectors, even when the tangent of the
		angle would be infinite.

		\word{FSINCOS} returns a Cartesian unit vector in the direction
		of the given angle, measured counter-clockwise from the positive
		X-axis. The order of results on the stack, namely $y$ underneath
		$x$, permits the 2-vector data type to be additionally viewed
		and used as a ratio approximating the tangent of the angle.
		Thus the phrase \word{FSINCOS} \word{F/} is functionally
		equivalent to \word{FTAN}, but is useful over only a limited
		and discontinuous range of angles, whereas \word{FSINCOS} and
		\word{FATAN2} are useful for all angles.
\remove{ed12}{This ordering has
		been found convenient for nearly two decades, and has the
		added benefit of being easy to remember. A corollary to this
		observation is that vectors in general should appear on the
		stack in this order.}

		The argument order for \word{FATAN2} is the same, converting a
		vector in the conventional representation to a scalar angle.
		Thus, for all angles, \word{FSINCOS} \word{FATAN2} is an identity
		within the accuracy of the arithmetic and the argument range of
		\word{FSINCOS}. Note that while \word{FSINCOS} always returns a
		valid unit vector, \word{FATAN2} will accept any non-null vector.
		An ambiguous condition exists if the vector argument to
		\word{FATAN2} has zero magnitude.
	\end{rationale}

	\begin{testing}\ttfamily
		\word[tools]{[UNDEFINED]} ~NaN \word[tools]{[IF]} ~0e 0e \word{F/} \word{FCONSTANT} ~NaN \word[tools]{[THEN]} \\
		\word[tools]{[UNDEFINED]} +Inf \word[tools]{[IF]} ~1e 0e \word{F/} \word{FCONSTANT} +Inf \word[tools]{[THEN]} \\
		\word[tools]{[UNDEFINED]} -Inf \word[tools]{[IF]} -1e 0e \word{F/} \word{FCONSTANT} -Inf \word[tools]{[THEN]}

		\word{TRUE} verbose \word{!} \\
		\word{DECIMAL}

		\textdf{The test harness default for \texttt{EXACT?} is TRUE.
			Uncomment the following line if your system needs it to
			be FALSE} \\
		\word{bs} SET-NEAR

		\word{VARIABLE} \#errors    0 \#errors \word{!}

		\word{:NONAME} \word{p} c-addr u -{}- ) \\
		\tab \word{p} \textdf{Display an error message followed by the
			line that had the error\@.} ) \\
		\tab 1 \#errors \word{+!} error1 \word{;} error-xt \word{!}

%		\word[tools]{[UNDEFINED]} $\backslash\backslash$ \word[tools]{[IF]} \\
%		\tab \word{:} $\backslash\backslash$ \word{p} -{}- )
%			-1 \word{PARSE} \word{2DROP} \word{BEGIN} \word{REFILL} \word{0=} \word{UNTIL} \word{;} \\
%		\word[tools]{[THEN]} \\

		\word[tools]{[UNDEFINED]} pi \word[tools]{[IF]} \\
		\tab 0.3141592653589793238463E1 \word{FCONSTANT} pi \\
		\word[tools]{[THEN]}

		\word[tools]{[UNDEFINED]} -pi \word[tools]{[IF]} \\
		\tab pi \word{FNEGATE} \word{FCONSTANT} -pi \\
		\word[tools]{[THEN]}

		\word{FALSE} \word[tools]{[IF]} \\
		\tab ~0.7853981633974483096157E0 \word{FCONSTANT} ~~pi/4 \\
		\tab -0.7853981633974483096157E0 \word{FCONSTANT} ~-pi/4 \\
		\tab ~0.1570796326794896619231E1 \word{FCONSTANT} ~~pi/2 \\
		\tab -0.1570796326794896619231E1 \word{FCONSTANT} ~-pi/2 \\
		\tab ~0.4712388980384689857694E1 \word{FCONSTANT} ~3pi/2 \\
		\tab ~0.2356194490192344928847E1 \word{FCONSTANT} ~3pi/4 \\
		\tab -0.2356194490192344928847E1 \word{FCONSTANT} -3pi/4 \\
		\word[tools]{[ELSE]} \\
		\tab ~pi~~ 4e \word{F/} \word{FCONSTANT} ~~pi/4 \\
		\tab -pi~~ 4e \word{F/} \word{FCONSTANT} ~-pi/4 \\
		\tab ~pi~~ 2e \word{F/} \word{FCONSTANT} ~~pi/2 \\
		\tab -pi~~ 2e \word{F/} \word{FCONSTANT} ~-pi/2 \\
		\tab ~pi/2 3e \word{F*} \word{FCONSTANT} ~3pi/2 \\
		\tab ~pi/4 3e \word{F*} \word{FCONSTANT} ~3pi/4 \\
		\tab -pi/4 3e \word{F*} \word{FCONSTANT} -3pi/4 \\
		\word[tools]{[THEN]}

		verbose \word{@} \word[tools]{[IF]} \\
		\tab \word[tools]{:NONAME} \word{p} -{}- fp.separate? ) \\
		\tab[2] \word{DEPTH} \word{toR} 1e \word{DEPTH} \word{Rfrom} \word{FDROP} \word{2Rfrom} \word{=} \word{;} \word{EXECUTE} \\
		\tab \word{CR} \word{.p} \textdf{floating-point and data stacks} ) \\
		\tab \word[tools]{[IF]} \word{.p} \textdf{*separate*} ) \word[tools]{[ELSE]} \word{.p} \textdf{*not separate*} ) \word[tools]{[THEN]} \\
		\tab \word{CR} \\
		\word[tools]{[THEN]}

		TESTING normal values

		\word{bs} ~~y~~~x~~~~~~~~~~~~~~rad~~~~~~~~~deg \\
		\test[R]{ 0e  1e \word{FATAN2}}{  0e  }\tab \word{bs} ~~0 \\
		\test[R]{ 1e  1e \word{FATAN2}}{  pi/4}\tab \word{bs} ~45 \\
		\test[R]{ 1e  0e \word{FATAN2}}{  pi/2}\tab \word{bs} ~90 \\
		\test[R]{-1e -1e \word{FATAN2}}{-3pi/4}\tab \word{bs} 135 \\
		\test[R]{ 0e -1e \word{FATAN2}}{  pi  }\tab \word{bs} 180 \\
		\test[R]{-1e  1e \word{FATAN2}}{ -pi/4}\tab \word{bs} 225 \\
		\test[R]{-1e  0e \word{FATAN2}}{ -pi/2}\tab \word{bs} 270 \\
		\test[R]{-1e  1e \word{FATAN2}}{ -pi/4}\tab \word{bs} 315

		TESTING Single UNIX 3 special values spec

		\word{bs} \textdf{ISO C / Single UNIX Specification Version 3:} \\
		\word{bs} \tab \url{http://www.unix.org/single_unix_specification/} \\
		\word{bs} \textdf{Select ``Topic'', then ``Math Interfaces'', then ``\texttt{atan2()}'':} \\
		\word{bs} \tab \href{http://www.opengroup.org/onlinepubs/009695399/functions/atan2f.html}
			{http://www.opengroup.org/onlinepubs/009695399/} \\
		\word{bs} \tab \href{http://www.opengroup.org/onlinepubs/009695399/functions/atan2f.html}
			{functions/atan2f.html}

		\word{bs} \textdf{If $y$ is +/-0 and $x$ is < 0, +/-pi shall be returned.} \\
		\test[R]{ 0e -1e \word{FATAN2}}{ pi} \\
		\test[R]{-0e -1e \word{FATAN2}}{-pi}

		\word{bs} \textdf{If $y$ is +/-0 and $x$ is > 0, +/-0 shall be returned.} \\
		\test[R]{ 0e  1e \word{FATAN2}}{ 0e}\\
		\test[R]{-0e  1e \word{FATAN2}}{-0e}\\

		\word{bs} \textdf{If $y$ is < 0 and $x$ is +/-0, -pi/2 shall be returned.} \\
		\test[R]{-1e  0e \word{FATAN2}}{-pi/2} \\
		\test[R]{-1e -0e \word{FATAN2}}{-pi/2} \\

		\word{bs} \textdf{If $y$ is > 0 and $x$ is +/-0, pi/2 shall be returned.} \\
		\test[R]{ 1e  0e \word{FATAN2}}{pi/2} \\
		\test[R]{ 1e -0e \word{FATAN2}}{pi/2} \\

		TESTING Single UNIX 3 special values optional spec

		\word{bs} \textdf{Optional ISO C / single UNIX specs:}

		\word{bs} \textdf{If either $x$ or $y$ is NaN, a NaN shall be returned.} \\
		\test[R]{NaN  1e \word{FATAN2}}{NaN}\\
		\test[R]{ 1e NaN \word{FATAN2}}{NaN}\\
		\test[R]{NaN NaN \word{FATAN2}}{NaN}

		\word{bs} \textdf{If $y$ is +/-0 and $x$ is -0, +/-pi shall be returned.} \\
		\test[R]{ 0e -0e \word{FATAN2}}{ pi}\\
		\test[R]{-0e -0e \word{FATAN2}}{-pi}

		\word{bs} \textdf{If $y$ is +/-0 and $x$ is +0, +/-0 shall be returned.} \\
		\test[R]{ 0e  0e \word{FATAN2}}{+0e}\\
		\test[R]{-0e  0e \word{FATAN2}}{-0e}

		\word{bs} \textdf{For finite values of +/-$y$ > 0, if $x$ is -Inf, +/-pi shall be returned.} \\
		\test[R]{ 1e -Inf \word{FATAN2}}{ pi}\\
		\test[R]{-1e -Inf \word{FATAN2}}{-pi}

		\word{bs} \textdf{For finite values of +/-$y$ > 0, if $x$ is +Inf, +/-0 shall be returned.} \\
		\test[R]{ 1e +Inf \word{FATAN2}}{+0e}\\
		\test[R]{-1e +Inf \word{FATAN2}}{-0e}

		\word{bs} \textdf{For finite values of $x$, if $y$ is +/-Inf, +/-pi/2 shall be returned.} \\
		\test[R]{+Inf  1e \word{FATAN2}}{ pi/2}\\
		\test[R]{+Inf -1e \word{FATAN2}}{ pi/2}\\
		\test[R]{+Inf  0e \word{FATAN2}}{ pi/2}\\
		\test[R]{+Inf -0e \word{FATAN2}}{ pi/2}\\
		\test[R]{-Inf  1e \word{FATAN2}}{-pi/2}\\
		\test[R]{-Inf -1e \word{FATAN2}}{-pi/2}\\
		\test[R]{-Inf  0e \word{FATAN2}}{-pi/2}\\
		\test[R]{-Inf -0e \word{FATAN2}}{-pi/2}

		\word{bs} \textdf{If $y$ is +/-Inf and $x$ is -Inf, +/-3pi/4 shall be returned.} \\
		\test[R]{+Inf -Inf \word{FATAN2}}{ 3pi/4}\\
		\test[R]{-Inf -Inf \word{FATAN2}}{-3pi/4}

		\word{bs} \textdf{If $y$ is +/-Inf and $x$ is +Inf, +/-pi/4 shall be returned.} \\
		\test[R]{+Inf +Inf \word{FATAN2}}{ pi/4}\\
		\test[R]{-Inf +Inf \word{FATAN2}}{-pi/4}

		verbose \word{@} \word[tools]{[IF]} \\
		\tab \word{CR} \word{.p} \textdf{\#ERRORS:} ) \#errors \word{@} \word{d} \word{CR} \\
		\word[tools]{[THEN]}
	\end{testing}
\end{worddef}

% ---------------------------------------------------------

\begin{worddef}{1491}{FATANH}[f-a-tan-h]
\item \stack[F]{r_1}{r_2} or
	\stack{r_1}{r_2}

	\param{r_2} is the floating-point value whose hyperbolic tangent is
	\param{r_1}. An ambiguous condition exists if \param{r_1} is outside
	the range of -1E0 to 1E0.
\end{worddef}


\begin{worddef}{1493}{FCOS}[f-cos]
\item \stack[F]{r_1}{r_2} or
	\stack{r_1}{r_2}

	\param{r_2} is the cosine of the radian angle \param{r_1}.
\end{worddef}


\begin{worddef}{1494}{FCOSH}[f-cosh]
\item \stack[F]{r_1}{r_2} or
	\stack{r_1}{r_2}

	\param{r_2} is the hyperbolic cosine of \param{r_1}.
\end{worddef}


\begin{worddef}[FEd]{1513}{FE.{}}[f-e-dot]
\item \stack{}{} \stack[F]{r}{} or
	\stack{r}{}

	Display, with a trailing space, the top number on the
	floating-point stack using engineering notation, where the
	significand is greater than or equal to 1.0 and less than
	1000.0 and the decimal exponent is a multiple of three.

	An ambiguous condition exists if the value of \word[core]{BASE}
	is not (decimal) ten or if the character string representation
	exceeds the size of the pictured numeric output string buffer.

\see \wref{core:BASE}{BASE},
	\xref[12.3.2 Floating-point operations]{float:ops},
	\wref{floating:REPRESENT}{REPRESENT}.
\end{worddef}


\begin{worddef}{1515}{FEXP}[f-e-x-p]
\item \stack[F]{r_1}{r_2} or
	\stack{r_1}{r_2}

	Raise $e$ to the power \param{r_1}, giving \param{r_2}.
\end{worddef}


\begin{worddef}{1516}{FEXPM1}[f-e-x-p-m-one]
\item \stack[F]{r_1}{r_2} or
	\stack{r_1}{r_2}

	Raise $e$ to the power \param{r_1} and subtract one, giving \param{r_2}.

\see \rref{floating:FEXPM1}{}.

	\begin{rationale} % A.12.6.1.1516 FEXPM1
		This function allows accurate computation when its arguments
		are close to zero, and provides a useful base for the standard
		exponential functions. Hyperbolic functions such as
\textsf{\replace{ed12}{cosh}{sinh}($x$)} can be efficiently and accurately
		implemented by using \word{FEXPM1}; accuracy is lost in this
		function for small values of $x$ if the word \word{FEXP} is
		used.

		An important application of this word is in finance; say a loan
		is repaid at 15\% per year; what is the daily rate? On a computer
		with single\place{ed12}{-}precision (six decimal digit) accuracy:

		\begin{enumerate}
		\item[1.] Using \word{FLN} and \word{FEXP}:

			\word{FLN} of 1.15 = 0.139762, \\
			divide by 365 = 3.82910E-4, \\
			form the exponent using \word{FEXP} = 1.00038, and \\
			subtract one (1) and convert to percentage = 0.038\%.
		\end{enumerate}
		Thus we only have two\place{ed12}{-}digit accuracy.
		\begin{enumerate}
		\item[2.] Using \word{FLNP1} and \word{FEXPM1}:

			\word{FLNP1} of 0.15 = 0.139762, (this is the same value
			as in the first example, although with the argument closer
			to zero it may not be so) \\
			divide by 365 = 3.82910E-4, \\
			form the exponent and subtract one (1) using
			\word{FEXPM1} = 3.82983E-4, and \\
			convert to percentage = 0.0382983\%.
		\end{enumerate}
		This is full six\place{ed12}{-}digit accuracy.

		\replace{ed12}{The presence of this}{ This} \remove{ed12}{word} allows the hyperbolic functions to
		be computed with usable accuracy. For example, the hyperbolic
		sine can be defined as:

		\begin{quote}\ttfamily
			\word[core]{:} \word{FSINH}~ \word[core]{p} r1 -{}- r2 ) \\
			\tab \word{FEXPM1}~ \word{FDUP}~
				 \word{FDUP} 1.0E0 \word{F+}~ \word{F/}~ \word{F+}~
				 2.0E0 \word{F/}
			\word[core]{;}
		\end{quote}
	\end{rationale}
\end{worddef}

% ---------------------------------------------------------

\begin{worddef}{1517}{FFIELD:}[f-field-colon][X:structures]
\item \stack{n_1 "<spaces>name"}{n_2}

	Skip leading space delimiters. Parse \param{name} delimited by
	a space. \param{Offset} is the first float aligned value greater
	than or equal to \param{n_1}.  \param{n_2 = offset + 1} float.

	Create a definition for \param{name} with the execution semantics
	given below.

\execute[name]
\cbstart\patch{ed12}
	\stack{\sout{a-}addr_{\uline{1}}}{\sout{f-}addr_{\uline{2}}}
\cbend

	Add the \param{offset} calculated during the compile time action to
\cbstart\param{\sout{a-}addr_{\uline{1}}}\cbend\ 
	giving the float aligned address
\cbstart\param{\sout{f-}addr_{\uline{2}}}.\cbend

\see \wref{facility:+FIELD}{+FIELD},
	\wref{facility:BEGIN-STRUCTURE}{BEGIN-STRUCTURE}, \\
	\wref{facility:END-STRUCTURE}{END-STRUCTURE},
	\rref{facility:FIELD:}{}.
\end{worddef}

% ---------------------------------------------------------

\begin{worddef}{1553}{FLN}[f-l-n]
\item \stack[F]{r_1}{r_2} or
	\stack{r_1}{r_2}

	\param{r_2} is the natural logarithm of \param{r_1}. An ambiguous
	condition exists if \param{r_1} is less than or equal to zero.
\end{worddef}


\begin{worddef}{1554}{FLNP1}[f-l-n-p-one]
\item \stack[F]{r_1}{r_2} or
	\stack{r_1}{r_2}

	\param{r_2} is the natural logarithm of the quantity \param{r_1}
	plus one. An ambiguous condition exists if \param{r_1} is less
	than or equal to negative one.

\see \rref{floating:FLNP1}{}.

	\begin{rationale} % A.12.6.1.1554 FLNP1
		This function allows accurate compilation when its arguments
		are close to zero, and provides a useful base for the standard
		logarithmic functions. For example, \word{FLN} can be
		implemented as:

		\begin{quote}\ttfamily
			\word[core]{:} \word{FLN} ~
				1.0E0 \word{F-} ~ \word{FLNP1}
			\word[core]{;}
		\end{quote}
	\see \rref{floating:FEXPM1}{FEXPM1}.
	\end{rationale}
\end{worddef}


\begin{worddef}{1557}{FLOG}[f-log]
\item \stack[F]{r_1}{r_2} or
	\stack{r_1}{r_2}

	\param{r_2} is the base-ten logarithm of \param{r_1}. An ambiguous
	condition exists if \param{r_1} is less than or equal to zero.
\end{worddef}


\begin{worddef}[FSd]{1613}{FS.{}}[f-s-dot]
\item \stack{}{} \stack[F]{r}{} or
	\stack{r}{}

	Display, with a trailing space, the top number on the
	floating-point stack in scientific notation:
	\arg{significand}\arg{exponent}
	where:
	\begin{center}
	  \begin{tabular}{r@{ \textsf{:=} }l}
		\arg{significand} & [\textbf{-}]\arg{digit}\textbf{.}\arg{digits0} \\
		\arg{exponent}	  & \textbf{E}[\textbf{-}]\arg{digits}
	  \end{tabular}
	\end{center}

	An ambiguous condition exists if the value of \word[core]{BASE}
	is not (decimal) ten or if the character string representation
	exceeds the size of the pictured numeric output string buffer.

\see \wref{core:BASE}{BASE},
	\xref[12.3.2 Floating-point operations]{float:ops},
	\wref{floating:REPRESENT}{REPRESENT}.
\end{worddef}


\begin{worddef}{1614}{FSIN}[f-sine]
\item \stack[F]{r_1}{r_2} or
	\stack{r_1}{r_2}

	\param{r_2} is the sine of the radian angle \param{r_1}.
\end{worddef}


\begin{worddef}{1616}{FSINCOS}[f-sine-cos]
\item \stack[F]{r_1}{r_2 r_3} or
	\stack{r_1}{r_2 r_3}

	\param{r_2} is the sine of the radian angle \param{r_1}. \param{r_3}
	is the cosine of the radian angle \param{r_1}.

\see \rref{floating:FATAN2}{FATAN2}.
\end{worddef}


\begin{worddef}{1617}{FSINH}[f-cinch]
\item \stack[F]{r_1}{r_2} or
	\stack{r_1}{r_2}

	\param{r_2} is the hyperbolic sine of \param{r_1}.
\end{worddef}


\begin{worddef}{1618}{FSQRT}[f-square-root]
\item \stack[F]{r_1}{r_2} or
	\stack{r_1}{r_2}

	\param{r_2} is the square root of \param{r_1}. An ambiguous
	condition exists if \param{r_1} is less than zero.
\end{worddef}


\begin{worddef}{1625}{FTAN}[f-tan]
\item \stack[F]{r_1}{r_2} or
	\stack{r_1}{r_2}

	\param{r_2} is the tangent of the radian angle \param{r_1}. An
	ambiguous condition exists if $\cos(r_1)$ is zero.
\end{worddef}


\begin{worddef}{1626}{FTANH}[f-tan-h]
\item \stack[F]{r_1}{r_2} or
	\stack{r_1}{r_2}

	\param{r_2} is the hyperbolic tangent of \param{r_1}.
\end{worddef}

% ---------------------------------------------------------

\begin{worddef}{1627}{FTRUNC}[f-trunc][x:ftrunc]
\item \stack[F]{r_1}{r_2} or \stack{r_1}{r_2}

	Round \param{r_1} to an integral value using the ``round
	towards zero'' rule, giving \param{r_2}.

\see \xref{float:ops}, \wref{floating:FROUND}{}, \wref{floating:FLOOR}{}.

	\begin{implement}
		\word{:} \word{FTRUNC}   \word{p} r1 -{}- r2 ) \\
		\tab[3.8] \word{FDUP} \word{F0=} \word{0=} \\
		\tab \word{IF}	\tab[1] \word{FDUP} \word{F0less} \\
		\tab \word{IF}	\tab[1] \word{FNEGATE} \word{FLOOR} \word{FNEGATE} \\
		\tab \word{ELSE}\tab \word{FLOOR} \\
		\tab \word{THEN} \\
		\tab \word{THEN}\tab \word{;}
	\end{implement}

	\begin{testing}
		\texttt{SET-EXACT}

		\test{-0E          \word{FTRUNC} \word{F0=}}{<TRUE>} \\
		\test{-1E-9        \word{FTRUNC} \word{F0=}}{<TRUE>} \\
		\test{-0.9E        \word{FTRUNC} \word{F0=}}{<TRUE>} \\
		\test{-1E  1E-5 F+ \word{FTRUNC} \word{F0=}}{<TRUE>} \\
		\test[R]{0E           \word{FTRUNC}    }{ 0E  } \\    
		\test[R]{1E-9         \word{FTRUNC}    }{ 0E  } \\
		\test[R]{-1E -1E-5 F+ \word{FTRUNC}    }{-1E  } \\
		\test[R]{3.14E        \word{FTRUNC}    }{ 3E  } \\
		\test[R]{3.99E        \word{FTRUNC}    }{ 3E  } \\  
		\test[R]{4E           \word{FTRUNC}    }{ 4E  } \\
		\test[R]{-4E          \word{FTRUNC}    }{-4E  } \\
		\test[R]{-4.1E        \word{FTRUNC}    }{-4E  }
	\end{testing}
\end{worddef}

% ---------------------------------------------------------

\begin{worddef}{1628}{FVALUE}[f-value][X:fvalue]
\item \stack[F]{r}{} \stack{"<spaces>name"}{} or
	\stack{r "<spaces>name"}{}

	Skip leading space delimiters.  Parse \param{name} delimited by a
	space.  Create a definition for \param{name} with the execution
	semantics defined below, with an initial value equal to \param{r}.

	\param{name} is referred to as a ``f-value''.

\execute[name]
	\stack[F]{}{r} or \stack{}{r}

	Place \param{r} on the floating point stack.  The value of
	\param{r} is that given when \param{name} was created, until the
	phrase ``\param{r} \word{TO} \param{name}'' is executed, causing
	a new value of \param{r} to be assigned to \param{name}.

\runtime[\word{TO} \param{name}]
	\stack[F]{r}{} or \stack{r}{}

	Assign the value \param{r} to \param{name}.

\see \xref{usage:parsing}, \wref{core:TO}{}

	\begin{implement}
		\dffamily
		The implementation of \word{FVALUE} requires detailed
		knowledge of the host implementation of \word{VALUE} and
		\word{TO}.

		\begin{quote}\ttfamily
			\word{VARIABLE} \%var \\
			\word{:} \word{TO} 1 \%var \word{!} \word{;}

			\word{:} \word{FVALUE} \word{p} F: r -{}- ) \word{p} "<spaces>name" -{}- ) \\
			\tab \word{CREATE} F, \\
			\tab \word{DOES} \%var \word{@} \word{IF} \word{F!} \word{ELSE} \word{F@} \word{THEN} \\
			\tab[4] 0 \%var \word{!} \word{;}

			\word{:} \word{VALUE} \word{p} x "<spaces>name" -{}- ) \\
			\tab \word{CREATE} \word{,} \\
			\tab \word{DOES} \%var \word{@} \word{IF} ~\word{!} \word{ELSE} ~\word{@} \word{THEN} \\
			\tab[4] 0 \%var \word{!} \word{;}
		\end{quote}
	\end{implement}

	\begin{testing}\ttfamily
		\test{0e0 \word{FVALUE} Tval}{} \\
		\test[R]{Tval}{0e0} \\
		\test{1e0 \word{TO} Tval}{} \\
		\test[R]{Tval}{1e0} \\[2ex]
		\word{:} setTval Tval \word{FSWAP} \word{TO} Tval \word{;} \\
		\test[RR]{2e0 setTval Tval}{1e0 2e0} \\[2ex]
		\test{5e0 \word{TO} Tval}{} \\
		\word{:} [execute] \word{EXECUTE} \word{;} \word{IMMEDIATE} \\
		\test[R]{\word{'} Tval \word{]} [execute] \word{[}}{2e0}
	\end{testing}
\end{worddef}

% ---------------------------------------------------------

\begin{worddef}[Ftilde]{1640}{F\tilde}[f-proximate]
\item \stack{}{flag} \stack[F]{r_1 r_2 r_3}{} or
	\stack{r_1 r_2 r_3}{flag}

	If \param{r_3} is positive, \emph{flag} is true if the absolute
	value of (\param{r_1} minus \param{r_2}) is less than \param{r_3}.

	If \param{r_3} is zero, \emph{flag} is true if the
	implementation-dependent encoding of \param{r_1} and \param{r_2}
	are exactly identical (positive and negative zero are unequal if
	they have distinct encodings).

	If \param{r_3} is negative, \emph{flag} is true if the absolute value
	of (\param{r_1} minus \param{r_2}) is less than the absolute value of
	\param{r_3} times the sum of the absolute values of \param{r_1} and
	\param{r_2}.

\see \rref{floating:Ftilde}{}.

	\begin{rationale} % A.12.6.1.1640 F\tilde
		This provides the three types of ``floating point equality''
		in common use --- ``close'' in absolute terms, exact equality
		as represented, and ``relatively close''.
	\end{rationale}
\end{worddef}


\begin{worddef}{2035}{PRECISION}
\item \stack{}{u}

	Return the number of significant digits currently used by
	\word{Fd}, \word{FEd}, or \word{FSd} as \param{u}.
\end{worddef}


% ========== x:stoftos ==========

\begin{worddef*}[StoF]{2175}{S>F}[S to F][x:stoftos]
\item \stack{n}{} \stack[F]{}{r} or \stack{n}{r}

	\uline{\param{r} is the floating-point equivalent of the single-cell
	value \param{n}.  An ambiguous condition exists if \param{n}
	can not be precisely represented as a floating-point value.}

\see \uline{\wref{floating:FtoS}{F>S}.}

	\begin{implement}
		\uline{\word{:} \word{StoF} \word{p} n -{}- r )} \\
		\tab \uline{\word{StoD} \word{DtoF}} \\
		\uline{\word{;}}
	\end{implement}
\end{worddef*}

% ==============================


\begin{worddef}{2200}{SET-PRECISION}
\item \stack{u}{}

	Set the number of significant digits currently used by \word{Fd},
	\word{FEd}, or \word{FSd} to \param{u}.
\end{worddef}


\begin{worddef}{2202}{SF!}[s-f-store]
\item \stack{sf-addr}{} \stack[F]{r}{} or
	\stack{r sf-addr}{}

	Store the floating-point number \param{r} as a 32-bit IEEE
	single-precision number at \param{sf-addr}. If the significand
	of the internal representation of \param{r} has more precision
	than the IEEE single-precision format, it will be rounded using
	the ``round to nearest'' rule. An ambiguous condition exists if
	the exponent of \param{r} is too large to be accommodated by the
	IEEE single-precision format.

\see \xref[12.3.1.1 Addresses]{float:addr},
	\xref[12.3.2 Floating-point operations]{float:ops}.
\end{worddef}


\begin{worddef}{2203}{SF@}[s-f-fetch]
\item \stack{sf-addr}{} \stack[F]{}{r} or
	\stack{sf-addr}{r}

	Fetch the 32-bit IEEE single-precision number stored at
	\param{sf-addr} to the floating-point stack as \param{r} in the
	internal representation. If the IEEE single-precision
	significand has more precision than the internal representation,
	it will be rounded to the internal representation using the
	``round to nearest'' rule. An ambiguous condition exists if the
	exponent of the IEEE single-precision representation is too
	large to be accommodated by the internal representation.

\see \xref[12.3.1.1 Addresses]{float:addr},
	\xref[12.3.2 Floating-point operations]{float:ops}.
\end{worddef}


\begin{worddef}{2204}{SFALIGN}[s-f-align]
\item \stack{}{}

	If the data-space pointer is not single-float aligned, reserve
	enough data space to make it so.

\see \xref[12.3.1.1 Addresses]{float:addr}.
\end{worddef}


\begin{worddef}{2206}{SFALIGNED}[s-f-aligned]
\item \stack{addr}{sf-addr}

	\param{sf-addr} is the first single-float-aligned address greater
	than or equal to \param{addr}.

\see \xref[12.3.1.1 Addresses]{float:addr}.
\end{worddef}

% ---------------------------------------------------------

\begin{worddef}{2206}[40]{SFFIELD:}[s-f-field-colon][X:structures]
\item \stack{n_1 "<spaces>name"}{n_2}

	Skip leading space delimiters. Parse \param{name} delimited by
	a space. \param{Offset} is the first single-float aligned value
	greater than or equal to \param{n_1}.  \param{n_2 = offset + 1}
	single-float.

	Create a definition for \param{name} with the execution semantics
	given below.

\execute[name]
\cbstart\patch{ed12}
	\stack{\sout{a-}addr_{\uline{1}}}{\sout{sf-}addr_{\uline{2}}}
\cbend

	Add the \param{offset} calculated during the compile time action to
\cbstart\param{\sout{a-}addr_{\uline{1}}}\cbend\ 
	giving the single-float aligned address
\cbstart\param{\sout{sf-}addr_{\uline{2}}}.\cbend

\see \wref{facility:+FIELD}{+FIELD},
	\wref{facility:BEGIN-STRUCTURE}{BEGIN-STRUCTURE}, \\
	\wref{facility:END-STRUCTURE}{END-STRUCTURE},
	\rref{facility:FIELD:}{}.
\end{worddef}

% ---------------------------------------------------------

\begin{worddef}{2207}{SFLOAT+}[s-float-plus]
\item \stack{sf-addr_1}{sf-addr_2}

	Add the size in address units of a 32-bit IEEE single-precision
	number to \param{sf-addr_1}, giving \param{sf-addr_2}.

\see \xref[12.3.1.1 Addresses]{float:addr}.
\end{worddef}


\begin{worddef}{2208}{SFLOATS}[s-floats]
\item \stack{n_1}{n_2}

	\param{n_2} is the size in address units of \param{n_1} 32-bit IEEE
	single-precision numbers.

\see \xref[12.3.1.1 Addresses]{float:addr}.
\end{worddef}
