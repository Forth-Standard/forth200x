\chapter{The optional Memory-Allocation word set} % 14
\wordlist{memory}

\begin{intro}
\section{The optional Memory-Allocation word set} % A.14

The Memory-Allocation word set provides a means for acquiring memory
other than the contiguous data space that is allocated by
\word[core]{ALLOT}. In many operating system environments it is
inappropriate for a process to pre-allocate large amounts of contiguous
memory (as would be necessary for the use of \word[core]{ALLOT}). The
Memory-Allocation word set can acquire memory from the system at any
time, without knowing in advance the address of the memory that will
be acquired.
\end{intro}

\section{Introduction} % 14.1

\section{Additional terms and notation} % 14.2
None.

\section{Additional usage requirements} % 14.3

\subsection{I/O Results data type} % 14.3.1
\label{mem:ior}

I/O results are single-cell numbers indicating the result of I/O
operations. A value of zero indicates that the I/O operation
completed successfully; other values and their meanings are
implementation-defined.

Append table \ref{mem:types} to table \ref{table:datatypes}.

\begin{table}[h]
  \begin{center}
	\caption{Data types}
	\label{mem:types}
	\begin{tabular}{llr}
	\hline\hline
	\emph{Symbol} & \emph{Data type} & \emph{Size on stack} \\
	\hline
	\emph{ior}		& I/O results			& 1 cell \\
	\hline\hline
	\end{tabular}
  \end{center}
\end{table}


\subsection{Environmental queries} % 14.3.2

Append table \ref{mem:env} to table \ref{table:env}.

See: \xref[3.2.6 Environmental queries]{usage:env}.

\begin{table}[ht]
  \begin{center}
	\caption{Environmental Query Strings}
	\label{mem:env}
	\begin{tabular}{p{9em}rcp{0.42\textwidth}}
		\hline\hline
		\multicolumn{2}{l}{String \hfill Value data type} & Constant? & Meaning \\
		\hline
		\texttt{MEMORY-ALLOC}		& \emph{flag}		& no	&
			memory-allocation word set present \\
		\texttt{MEMORY-ALLOC-EXT}	& \emph{flag}		& no	&
			memory-allocation extensions word set present \\
		\hline\hline
	\end{tabular}
  \end{center}
\end{table}

\subsection{Allocated regions} % 14.3.3

A program may address memory in data space regions made available
by \word{ALLOCATE} or \word{RESIZE} and not yet released by \word{FREE}.

See: \xref[3.3.3 Data space]{usage:dataspace}.


\section{Additional documentation requirements} % 14.4

\subsection{System documentation} % 14.4.1

\subsubsection{Implementation-defined options} % 14.4.1.1
\begin{itemize}
\item values and meaning of ior
	(\xref[14.3.1 I/O Results data type]{mem:ior},
	 \wref{memory:ALLOCATE}{ALLOCATE},
	 \wref{memory:FREE}{FREE},
	 \wref{memory:RESIZE}{RESIZE}.
\end{itemize}

\subsubsection{Ambiguous conditions} % 14.4.1.2
\begin{itemize}
\item no additional requirements.
\end{itemize}

\subsubsection{Other system documentation} % 14.4.1.3
\begin{itemize}
\item no additional requirements.
\end{itemize}

\subsection{Program documentation} % 14.4.2
\begin{itemize}
\item no additional requirements.
\end{itemize}

\section{Compliance and labeling} % 14.5

\subsection{ANS Forth systems} % 14.5.1

The phrase ``Providing the Memory-Allocation word set'' shall be
appended to the label of any Standard System that provides all of
the Memory-Allocation word set.

The phrase ``Providing \emph{name(s)} from the Memory-Allocation
Extensions word set'' shall be appended to the label of any Standard
System that provides portions of the Memory-Allocation Extensions
word set.

The phrase ``Providing the Memory-Allocation Extensions word set''
shall be appended to the label of any Standard System that provides
all of the Memory-Allocation and Memory-Allocation Extensions word
sets.

\subsection{ANS Forth programs} % 14.5.2

The phrase ``Requiring the Memory-Allocation word set'' shall be
appended to the label of Standard Programs that require the system
to provide the Memory-Allocation word set.

The phrase ``Requiring \emph{name(s)} from the Memory-Allocation
Extensions word set'' shall be appended to the label of Standard
Programs that require the system to provide portions of the
Memory-Allocation Extensions word set.

The phrase ``Requiring the Memory-Allocation Extensions word set''
shall be appended to the label of Standard Programs that require
the system to provide all of the Memory-Allocation and
Memory-Allocation Extensions word sets.


\section{Glossary} % 14.6

\subsection{Memory-Allocation words} % 14.6.1


\begin{worddef}{0707}{ALLOCATE}
\item \stack{u}{a-addr ior}

	Allocate \param{u} address units of contiguous data space. The
	data-space pointer is unaffected by this operation. The initial
	content of the allocated space is undefined.

	If the allocation succeeds, \param{a-addr} is the aligned
	starting address of the allocated space and \param{ior} is
	zero.

	If the operation fails, \param{a-addr} does not represent a valid
	address and \param{ior} is the implementation-defined I/O result
	code.

\see \wref{core:HERE}{HERE},
	\wref{memory:FREE}{FREE},
	\wref{memory:RESIZE}{RESIZE}.
\end{worddef}


\begin{worddef}{1605}{FREE}
\item \stack{a-addr}{ior}

	Return the contiguous region of data space indicated by
	\param{a-addr} to the system for later allocation. \param{a-addr}
	shall indicate a region of data space that was previously
	obtained by \word{ALLOCATE} or \word{RESIZE}. The data-space
	pointer is unaffected by this operation.

	If the operation succeeds, \param{ior} is zero. If the operation
	fails, \param{ior} is the implementation-defined I/O result code.

\see \wref{core:HERE}{HERE},
	\wref{memory:ALLOCATE}{ALLOCATE},
	\wref{memory:RESIZE}{RESIZE}.
\end{worddef}


\begin{worddef}{2145}{RESIZE}
\item \stack{a-addr_1 u}{a-addr_2 ior}

	Change the allocation of the contiguous data space starting at
	the address \param{a-addr_1}, previously allocated by
	\word{ALLOCATE} or \word{RESIZE}, to \param{u} address units.
	\param{u} may 	be either larger or smaller than the current
	size of the region. The data-space pointer is unaffected by
	this operation.

	If the operation succeeds, \param{a-addr_2} is the aligned
	starting address of \param{u} address units of allocated memory
	and \param{ior} is zero. \param{a-addr_2} may be, but need
	not be, the same as \param{a-addr_1}. If they are not the same,
	the values contained in the region at \param{a-addr_1} are copied
	to \param{a-addr_2}, up to the minimum size of either of the
	two regions. If they are the same, the values contained in the
	region are preserved to the minimum of \param{u} or the original
	size.  If \param{a-addr_2} is not the same as \param{a-addr_1},
	the region of memory at \param{a-addr_1} is returned to the system
	according to the operation of \word{FREE}.

	If the operation fails, \param{a-addr_2} equals \param{a-addr_1},
	the region of memory at \param{a-addr_1} is unaffected, and
	\param{ior} is the implementation-defined I/O result code.

\see \wref{core:HERE}{HERE},
	\wref{memory:ALLOCATE}{ALLOCATE},
	\wref{memory:FREE}{FREE}.
\end{worddef}


\subsection{Memory-Allocation extension words} % 14.6.2

None
