% !TeX root = forth.tex

\chapter*{Proposals Process}
\label{process}
\addcontentsline{toc}{section}{Proposals Process}
\markboth{Proposals Process}{Proposals Process}

In developing a standard it is necessary for the standards committee to
know what the system implementors and the programmers are already doing
in that area, and what they would be willing to do, or wish for.

To that end we have introduced a system of consultation with the Forth
community:

\begin{enumerate}
\item
	A proponent of an extension or change to the standard writes a
	proposal.

\item
	The proponent publishes the proposal as an \emph{RfD} (Request for
	Discussion) by sending a copy to the \texttt{forth200x@yahoogroups.com}
	email list \emph{and} to the \texttt{comp.lang.forth} usenet news group
	where it can be discussed.  The maintainers of the
	\texttt{www.forth200x.org} web site will then place a copy of the
	proposal on that web site.

	Be warned, this will generate a lot of heated discussion.

	In order for the results to be available in time for a standards
	meeting, an RfD should be published at least 12 weeks before the
	next meeting.

	If a proposal does not propose extensions or changes to the Forth
	language, but a rewording of the current document, there is nothing
	for a system implementor to implement, or a programmer to use.  In
	such a case, the proposal should be published as a Request for Comment
	(RfC). The proposal will be considered, along with any comments, at
	the next committee meeting.

\item
	The proponent can modify the proposal, taking any comments into
	consideration.  Where comments have been
	\replace{ed14}{dismiss}{ dismissed},
	both the comment
	and the reasons for its dismissal should be given.  The revised
	proposal is published as a revised RfD/RfC.

\item
	Once a proposal has settled down, it is frozen, and submitted to a
	vote taker, who then publishes a \emph{CfV} (Call for Votes) on the
	proposal. The vote taker will normally be a member of the standards
	committee.  In the poll, system implementors can state, whether
	their systems implement the proposal, or what the chances are that
	it ever will.  Similarly, programmers can state whether they have
	used something similar to the proposed extension and whether they
	would use the proposed extension once it is standardized.  The
	results of this poll are used by the standards committee when
	deciding whether to accept the proposal into the standards document.

	In order for the results to be available in time for a standards
	meeting, the CfV should be started at least 6 weeks before that
	meeting.

\item
	One to two weeks after publishing the CfV, the vote taker will
	publish a \emph{Current Standings}.  Note that the poll will remain
	open, especially for information on additional systems, and the
	results will be updated on the Forth200\emph{x} web page.  The
	results considered at a standards meeting are those from four weeks
	prior to that meeting.  If no poll results are available by that
	deadline, the proposal will be considered at a later meeting.

\item
	A proposal will only be accepted into the new basis document by
	consensus of those present at a standards meeting. If you can
	not attend a meeting, you should ask somebody who is attending
	to champion the proposal on your behalf.
\end{enumerate}

Should a contributor consider their comments to have been dismissed
without due consideration, they are encouraged to submit a counter
proposal.

Proposals which have passed the poll will be integrated into the basis
document in preparation for the approaching standards committee meeting.
Proposals often require some rewording in this process, so the proponent
should work with the editor to integrate the proposal into the document.

A proposal should give a rationale for the proposal, so that system
implementors and programmers may see the relevance of the proposal and
why they should adopt (and vote for) it. The proposal should include
the following sections, where appropriate.

\begin{description}
\item[Author:] ~\\
	The name of the author(s) of the proposal.

\item[Change Log:] ~\\
	A list of changes to the last published edition on the proposal.

\item[Problem:] ~\\
	This states what problem the proposal addresses.

\item[Solution:] ~\\
	An informal description of the proposed solution to the problem
	identified by the proposal.

\item[Typical use:] ~\\
	Shows a typical use of the word/feature you propose; this should
	make the formal wording easier to understand.

\item[Remarks:] ~\\
	This gives the rationale for specific decisions you have taken in
	the proposal (often in response to comments in the RfD phase), or
	discusses specific issues that have not been decided yet.

\item[Proposal:] ~\\
	This is the formal or normative part of the proposal and should be
	as well specified as possible.

	Some issues could be left undecided in the initial RfDs, leaving
	the issue open for discussion.  These issues should be mentioned
	in the Remarks section as well as in the Proposal section.

	If you want to leave something open to the system implementor, make
	that explicit, e.g., by making it an ambiguous condition.

	For the wording of word definitions, it is normally a good idea to
	take your inspiration from existing word definitions in the basis
	document.  Where possible you should include the rationale for the
	definition.  Should a proposal be accepted where no rationale has
	been provided, the editor will construct a rationale from other
	parts of the proposal.  The proponent should work with the editor
	in the development of this rationale.

\item[Reference implementation:] ~\\
	This makes it easier for system implementors to adopt your proposal.
	Where possible they should be provided in standard Forth, as defined
	by this document.
	Where this is not possible, system specific knowledge is required or
	non standard words are used, this should be documented.

\item[Testing:] ~\\
	This should test the feature/words you propose, in particular, it
	should test boundary conditions.  Where possible test cases should
	be written to conform with John Hayes tester.f test harness%
	\ifrelease.\else, see Appendix \ref{annex:test}.\fi

\item[Experience:] ~\\
	Indicate where the proposal has already been implemented and/or
	used.

\item[Comments:] ~\\
	Initially this is blank.  As comments are made on the proposal,
	they should be incorporated into the proposal.  Comment which can
	not be incorporated should be included in this section.  A response
	to the comment may be included after the comment itself.

\item[Instructions for responding to the poll:] ~\\
	Once the proposal enters the CfV stage, the vote taker will add
	these instructions to the proposal.
\end{description}
