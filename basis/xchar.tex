\chapter{The optional Extended Characters wordset} % 18
\wordlist{xchar}

\section{Introduction} % 18.1

This wordset deals with variable width character encodings.  It also
works with fixed width encodings.

Since the standard specifies ASCII encoding for characters, only
ASCII-compatible encodings may be used.  Because ASCII compatibility
has so many benefits, most encodings actually are ASCII compatible.
The characters beyond the ASCII encoding are called ``extended
characters'' (xchars).

All words dealing with strings shall handle xchars when the xchar
wordset is present.  This includes dictionary definitions, so the
dictionary entries should not use bit 8 for other purposes.  White
space parsing does not have to treat code points greater than \$20
as white space.

\section{Additional terms and notation} % 18.2

\subsection{Definition of Terms} % 18.2.1

\begin{description}
\item[code point:] A member of an extended character set.
\end{description}

\subsection{Parsed-text notation}

Append table \ref{xchar:syntax} to table \ref{table:syntax}.

\begin{table}[ht]
	\begin{center}
		\caption{Parsed text abbreviations}
		\label{xchar:syntax}
		\begin{tabular}{ll}
		\hline\hline
			\emph{Abbreviation} & \emph{Description} \\ \hline
			\param{<xchar>} & the delimiting extended character \\
		\hline\hline
		\end{tabular}
	\end{center}
\end{table}

See: \xref{notation:parsed}.

\section{Additional usage requirements} % 18.3

\subsection{Data types} % 18.3.1

Append table \ref{xchar:types} to table \ref{table:datatypes}.

\begin{table}[ht]
	\begin{center}
		\caption{Data Types}
		\label{xchar:types}
		\begin{tabular}{llr}
			\hline\hline
			\emph{Symbol}		& \emph{Data type} & \emph{Size on stack} \\ \hline
			\emph{pchar}		& primitive character		& 1 cell \\
			\emph{xchar}		& extended character			& 1 cell \\
			\emph{xc-addr}		& xchar-aligned address		& 1 cell \\
%			\emph{xc-addr u}	& character-aligned string		& 2 cells \\
			\hline\hline
		\end{tabular}
	\end{center}
\end{table}

See: \xref{usage:data}.

\subsubsection{Extended Characters} % 18.3.1.1

An extended character (xchar) is the code point of a character within an
extended character set; on the stack it is a subset of \emph{u}.  Extended
characters are stored in memory encoded as one or more primitive characters
(pchars).

%\begin{description}
%\item[xc-addr] is the address of an xchar in memory.
%	Alignment requirements are the same as \emph{c-addr}.

%\item[xc-addr u] is a string of xchars in memory, starting at
%  \emph{xc-addr}, \emph{u} pchars long.
%\end{description}


\subsection{Environmental queries} % 18.3.2.

Append table \ref{xchar:env} to table \ref{table:env}.

\begin{table}[ht]
	\begin{center}
		\caption{Environmental Query Strings}
		\label{xchar:env}
		\begin{tabular}{p{9em}rcp{0.42\textwidth}}
			\hline\hline
			\multicolumn{2}{l}{String \hfill Value data type} & Constant? & Meaning \\
			\hline
			\texttt{XCHAR-ENCODING}		& \emph{c-addr u} & no &
				Returns a printable ASCII string that represents the encoding,
				and use the preferred MIME name (if any) or the name in the
				IANA character-set register\footnotemark[1] (RFC-1700) such
				as ``\texttt{ISO-LATIN-1}'' or ``\texttt{UTF--8}'',
				with the exception of ``\texttt{ASCII}'', where the alias
				``\texttt{ASCII}'' is preferred. \\
			\texttt{MAX-XCHAR}			& \emph{u} & no &
				Maximal value for \emph{xchar} \\
			\texttt{XCHAR-MAXMEM}		& \emph{u} & no &
				Maximal memory consumed by an \emph{xchar} in address units \\
			\hline\hline \\[-1ex]
			\multicolumn{4}{l}{\footnotemark[1]
				\url{http://www.iana.org/assignments/character-sets}} \\
		\end{tabular}
	\end{center}
\end{table}

See: \xref[3.2.6 Environmental queries]{usage:env}.

\subsection{Common encodings} % 18.3.3

Input and files are often encoded iso--latin--1 or utf--8.  The encoding
depends on settings of the computer system such as the LANG environment
variable on Unix.  You can use the system consistently only when you do
not change the encoding, or only use the ASCII subset.
The typical practice in environments requiring more than one encoding
is that the base system is ASCII only, and the character set is then
extended to specify the required encoding.

%\subsection{Exception Handling} % 18.3.5
%
%Append table \ref{xchar:throw} to table \ref{table:throw}:
%
%\begin{table}[ht]
%	\begin{center}
%		\caption{\word[exception]{THROW} code assignments}
%		\label{xchar:throw}
%		\begin{tabular}{ll}
%			\hline\hline
%			Code & Reserved for \\\hline
%			-77 & Malformed xchar
%			\\ \hline\hline
%		\end{tabular}
%	\end{center}
%\end{table}

\subsection{The Forth text interpreter} % 18.3.4

In section \xref{usage:numbers}{}, \arg{cnum} should be redefined to be:

\begin{center}
	\begin{tabular}{ll}
		\arg{cnum}		& the number is the value of \arg{xchar}
	\end{tabular}
\end{center}

\section{Additional documentation requirements} % 18.4

\subsection{System documentation} % 18.4.1

\subsubsection{Implementation-defined options} % 18.4.1.1

Terminal string IO like \word{TYPE} and \word{ACCEPT} also are extended to
work with xchars in the string.  Since Unicode input and display poses a
number of challenges like input method editors for different languages,
left-to-right and right-to-left writing, and most fonts contain only a subset
of Unicode glyphs, systems should document their capabilities.  File IO and
in-memory string handling should work transparently with xchars.

\subsubsection{Ambiguous conditions} % 18.4.1.2

\begin{itemize}
\item the data in memory does not encode a valid xchar
	(\wref{xchar:X-SIZE}{});
\item the xchar value is outside the range of allowed code points of
	the current character set used.
\item words improperly used outside
	\wref{core:num-start}{<\num} and \wref{core:num-end}{\num>}
	(\wref{xchar:XHOLD}{}).
\end{itemize}

\subsubsection{Other system documentation} % 18.4.1.3
\begin{itemize}
\item no additional requirements.
\end{itemize}

\subsection{Program documentation} % 18.4.2
\begin{itemize}
\item no additional requirements.
\end{itemize}

\section{Compliance and labeling} % 18.5

\cbstart\patch{F12}
\subsection[Forth-\snapshot systems]{Forth-\snapshot\strike{-1}{25} systems} % 18.5.1
\cbend

The phrase ``Providing the Extended Character word set'' shall be
appended to the label of any Standard System that provides all of
the Extended Character word set.

The phrase ``Providing \emph{name(s)} from the Extended Character
Extensions word set'' shall be appended to the label of any Standard
System that provides portions of the Extended Character Extensions
word set.

The phrase ``Providing the Extended Character Extensions word set''
shall be appended to the label of any Standard System that provides
all of the Extended Character and Extended Character Extensions
word sets.

\cbstart\patch{F12}
\subsection[Forth-\snapshot programs]{Forth-\snapshot\strike{-1}{25} programs} % 18.5.2
\cbend

The phrase ``Requiring the Extended Character word set'' shall be
appended to the label of Standard Programs that require the system
to provide the Extended Character word set.

The phrase ``Requiring \emph{name(s)} from the Extended Character
Extensions word set'' shall be appended to the label of Standard Programs
that require the system to provide portions of the Extended Character
Extensions word set.

The phrase ``Requiring the Extended Character Extensions word set''
shall be appended to the label of Standard Programs that require the
system to provide all of the Extended Character Exception and
Extended Character Extensions word sets.

\section{Glossary} % 18.6

\subsection{Extended Character words} % 18.6.1

\begin{worddef}{2486}[50]{X-SIZE}[][x:xchar]
\item \stack{xc-addr u_1}{u_2}

	\param{u_2} is the number of pchars used to encode the first xchar
	stored in the string \param{xc-addr u1}.  To calculate the size of
	the xchar, only the bytes inside the buffer may be accessed.  An
	ambiguous condition exists if the xchar is incomplete or malformed.

	\begin{implement}
	\word{:} \word{X-SIZE} \word{p} xc-addr u1 -{}- u2 ) \\
	\tab \word{0=} ~~~~~~~~\word{IF} \word{DROP} 0 \word{EXIT} \word{THEN} \\
	\tab \word{bs} \textdf{length of UTF-8 char starting at u8-addr (accesses only u8-addr)} \\
	\tab \word{C@} \\
	\tab \word{DUP} \$80 \word{Uless} \word{IF} \word{DROP} 1 \word{EXIT} \word{THEN} \\
	\tab \word{DUP} \$c0 \word{Uless} \word{IF} -77 \word[exception]{THROW} ~~\word{THEN} \\
	\tab \word{DUP} \$e0 \word{Uless} \word{IF} \word{DROP} 2 \word{EXIT} \word{THEN} \\
	\tab \word{DUP} \$f0 \word{Uless} \word{IF} \word{DROP} 3 \word{EXIT} \word{THEN} \\
	\tab \word{DUP} \$f8 \word{Uless} \word{IF} \word{DROP} 4 \word{EXIT} \word{THEN} \\
	\tab \word{DUP} \$fc \word{Uless} \word{IF} \word{DROP} 5 \word{EXIT} \word{THEN} \\
	\tab \word{DUP} \$fe \word{Uless} \word{IF} \word{DROP} 6 \word{EXIT} \word{THEN} \\
	\tab -77 \word[exception]{THROW} \word{;}
	\end{implement}
\end{worddef}


\begin{worddef}{2487}[10]{XC!+}[x-c-store-plus][x:xchar]
\item \stack{xchar xc-addr_1}{xc-addr_2}

	Stores the \param{xchar} at \param{xc-addr_1}.  \param{xc-addr_2}
	points to the first memory location after the stored \param{xchar}.

	\begin{implement}
	\word{:} \word{XC!+} \word{p} xchar xc-addr -{}- xc-addr' ) \\
	\tab \word{OVER} \$80 \word{Uless} \word{IF} \word{TUCK} \word{C!} \word{CHAR+} \word{EXIT} \word{THEN} \word{bs} \textdf{special case ASCII} \\
	\tab \word{toR} 0 \word{SWAP} \$3F \\
	\tab \word{BEGIN} \word{2DUP} \word{Umore} \word{WHILE} \\
	\tab[2] \word{2/} \word{toR} \word{DUP} \$3F \word{AND} \$80 \word{OR} \word{SWAP} 6 \word{RSHIFT} \word{Rfrom} \\
	\tab \word{REPEAT} \$7F \word{XOR} \word{2*} \word{OR} \word{Rfrom} \\
	\tab \word{BEGIN} \word{OVER} \$80 \word{Uless} \word{0=} \word{WHILE} \word{TUCK} \word{C!} \word{CHAR+} \word{REPEAT} \word{NIP} \\
	\word{;}
	\end{implement}
\end{worddef}


\begin{worddef}[XC!+q]{2487}[15]{XC!+?}[x-c-store-plus-query][x:xchar]
\item \stack{xchar xc-addr_1 u_1}{xc-addr_2 u_2 flag}

	Stores the \param{xchar} into the string buffer specified by
	\param{xc-addr_1 u_1}.  \param{xc-addr_2 u_2} is the remaining string
	buffer.  If the \param{xchar} did fit into the buffer, \param{flag}
	is true, otherwise \param{flag} is false, and \param{xc-addr_2 u_2}
	equal \param{xc-addr_1 u_1}.  \word{XC!+q} is safe for buffer overflows.

	\begin{implement}
	\word{:} \word{XC!+q} \word{p} xchar xc-addr u -{}- xc-addr' u' flag ) \\
	\tab \word{toR} \word{OVER} \word{XC-SIZE} \word{R@} \word{OVER} \word{Uless} \word{IF} \word{p} xchar xc-addr1 len r: u1 ) \\
	\tab[2] \word{bs} \textdf{not enough space} \\
	\tab[2] \word{DROP} \word{NIP} \word{Rfrom} \word{FALSE} \\
	\tab \word{ELSE} \\
	\tab[2] \word{toR} \word{XC!+} \word{Rfrom} \word{Rfrom} \word{SWAP} \word{-} \word{TRUE} \\
	\tab \word{THEN} \word{;}
	\end{implement}

	\begin{testing}
	\test{\$ffff \word{PAD} 4 \word{XC!+q}}{\word{PAD} 3 \word{+} 1 <TRUE>}
	\end{testing}
\end{worddef}


\begin{worddef}{2487}[20]{XC,}[x-c-comma][x:xchar]
\item \stack{xchar}{}

	Append the encoding of \param{xchar} to the dictionary.

\see \wref{core:C,}{}.

	\begin{implement}
	\word{:} \word{XC,} \word{p} xchar -{}- ) \word{HERE} \word{XC!+} DP \word{!} \word{;}
	\end{implement}
\end{worddef}


\begin{worddef}{2487}[25]{XC-SIZE}[x-c-size][x:xchar]
\item \stack{xchar}{u}

	\param{u} is the number of pchars used to encode \param{xchar} in memory.

	\begin{implement}
	\word{:} \word{XC-SIZE} \word{p} xchar -{}- n ) \\
	\tab \word{DUP} \$80 \word{Uless} \word{IF} \word{DROP} 1 \word{EXIT} \word{THEN} \word{bs} \textdf{special case ASCII} \\
	\tab \$800  2 \word{toR} \\
	\tab \word{BEGIN} \word{2DUP} U>= \word{WHILE}  5 \word{LSHIFT} \word{Rfrom} \word{1+} \word{toR} \word{DUP} \word{0=} \word{UNTIL} \word{THEN} \\
	\tab \word{2DROP} \word{Rfrom} \\
	\word{;}
	\end{implement}

	\begin{testing}
		This test assumes UTF-8 encoding is being used.

		\ttfamily
		\word{HEX} \\
		\test{     0 \word{XC-SIZE}}{1} \\
		\test{    7f \word{XC-SIZE}}{1} \\
		\test{    80 \word{XC-SIZE}}{2} \\
		\test{   7ff \word{XC-SIZE}}{2} \\
		\test{   800 \word{XC-SIZE}}{3} \\
		\test{  ffff \word{XC-SIZE}}{3} \\
		\test{ 10000 \word{XC-SIZE}}{4} \\
		\test{1fffff \word{XC-SIZE}}{4}
  \end{testing}
\end{worddef}


\begin{worddef}{2487}[35]{XC@+}[x-c-fetch-plus][x:xchar]
\item \stack{xc-addr_1}{xc-addr_2 xchar}

	Fetches the \param{xchar} at \param{xc-addr_1}.  \param{xc-addr_2}
	points to the first memory location after the retrieved \param{xchar}.

	\begin{implement}
	\word{:} \word{XC@+} \word{p} xc-addr -{}- xc-addr' u ) \\
	\tab \word{COUNT} \word{DUP} \$80 \word{Uless} \word{IF} \word{EXIT} \word{THEN}  \word{bs} \textdf{special case ASCII} \\
   \tab \$7F \word{AND} \$40 \word{toR} \\
	\tab \word{BEGIN} \word{DUP} \word{R@} \word{AND} \word{WHILE} \word{R@} \word{XOR} \\
	\tab[2] 6 \word{LSHIFT} \word{Rfrom} 5 \word{LSHIFT} \word{toR} \word{toR} \word{COUNT} \\
	\tab[2] \$3F \word{AND} \word{Rfrom} \word{OR} \\
	\tab \word{REPEAT} \word{Rfrom} \word{DROP} \\
	\word{;}
	\end{implement}

%	\begin{testing}
%		\ttfamily
%		\word{HEX} \\
%		\word{:} test-string \word{Sq}
%			\begin{CJK}{UTF8}{cyberbit}
%				恭喜发财
%			\end{CJK}!"
%		\word{;} \\
%		\word{:} TXC@ \word{XC@+} \word{SWAP} \word{;}
%
%		\test{test-string \word{DROP} TXC@ TXC@ TXC@ TXC@ TXC@ \word{DROP} \\}{606D 559C 53D1 8D22 21}
%  \end{testing}
\end{worddef}


\begin{worddef}{2487}[40]{XCHAR+}[x-char-plus][x:xchar]
\item \stack{xc-addr_1}{xc-addr_2}

	Adds the size of the xchar stored at \param{xc-addr_1} to this address,
	giving \param{xc-addr_2}.

\see \remove{ed12}{\wref{core:CHAR+}{}.}

	\begin{implement}
	\word{:} \word{XCHAR+} \word{p} xc-addr -{}- xc-addr' )  \word{XC@+} \word{DROP} \word{;}
	\end{implement}
\end{worddef}


\begin{worddef}{2488}[10]{XEMIT}[x-emit][x:xchar]
\item \stack{xchar}{}

	Prints an \param{xchar} on the terminal.

\see \wref{core:EMIT}.

	\begin{implement}
	\word{:} \word{XEMIT} \word{p} xchar -{}- ) \\
	\tab \word{DUP} \$80 \word{Uless} \word{IF} \word{EMIT} \word{EXIT} \word{THEN} \word{bs} \textdf{special case ASCII} \\
	\tab 0 \word{SWAP} \$3F \\
	\tab \word{BEGIN} \word{2DUP} \word{Umore} \word{WHILE} \\
	\tab[2] \word{2/} \word{toR} \word{DUP} \$3F \word{AND} \$80 \word{OR} \word{SWAP} 6 \word{RSHIFT} \word{Rfrom} \\
	\tab \word{REPEAT} \$7F \word{XOR} \word{2*} \word{OR} \\
	\tab \word{BEGIN} \word{DUP} \$80 \word{Uless} \word{0=} \word{WHILE} \word{EMIT} \word{REPEAT} \word{DROP}\\
	\word{;}
	\end{implement}
\end{worddef}


\begin{worddef}{2488}[30]{XKEY}[x-key][x:xchar]
\item \stack{}{xchar}

	Reads an \param{xchar} from the terminal.  This will discard all input
	events up to the completion of the \param{xchar}.

\see \wref{core:KEY}{}.

	\begin{implement}
	\word{:} \word{XKEY} \word{p} -{}- xchar ) \\
	\tab \word{KEY} \word{DUP} \$80 \word{Uless} \word{IF} \word{EXIT} \word{THEN} \word{bs} \textdf{special case ASCII} \\
	\tab \$7F \word{AND} \$40 \word{toR} \\
	\tab \word{BEGIN} \word{DUP} \word{R@} \word{AND} \word{WHILE} \word{R@} \word{XOR} \\
	\tab[2] 6 \word{LSHIFT} \word{Rfrom} 5 \word{LSHIFT} \word{toR} \word{toR} \word{KEY} \\
	\tab[2] \$3F \word{AND} \word{Rfrom} \word{OR} \\
	\tab \word{REPEAT} \word{Rfrom} \word{DROP} \word{;}
	\end{implement}
\end{worddef}


\begin{worddef}[XKEYq]{2488}[35]{XKEY?}[x-key-query][x:xchar]
\item \stack{}{flag}

	\param{Flag} is true when it's possible to do \word{XKEY} without
	blocking.  Subsequent \word[facility]{KEYq}, \word{KEY},
	\word[facility]{EKEYq}, and \word[facility]{EKEY} may be affected
	by \word{XKEYq}.

\see \wref{facility:KEYq}{}.
\end{worddef}


\subsection{Extended Character extension words} % 18.6.2
\extended

\begin{worddef}{0145}{+X/STRING}[plus-x-string][x:xchar]
\item \stack{xc-addr_1 u_1}{xc-addr_2 u_2}

	Step forward by one xchar in the buffer defined by \param{xc-addr_1 u_1}.
	\param{xc-addr_2 u_2} is the remaining buffer after stepping over the
	first xchar in the buffer.

	\begin{implement}
	\word{:} \word{+X/STRING} \word{p} xc-addr1 u1 -{}- xc-addr2 u2 ) \\
	\tab \word{OVER} \word{DUP} \word{XCHAR+} \word{SWAP} \word{-} \word[string]{/STRING} \word{;}
	\end{implement}
\end{worddef}


\begin{worddef}{0175}{-TRAILING-GARBAGE}[minus-trailing-garbage][x:xchar]
\item \stack{xc-addr u_1}{xc-addr u_2}

	Examine the last xchar in the string \param{xc-addr u_1} --- if the
	encoding is correct and it represents a full xchar, \param{u_2} equals
	\param{u_1}, otherwise, \param{u_2} represents the string without the
	last (garbled) xchar.  \word{-TRAILING-GARBAGE} does not change this
	garbled xchar.

	\begin{implement}
	\word{:} \word{-TRAILING-GARBAGE} \word{p} xc-addr u1 -{}- xc-addr u2 ) \\
	\tab \word{2DUP} \word{+} \word{DUP} \word{XCHAR-} \word{p} addr u1 end1 end2 ) \\
	\tab \word{2DUP} \word{DUP} \word{OVER} \word{OVER} \word{-} \word{X-SIZE} \word{+} \word{=} \word{IF} \word{bs} \textdf{last xchar ok} \\
	\tab[2] \word{2DROP} \\
	\tab \word{ELSE} \\
	\tab[2] \word{NIP} \word{NIP} \word{OVER} \word{-} \\
	\tab \word{THEN} \word{;}
	\end{implement}
\end{worddef}


\begin{worddef}{0895}{CHAR}[][x:xchar]
\item \stack{"<spaces>name"}{xchar}

	Skip leading space delimiters.  Parse \param{name} delimited by a space.
	Put the value of its first \param{xchar} onto the stack.

\see \wref{core:CHAR}.

	\begin{rationale}
		The behavior of the extended version of \word{CHAR} is fully backward
		compatible with \wref{core:CHAR}{}.
	\end{rationale}

	\begin{implement}
	\word{:} \word{CHAR} \word{p} "name" -{}- xchar )
		\word{BL} \word{WORD} \word{COUNT} \word{DROP} \word{XC@+} \word{NIP} \word{;}
	\end{implement}
\end{worddef}


\begin{worddef}[EKEYtoXCHAR]{1306}[60]{EKEY>XCHAR}[e-key-to-x-char][x:xchar]
\item \stack{x}{xchar true | x false}

	If the keyboard event \param{x} corresponds to an xchar, return the
	\param{xchar} and \param{true}.  Otherwise, return \param{x} and
	\param{false}.

\see \wref{facility:EKEY}{},
	\wref{facility:EKEYtoCHAR}{},
	\wref{facility:EKEYtoFKEY}{}.
\end{worddef}


\begin{worddef}{2008}{PARSE}[][x:xchar]
\item \stack{xchar "ccc<xchar>"}{c-addr u}

  Parse \param{ccc} in the input stream delimited by \param{xchar}.

	\param{c-addr} is the address (within the input buffer) and \param{u}
	is the length of the parsed string. If the parse area was empty, the
	resulting string has a zero length.

\see \xref{usage:parsing}{},
	\wref{core:PARSE}{},
	\rref{core:PARSE}{}.
\end{worddef}


\begin{worddef}{2486}[70]{X-WIDTH}[][x:xchar]
\item \stack{xc-addr u}{n}

	\param{n} is the number of monospace ASCII characters that take the same
	space to display as the xchar string \param{xc-addr u}; assuming a
	monospaced display font, i.e., xchar width is always an integer multiple
	of the width of an ASCII character.

	\begin{implement}
	\word{:} \word{X-WIDTH} \word{p} xc-addr u -{}- n ) \\
	\tab 0 \word{ROT} \word{ROT} \word{OVER} \word{+} \word{SWAP} \word{qDO} \\
	\tab[2] \word{I} \word{XC@+} \word{SWAP} \word{toR} \word{XC-WIDTH} \word{+} \\
	\tab \word{Rfrom} \word{I} \word{-} \word{+LOOP} \word{;}
	\end{implement}
\end{worddef}


\begin{worddef}{2487}[30]{XC-WIDTH}[x-c-width][x:xchar]
\item \stack{xchar}{n}

	\param{n} is the number of monospace ASCII characters that take the same
	space to display as the \param{xchar}; i.e., \param{xchar} width is always
	an integer multiple of the width of an ASCII char.

	\begin{implement}
	\word{:} wc, \word{p} n low high -{}- )  \word{1+} \word{,} \word{,} \word{,} \word{;}

	\word{CREATE} wc-table \word{bs} \textdf{derived from wcwidth source code, for UCS32} \\
	0 ~0300 ~0357 wc,\tab[2]
	0 ~035D ~036F wc,\tab[2]
	0 ~0483 ~0486 wc,\\
	0 ~0488 ~0489 wc,\tab[2]
	0 ~0591 ~05A1 wc,\tab[2]
	0 ~05A3 ~05B9 wc,\\
	0 ~05BB ~05BD wc,\tab[2]
	0 ~05BF ~05BF wc,\tab[2]
	0 ~05C1 ~05C2 wc,\\
	0 ~05C4 ~05C4 wc,\tab[2]
	0 ~0600 ~0603 wc,\tab[2]
	0 ~0610 ~0615 wc,\\
	0 ~064B ~0658 wc,\tab[2]
	0 ~0670 ~0670 wc,\tab[2]
	0 ~06D6 ~06E4 wc,\\
	0 ~06E7 ~06E8 wc,\tab[2]
	0 ~06EA ~06ED wc,\tab[2]
	0 ~070F ~070F wc,\\
	0 ~0711 ~0711 wc,\tab[2]
	0 ~0730 ~074A wc,\tab[2]
	0 ~07A6 ~07B0 wc,\\
	0 ~0901 ~0902 wc,\tab[2]
	0 ~093C ~093C wc,\tab[2]
	0 ~0941 ~0948 wc,\\
	0 ~094D ~094D wc,\tab[2]
	0 ~0951 ~0954 wc,\tab[2]
	0 ~0962 ~0963 wc,\\
	0 ~0981 ~0981 wc,\tab[2]
	0 ~09BC ~09BC wc,\tab[2]
	0 ~09C1 ~09C4 wc,\\
	0 ~09CD ~09CD wc,\tab[2]
	0 ~09E2 ~09E3 wc,\tab[2]
	0 ~0A01 ~0A02 wc,\\
	0 ~0A3C ~0A3C wc,\tab[2]
	0 ~0A41 ~0A42 wc,\tab[2]
	0 ~0A47 ~0A48 wc,\\
	0 ~0A4B ~0A4D wc,\tab[2]
	0 ~0A70 ~0A71 wc,\tab[2]
	0 ~0A81 ~0A82 wc,\\
	0 ~0ABC ~0ABC wc,\tab[2]
	0 ~0AC1 ~0AC5 wc,\tab[2]
	0 ~0AC7 ~0AC8 wc,\\
	0 ~0ACD ~0ACD wc,\tab[2]
	0 ~0AE2 ~0AE3 wc,\tab[2]
	0 ~0B01 ~0B01 wc,\\
	0 ~0B3C ~0B3C wc,\tab[2]
	0 ~0B3F ~0B3F wc,\tab[2]
	0 ~0B41 ~0B43 wc,\\
	0 ~0B4D ~0B4D wc,\tab[2]
	0 ~0B56 ~0B56 wc,\tab[2]
	0 ~0B82 ~0B82 wc,\\
	0 ~0BC0 ~0BC0 wc,\tab[2]
	0 ~0BCD ~0BCD wc,\tab[2]
	0 ~0C3E ~0C40 wc,\\
	0 ~0C46 ~0C48 wc,\tab[2]
	0 ~0C4A ~0C4D wc,\tab[2]
	0 ~0C55 ~0C56 wc,\\
	0 ~0CBC ~0CBC wc,\tab[2]
	0 ~0CBF ~0CBF wc,\tab[2]
	0 ~0CC6 ~0CC6 wc,\\
	0 ~0CCC ~0CCD wc,\tab[2]
	0 ~0D41 ~0D43 wc,\tab[2]
	0 ~0D4D ~0D4D wc,\\
	0 ~0DCA ~0DCA wc,\tab[2]
	0 ~0DD2 ~0DD4 wc,\tab[2]
	0 ~0DD6 ~0DD6 wc,\\
	0 ~0E31 ~0E31 wc,\tab[2]
	0 ~0E34 ~0E3A wc,\tab[2]
	0 ~0E47 ~0E4E wc,\\
	0 ~0EB1 ~0EB1 wc,\tab[2]
	0 ~0EB4 ~0EB9 wc,\tab[2]
	0 ~0EBB ~0EBC wc,\\
	0 ~0EC8 ~0ECD wc,\tab[2]
	0 ~0F18 ~0F19 wc,\tab[2]
	0 ~0F35 ~0F35 wc,\\
	0 ~0F37 ~0F37 wc,\tab[2]
	0 ~0F39 ~0F39 wc,\tab[2]
	0 ~0F71 ~0F7E wc,\\
	0 ~0F80 ~0F84 wc,\tab[2]
	0 ~0F86 ~0F87 wc,\tab[2]
	0 ~0F90 ~0F97 wc,\\
	0 ~0F99 ~0FBC wc,\tab[2]
	0 ~0FC6 ~0FC6 wc,\tab[2]
	0 ~102D ~1030 wc,\\
	0 ~1032 ~1032 wc,\tab[2]
	0 ~1036 ~1037 wc,\tab[2]
	0 ~1039 ~1039 wc,\\
	0 ~1058 ~1059 wc,\tab[2]
	1 ~0000 ~1100 wc,\tab[2]
	2 ~1100 ~115f wc,\\
	0 ~1160 ~11FF wc,\tab[2]
	0 ~1712 ~1714 wc,\tab[2]
	0 ~1732 ~1734 wc,\\
	0 ~1752 ~1753 wc,\tab[2]
	0 ~1772 ~1773 wc,\tab[2]
	0 ~17B4 ~17B5 wc,\\
	0 ~17B7 ~17BD wc,\tab[2]
	0 ~17C6 ~17C6 wc,\tab[2]
	0 ~17C9 ~17D3 wc,\\
	0 ~17DD ~17DD wc,\tab[2]
	0 ~180B ~180D wc,\tab[2]
	0 ~18A9 ~18A9 wc,\\
	0 ~1920 ~1922 wc,\tab[2]
	0 ~1927 ~1928 wc,\tab[2]
	0 ~1932 ~1932 wc,\\
	0 ~1939 ~193B wc,\tab[2]
	0 ~200B ~200F wc,\tab[2]
	0 ~202A ~202E wc,\\
	0 ~2060 ~2063 wc,\tab[2]
	0 ~206A ~206F wc,\tab[2]
	0 ~20D0 ~20EA wc,\\
	2 ~2329 ~232A wc,\tab[2]
	0 ~302A ~302F wc,\tab[2]
	2 ~2E80 ~303E wc,\\
	0 ~3099 ~309A wc,\tab[2]
	2 ~3040 ~A4CF wc,\tab[2]
	2 ~AC00 ~D7A3 wc,\\
	2 ~F900 ~FAFF wc,\tab[2]
	0 ~FB1E ~FB1E wc,\tab[2]
	0 ~FE00 ~FE0F wc,\\
	0 ~FE20 ~FE23 wc,\tab[2]
	2 ~FE30 ~FE6F wc,\tab[2]
	0 ~FEFF ~FEFF wc,\\
	2 ~FF00 ~FF60 wc,\tab[2]
	2 ~FFE0 ~FFE6 wc,\tab[2]
	0 ~FFF9 ~FFFB wc,\\
	0 1D167 1D169 wc,\tab[2]
	0 1D173 1D182 wc,\tab[2]
	0 1D185 1D18B wc,\\
	0 1D1AA 1D1AD wc,\tab[2]
	2 20000 2FFFD wc,\tab[2]
	2 30000 3FFFD wc,\\
	0 E0001 E0001 wc,\tab[2]
	0 E0020 E007F wc,\tab[2]
	0 E0100 E01EF wc,\\
	\word{HERE} wc-table \word{-} \word{CONSTANT} \#wc-table

	\word{bs} \textdf{inefficient table walk:}

	\word{:} \word{XC-WIDTH} \word{p} xchar -{}- n ) \\
	\tab wc-table \#wc-table \word{OVER} \word{+} \word{SWAP} \word{qDO} \\
	\tab[2] \word{DUP} \word{I} \word{2@} \word{WITHIN} \word{IF}
		\word{DROP} \word{I} 2 \word{CELLS} \word{+} \word{@} \word{UNLOOP} \word{EXIT} \word{THEN} \\
	\tab 3 \word{CELLS} \word{+LOOP} \word{DROP} 1 \word{;}
	\end{implement}

	\begin{testing}
		\test{\$606D \word{XC-WIDTH}}{2} \\
		\test{  \$41 \word{XC-WIDTH}}{1} \\
		\test{\$2060 \word{XC-WIDTH}}{0}
	\end{testing}
\end{worddef}


\begin{worddef}{2487}[45]{XCHAR-}[x-char-minus][x:xchar]
\item \stack{xc-addr_1}{xc-addr_2}

	Goes backward from \param{xc-addr_1} until it finds an xchar so that the
	size of this xchar added to \param{xc-addr_2} gives \param{xc-addr1}.
	There is an ambiguous condition when the encoding doesn't permit reliable
	backward stepping through the text.

	\begin{implement}
	\word{:} \word{XCHAR-} \word{p} xc-addr -{}- xc-addr' ) \\
	\tab \word{BEGIN}  1 \word{CHARS} \word{-} \word{DUP} \word{C@} \$C0 \word{AND} \$80 \word{ne} \word{UNTIL} \word{;}
	\end{implement}
\end{worddef}


\begin{worddef}{2488}[20]{XHOLD}[x-hold][x:xchar]
\item \stack{xchar}{}

	Adds \param{xchar} to the picture numeric output string.  An ambiguous
	condition exists if \word{XHOLD} executes outside of a \word{num-start}
	\word{num-end} delimited number conversion.

\see \wref{core:HOLD}{}.

	\begin{implement}
	\word{CREATE} xholdbuf 8 \word{ALLOT}

	\word{:} \word{XHOLD} \word{p} xchar -{}- )
		xholdbuf \word{TUCK} \word{XC!+} \word{OVER} \word{-} \word{HOLDS} \word{;}
	\end{implement}
\end{worddef}


\begin{worddef}[XSTRING-]{2495}{X\bs{}STRING-}[x-string-minus][x:xchar]
\item \stack{xc-addr u_1}{xc-addr u_2}

	Search for the penultimate xchar in the string \param{xc-addr u_1}.
	The string \param{xc-addr u_2} contains all xchars of
	\param{xc-addr u_1}, but the last.  Unlike \word{XCHAR-},
	\word{XSTRING-} can be implemented in encodings where xchar
	boundaries can only reliably detected when scanning in forward
	direction.

	\begin{implement}
	\word{:} \word{XSTRING-} \word{p} xc-addr u -{}- xc-addr u' ) \\
	\tab \word{OVER} \word{+} \word{XCHAR-} \word{OVER} \word{-} \word{;}
	\end{implement}
\end{worddef}


\begin{worddef}{2520}{[CHAR]}[bracket-char][x:xchar]
\interpret
	Interpretation semantics for this word are undefined.

\compile \stack{"<spaces>name"}{}

	Skip leading space delimiters.  Parse \param{name} delimited by a space.
	Append the run-time semantics given below to the current definition.

\runtime \stack{}{xchar}

	Place \param{xchar}, the value of the first xchar of \param{name}, on
	the stack.

\see \wref{core:[CHAR]}.

	\begin{implement}
	\word{:} \word{[CHAR]} \word{p} "name" -{}- rt:xchar ) \\
	\tab \word{CHAR} \word{POSTPONE} \word{LITERAL} \word{;} \word{IMMEDIATE}
	\end{implement}
\end{worddef}

\endinput
