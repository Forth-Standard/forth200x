\section{Forth 200X Treffen auf der EuroForth 2006}

Am Tag vor der EuroForth 2006 fand wieder einmal das Treffen des
Forth~200X Komittees statt, das am n�chsten Forth-Standard arbeitet.
Die Teilnehmer waren Willem Botha, Federico de Ceballos, Anton Ertl,
Peter Knaggs, Nick Nelson (Beobachter), Stephen Pelc, Jannus Poial
(Beobachter), und Bill Stoddart.

Der RfD/CfV-Prozess und seine Beziehung zum Treffen wurde diskutiert,
und folgender Zeitplan festgelegt:

\begin{itemize}
\item Ein RfD sollte mindestens 12 Wochen vor dem Treffen erstmals
ver�ffentlicht werden, um die anderen Termine wahrscheinlich einhalten
zu k�nnen.

\item Der CfV muss sp�testens 6 Wochen vor dem Treffen eingereicht
werden, bei dem er besprochen werden soll.

\item Der Wahlleiter muss den Zwischenstand sp�testens 4 Wochen vor
dem Treffen ver�ffentlichen.

\item Beim Treffen wird dann �ber den Vorschlag abgestimmt.  Die
Mehrheit entscheidet.
\end{itemize}

Die Kontakt-Daten der Kommittee-Mitglieder werden ver�ffentlicht,
damit sich die Forth-Gemeinde an sie wenden kann und nicht selbst zum
Treffen kommen muss.

Peter Knaggs, der Editor des Standard-Dokuments, stellte das sehr
gelungene Dokument vor, das als \latex-Quellcode und in PDF-Form
vorlag (und in beiden Formen ver�ffentlicht werden wird).  Es kann
u.a. in einer Inline-Version erzeugt werden, wo bei jedem Wort die
Begr�ndung direkt dabei steht statt in einem Anhang.  Nat�rlich gab es
immer noch Verbesserungsvorschl�ge.

Ein zentraler Punkt war nat�rlich die Diskussion der Vorschl�ge, die
den RfD/CfV durchlaufen hatten (siehe
@url{http://www.forth200x.org/rfds.html}):

\begin{description}

\item[EKEY return values] Hierbei geht es darum, dass man mit den von
EKEY gelieferten Werten auch etwas anfangen kann, z.B. erkennen, ob
eine Cursor-Taste gedr�ckt wurde und welche.  Bei diesem Vorschlag
verlangte das Kommittee noch eine Klarstellung, welche Beziehung
zwischen EKEY und KEY herrscht, ansonsten wurde der Vorschlag
akzeptiert.

\item[FP-stack] Forth-Systeme sollen garantieren, dass es einen
separaten Gleitkomma-Stack gibt.  Bei diesem Vorschlag verlangte das
Kommittee, dass er so umformuliert werden soll, sodass die
Auswirkungen auf existierende Programme und Systeme deutlicher werden.

\item[One-time file loading] Dieser Vorschlag standardisiert das Wort
\texttt{required}, das so funktioniert wie \texttt{included}, wenn die
Datei noch nicht geladen wurde, aber nichts tut, wenn sie schon
geladen wurde; zus�tzlich werden noch \texttt{include} und
\texttt{require} standardisiert.  Dieser Vorschlag wurde angenommen.

\end{description}

Ausserdem wurden noch Vorschl�ge diskutiert, die noch nicht die
CfV-Stufe erreicht hatten.  U.a. resultierte daraus eine
Umformulierung von \texttt{TO}, sodass k�nftige �nderungen
(\texttt{2VALUE}, \texttt{FVALUE}) leichter gemacht werden k�nnen.

Weiters wurde beschlossen, dass der Sitzungsbericht in diversen
Papiermedien ver�ffentlicht werden sollte (wichtig f�r eine offizielle
Standardisierungsorganisation), was u.a. hiermit geschieht.

Das n�chste Treffen findet wieder am Tag vor der EuroForth statt, und
Ihr seid herzlich eingeladen, ihm beizuwohnen.
