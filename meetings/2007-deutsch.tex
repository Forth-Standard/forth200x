\section{Forth 200X Treffen auf der EuroForth 2007}

Am Tag vor der EuroForth 2007 in Schloss Dagstuhl fand wieder einmal
das Treffen des Forth~200X Komittees statt, das am n�chsten
Forth-Standard arbeitet.  Die Teilnehmer waren Sergey N. Baranov,
Willem Botha, Federico de Ceballos, Anton Ertl, Ulrich Hoffmann, Peter
Knaggs, Dagobert Michelsen, Bernd Paysan, Stephen Pelc, und Carsten
Strotmann.

Der Grossteil der Diskussion drehte sich nat�rlich um die
Vorschl�ge f�r Forth-Erweiterungen, die den RfD/CfV-Prozess
durchlaufen hatten (siehe @url{http://www.forth200x.org/rfds.html}).
Die folgenden Vorschl�ge wurden beschlossen:

\begin{description}

\item[Structures] Diese Erweiterung erleichtert es, Datenstrukturen
  mit benannten Feldern zu definieren, �hnlich wie \code{struct}s in C.

\item[Throw IORs] Dieser Vorschlag definiert einige weitere Werte, die
  ein System bei bestimmten Fehlern mit \code{throw} werfen kann.

\item[EKEY return values] Mit dieser Erweiterungen k�nnen
  Standard-Programme bestimmen, ob z.B. eine Cursor-Taste gedr�ckt
  wurde.  Nachdem dieser Vorschlag beim Treffen 2006 noch f�r eine
  Klarstellung zur�ckgestellt worden war, wurde er diesmal angenommen
  (die Klarstellung wurde allerdings beim Treffen vor der Forth-Tagung
  2009 noch einmal �berarbeitet).

\item[Number Prefixes] standardisiert die Schreibweise von Zahlen
  o.�. mit einem Prefix, der die Basis angibt, z.B. \texttt{\$1f} f�r
  einen Hex-Wert.

\end{description}

Weiters wurden noch zwei �nderungsvorschl�ge angenommen, die den
Text des Standarddokuments �ndern, ohne den Inhalt signifikant zu
�ndern, darunter eine �nderung bei der Definition von
\texttt{to}, die k�nftige Erweiterungen (z.B. \code{2value})
einfacher machen soll.

Schlie�lich wurde noch der Vorschlag \textbf{separate fp stack}
diskutiert, der den separaten Gleitkomma-Stack zum Standard erkl�ren
soll.  Dieser Vorschlag wurde bei diesem Treffen noch zur�ckgestellt,
um eine Beschreibung der Auswirkungen in das Standard-Dokument
einzuarbeiten, und wurde dann beim Treffen 2008 angenommen.

Au�erdem wurden bei dem Treffen noch Vorschl�ge diskutiert, die noch
nicht die CfV-Stufe erreicht hatten:

\begin{description}

\item[Enhanced locals] schlagen einerseits eine bessere Syntax f�r
  locals vor, andererseits auch Erweiterungen wir lokale Buffer.

\item[Escaped strings] Hier geht es um M�glichkeiten, Sonderzeichen
  oder \texttt{"} in String-Literale einzubauen.

\item[Synonyms] erlauben es, einen neuen Namen f�r ein existierendes
  Wort zu definieren.

\item[Extended characters] Die xchar-Erweiterung erlaubt die
  Verwendung von Zeichenkodierungen wie UTF-8 (siehe VD 1/2006, S. 19).

\item[2VALUE, FVALUE] �hnlich wie VALUE f�r andere Datentypen.

\item[Directories] Wie referenziert man eine andere Datei innerhalb
  des gleichen Forth-Programms?

\end{description}

Weiters wurden noch m�gliche zuk�nftige RfDs pr�sentiert.

Zus�tzlich gab's noch Diskussionen �ber das vorl�ufige
Standard-Dokument und �ber diverse organisatorische Themen.

Wer's genauer wissen will, findet noch die Minutes von Peter Knaggs
und meinen englischen Bericht auf
\url{http://www.forth200x.org/forth200x.html}.
