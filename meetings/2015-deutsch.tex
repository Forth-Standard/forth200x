\section{Forth 200X Treffen auf der EuroForth 2015}

An den beiden Tagen vor der EuroForth 2015 in Bath fand wieder einmal
das Treffen des Forth~200X Komittees statt, das am n�chsten
Forth-Standard arbeitet.

Diesmal standen zwei Erweiterungsvorschl�ge zur Abstimmung:

\textbf{2s-Complement Wrap-Around Integers} legt die
Zweierkomplementdarstellung als einzige Darstellung f�r negative ganze
Zahlen fest (bisher waren theoretisch auch Einerkomplement und
Sign-Magnitude standard, wurden aber nicht genutzt).  Au�erdem soll
bei einem �berlauf vieler ganzzahliger Operationen das Ergebnis
mittels Modulo-Arithmetik berechnet werden, wie das die Forth-Systeme
auch jetzt schon machen (bisher war �berlauf eine \emph{ambigouous
  condition}, ein Standard-Programm durfte also keinen �berlauf
haben).  Die vorgeschlagene �nderung erm�glicht es, Hash-Funktionen
und Kryptographie in Standard-Forth ohne Verrenkungen zu
programmieren.  Dieser Vorschlag wurde einstimmig angenommen.

\textbf{1 chars = 1} legt fest, dass ein Character\footnote{Hierbei
  sind \emph{Characters} im Sinne einer kleinen Speichereinheit
  gemeint, darstellbare Zeichen werden k�nnen seit dem xchar-Wordset
  von Forth-2012 mit mehreren Characters repr�sentiert werden.}  genau
eine address unit (typischerweise ein Byte, auf wort-adressierten
Maschinen eine Cell) gro� ist.  Dadurch w�ren Forth-Systeme mit
address units $<8$ bits nicht mehr standard (solche Systeme gab es
aber ohnehin nicht), und auch nicht Forth-Systeme mit Bytes als
address units und 16-bit Characters (da gab es nur experimentelle
Forth-Systeme); daf�r w�rden viele Programme Standard, in denen
\code{chars} vergessen oder absichtlich weggelassen wurde.  Die
urspr�ngliche Idee von \code{chars} war wohl, dass die Welt in
Richtung 16-bit Zeichen gehen w�rde, was aber durch Unicode 2.0 (f�r
das 16 bits nicht mehr ausreichen) und die Erfindung von UTF-8 (sodass
man auch mit 8-bit-Einheiten Unicode darstellen kann) �berholt war; in
Forth-2012 spiegelt sich diese Entwicklung in der Einf�hrung des
xchar-Wordsets wider, die M�glichkeit zu 16-bit Characters auf
byte-adressierten Maschinen w�rde mit dem vorliegenden Vorschlag
entfernt.

Bei der Abstimmung zum Vorschlag \emph{1 chars = 1} gab es drei
Enthaltungen, weil diese Mitglieder sich noch nicht ausreichend klar
�ber das Thema waren; da das nicht f�r einen Konsens spricht, haben
wir den Vorschlag als vorerst nicht akzeptiert eingestuft und werden
ihn n�chstes Jahr erneut aufgreifen.

Weiters wurden noch unfertige Vorschl�ge diskutiert, allen voran
\textbf{Quotations} und \textbf{Recognizers}. Bei beiden wollte das
Kommittee auf mehr Erfahrungen warten.  Bei den meisten �lteren
Vorschl�gen gab es keine Fortschritte.

Neben diesen konkreten Vorschl�gen drehte sich in diesem Treffen viel
darum, wie wir den Standard k�nftig entwickeln wollen:

Gerald Wodni stellte die Web-Version des Standards vor
\url{http://forth-standard.org}, die in Zukunft zu einer
Web~2.0-Version werden soll, wo Benutzer Fragen zu einem Wort stellen
k�nnen, Beispiel daf�r posten k�nnen, oder �nderungsvorschl�ge posten
k�nnen.

Er stellte auch die �berarbeitete Form des Paket-Repositories
\url{http://theforth.net} vor, und wir gaben da noch einige Anregungen
zum Format der Paket-Metadaten.

Wir wollen auch ausprobieren, an k�nftigen elektronischen
Standardisierungstreffen auch Beobachter (oder genauer Zuh�rer)
teilnehmen zu lassen, die nicht Mitglieder des Kommittees sind.
Solche Treffen finden gelegentlich statt, um Dinge abzuschlie�en, die
beim physischen Treffen angefangen wurden, aber wo zum Beispiel die
genaue Formulierung nicht fertig war.

Das n�chste Treffen findet in den zwei Tagen vor der EuroForth 2016 in
Konstanz statt.  Ihr seid herzlich eingeladen, diesem Treffen
beizuwohnen.
